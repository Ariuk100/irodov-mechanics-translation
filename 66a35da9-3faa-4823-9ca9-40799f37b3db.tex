% This LaTeX document needs to be compiled with XeLaTeX.
\documentclass[10pt]{article}
\usepackage[utf8]{inputenc}
\usepackage{graphicx}
\usepackage[export]{adjustbox}
\graphicspath{ {./images/} }
\usepackage{hyperref}
\hypersetup{colorlinks=true, linkcolor=blue, filecolor=magenta, urlcolor=cyan,}
\urlstyle{same}
\usepackage{amsmath}
\usepackage{amsfonts}
\usepackage{amssymb}
\usepackage[version=4]{mhchem}
\usepackage{stmaryrd}
\usepackage{caption}
\usepackage{fvextra, csquotes}
\usepackage{multirow}
\usepackage[fallback]{xeCJK}
\usepackage{polyglossia}
\usepackage{fontspec}
\usepackage{newunicodechar}
\IfFontExistsTF{Noto Serif CJK JP}
{\setCJKmainfont{Noto Serif CJK JP}}
{\IfFontExistsTF{STSong}
  {\setCJKmainfont{STSong}}
  {\IfFontExistsTF{Droid Sans Fallback}
    {\setCJKmainfont{Droid Sans Fallback}}
    {\setCJKmainfont{SimSun}}
}}

\setmainlanguage{russian}
\IfFontExistsTF{CMU Serif}
{\setmainfont{CMU Serif}}
{\IfFontExistsTF{DejaVu Sans}
  {\setmainfont{DejaVu Sans}}
  {\setmainfont{Georgia}}
}

\title{Релятивистская динамика }

\author{Ведущие редакторы Б. И. Копылов, Т. Г. Хохлова\\
Художник Н. А. Лозинская\\
Технический редактор Е. В. Денюкова\\
Компьютерная верстка: В. А. Носенко\\
Подписано 11.02.14. Формат \(60 \times 90 / 16\).\\
Усл. печ. л. 19,31.\\
Издательство «БИНОМ. Лаборатория знаний»\\
125167, Москва, проезд Аэропорта, д. 3\\
Телефон: (499) 157-5272\\
e-mail: binom@Lbz.ru, http://www.Lbz.ru}
\date{}


%New command to display footnote whose markers will always be hidden
\let\svthefootnote\thefootnote
\newcommand\blfootnotetext[1]{%
  \let\thefootnote\relax\footnote{#1}%
  \addtocounter{footnote}{-1}%
  \let\thefootnote\svthefootnote%
}

%Overriding the \footnotetext command to hide the marker if its value is `0`
\let\svfootnotetext\footnotetext
\renewcommand\footnotetext[2][?]{%
  \if\relax#1\relax%
    \ifnum\value{footnote}=0\blfootnotetext{#2}\else\svfootnotetext{#2}\fi%
  \else%
    \if?#1\ifnum\value{footnote}=0\blfootnotetext{#2}\else\svfootnotetext{#2}\fi%
    \else\svfootnotetext[#1]{#2}\fi%
  \fi
}

\newunicodechar{τ}{\ifmmode\tau\else{$\tau$}\fi}
\newunicodechar{ω}{\ifmmode\omega\else{$\omega$}\fi}
\newunicodechar{φ}{\ifmmode\varphi\else{$\varphi$}\fi}
\newunicodechar{β}{\ifmmode\beta\else{$\beta$}\fi}
\newunicodechar{𝜄}{\ifmmode\iota\else{$\iota$}\fi}
\newunicodechar{α}{\ifmmode\alpha\else{$\alpha$}\fi}
\newunicodechar{𝜂}{\ifmmode\eta\else{$\eta$}\fi}
\newunicodechar{→}{\ifmmode\rightarrow\else{$\rightarrow$}\fi}
\newunicodechar{ε}{\ifmmode\varepsilon\else{$\varepsilon$}\fi}
\newunicodechar{ρ}{\ifmmode\rho\else{$\rho$}\fi}
\newunicodechar{Σ}{\ifmmode\Sigma\else{$\Sigma$}\fi}
\newunicodechar{𝜇}{\ifmmode\mu\else{$\mu$}\fi}
\newunicodechar{π}{\ifmmode\pi\else{$\pi$}\fi}
\newunicodechar{⋅}{\ifmmode\cdot\else{$\cdot$}\fi}

\begin{document}
\maketitle
\captionsetup{singlelinecheck=false}
(i)

MEXAHUKA\\
OCHOBHST 3aKOHEI\\
\includegraphics[max width=\textwidth, alt={}, center]{66a35da9-3faa-4823-9ca9-40799f37b3db-001_273_236_685_3}\\
\includegraphics[max width=\textwidth, alt={}, center]{66a35da9-3faa-4823-9ca9-40799f37b3db-001_651_564_1048_465}

\section*{И.Е. Иродов}
\section*{МЕХАНИКА}
\section*{основные законы}
12-е издание (электронное)

Рекомендовано учебно-методическим объединением в области «Ядерные физика и технологии» в качестве учебного пособия для студентов физических специальностей высших учебных заведений

Деривативное электронное издание на основе печатного издания: Механика. Основные законы / И. Е. Иродов. - 12-е изд. -М. : БИНОМ. Лаборатория знаний, 2014. - 309 с. : ил.

И83 Механика. Основные законы [Электронный peсурс] / И. Е. Иродов. - 12-е изд. (эл.). -М. : БИНОМ. Лаборатория знаний, 2014. - 309 с. : ил.

ISBN 978-5-9963-2350-0\\
В книге рассмотрены основные законы как нерелятивистской (ньютоновской), так и релятивистской механики - законы движения и законы сохранения импульса, энергии и момента импульса. На большом количестве примеров и задач показано, как следует применять эти законы при решении различных конкретных вопросов.

Для студентов физических специальностей вузов.\\
УДК 531\\
ББК 22.2

По вопросам приобретения обращаться:\\
«БИНОМ. Лаборатория знаний»\\
Телефон: (499) 157-5272\\
e-mail: \href{mailto:binom@Lbz.ru}{binom@Lbz.ru}, \href{http://www.Lbz.ru}{http://www.Lbz.ru}

\section*{Содержание}
Предисловие ..... 5\\
Система обозначений ..... 6\\
Введение ..... 7\\
Глава 1. Основы кинематики ..... 9\\
§ 1.1. Кинематика точки ..... 9\\
§ 1.2. Кинематика твердого тела ..... 16\\
§ 1.3. Преобразование скорости и ускорения при пере- ходе к другой системе отсчета ..... 24\\
Задачи ..... 28\\
Глава 2. Основное уравнение динамики ..... 36\\
§ 2.1. Инерциальные системы отсчета ..... 36\\
§ 2.2. Основные законы ньютоновской динамики ..... 40\\
§ 2.3. Силы ..... 45\\
§ 2.4. Основное уравнение динамики ..... 48\\
§ 2.5. Неинерциальные системы отсчета. Силы инерции ..... 51\\
Задачи ..... 57\\
Глава 3. Закон сохранения импульса ..... 68\\
§ 3.1. О законах сохранения ..... 68\\
§ 3.2. Импульс системы ..... 70\\
§ 3.3. Закон сохранения импульса ..... 73\\
§ 3.4. Центр масс. Ц-система ..... 77\\
§ 3.5. Движение тела переменной массы ..... 82\\
Задачи ..... 85\\
Глава 4. Закон сохранения энергии ..... 93\\
§ 4.1. Работа и мощность ..... 93\\
§ 4.2. Консервативные силы. Потенциальная энергия ..... 98\\
§ 4.3. Механическая энергия частицы в поле ..... 108\\
§ 4.4. Потенциальная энергия системы ..... 112\\
§ 4.5. Закон сохранения механической энергии системы ..... 117\\
§ 4.6. Столкновение двух частиц ..... 126\\
§ 4.7. Механика несжимаемой жидкости ..... 136\\
Задачи ..... 143\\
Глава 5. Закон сохранения момента импульса ..... 157\\
§ 5.1. Момент импульса частицы. Момент силы ..... 157\\
§ 5.2. Закон сохранения момента импульса ..... 163\\
§ 5.3. Собственный момент импульса ..... 169\\
§ 5.4. Динамика твердого тела ..... 173\\
Задачи ..... 189\\
Глава 6. Колебания ..... 200\\
§ 6.1. Гармонические колебания ..... 200\\
§ 6.2. Сложение гармонических колебаний ..... 207\\
§ 6.3. Затухающие колебания ..... 211\\
§ 6.4. Вынужденные колебания ..... 214\\
Задачи ..... 218\\
Глава 7. Кинематика специальной теории относи- тельности ..... 224\\
§ 7.1. Трудности дорелятивистской физики ..... 224\\
§ 7.2. Постулаты Эйнштейна ..... 229\\
§ 7.3. Замедление времени и сокращение длины ..... 233\\
§ 7.4. Преобразования Лоренца ..... 243\\
§ 7.5. Следствия из преобразований Лоренца ..... 247\\
Задачи ..... 255\\
Глава 8. Релятивистская динамика ..... 262\\
§ 8.1. Релятивистский импульс ..... 262\\
§ 8.2. Основное уравнение релятивистской динамики ..... 266\\
§ 8.3. Закон взаимосвязи массы и энергии ..... 269\\
§ 8.4. Связь между энергией и импульсом частицы ..... 273\\
§ 8.5. Система релятивистских частиц ..... 277\\
Задачи ..... 285\\
Приложения ..... 293

\begin{enumerate}
  \item Движение точки в полярных координатах ..... 293
  \item О задаче Кеплера ..... 295
  \item Доказательство теоремы Штейнера ..... 297
  \item Греческий алфавит ..... 298
  \item Основные единицы СИ в механике ..... 298
  \item Формулы алгебры и тригонометрии ..... 299
  \item Таблица производных и интегралов ..... 299
  \item Некоторые сведения о векторах ..... 300
  \item Единицы механических величин в системах СИ и СГС ..... 301
  \item Десятичные приставки к названиям единиц ..... 302
  \item Некоторые внесистемные единицы ..... 302
  \item Астрономические величины ..... 303
  \item Физические постоянные ..... 303\\
Предметный указатель ..... 304
\end{enumerate}

\section*{Предисловие}
Цель этой книги - сосредоточить внимание на основных законах механики (законах движения и законах сохранения импульса, энергии и момента импульса), а также показать, как следует применять эти законы при решении различных конкретных задач. При этом автор стремился помочь студентам, приступившим к изучению физики, начать вырабатывать в себе необходимую для будущего специалиста культуру физического мышления, а также определенную смелость в самостоятельном подходе к решению проблемных задач.

Книга содержит две части: ньютоновская механика ( \(1-6\) главы); релятивистская механика ( \(7-8\) главы). В первой части законы механики рассматриваются в ньютоновском приближении, т. е. при скоростях движения, значительно меньших скорости света, во второй при скоростях, сравнимых со скоростью света.

В каждой главе сначала излагается теория соответствующего вопроса, а затем на ряде наиболее поучительных и интересных в физическом отношении примеров и задач показывается, как следует подходить к их решению. Задачи (их около 90) тесно связаны с основным текстом, часто являются его развитием и дополнением, поэтому работа над ними не менее важна, чем изучение основного текста.

Курсивом выделены важнейшие положения и термины. Петит используется для примеров и задач, а также для материала повышенной трудности (этот материал при первом чтении можно безболезненно опустить).

В настоящем издании сделаны некоторые изменения чисто технического характера, внесены небольшие дополнения и уточнения, а также исправлены замеченные опечатки.

Книга как учебное пособие рассчитана в основном на студентов первых курсов вузов с расширенной программой по курсу общей физики. Она может быть полезной и студентам старших курсов, а также преподавателям вузов.\\
И. Иродов

\section*{Система обозначений}
Векторы обозначены жирным прямым шрифтом (например \(\mathbf{r}, \mathbf{F}\) ); та же буква светлым шрифтом ( \(r, F\) ) означает модуль вектора.

Орты - единичные векторы:\\
\(\mathbf{i}, \mathbf{j}, \mathbf{k}\) - орты декартовых координат \(x, y, z\);\\
\(\mathbf{e}_{\rho}, \mathbf{e}_{\varphi}, \mathbf{e}_{z}\) - орты цилиндрических координат \(\rho, \varphi, z\);\\
\(\mathbf{n}, \boldsymbol{\tau}\) - орты нормали и касательной к траектории.\\
Средние величины заключены в угловые скобки < >, например 〈v>, \(\langle N\rangle\).

Символы \(\Delta, \mathrm{d}, \delta\) перед величинами означают:\\
\(\Delta\) - приращение величины, т. е. разность между ее конечным и начальным значениями, например \(\Delta \mathbf{r}=\mathbf{r}_{2}-\mathbf{r}_{1}, \Delta U=U_{2}-U_{1}\);

\begin{itemize}
  \item \(\Delta\) - убыль величины, т. е. разность между ее начальным и конечным значениями, например \(-\Delta \mathbf{r}=\mathbf{r}_{1}-\mathbf{r}_{2},-\Delta U=U_{1}-U_{2}\);\\
d - дифференциал, например \(\mathrm{d} \mathbf{r}, \mathrm{d} U\);\\
\(\delta\) - элементарное значение величины, например \(\delta A\) - элементарная работа;
\end{itemize}

っ - знак пропорциональности, например \(E\) ა \(a^{2}\);\\
\(\sim\) - величина порядка..., например \(l \sim 10^{-4}\) м.\\
Производная по времени от произвольной функции обозначена \(\mathrm{d} f / \mathrm{d} t\) или точкой над функцией ( \(\dot{f}\) ).

Системы отсчета обозначены курсивными буквами \(K, K^{\prime}, Ц\).\\
Ц-система - это система отсчета, связанная с центром масс и движущаяся поступательно по отношению к инерциальным системам (ее же называют системой центра инерции). Все величины в Ц-системе отмечены сверху значком \(\sim\) (тильда), например \(\tilde{\mathbf{p}}, \tilde{E}\).

\section*{Введение}
Механика - это раздел физики, в котором изучается движение тел в пространстве и времени. Тот факт, что механические явления протекают в пространстве и времени, находит свое отражение в любом механическом законе, содержащем явно или неявно про-странственно-временные соотношения - расстояния и промежутки времени.

Положение тела в пространстве может быть определено только по отношению к каким-либо другим телам. Это же относится и к движению тела, т. е. к изменению его положения с течением времени. Тело (или система неподвижных друг относительно друга тел), которое служит для определения положения интересующего нас тела, называют телом отсчета.

Практически для описания движения с телом отсчета связывают какую-нибудь систему координат, например декартову. Координаты тела позволяют установить его положение в пространстве. Так как движение происходит не только в пространстве, но и во времени, то для описания движения необходимо отсчитывать также и время. Это делается с помощью часов того или иного типа.

Совокупность тела отсчета и связанных с ним координат и синхронизированных между собой часов образует систему отсчета. Понятие системы отсчета является фундаментальным в физике. Пространственно-временное описание движения при помощи расстояний и промежутков времени возможно только тогда, когда выбрана определенная система отсчета.

Пространство и время сами являются физическими объектами, как и любые другие, однако неизмеримо более важными и существенными. Чтобы изучить свойства пространства и времени, нужно наблюдать движение тел, которые в них находятся. Исследуя характер движения тел, мы тем самым познаем и свойства пространства и времени.

Опыт показывает, что, пока скорости тел малы по сравнению со скоростью света, линейные масштабы и промежутки времени остаются неизменными при переходе от одной системы отсчета к другой, т. е. не зависят от выбора системы отсчета. Это нашло свое выражение в ньютоновской концепции абсолютности пространства и времени. Механику, изучающую движения тел именно в этих случаях, называют ньютоновской.

При переходе же к скоростям, сравнимым со скоростью света, обнаруживается, что характер движения тел радикально меняется. При этом линейные масштабы и промежутки времени уже зависят от выбора системы отсчета и в разных системах отсчета будут разными. Механику, основанную на этих представлениях, называют релятивистской. Естественно, что релятивистская механика является более общей и в частном случае малых скоростей переходит в ньютоновскую.

Реальные движения тел настолько сложны, что, изучая их, необходимо отвлечься от несущественных для рассматриваемого движения деталей (в противном случае задача так усложнилась бы, что решить ее практически было бы невозможно). С этой целью используют понятия (абстракции, идеализации), применимость которых зависит от конкретного характера интересующей нас задачи, а также от той степени точности, с которой мы хотим получить результат. Среди этих понятий большую роль играют понятия материальной точки и абсолютно твердого тела.

Материальная точка - это тело, размерами которого в условиях данной задачи можно пренебречь. Ясно, что одно и то же тело в одних случаях можно рассматривать как материальную точку, в других же - как протяженное тело.

Абсолютно твердое тело, или, короче, твердое тело, - это система материальных точек, расстояния между которыми не меняются в процессе движения. Реальное тело можно считать абсолютно твердым, если в условиях рассматриваемой задачи его деформации пренебрежимо малы.

Механика ставит перед собой две основные задачи:

\begin{enumerate}
  \item Изучение различных движений и обобщение полученных результатов в виде законов движения - законов, с помощью которых может быть предсказан характер движения в каждом конкретном случае.
  \item Отыскание общих механических свойств, т. е. общих теорем или принципов, присущих любой системе, независимо от конкретного рода взаимодействий между телами системы.
\end{enumerate}

Решение первой задачи привело к установлению Ньютоном и Эйнштейном так называемых динамических законов, решение же второй задачи - к обнаружению законов сохранения таких фундаментальных величин, как энергия, импульс и момент импульса.

Динамические законы и законы сохранения энергии, импульса и момента импульса представляют собой основные законы механики. Изучение их и составляет содержание этой книги.

\section*{Основы кинематики}
Кинематика-это раздел механики,где изучаются способы описания движений независимо от причин,обусловливающих эти движения.В этой главе рассмотрены три вопроса:кинемати- ка точки,кинематика твердого тела,преобразование скорости и ускорения при переходе от одной системы отсчета к другой.

\section*{§ 1.1.Кинематика точки}
Существует три способа описания движения точки:вектор- ный,координатный и естественный.Рассмотрим их последова- тельно.

\section*{Векторный способ}
В этом способе положение интересующей нас точки \(A\) задают радиусом-вектором \(\mathbf{r}\) ,проведенным из некоторой неподвижной точки \(O\) выбранной системы отсчета в точку \(A\) .При движении точки \(A\) ее радиус-вектор меняется в общем случае как по мо- дулю,так и по направлению,т.е.радиус-вектор \(\mathbf{r}\) зависит от времени \(t\) .Геометрическое место концов радиуса-вектора \(\mathbf{r}\) на- зывают траекторией точки \(A\) .

Введем понятие скорости точки. Пусть за промежуток времени \(\Delta t\) точка \(A\) переместилась из точки 1 в точку 2 (рис.1.1).Из рисунка видно,что век- тор перемещения \(\Delta \mathbf{r}\) точки \(A\) представ- ляет собой приращение радиуса-векто- ра \(\mathbf{r}\) за время \(\Delta t: \Delta \mathbf{r}=\mathbf{r}_{2}-\mathbf{r}_{1}\) .Отноше-

\begin{figure}[h]
\begin{center}
  \includegraphics[alt={},max width=\textwidth]{66a35da9-3faa-4823-9ca9-40799f37b3db-010_245_402_1194_758}
\captionsetup{labelformat=empty}
\caption{Рис. 1.1}
\end{center}
\end{figure}

ние \(\Delta \mathbf{r} / \Delta t\) называют средним вектором скорости 〈v〉за время \(\Delta t\) .Вектор 〈v〉 совпадает по направлению с \(\Delta \mathbf{r}\) .

Определим вектор скорости \(\mathbf{v}\) точки в данный момент време- ни как предел отношения \(\Delta \mathbf{r} / \Delta t\) при \(\Delta t \rightarrow 0\) ,т.е.


\begin{equation*}
\mathbf{v}=\lim _{\Delta t \rightarrow 0} \frac{\Delta \mathbf{r}}{\Delta t}=\frac{\mathrm{d} \mathbf{r}}{\mathrm{~d} t} \tag{1.1}
\end{equation*}


Это значит, что вектор скорости \(\mathbf{v}\) точки в данный момент времени равен производной от радиуса-вектора \(\mathbf{r}\) по времени и направлен по касательной к траектории в данной точке в сторону движения точки \(A\) (как и вектор dr). Модуль вектора \(\mathbf{v}\) равен*

\[
v=|\mathbf{v}|=|\mathrm{d} \mathbf{r} / \mathrm{d} t| .
\]

Движение точки характеризуется также ускорением. Вектор ускорения а определяет скорость изменения вектора скорости точки со временем:


\begin{equation*}
\mathbf{a}=\mathrm{d} \mathbf{v} / \mathrm{d} t, \tag{1.2}
\end{equation*}


т. е. равен производной от вектора скорости по времени. Направление вектора а совпадает с направлением вектора dv приращением вектора \(\mathbf{v}\) за время \(\mathrm{d} t\). Модуль вектора \(\mathbf{a}\) определяется аналогично модулю вектора \(\mathbf{v}\).\\
Пример. Радиус-вектор точки зависит от времени \(t\) по закону

\[
\mathbf{r}=\mathbf{A} t+\mathbf{B} t^{2} / 2,
\]

где \(\mathbf{A}\) и \(\mathbf{B}\) - постоянные векторы. Найдем скорость \(\mathbf{v}\) и ускорение а точки:

\[
\mathbf{v}=\mathrm{d} \mathbf{r} / \mathrm{d} t=\mathbf{A}+\mathbf{B} t, \quad \mathbf{a}=\mathrm{d} \mathbf{v} / \mathrm{d} t=\mathbf{B}=\text { const } .
\]

Модуль вектора скорости

\[
v=\sqrt{\mathbf{v}^{2}}=\sqrt{\mathbf{A}^{2}+2 \mathbf{A} \mathbf{B} t+\mathbf{B}^{2} t^{2}} .
\]

Таким образом, зная зависимость \(\mathbf{r}(t)\), можно найти скорость \(\mathbf{v}\) и ускорение \(\mathbf{a}\) точки в каждый момент времени.

Возникает и обратная задача: можно ли найти \(\mathbf{v}(t)\) и \(\mathbf{r}(t)\), зная зависимость от времени ускорения \(\mathbf{a}(t)\) ?

Оказывается, для получения однозначного решения этой задачи одной зависимости \(\mathbf{a}(t)\) недостаточно, необходимо еще знать начальнье условия, а именно скорость \(\mathbf{v}_{0}\) и радиус-вектор \(\mathbf{r}_{0}\) точки в некоторый начальный момент \(t=0\). Чтобы в

\footnotetext{\begin{itemize}
  \item Заметим, что в общем случае \(|\mathrm{d} \mathbf{r}| \neq \mathrm{d} r\), где \(r\) - модуль радиуса-вектора \(\mathbf{r}\) и \(v \neq \mathrm{d} r / \mathrm{d} t\). Например, если \(\mathbf{r}\) меняется только по направлению (точка движется по окружности), то \(r=\) const, \(\mathrm{d} r=0\), но \(|\mathrm{d} \mathbf{r}| \neq 0\).
\end{itemize}
}этом убедиться, рассмотрим простейший случай, когда в процессе движения ускорение точки \(\mathbf{a}=\) const.

Сначала определим скорость точки \(\mathbf{v}(t)\). Согласно (1.2), за промежуток времени \(\mathrm{d} t\) элементарное приращение скорости \(\mathrm{d} \mathbf{v}=\mathbf{a d} t\). Проинтегрировав это выражение по времени от \(t=0\) до \(t\), найдем приращение вектора скорости за это время:

\[
\Delta \mathbf{v}=\int_{0}^{t} \mathbf{a} \mathrm{~d} t=\mathbf{a} t
\]

Но величина \(\Delta \mathbf{v}\) - это еще не искомая скорость \(\mathbf{v}\). Чтобы найти \(\mathbf{v}\), необходимо знать скорость \(\mathbf{v}_{0}\) в начальный момент времени. Тогда \(\mathbf{v}=\mathbf{v}_{0}+\Delta \mathbf{v}\), или

\[
\mathbf{v}=\mathbf{v}_{0}+\mathbf{a} t .
\]

Аналогично решается вопрос и о радиусе-векторе \(\mathbf{r}(t)\) точки. Согласно (1.1), за промежуток времени \(\mathrm{d} t\) элементарное приращение радиуса-вектора \(\mathrm{d} \mathbf{r}=\mathbf{v d} t\). Интегрируя это выражение с учетом найденной зависимости \(\mathbf{v}(t)\), определим приращение ра-диуса-вектора за время от \(t=0\) до \(t\) :

\[
\Delta \mathbf{r}=\int_{0}^{t} \mathbf{v}(t) \mathrm{d} t=\mathbf{v}_{0} t+\mathbf{a} t^{2} / 2
\]

Для нахождения самого радиуса-вектора \(\mathbf{r ( t )}\) необходимо знать еще положение точки \(\mathbf{r}_{0}\) в начальный момент времени. Тогда \(\mathbf{r}=\mathbf{r}_{0}+\Delta \mathbf{r}\), или

\[
\mathbf{r}=\mathbf{r}_{0}+\mathbf{v}_{0} t+\mathbf{a} t^{2} / 2
\]

Рассмотрим, например, движение камня, брошенного под некоторым углом к горизонту с начальной скоростью \(\mathbf{v}_{0}\). Если считать, что камень движется с постоянным ускорением \(\mathbf{a}=\mathbf{g}\), то его положение относительно точки бросания ( \(\mathbf{r}_{0}=0\) ) определяется радиусом-вектором

\[
\mathbf{r}=\mathbf{v}_{0} t+\mathbf{g} t^{2} / 2
\]

\begin{figure}[h]
\begin{center}
  \includegraphics[alt={},max width=\textwidth]{66a35da9-3faa-4823-9ca9-40799f37b3db-012_275_412_1389_752}
\captionsetup{labelformat=empty}
\caption{Рис. 1.2}
\end{center}
\end{figure}

т. е. в данном случае \(\mathbf{r}\) представляет собой сумму двух векторов, что показано на рис. 1.2.

Итак, для полного решения задачи о движении точки определения ее скорости \(\mathbf{v}\) и положения \(\mathbf{r}\) в зависимости от времени - недостаточно знать зависимость \(\mathbf{a}(t)\), но еще необходимо знать и начальные условия, т. е. скорость \(\mathbf{v}_{0}\) и положение \(\mathbf{r}_{0}\) точки в начальный момент времени.

В заключение напомним, что в СИ единицами длины, скорости и ускорения являются соответственно метр (м), метр на секунду (м/с) и метр на секунду в квадрате (м/с²).

\section*{Координатный способ}
В этом способе с выбранным телом отсчета жестко связывают определенную систему координат (декартову, косоугольную или криволинейную). Выбор той или иной системы координат определяется рядом соображений: характером или симметрией задачи, постановкой вопроса, а также стремлением упростить само решение. Ограничимся здесь* декартовой системой координат \(x, y, z\).

Запишем проекции на оси \(X, Y, Z\) радиуса-вектора \(\mathbf{r}(t)\), характеризующего положение интересующей нас точки относительно начала координат \(O\) в момент \(t\) :

\[
x=x(t) ; \quad y=y(t) ; \quad z=z(t) .
\]

Зная зависимость этих координат от времени - закон движения точки, можно найти положение точки в каждый момент времени, ее скорость и ускорение. Действительно, спроецировав (1.1) и (1.2), например, на ось \(X\), получим формулы, определяющие проекции векторов скорости и ускорения на эту ось:


\begin{equation*}
v_{x}=\mathrm{d} x / \mathrm{d} t \tag{1.3}
\end{equation*}


где \(\mathrm{d} x\) - проекция вектора перемещения \(\mathrm{d} \mathbf{r}\) на ось \(X\);


\begin{equation*}
a_{x}=\frac{\mathrm{d} v_{x}}{\mathrm{~d} t}=\frac{\mathrm{d}^{2} x}{\mathrm{~d} t^{2}} \tag{1.4}
\end{equation*}


где \(d v_{x}\) - проекция вектора приращения скорости dv на ось \(X\). Аналогичные соотношения получаются для \(y\) - и \(z\)-проекций со-

\footnotetext{\begin{itemize}
  \item В приложении 1 рассмотрено движение точки в полярных координатах.
\end{itemize}
}ответствующих векторов. Из этих формул видно, что проекции векторов скорости и ускорения равны соответственно первой и второй производным координат по времени.

Таким образом, зависимости \(x(t), y(t), z(t)\), по существу, полностью определяют движение точки. Зная их, можно найти не только положение точки, но и проекции ее скорости и ускорения, а следовательно, модуль и направление векторов \(\mathbf{v}\) и \(\mathbf{a}\) любой момент времени. Например, модуль вектора скорости

\[
v=\sqrt{v_{x}^{2}+v_{y}^{2}+v_{z}^{2}},
\]

направление же вектора \(\mathbf{v}\) задается направляющими косинусами по формулам

\[
\cos \alpha=v_{x} / v, \quad \cos \beta=v_{y} / v, \quad \cos \gamma=v_{z} / v,
\]

где \(\alpha, \beta, \gamma\) - углы меду вектором \(\mathbf{v}\) и осями \(X, Y, Z\) соответственно. Аналогичными формулами определяются модуль и направление вектора ускорения.

Кроме того, можно решить и ряд других вопросов: найти траекторию точки, зависимость пройденного ею пути от времени, зависимость скорости от положения точки и пр.

Решение обратной задачи - нахождение скорости и закона движения точки по заданному ускорению - проводится, как и в векторном способе, путем интегрирования (в данном случае проекций ускорения по времени), причем задача и здесь имеет однозначное решение, если кроме ускорения заданы еще и начальные условия: проекции скорости и координаты точки в начальный момент.

\section*{«Естественный» способ}
Этот способ применяют тогда, когда траектория точки известна заранее. Положение точки \(A\) определяют дуговой координатой \(l\) - расстоянием вдоль траектории от выбранного начала отсчета \(O\) (рис. 1.3). При этом произвольно устанавливают положительное направление отсчета координаты \(l\) (например, так, как показано стрелкой на рисунке).

Движение точки определено, если известны ее траектория, начало отсчета \(O\), положительное направление отсчета дуговой координаты \(l\) и закон движения точки, т. е. зависимость \(l(t)\).

Скорость точки. Введем единичный вектор τ, связанный с движущейся точкой \(A\) и направленный по касательной к траектории в сторону возрастания дуговой координаты \(l\) (рис.1.3).

\begin{figure}[h]
\begin{center}
  \includegraphics[alt={},max width=\textwidth]{66a35da9-3faa-4823-9ca9-40799f37b3db-015_342_991_302_169}
\captionsetup{labelformat=empty}
\caption{Рис. 1.4}
\end{center}
\end{figure}

Рис. 1.3

Очевидно, что \(\boldsymbol{\tau}\) - переменный вектор: он зависит от \(l\). Вектор скорости \(\mathbf{v}\) точки \(A\) направлен по касательной к траектории, поэтому его можно представить так:


\begin{equation*}
\mathbf{v}=v_{\tau} \boldsymbol{\tau}, \tag{1.5}
\end{equation*}


где \(v_{\tau}=\mathrm{d} l / \mathrm{d} t\) - проекция вектора \(\mathbf{v}\) на направление вектора \(\boldsymbol{\tau}\), причем \(v_{\tau}\) - величина алгебраическая. Кроме того,

\[
\left|v_{\tau}\right|=|\mathbf{v}|=v .
\]

Ускорение точки. Продифференцируем (1.5) по времени:


\begin{equation*}
\mathbf{a}=\frac{\mathrm{d} \mathbf{v}}{\mathrm{~d} t}=\frac{\mathrm{d} v_{\tau}}{\mathrm{d} t} \boldsymbol{\tau}+v_{\tau} \frac{\mathrm{d} \boldsymbol{\tau}}{\mathrm{~d} t} . \tag{1.6}
\end{equation*}


Затем преобразуем второе слагаемое этого выражения:


\begin{equation*}
v_{\tau} \frac{\mathrm{d} \boldsymbol{\tau}}{\mathrm{~d} t}=v_{\tau} \frac{\mathrm{d} \boldsymbol{\tau}}{\mathrm{~d} t} \frac{\mathrm{~d} l}{\mathrm{~d} t}=v_{\tau}^{2} \frac{\mathrm{~d} \boldsymbol{\tau}}{\mathrm{~d} l}=v^{2} \frac{\mathrm{~d} \boldsymbol{\tau}}{\mathrm{~d} l} . \tag{1.7}
\end{equation*}


Определим приращение вектора \(\tau\) на участке \(\mathrm{d} l\) (рис. 1.4). Можно строго показать, что при стремлении точки 2 к точке 1 отрезок траектории между ними стремится к дуге окружности с центром в некоторой точке \(O\). Эту точку называют центром кривизнь траектории в данной точке, а радиус \(\rho\) соответствующей окружности - радиусом кривизны траектории в той же точке.

Как видно из рис. 1.4, угол \(\delta \alpha=|\mathrm{d} l| / \rho=|\mathrm{d} \tau| / 1\), откуда

\[
|\mathrm{d} \boldsymbol{\tau} / \mathrm{d} l|=1 / \rho,
\]

причем при \(\mathrm{d} l \rightarrow 0 \mathrm{~d} \boldsymbol{\tau} \perp \boldsymbol{\tau}\). Введя единичный вектор \(\mathbf{n}\) нормали к траектории в точке 1 , направленный к центру кривизны, запишем последнее равенство в векторном виде:


\begin{equation*}
\mathrm{d} \boldsymbol{\tau} / \mathrm{d} l=\mathbf{n} / \rho . \tag{1.8}
\end{equation*}


Подставим (1.8) в (1.7) и полученное выражение в (1.6). В результате найдем


\begin{equation*}
\mathbf{a}=\frac{\mathrm{d} v_{\tau}}{\mathrm{d} t} \boldsymbol{\tau}+\frac{v^{2}}{\rho} \mathbf{n} . \tag{1.9}
\end{equation*}


Здесь первое слагаемое называют тангенииальньлм ускорением, второе - нормальньлм ускорением. Таким образом, полное ускорение а точки может быть представлено как векторная сумма тангенциального и нормального ускорений.

Проекции вектора \(\mathbf{a}\) на орты \(\boldsymbol{\tau}\) и \(\mathbf{n}\), как видно из (1.9), равны


\begin{equation*}
a_{\tau}=\mathrm{d} v_{\tau} / \mathrm{d} t, \quad a_{n}=v^{2} / \rho . \tag{1.10}
\end{equation*}


Модуль полного ускорения точки

\[
a=\sqrt{a_{\tau}^{2}+a_{n}^{2}}=\sqrt{\dot{v}^{2}+\left(v^{2} / \rho\right)^{2}}
\]

где \(\dot{v}\) - производная модуля скорости по времени.

Пример. Точка \(A\) движется по дуге радиусом \(\rho\) (рис. 1.5). Ее скорость зависит от дуговой координаты \(l\) по закону \(v=k \sqrt{l}\), где \(k\) - постоянная. Найдем угол \(\alpha\) между векторами полного ускорения и скорости точки как функцию координаты \(l\).

\begin{figure}[h]
\begin{center}
  \includegraphics[alt={},max width=\textwidth]{66a35da9-3faa-4823-9ca9-40799f37b3db-016_306_428_1407_732}
\captionsetup{labelformat=empty}
\caption{Рис. 1.5}
\end{center}
\end{figure}

Из рис. 1.5 видно, что угол \(\alpha\) можно определить по формуле \(\operatorname{tg} \alpha=a_{n} / a_{\tau}\). Найдем \(a_{n}\) и \(a_{\tau}\). Нормальное ускорение

\[
a_{n}=v^{2} / \rho=k^{2} l / \rho .
\]

В нашем случае \(v_{\tau}=v\), поэтому тангенциальное ускорение

\[
a_{\tau}=\frac{\mathrm{d} v}{\mathrm{~d} t}=\frac{\mathrm{d} v}{\mathrm{~d} l} \frac{\mathrm{~d} l}{\mathrm{~d} t}=\frac{\mathrm{d} v}{\mathrm{~d} l} v .
\]

Учитывая зависимость \(v\) от \(l\), получим

\[
a_{\tau}=\frac{k}{2 \sqrt{l}} k \sqrt{l}=\frac{k^{2}}{2} .
\]

В результате \(\operatorname{tg} \alpha=2 l / \rho\).

\section*{§ 1.2. Кинематика твердого тела}
Теория движения твердого тела помимо самостоятельного значения играет важную роль еще и в другом отношении. С твердым телом, как известно, может быть связана система отсчета, служащая для пространственно-временного описания различных движений. Поэтому изучение характера движения твердых тел равносильно, по существу, изучению движений соответствующих систем отсчета. Результаты, которые мы получим в этом параграфе, будут неоднократно использоваться в дальнейшем.

Различают пять видов движения твердого тела: 1) поступательное, 2) вращение вокруг неподвижной оси, 3) плоское движение, 4) движение вокруг неподвижной точки и 5) свободное движение. Первые два движения (поступательное и вращение вокруг неподвижной оси) являются основными движениями твердого тела. Остальные виды движения твердого тела, оказывается, можно свести к одному из основных движений или к их совокупности (это будет показано на примере плоского движения).

В данном параграфе рассмотрены первые три вида движения и вопрос сложения угловых скоростей.

\section*{Поступательное движение}
Это такое движение твердого тела, при котором любая прямая, связанная с телом, все время остается параллельной свое-

му начальному положению, например вагон, движущийся по прямому участку пути; кабина колеса обозрения и др.

При поступательном движении все точки твердого тела совершают за один и тот же промежуток времени равные перемещения. Поэтому скорости и ускорения всех точек тела в данный момент времени одинаковы. Это обстоятельство позволяет свести изучение поступательного движения твердого тела к изучению движения отдельной точки тела, т. е. к задаче кинематики точки.

Таким образом, поступательное движение твердого тела может быть полностью описано, если известны зависимость от времени радиуса-вектора \(\mathbf{r}(t)\) любой точки этого тела и положение последнего в начальный момент.

\section*{Вращение вокруг неподвижной оси}
Пусть твердое тело, вращаясь вокруг неподвижной в данной системе отсчета оси \(O O^{\prime}\), совершило за время \(\mathrm{d} t\) бесконечно малый поворот. Соответствующий угол поворота будем характеризовать вектором \(\mathrm{d} \boldsymbol{\varphi}\), модуль которого равен углу поворота, а направление совпадает с осью \(O O^{\prime}\), причем так, что направление поворота отвечает правилу правого винта по отношению к направлению вектора \(\mathrm{d} \boldsymbol{\varphi}\) (рис. 1.6).

Теперь найдем элементарное перемещение любой точки \(A\) твердого тела при таком повороте. Положение точки \(A\) зададим радиусом-вектором \(\mathbf{r}\), проведенным из некоторой точки \(O\) на оси вращения. Тогда линейное перемещение конца радиуса-вектора \(\mathbf{r}\)

\begin{figure}[h]
\begin{center}
  \includegraphics[alt={},max width=\textwidth]{66a35da9-3faa-4823-9ca9-40799f37b3db-018_451_408_992_750}
\captionsetup{labelformat=empty}
\caption{Рис. 1.6}
\end{center}
\end{figure}

\[
|\mathrm{d} \mathbf{r}|=r \sin \vartheta \mathrm{~d} \varphi,
\]

или в векторном виде


\begin{equation*}
\mathrm{d} \mathbf{r}=[\mathrm{d} \boldsymbol{\varphi}, \mathbf{r}] . \tag{1.11}
\end{equation*}


Отметим, что это равенство справедливо лишь для бесконечно малого поворота d \(\boldsymbol{\varphi}\). Другими словами, только бесконечно малые повороты можно рассматривать как векторы*.

Кроме того, введенный нами вектор d \(\boldsymbol{\varphi}\) удовлетворяет основному свойству векторов - векторному сложению. В самом деле, пусть твердое тело совершает два элементарных поворота \(\mathrm{d} \boldsymbol{\varphi}_{1}\) и \(\mathrm{d} \boldsymbol{\varphi}_{2}\) вокруг разных осей, проходящих через неподвижную точку \(O\). Тогда результирующее перемещение \(\mathrm{d} \mathbf{r}\) произвольной точки \(A\) тела, радиус-вектор которой относительно точки \(O\) равен \(\mathbf{r}\), можно представить так:

\[
\mathrm{d} \mathbf{r}=\mathrm{d} \mathbf{r}_{1}+\mathrm{d} \mathbf{r}_{2}=\left[\mathrm{d} \boldsymbol{\varphi}_{1}, \mathbf{r}\right]+\left[\mathrm{d} \boldsymbol{\varphi}_{2}, \mathbf{r}\right]=[\mathrm{d} \boldsymbol{\varphi}, \mathbf{r}],
\]

где


\begin{equation*}
\mathrm{d} \boldsymbol{\varphi}=\mathrm{d} \boldsymbol{\varphi}_{1}+\mathrm{d} \boldsymbol{\varphi}_{2}, \tag{1.12}
\end{equation*}


т. е. два данных поворота ( \(\mathrm{d} \boldsymbol{\varphi}_{1}\) и \(\mathrm{d} \boldsymbol{\varphi}_{2}\) ) эквивалентны одному повороту на угол \(\mathrm{d} \boldsymbol{\varphi}=\mathrm{d} \boldsymbol{\varphi}_{1}+\mathrm{d} \boldsymbol{\varphi}_{2}\) вокруг оси, совпадающей с вектором \(\mathrm{d} \boldsymbol{\varphi}\) и проходящей через точку \(O\).

Заметим, что при рассмотрении таких величин, как ради-ус-вектор \(\mathbf{r}\), скорость \(\mathbf{v}\), ускорение \(\mathbf{a}\), не возникал вопрос о выборе их направления: оно вытекало естественным образом из природы самих величин. Подобные векторы называют полярными. В отличие от них векторы типа \(\mathrm{d} \varphi\), направление которых связывают с направлением вращения, называют аксиальньлми.

Введем векторы угловой скорости и углового ускорения. Вектор угловой скорости ω определяют как


\begin{equation*}
\boldsymbol{\omega}=\mathrm{d} \boldsymbol{\varphi} / \mathrm{d} t, \tag{1.13}
\end{equation*}


где \(\mathrm{d} t\) - промежуток времени, за который тело совершает поворот \(\mathrm{d} \boldsymbol{\varphi}\). Вектор \(\boldsymbol{\omega}\) совпадает по направлению с вектором \(\mathrm{d} \boldsymbol{\varphi}\) и представляет собой аксиальный вектор.

\footnotetext{\begin{itemize}
  \item Как следует из рис. 1.6, для конечного поворота на угол \(\Delta \varphi\) линейное перемещение точки \(A\)
\end{itemize}

\[
|\Delta \mathbf{r}|=r \sin \vartheta \cdot 2 \sin (\Delta \varphi / 2) .
\]

Отсюда сразу видно, что перемещение \(\Delta \mathbf{r}\) нельзя представить как векторное произведение векторов \(\Delta \boldsymbol{φ}\) и \(\mathbf{r}\). Это возможно лишь в случае бесконечно малого поворота \(\mathrm{d} \varphi\), в пределах которого радиус-вектор \(\mathbf{r}\) можно считать неизменным.
}Изменение вектора \(\omega\) со временем характеризуют вектором углового ускорения \(\boldsymbol{\beta}\) :


\begin{equation*}
\boldsymbol{\beta}=\mathrm{d} \boldsymbol{\omega} / \mathrm{d} t . \tag{1.14}
\end{equation*}


Направление вектора \(\boldsymbol{\beta}\) совпадает с направлением \(\mathrm{d} \boldsymbol{\omega}\) - приращения вектора \(\boldsymbol{\omega}\). Вектор \(\boldsymbol{\beta}\), как и \(\boldsymbol{\omega}\), является аксиальным.

Единицей угловой скорости в СИ является радиан в секунду (рад/с), а единицей углового ускорения - радиан на секунду в квадрате (рад/c²).

Представление угловой скорости и углового ускорения в виде векторов оказывается чрезвычайно плодотворным, особенно при изучении более сложных движений твердого тела. Это дает возможность во многих случаях получить большую наглядность, а также резко упростить как анализ движения, так и соответствующие расчеты.

Запишем выражения для угловой скорости и углового ускорения и проекциях на ось вращения \(Z\), положительное направление которой свяжем с положительным направлением отсчета координаты \(\varphi\) - угла поворота - правилом правого винта (рис. 1.7). Тогда проекции \(\omega_{z}\) и \(\beta_{z}\) векторов \(\omega\) и \(\boldsymbol{\beta}\) на ось \(Z\) определяются формулами


\begin{gather*}
\omega_{z}=\mathrm{d} \varphi / \mathrm{d} t  \tag{1.15}\\
\beta_{z}=\mathrm{d} \omega_{z} / \mathrm{d} t . \tag{1.16}
\end{gather*}


Здесь \(\omega_{z}\) и \(\beta_{z}\) - величины алгебраические. Их знак характеризует направление соответствующего вектора. Например, если \(\omega_{z}>0\), то направление вектора \(\omega\) совпадает с положительным направлением оси \(Z\); если же \(\omega_{z}<0\), то направление вектора \(\omega\) противоположно. Аналогично и для углового ускорения.

Таким образом, зная зависимость \(\varphi(t)\) - закон вращения тела, по формулам (1.15) и (1.16) можно найти угловую скорость и угловое ускорение в каждый момент времени. И наоборот, если известны зависимость углового ускорения от времени и начальные условия, т. е. угловая скорость \(\omega_{0}\) и

\begin{figure}[h]
\begin{center}
  \includegraphics[alt={},max width=\textwidth]{66a35da9-3faa-4823-9ca9-40799f37b3db-020_170_188_1494_972}
\captionsetup{labelformat=empty}
\caption{Рис. 1.7}
\end{center}
\end{figure}

угол \(\varphi_{0}\) в начальный момент времени, то можно найти \(\omega(t)\) и \(\varphi(t)\).

Пример. Твердое тело вращается вокруг неподвижной оси по закону \(\varphi=a t-b t^{2} / 2\), где \(a\) и \(b\) - некоторые положительные постоянные. Найдем характер движения этого тела.\\
Согласно (1.15) и (1.16),

\[
\omega_{z}=a-b t ; \quad \beta_{z}=-b=\text { const. }
\]

Отсюда видно, что тело, вращаясь равнозамедленно ( \(\beta_{z}<0\) ), останавливается в момент \(t_{0}=a / b\), а затем направление вращения (знак \(\omega_{z}\) ) изменяется на противоположное.\\
Отметим, что решение всех задач на вращение твердого тела вокруг неподвижной оси аналогично по форме задачам на прямолинейное движение точки. Достаточно заменить линейные величины \(x, v_{x}\) и \(a_{x}\) на соответствующие угловые \(\varphi, \omega_{z}\) и \(\beta_{z}\), и мы получим все закономерности и соотношения для вращающегося тела.

\section*{Связь между линейными и угловыми величинами}
Найдем скорость \(\mathbf{v}\) произвольной точки \(A\) твердого тела, вращающегося вокруг неподвижной оси \(O O^{\prime}\) с угловой скоростью \(\omega\). Пусть положение точки \(A\) относительно некоторой точки \(O\) оси вращения характеризуется радиусом-вектором \(\mathbf{r}\) (рис. 1.8). Воспользуемся формулой (1.11), поделив ее на соответствующий промежуток времени \(\mathrm{d} t\). Так как \(\mathrm{d} \mathbf{r} / \mathrm{d} t=\mathbf{v}\) и \(\mathrm{d} \boldsymbol{\varphi} / \mathrm{d} t=\boldsymbol{\omega}\), то


\begin{equation*}
\mathbf{v}=[\omega \mathbf{r}], \tag{1.17}
\end{equation*}


\begin{figure}[h]
\begin{center}
  \includegraphics[alt={},max width=\textwidth]{66a35da9-3faa-4823-9ca9-40799f37b3db-021_467_330_1242_72}
\captionsetup{labelformat=empty}
\caption{Рис. 1.8}
\end{center}
\end{figure}

т. е. скорость \(\mathbf{v}\) любой точки \(A\) твердого тела, вращающегося вокруг некоторой оси с угловой скоростью \(\omega\), равна векторному произведению \(\boldsymbol{\omega}\) на радиус-вектор \(\mathbf{r}\) точки \(A\) относительно произвольной точки \(O\) оси вращения (рис. 1.8).

Модуль вектора (1.17) \(v=\omega r \sin \vartheta\), или

\[
v=\omega \rho,
\]

где \(\rho\) - радиус окружности, по которой движется точка \(A\).

Продифференцировав (1.17) по времени, найдем полное ускорение а точки \(A: \mathbf{a}=[\mathrm{d} \omega / \mathrm{d} t, \mathbf{r}]+[\omega, \mathrm{d} \mathbf{r} / \mathrm{d} t]\), или


\begin{equation*}
\mathbf{a}=[\beta \mathbf{r}]=[\omega[\omega \mathbf{r}]] . \tag{1.18}
\end{equation*}


В данном случае (ось вращения неподвижна) \(\boldsymbol{\beta} \| \boldsymbol{\omega}\), поэтому вектор [βr] представляет собой тангенциальное ускорение \(\mathbf{a}_{\tau}\). Вектор же [ω[ωr]] - это нормальное ускорение \(\mathbf{a}_{n}\). Проекции вектора а на орты \(\tau\) и \(\mathbf{n}\) равны:

\[
a_{\tau}=\beta_{z} \rho, \quad a_{n}=\omega^{2} \rho
\]

Отсюда модуль полного ускорения

\[
a=\sqrt{a_{\tau}^{2}+a_{n}^{2}}=\rho \sqrt{\beta^{2}+\omega^{4}} .
\]

\section*{Плоское движение твердого тела}
Это такое движение, при котором каждая точка твердого тела движется в плоскости, параллельной некоторой неподвижной (в данной системе отсчета) плоскости. При этом плоская фигура \(\Phi\), образованная сечением тела этой неподвижной плоскостью \(P\) (рис. 1.9), в процессе движения все время остается в этой плоскости, например цилиндр, катящийся по плоскости без скольжения (но конус в подобном случае совершает уже более сложное движение).

\begin{figure}[h]
\begin{center}
  \includegraphics[alt={},max width=\textwidth]{66a35da9-3faa-4823-9ca9-40799f37b3db-022_270_307_947_857}
\captionsetup{labelformat=empty}
\caption{Рис. 1.9}
\end{center}
\end{figure}

Положение твердого тела при плоском движении однозначно определяется положением плоской фигуры \(\Phi\) в неподвижной плоскости \(P\). Это позволяет свести изучение плоского движения твердого тела к изучению движения плоской фигуры в ее плоскости.

Пусть плоская фигура \(\Phi\) движется в своей плоскости \(P\), неподвижной в \(K\)-системе отсчета (рис. 1.10). Положение фигуры \(\Phi\) на плоскости можно определить, задав радиус-вектор \(\mathbf{r}_{0}\) произвольной точки \(O^{\prime}\) фигуры и угол \(\varphi\)

\begin{figure}[h]
\begin{center}
  \includegraphics[alt={},max width=\textwidth]{66a35da9-3faa-4823-9ca9-40799f37b3db-022_311_386_1438_778}
\captionsetup{labelformat=empty}
\caption{Рис. 1.10}
\end{center}
\end{figure}

между радиусом-вектором \(\mathbf{r}^{\prime}\), жестко связанным с фигурой, и некоторым фиксированным направлением в \(K\)-системе отсчета. Тогда плоское движение твердого тела будет описываться двумя уравнениями:

\[
\mathbf{r}_{0}=\mathbf{r}_{0}(t), \quad \varphi=\varphi(t) .
\]

Если за промежуток времени \(\mathrm{d} t\) радиус-вектор \(\mathbf{r}^{\prime}\) точки \(A\) (рис.1.10) повернется на угол \(\mathrm{d} \boldsymbol{\varphi}\), то на такой же угол повернется и любой отрезок, связанный с фигурой. Другими словами, поворот фигуры на угол d \(\boldsymbol{\varphi}\) не зависит от выбора точки \(O^{\prime}\). А это значит, что и угловая скорость \(\omega\) фигуры тоже не зависит от выбора точки \(O^{\prime}\), и мы имеем право называть \(\omega\) угловой скоростью твердого тела как такового.

Найдем скорость \(\mathbf{v}\) произвольной точки \(A\) тела при плоском движении. Введем вспомогательную \(K^{\prime}\)-систему отсчета, которая жестко связана с точкой \(O^{\prime}\) тела и перемещается поступательно относительно \(K\)-системы (рис. 1.10). Тогда элементарное перемещение \(\mathrm{d} \mathbf{r}\) точки \(A\) в \(K\)-системе можно записать в виде

\[
\mathrm{d} \mathbf{r}=\mathrm{d} \mathbf{r}_{0}+\mathrm{d} \mathbf{r}^{\prime}
\]

где \(\mathrm{d} \mathbf{r}_{0}\) - перемещение \(K^{\prime}\)-системы (точки \(O^{\prime}\) ), а \(\mathrm{d} \mathbf{r}^{\prime}\) - перемещение точки \(A\) относительно \(K^{\prime}\)-системы. Перемещение dr' обусловлено вращением тела вокруг неподвижной в \(K^{\prime}\)-системе оси, проходящей через точку \(O^{\prime}\); согласно (1.11), \(\mathrm{d} \mathbf{r}^{\prime}=\left[\mathrm{d} \boldsymbol{\varphi}, \mathbf{r}^{\prime}\right]\). Подставив это выражение в предыдущее и разделив обе части полученного равенства на \(\mathrm{d} t\), найдем


\begin{equation*}
\mathbf{v}=\mathbf{v}_{0}+\left[\omega \mathbf{r}^{\prime}\right], \tag{1.19}
\end{equation*}


т. е. скорость любой точки \(A\) твердого тела при плоском движении* складывается из скорости \(\mathbf{v}_{0}\) произвольной точки \(O^{\prime}\) этого тела и скорости \(\mathbf{v}^{\prime}=\left[\boldsymbol{\omega} \mathbf{r}^{\prime}\right]\), обусловленной вращением тела вокруг оси, проходящей через точку \(O^{\prime}\). Подчеркнем еще раз, что \(\mathbf{v}^{\prime}\) - это скорость точки \(A\) относительно поступательно движущейся \(K^{\prime}\)-системы отсчета, жестко связанной с точкой \(O^{\prime}\).

\footnotetext{\begin{itemize}
  \item Заметим, что формула (1.19) оказывается справедливой и для любого сложного движения твердого тела.
\end{itemize}
}Иначе говоря, плоское движение твердого тела можно представить как совокупность двух основных видов движения поступательного (вместе с произвольной точкой \(O^{\prime}\) тела) и вращательного (вокруг оси, проходящей через точку \(O^{\prime}\) ).

Покажем, что плоское движение можно свести к чисто вращательному. Действительно, при плоском движении скорость \(\mathbf{v}_{0}\) произвольной точки \(O^{\prime}\) тела перпендикулярна вектору \(\boldsymbol{\omega}\), а это значит, что всегда найдется такая точка \(M\), жестко связанная с телом \({ }^{*}\), скорость которой \(\mathbf{v}=0\) в данный момент. Из условия \(0=\mathbf{v}_{0}=\left[\omega \mathbf{r}_{M}^{\prime}\right]\) можно найти положение точки \(M\), т. е. ее радиус-вектор \(\mathbf{r}_{M}^{\prime}\) относительно точки \(O^{\prime}\) (рис. 1.11). Этот вектор перпендикулярен векторам

\begin{figure}[h]
\begin{center}
  \includegraphics[alt={},max width=\textwidth]{66a35da9-3faa-4823-9ca9-40799f37b3db-024_366_366_371_794}
\captionsetup{labelformat=empty}
\caption{Рис. 1.11}
\end{center}
\end{figure}

Точка \(M\) определяет и положение соответствующей оси (она совпадает по направлению с вектором ω). Движение твердого тела в данный момент времени представляет собой чистое вращение вокруг этой оси. Такую ось называют мгновенной осью вращения.

Положение мгновенной оси, вообще говоря, меняется со временем. Например, в случае катящегося по плоскости цилиндра мгновенная ось в каждый момент совпадает с линией касания цилиндра и плоскости.

\section*{Сложение угловых скоростей}
Рассмотрим движение твердого тела, вращающегося одновременно вокруг двух пересекающихся осей. Сообщим некоторому телу вращение с угловой скоростью \(\omega^{\prime}\) вокруг оси \(O A\) (рис. 1.12) и затем эту ось приведем во вращение с угловой скоростью \(\omega_{0}\) вокруг оси \(O B\), неподвижной в \(K\)-системе отсчета. Найдем результирующее движение тела в \(K\)-сис-

\begin{figure}[h]
\begin{center}
  \includegraphics[alt={},max width=\textwidth]{66a35da9-3faa-4823-9ca9-40799f37b3db-024_305_303_1351_851}
\captionsetup{labelformat=empty}
\caption{Рис. 1.12}
\end{center}
\end{figure}

\footnotetext{\begin{itemize}
  \item Точка \(M\) может оказаться и вне тела.
\end{itemize}
}Введем вспомогательную \(K^{\prime}\)-систему отсчета, жестко связанную с осями \(O A\) и \(O B\). Ясно, что эта система вращается с угловой скоростью \(\omega_{0}\), и тело вращается относительно нее с угловой скоростью \(\boldsymbol{\omega}^{\prime}\).

За промежуток времени \(\mathrm{d} t\) тело совершит поворот \(\mathrm{d} \varphi^{\prime}\) вокруг оси \(A O\) в \(K^{\prime}\)-системе и одновременно поворот \(\mathrm{d} \varphi_{0}\) вокруг оси \(O B\) вместе с \(K^{\prime}\)-системой. Суммарный поворот, согласно (1.12), есть \(\mathrm{d} \boldsymbol{\varphi}=\mathrm{d} \boldsymbol{\varphi}_{0}+\mathrm{d} \boldsymbol{\varphi}^{\prime}\). Разделив обе части этого равенства на \(\mathrm{d} t\), получим


\begin{equation*}
\omega=\omega_{0}+\omega^{\prime} . \tag{1.20}
\end{equation*}


Таким образом, результирующее движение твердого тела в \(K\)-системе представляет собой чистое вращение с угловой скоростью \(\omega\) вокруг оси, совпадающей в каждый момент с вектором \(\omega\) и проходящей через точку \(O\) (рис. 1.12). Эта ось перемещается относительно \(K\)-системы - она поворачивается с угловой скоростью \(\omega_{0}\) вместе с осью \(O A\) вокруг оси \(O B\).

Нетрудно сообразить, что даже в том случае, когда угловые скорости \(\boldsymbol{\omega}^{\prime}\) и \(\boldsymbol{\omega}_{0}\) не меняются по модулю, тело будет обладать в \(K\)-системе угловым ускорением \(\beta\), направленным, согласно (1.14), за плоскость (рис. 1.12). Вопрос об угловом ускорении твердого тела более подробно рассмотрен в задаче 1.10 .

И последнее замечание. Поскольку вектор угловой скорости \(\omega\) удовлетворяет основному свойству векторов - векторному сложению, \(\omega\) можно представить как векторную сумму составляющих на определенные направления, т. е. \(\boldsymbol{\omega}=\boldsymbol{\omega}_{1}+\boldsymbol{\omega}_{2}+\ldots\), где все векторы относятся к одной и той же системе отсчета. Этим удобным и полезным приемом часто пользуются при анализе сложного движения твердого тела.

\section*{§ 1.3. Преобразование скорости и ускорения при переходе к другой системе отсчета}
Приступая к изучению этого вопроса, напомним, что в рамках ньютоновской механики длина масштабов и время считаются абсолютными. Любой масштаб одинаков в разных системах отсчета, т. е. не зависит от движения. Это же касается и течения времени, которое также одинаково во всех системах.

Постановка вопроса. Имеются две произвольные системы отсчета \(K\) и \(K^{\prime}\), движущиеся определенным образом относительно друг друга. Известны скорость \(\mathbf{v}\) и ускорение \(\mathbf{a}\) некоторой точки \(A\) в \(K\)-системе. Каковы соответствующие значения \(\mathbf{v}^{\prime}\) и \(\mathbf{a}^{\prime}\) этой точки в \(K^{\prime}\)-системе?

Рассмотрим последовательно три наиболее важных случая движения одной системы отсчета относительно другой.

\begin{enumerate}
  \item \(K^{\prime}\)-система движется поступательно по отношению к \(\boldsymbol{K}\)-системе. Пусть в \(K\)-системе начало отсчета \(K^{\prime}\)-системы характеризуется радиусом-вектором \(\mathbf{r}_{0}\), а ее скорость и ускорение - векторами \(\mathbf{v}_{0}\) и \(\mathbf{a}_{0}\). Если положение точки \(A\) в \(K\)-системе определяется радиусом-вектором \(\mathbf{r}\), то \(\mathbf{r}=\mathbf{r}_{0}+\mathbf{r}^{\prime}\) (рис. 1.13). Пусть далее за промежуток времени \(\mathrm{d} t\) точка \(A\) совершит в \(K\)-системе элементарное перемещение dr. Это перемещение складывается из перемещения \(\mathrm{d} \mathbf{r}_{0}\) вместе с \(K^{\prime}\)-системой и перемещения \(\mathrm{d} \mathbf{r}^{\prime}\) относительно \(K^{\prime}\)-системы, т. е. \(\mathrm{d} \mathbf{r}=\mathrm{d} \mathbf{r}_{0}+\mathrm{d} \mathbf{r}^{\prime}\). Разделив данное выраже-
\end{enumerate}

\begin{figure}[h]
\begin{center}
  \includegraphics[alt={},max width=\textwidth]{66a35da9-3faa-4823-9ca9-40799f37b3db-026_360_384_590_780}
\captionsetup{labelformat=empty}
\caption{Рис. 1.13}
\end{center}
\end{figure}

ние на \(\mathrm{d} t\), получим следующую формулу преобразования скорости:


\begin{equation*}
\mathbf{v}=\mathbf{v}_{0}+\mathbf{v}^{\prime} . \tag{1.21}
\end{equation*}


Продифференцировав (1.21) по времени, найдем формулу преобразования ускорения:


\begin{equation*}
\mathbf{a}=\mathbf{a}_{0}+\mathbf{a}^{\prime} . \tag{1.22}
\end{equation*}


Отсюда видно, в частности, что при \(\mathbf{a}_{0}=0 \mathbf{a}=\mathbf{a}^{\prime}\), т. е. при движении \(K^{\prime}\)-системы без ускорения относительно \(K\)-системы, ускорения точки \(A\) в обеих системах отсчета будут одинаковы.\\
2. \(K^{\prime}\)-система вращается с постоянной угловой скоростью ω вокруг оси, неподвижной в \(K\)-системе. Возьмем начала отсчета \(K\) - и \(K^{\prime}\)-систем в произвольной точке \(O\) на оси вращения (рис. 1.14, \(a\) ). Тогда радиус-вектор точки \(A\) в обеих системах отсчета будет один и тот же: \(\mathbf{r} \equiv \mathbf{r}^{\prime}\).

\begin{figure}[h]
\begin{center}
  \includegraphics[alt={},max width=\textwidth]{66a35da9-3faa-4823-9ca9-40799f37b3db-027_410_676_131_278}
\captionsetup{labelformat=empty}
\caption{Рис. 1.14}
\end{center}
\end{figure}

Если точка \(A\) неподвижна в \(K^{\prime}\)-системе, то это значит, что ее перемещение \(\mathrm{d} \mathbf{r}\) в \(K\)-системе за время \(\mathrm{d} t\) обусловлено только поворотом радиуса-вектора \(\mathbf{r}\) на угол \(\mathrm{d} \varphi\) (вместе \(\mathrm{K}^{\prime}\)-системой) и равно, согласно (1.11), векторному произведению [ \(\mathrm{d} \boldsymbol{\varphi}, \mathrm{r}\) ].

Если же точка \(A\) движется относительно \(K^{\prime}\)-системы со скоростью \(\mathbf{v}^{\prime}\), то за время \(\mathrm{d} t\) она совершит дополнительное перемещение \(\mathbf{v}^{\prime} \mathrm{d} t\) (рис. 1.14, a) и тогда


\begin{equation*}
\mathrm{d} \mathbf{r}=\mathbf{v}^{\prime} \mathrm{d} t+[\mathrm{d} \boldsymbol{\varphi}, \mathbf{r}] . \tag{1.23}
\end{equation*}


Разделив это выражение на \(\mathrm{d} t\), получим следующую формулу преобразования скорости:


\begin{equation*}
\mathbf{v}=\mathbf{v}^{\prime}+[\omega \mathbf{r}], \tag{1.24}
\end{equation*}


где \(\mathbf{v}\) и \(\mathbf{v}^{\prime}\) - скорости точки \(A\) в \(K\) - и \(K^{\prime}\)-системах отсчета соответственно.

Теперь перейдем к ускорениям. В соответствии с (1.24) приращение \(\mathrm{d} \mathbf{v}\) вектора \(\mathbf{v}\) за время \(\mathrm{d} t\) в \(K\)-системе должно складываться из суммы приращений векторов \(\mathbf{v}^{\prime}\) и [ωr], т. е.


\begin{equation*}
\mathrm{d} \mathbf{v}=\mathrm{d} \mathbf{v}^{\prime}+[\omega, \mathrm{d} \mathbf{r}] . \tag{1.25}
\end{equation*}


Найдем \(\mathrm{dv}^{\prime}\). Если точка \(A\) движется в \(K^{\prime}\)-системе с \(\mathbf{v}^{\prime}=\) const, то приращение этого вектора в \(K\)-системе обусловлено только его поворотом на угол d \(\boldsymbol{\varphi}\) (вместе с \(K^{\prime}\)-системой) и равно, как и в случае с \(\mathbf{r}\), векторному произведению [ \(\mathrm{d} \boldsymbol{\varphi}, \mathrm{v}^{\prime}\) ]. В этом нетрудно убедиться, совместив начало вектора \(\mathbf{v}^{\prime}\) с осью вращения (рис.1.14, б). Если же точка \(A\) имеет ускорение \(\mathbf{a}^{\prime}\)

в \(K^{\prime}\)-системе, то за время \(\mathrm{d} t\) вектор \(\mathbf{v}^{\prime}\) получит еще дополнительное приращение \(\mathbf{a}^{\prime} \mathrm{d} t\) и тогда


\begin{equation*}
\mathrm{d} \mathbf{v}^{\prime}=\mathbf{a}^{\prime} \mathrm{d} t+\left[\mathrm{d} \boldsymbol{\varphi}, \mathbf{v}^{\prime}\right] . \tag{1.26}
\end{equation*}


Подставим (1.26) и (1.23) в равенство (1.25) и полученное выражение разделим на \(\mathrm{d} t\). В результате найдем следующую формулу преобразования ускорения:


\begin{equation*}
\mathbf{a}=\mathbf{a}^{\prime}+2\left[\omega \mathbf{v}^{\prime}\right]+[\omega[\omega \mathbf{r}]], \tag{1.27}
\end{equation*}


где \(\mathbf{a}\) и \(\mathbf{a}^{\prime}\) - ускорения точки \(A\) в \(K\) - и \(K^{\prime}\)-системах отсчета. Второе слагаемое в правой части этой формулы называют кориолисовыл (или поворотным) ускорением \(\mathbf{a}_{\text {кор }}\), а третье слагаемое осестремительным ускорением* \(\mathbf{a}_{\text {ос }}\) :


\begin{equation*}
\mathbf{a}_{\text {кор }}=2\left[\omega \mathbf{v}^{\prime}\right], \quad \mathbf{a}_{\mathrm{oc}}=[\omega[\omega \mathbf{r}]] . \tag{1.28}
\end{equation*}


Таким образом, ускорение а точки относительно \(K\)-системы равно сумме трех ускорений: ускорения \(\mathbf{a}^{\prime}\) относительно \(K^{\prime}\)-системы, кориолисова ускорения \(\mathbf{a}_{\text {кор }}\) и осестремительного ускорения \(\mathbf{a}_{\mathrm{oc}}\).

Осестремительное ускорение можно представить в виде \(\mathbf{a}_{\text {oc }}=-\omega^{2} \boldsymbol{\rho}\), где \(\boldsymbol{\rho}\) - радиус-вектор, перепендикулярный оси вращения и характеризующий положение точки \(A\) относительно этой оси. Тогда формулу (1.27) можно записать так:


\begin{equation*}
\mathbf{a}=\mathbf{a}^{\prime}+2\left[\omega \mathbf{v}^{\prime}\right]-\omega^{2} \boldsymbol{\rho} . \tag{1.29}
\end{equation*}


\begin{enumerate}
  \setcounter{enumi}{2}
  \item \(K^{\prime}\)-система вращается с постоянной угловой скоростью \(\omega\) вокруг оси, перемещающейся поступательно со скоростью \(\mathbf{v}_{0}\) и ускорением \(\mathbf{a}_{0}\) по отношению к \(\boldsymbol{K}\)-системе. Этот случай объединяет два предыдущих. Введем вспомогательную \(S\)-систему отсчета, которая жестко связана с осью вращения \(K^{\prime}\)-системы и перемещается поступательно в \(K\)-системе. Пусть \(\mathbf{v}\) и \(\mathbf{v}_{S}\) - скорости точки \(A\) в \(K\) - и \(S\)-системах отсчета, тогда в соответствии с (1.21) \(\mathbf{v}=\mathbf{v}_{0}+\mathbf{v}_{S}\). Заменив \(\mathbf{v}_{S}\), согласно (1.24), выражением \(\mathbf{v}_{S}=\mathbf{v}^{\prime}+[\omega \mathbf{r}]\), где \(\mathbf{r}\) - радиус-вектор точки \(A\) относительно про-
\end{enumerate}

\footnotetext{\begin{itemize}
  \item Осестремительное ускорение не следует путать с нормальным ускорением.
\end{itemize}
}извольной точки на оси вращения \(K^{\prime}\)-системы, получим следующую формулу преобразования скорости:


\begin{equation*}
\mathbf{v}=\mathbf{v}^{\prime}+\mathbf{v}_{0}+[\boldsymbol{\omega r}] . \tag{1.30}
\end{equation*}


Аналогичным образом, используя (1.22) и (1.29), найдем формулу преобразования ускорения:


\begin{equation*}
\mathbf{a}=\mathbf{a}^{\prime}+\mathbf{a}_{0}+2\left[\omega \mathbf{v}^{\prime}\right]-\omega^{2} \boldsymbol{\rho} . \tag{1.31}
\end{equation*}


Напомним, что в последних двух формулах \(\mathbf{v}, \mathbf{v}^{\prime}\) и \(\mathbf{a}, \mathbf{a}^{\prime}\) скорости и ускорения точки \(A\) соответственно в \(K\) - и \(K^{\prime}\)-системах отсчета, \(\mathbf{v}_{0}\) и \(\mathbf{a}_{0}\) - скорость и ускорение оси вращения \(K^{\prime}\)-системы в \(K\)-системе, \(\mathbf{r}\) - радиус-вектор точки \(A\) относительно произвольной точки на оси вращения системы, \(\rho\) - ради-ус-вектор, перпендикулярный оси вращения и характеризующий положение точки \(A\) относительно этой оси.\\
Пример. Диск вращается с постоянной угловой скоростью \(\omega\) вокруг собственной оси, укрепленной на столе. По диску движется точка \(A\) с постоянной относительно стола скоростью \(\mathbf{v}\). Найдем скорость \(\mathbf{v}^{\prime}\) и ускорение \(\mathbf{a}^{\prime}\) точки \(A\) относительно диска в момент, когда радиус-вектор, характеризующий ее положение по отношению к оси вращения, равен \(\rho\).\\
Скорость \(\mathbf{v}^{\prime}\) точки \(A\), согласно (1.24),

\[
\mathbf{v}^{\prime}=\mathbf{v}-[\omega \rho] .
\]

Ускорение же \(\mathbf{a}^{\prime}\) найдем с помощью (1.29), учтя, что в данном случае \(\mathbf{a}=0\), ибо \(\mathbf{v}=\) const. Тогда \(\mathbf{a}^{\prime}=-2\left[\omega \mathbf{v}^{\prime}\right]+\omega^{2} \rho\). После подстановки в эту формулу выражения для \(\mathbf{v}^{\prime}\) получим

\[
\mathbf{a}^{\prime}=2[\mathbf{v} \omega]-\omega^{2} \rho .
\]

\section*{Задачи}
1.1. Радиус-вектор, характеризующий положение частицы \(M\) относительно неподвижной точки \(O\), меняется со временем по закону \(\mathbf{r}=\mathbf{A} \sin \omega t+\mathbf{B} \cos \omega t, \quad \mathbf{A}\) и \(\mathbf{B}\) - постоянные векторы, причем \(\mathbf{A} \perp \mathbf{B} ; \omega\) - положительная постоянная. Найти ускорение а частицы и уравнение ее траектории \(y(x)\), взяв оси \(X\) и \(Y\) совпадающими по направлению с векторами \(\mathbf{A}\) и \(\mathbf{B}\) соответственно и имеющими начало в точке \(O\).

P е ш е н и е. Продифференцировав \(\mathbf{r}\) по времени дважды, получим

\[
\mathbf{a}=-\omega^{2}(\mathbf{A} \sin \omega t+\mathbf{B} \cos \omega t)=-\omega^{2} \mathbf{r},
\]

т. е. вектор а все время направлен к точке \(O\), а его модуль пропорционален расстоянию частицы до этой точки.

Теперь найдем уравнение траектории. Спроецировав \(\mathbf{r}\) на оси \(X\) и \(Y\), получим

\[
x=\mathbf{A} \sin \omega t, \quad y=\mathbf{B} \cos \omega t .
\]

Исключив \(\omega t\) из этих двух уравнений, найдем

\[
x^{2} / A^{2}+y^{2} / B^{2}=1 .
\]

Это уравнение эллипса, \(A\) и \(B\) - его полуоси (рис. 1.15, где стрелкой показано направление движения частицы \(M\) ).

\begin{figure}[h]
\begin{center}
  \includegraphics[alt={},max width=\textwidth]{66a35da9-3faa-4823-9ca9-40799f37b3db-030_257_345_702_802}
\captionsetup{labelformat=empty}
\caption{Рис. 1.15}
\end{center}
\end{figure}

1.2. Перемещение и путь. Частице в момент \(t=0\) сообщили скорость \(\mathbf{v}_{0}\), после чего ее скорость стала меняться со временем \(t\) по закону \(\mathbf{v}=\mathbf{v}_{0}(1-t / \tau)\), где \(\tau\) - положительная постоянная. Найти за первые \(t\) секунд движения: 1) вектор перемещения \(\Delta \mathbf{r}\) частицы; 2) пройденный ею путь \(s\).

Решение. 1. Согласно (1.1), \(\mathrm{d} \mathbf{r}=\mathbf{v} \mathrm{d} t=\mathbf{v}_{0}(1-t / \tau) \mathrm{d} t\). Проинтегрировав это уравнение по времени от 0 до \(t\), получим

\[
\Delta \mathbf{r}=\mathbf{v}_{0} t(1-t / 2 \tau) .
\]

\begin{enumerate}
  \setcounter{enumi}{1}
  \item Путь \(s\), пройденный частицей за время \(t\), равен
\end{enumerate}

\[
s=\int_{0}^{t} v \mathrm{~d} t
\]

где \(v\) - модуль вектора v. В данном случае

\[
v=v_{0}|1-t / \tau|=\left\{\begin{array}{l}
v_{0}(1-t / \tau), \text { если } t \leqslant \tau, \\
v_{0}(t / \tau-1), \text { если } t \geqslant \tau .
\end{array}\right.
\]

Отсюда следует, что при \(t>\tau\) интеграл для вычисления пути необходимо разбить на две части: от 0 до \(\tau\) и от \(\tau\) до \(t\). Проведя интегрирование для обоих случаев, получим

\[
s=\left\{\begin{array}{c}
v_{0} t(1-t / 2 \tau), \text { если } t \leqslant \tau, \\
\frac{1}{2} v_{0} \tau\left[1+(1-t / \tau)^{2}\right], \text { если } t \geqslant \tau .
\end{array}\right.
\]

\begin{figure}[h]
\begin{center}
  \includegraphics[alt={},max width=\textwidth]{66a35da9-3faa-4823-9ca9-40799f37b3db-031_325_322_406_123}
\captionsetup{labelformat=empty}
\caption{Рис. 1.16}
\end{center}
\end{figure}

На рис. 1.16 показаны графики зависимостей \(v(t)\) и \(s(t)\). Здесь же штриховыми линиями показаны графики зависимостей от \(t\) проекций \(v_{x}\) и \(\Delta x\) векторов \(\mathbf{v}\) и \(\Delta \mathbf{r}\) на ось \(X\), направленную вдоль вектора \(\mathrm{v}_{0}\) 。\\
1.3. Трамвай движется прямолинейно от остановки \(A\) до следующей остановки \(B\) с ускорением, изменяющимся по закону \(a=a_{0}-b s\), где \(a_{0}\) и \(b\) - положительные постоянные, \(s\) - расстояние от остановки \(A\) до трамвая. Найти расстояние между этими остановками и максимальную скорость трамвая.

Решение. Сначала найдем зависимость скорости от расстояния \(s\). За промежуток времени \(\mathrm{d} t\) приращение скорости \(\mathrm{d} v=a \mathrm{~d} t\). Приведем это выражение к виду, удобному для интегрирования, воспользовавшись тем, что \(\mathrm{d} t=\mathrm{d} s / v\). Тогда

\[
v \mathrm{~d} v=\left(a_{0}-b s\right) \mathrm{d} s .
\]

Проинтегрировав это уравнение (левую часть - от 0 до \(v\), правую - от нуля до \(s\) ), получим

\[
v^{2} / 2=a_{0} s-b s^{2} / 2, \text { или } v=\sqrt{\left(2 a_{0}-b s\right) s} .
\]

Отсюда видно, что расстояние между остановками, т. е. значение \(s_{0}\), при котором \(v=0\), есть \(s_{0}=2 a_{0} / b\). Максимальную скорость найдем из условия \(\mathrm{d} v / \mathrm{d} s=0\) или, проще, из условия максимума подкоренного выражения. Отсюда значение \(s_{m}\), соответствующее \(v_{\text {макс }}\), определяется как \(s_{m}=a_{0} / b\) и \(v_{\text {макс }}=a_{0} / \sqrt{b}\).\\
1.4. Частица движется в плоскости \(X, Y\) из точки с координатами \(x= =y=\mathbf{0}\) со скоростью \(\mathbf{v}=a \mathbf{i}+b x \mathbf{j}\), где \(a\) и \(b\) - некоторые постоянные, \(\mathbf{i}\) и \(\mathbf{j}\) - орты осей \(X\) и \(Y\). Найти уравнение ее траектории \(y(x)\).

Решение. Запишем приращения \(y\) - и \(x\)-координат частицы за промежуток времени \(\mathrm{d} t\) :

\[
\mathrm{d} y=v_{y} \mathrm{~d} t, \quad \mathrm{~d} x=v_{x} \mathrm{~d} t,
\]

где \(v_{y}=b x, v_{x}=a\). Взяв их отношение, получим

\[
\mathrm{d} y=(b / a) x \mathrm{~d} x .
\]

Интегрируем это уравнение:

\[
y=\int_{0}^{x}(b / a) x \mathrm{~d} x=(b / 2 a) x^{2}
\]

т.е. траекторией точки является парабола.\\
1.5. Закон движения точки \(A\) обода колеса, катящегося равномерно по горизонтальному пути (ось \(x\) ), имеет вид

\[
x=b(\omega t-\sin \omega t) ; \quad y=b(1-\cos \omega t)
\]

где \(b\) и \(\omega\) - положительные постоянные. Найти скорость \(v\) точки \(A\), путь \(s\), пройденный ею между двумя последовательными касаниями полотна дороги, а также модуль и направление вектора ускорения точки \(A\).\\
Р е ш е н и е. Скорость \(v\) точки \(A\) и пройденный ею путь \(s\) определяются следующими формулами:

\[
\begin{gathered}
v=\sqrt{v_{x}^{2}+v_{y}^{2}}=b \omega \sqrt{2(1-\cos \omega t)}=2 b \omega|\sin (\omega t / 2)| \\
s=\int_{0}^{t_{1}} v \mathrm{~d} t=4 b\left[1-\cos \left(\omega t_{1} / 2\right)\right]
\end{gathered}
\]

где \(t_{1}\) - промежуток времени между двумя последовательными касаниями. Из уравнения \(y=y(t)\) находим, что \(y\left(t_{1}\right)=0\) при \(\omega t_{1}=2 \pi\). Поэтому \(s=8 b\).

Модуль ускорения точки \(A\)

\[
a=\sqrt{a_{x}^{2}+a_{y}^{2}}=b \omega^{2} .
\]

Покажем, что вектор \(\mathbf{a}\), постоянный по модулю, все время направлен к центру колеса - точке \(C\). Действительно, в \(K^{\prime}\)-системе отсчета, связанной с точкой \(C\) и перемещающейся поступательно и равномерно относительно полотна дороги, точка \(A\) движется рав-

номерно по окружности с центром в точке \(C\). Поэтому ускорение точки \(A\) в \(K^{\prime}\)-системе направлено к центру колеса. А так как \(K^{\prime}\)-система движется равномерно, то вектор а будет таким же и относительно полотна дороги.\\
1.6. Тангенциальное и нормальное ускорения. Точка движется замедленно по окружности радиуса \(r\) так, что ее тангенциальное и нормальное ускорения в каждый момент равны друг другу по модулю. В начальный момент точке была сообщена скорость \(v_{0}\). Найти скорость \(v\) и модуль полного ускорения \(a\) точки в зависимости от пройденного пути \(s\).\\
Решение. По условию, \(\mathrm{d} v / \mathrm{d} t=-v^{2} / r\). Представив \(\mathrm{d} t\) как \(\mathrm{d} s / v\), преобразуем исходное уравнение к виду

\[
\mathrm{d} v / v=-\mathrm{d} s / r .
\]

Интегрирование этого уравнения с учетом начальной скорости приводит к следующему результату:

\[
v=v_{0} \mathrm{e}^{-s / r} .
\]

В данном случае \(\left|a_{\tau}\right| \equiv a_{n}\), поэтому модуль полного ускорения \(a=\sqrt{2} a_{n}=\sqrt{2} v^{2} / r\), или

\[
a=\sqrt{2}\left(v_{0}^{2} / r\right) \mathrm{e}^{-2 s / r} .
\]

1.7. Точка движется по плоской траектории так, что ее тангенциальное ускорение \(a_{\tau}=a_{0}\), а нормальное ускорение \(a_{n}=b t^{4}\), где \(a_{0}\) и \(b\) - положительные постоянные, \(t\) - время. В момент \(t=0\) точка начала двигаться с нулевой начальной скоростью. Найти радиус кривизны \(\rho\) траектории точки и модуль ее полного ускорения в зависимости от пройденного пути \(s\).\\
Р е ш е н и е. Элементарное приращение скорости точки \(\mathrm{d} v=a_{\tau} \mathrm{d} t\). Проинтегрировав это уравнение, получим \(v=a_{0} t\). Пройденный путь \(s=a_{0} t^{2} / 2\).

Радиус кривизны траектории, согласно (1.10), можно представить как \(\rho=v^{2} / a_{n}=a_{0}^{2} / b t^{2}\), или

\[
\rho=a_{0}^{3} / 2 b s .
\]

Модуль полного ускорения

\[
a=\sqrt{a_{\tau}^{2}+a_{n}^{2}}=a_{0} \sqrt{1+\left(4 b s^{2} / a_{0}^{3}\right)^{2}} .
\]

1.8. Частица движется равномерно со скоростью \(v\) по параболической траектории \(y=k x^{2}\), где \(k\) - положительная постоянная. Найти модуль ускорения частицы в точке \(x=0\).\\
Решение. Продифференцируем дважды уравнение траектории по времени:

\[
\dot{y}=2 k x \dot{x}, \quad \ddot{y}=2 k\left(\dot{x}^{2}+x \ddot{x}\right) .
\]

Так как частица движется равномерно, то это значит, что ее ускорение во всех точках траектории чисто нормальное и в точке \(x=0\) совпадает с производной \(\ddot{y}\) в этой точке. Имея в виду, что в точке \(x=0\) величина \(|\dot{x}|=v\), получим

\[
a=(\ddot{y})_{x=0}=2 k v^{2} .
\]

Заметим, что в приведенном способе решения мы обошли вычисление радиуса кривизны траектории в точке \(x=0\), который обычно бывает необходимо знать для определения нормального ускорения ( \(a_{n}=v^{2} / \rho\) ).\\
1.9. Вращение твердого тела. Твердое тело вращается вокруг неподвижной оси с угловым ускорением \(\boldsymbol{\beta}=\boldsymbol{\beta}_{0} \cos \varphi\), где \(\boldsymbol{\beta}_{0}\) - постоянный вектор, \(\varphi\) - угол поворота тела из начального положения. Найти угловую скорость \(\omega_{z}\) тела в зависимости от угла \(\varphi\), если при \(\varphi=0\) она была равна нулю.\\
Решение. Выберем положительное направление оси \(z\) вдоль вектора \(\boldsymbol{\beta}_{0}\). Согласно (1.16), \(\mathrm{d} \omega_{z}=\beta_{z} \mathrm{~d} t\). Представив \(\mathrm{d} t\) по формуле (1.15) как \(\mathrm{d} \varphi / \omega_{z}\), преобразуем предыдущее уравнение к виду

\[
\omega_{z} \mathrm{~d} \omega_{z}=\beta_{0} \cos \varphi \mathrm{~d} \varphi
\]

Интегрирование этого уравнения с учетом начального условия \(\left(\omega_{z}=0\right.\) при \(\left.\varphi=0\right)\) дает \(\omega_{z}^{2} / 2=\beta_{0} \sin \varphi\).\\
Отсюда

\[
\omega_{z}= \pm \sqrt{2 \beta_{0} \sin \varphi} .
\]

График зависимости \(\omega_{z}(\varphi)\) показан на рис. 1.17. Из него видно, что с ростом угла φ вектор ω сначала увеличивается, совпадая по направлению с вектором \(\boldsymbol{\beta}_{0}\left(\omega_{z}>0\right)\), достигает максимума при

\begin{figure}[h]
\begin{center}
  \includegraphics[alt={},max width=\textwidth]{66a35da9-3faa-4823-9ca9-40799f37b3db-034_265_334_1472_808}
\captionsetup{labelformat=empty}
\caption{Рис. 1.17}
\end{center}
\end{figure}

\(\varphi=\pi / 2\) и затем начинает уменьшаться, обращаясь в нуль при \(\varphi=\pi\). После этого тело подобным же образом начинает вращаться в противоположном направлении ( \(\omega_{z}<0\) ). В результате тело будет совершать колебания около положения \(\varphi=\pi / 2\) с амплитудой, равной \(\pi / 2\).\\
1.10. Круглый конус с радиусом основания \(r\) и высотой \(h\) катится без

\begin{figure}[h]
\begin{center}
  \includegraphics[alt={},max width=\textwidth]{66a35da9-3faa-4823-9ca9-40799f37b3db-035_299_444_391_151}
\captionsetup{labelformat=empty}
\caption{Рис. 1.18}
\end{center}
\end{figure}

скольжения по поверхности стола, как показано на рис. 1.18. Вершина конуса закреплена шарнирно в точке \(O\) на уровне точки \(C\) - центра основания конуса. Точка \(C\) движется с постоянной скоростью \(v\). Найти относительно стола:

\begin{enumerate}
  \item угловую скорость ω конуса;
  \item его угловое ускорение \(\boldsymbol{\beta}\).
\end{enumerate}

Решение. 1. Согласно (1.20), \(\boldsymbol{\omega}=\boldsymbol{\omega}_{0}+\boldsymbol{\omega}^{\prime}\), где \(\boldsymbol{\omega}_{0}\) и \(\boldsymbol{\omega}^{\prime}\) - угловые скорости вращения вокруг осей \(O O^{\prime}\) и \(O C\) соответственно. Модули векторов \(\omega_{0}\) и \(\omega^{\prime}\) легко найти с помощью рис. 1.18:

\[
\omega_{0}=v / h, \quad \omega^{\prime}=v / r .
\]

Их отношение \(\omega_{0} / \omega^{\prime}=r / h\). Отсюда следует, что вектор \(\omega\) совпадает в каждый момент с образующей конуса, которая проходит через точку касания \(A\).\\
Модуль вектора \(\omega\)

\[
\omega=\sqrt{\omega_{0}^{2}+\omega^{\prime 2}}=(v / r) \sqrt{1+(r / h)^{2}} .
\]

\begin{enumerate}
  \setcounter{enumi}{1}
  \item Угловое ускорение \(\boldsymbol{\beta}\) конуса, согласно (1.14), есть производная вектора \(\omega\) по времени. Так как \(\omega_{0}=\) const, то
\end{enumerate}

\[
\boldsymbol{\beta}=\mathrm{d} \boldsymbol{\omega} / \mathrm{d} t=\mathrm{d} \boldsymbol{\omega}^{\prime} / \mathrm{d} t .
\]

Вектор \(\boldsymbol{\omega}^{\prime}\), оставаясь постоянным по модулю, поворачивается вокруг оси \(O O^{\prime}\) с угловой скоростью \(\omega_{0}\). Его приращение за промежуток времени \(\mathrm{d} t\) равно по модулю \(\left|\mathrm{d} \omega^{\prime}\right|=\omega^{\prime} \omega_{0} \mathrm{~d} t\), или в векторном виде \(\mathrm{d} \boldsymbol{\omega}^{\prime}=\left[\boldsymbol{\omega}_{0} \boldsymbol{\omega}^{\prime}\right] \mathrm{d} t\). Таким образом,

\[
\beta=\left[\omega_{0} \omega^{\prime}\right] .
\]

Модуль этого вектора \(\beta=v^{2} / r h\).\\
1.11. Преобразования скорости и ускорения. Горизонтально расположенный стержень вращается с постоянной угловой скоростью \(\omega\)

вокруг вертикальной оси, укрепленной на столе и проходящей через один из концов стержня. По стержню движется небольшая муфта. Ее скорость относительно стержня меняется по закону \(\mathbf{v}^{\prime}=b \mathbf{r}\), где \(b\) - постоянная, \(\mathbf{r}\) - радиус-вектор, характеризующий расстояние муфты от оси вращения. Найти: 1) скорость \(\mathbf{v}\) и ускорение а муфты относительно стола в зависимости от \(\mathbf{r}\); 2) угол между векторами \(\mathbf{v}\) и \(\mathbf{a}\) в процессе движения.

Решение. 1. Согласно (1.24),

\[
\mathbf{v}=b \mathbf{r}+[\omega \mathbf{r}]
\]

Модуль этого вектора \(v=r \sqrt{b^{2}+\omega^{2}}\).\\
Ускорение а находим по формуле (1.29), где в нашем случае \(\mathbf{a}^{\prime}=\mathrm{d} \mathbf{v}^{\prime} / \mathrm{d} t=b^{2} \mathbf{r}\).

Тогда

\[
\mathbf{a}=\left(b^{2}-\omega^{2}\right) \mathbf{r}+2 b[\omega \mathbf{r}]
\]

Модуль этого вектора \(a=\left(b^{2}+\omega^{2}\right) r\).\\
2. Для определения угла \(\alpha\) между векторами \(\mathbf{v}\) и \(\mathbf{a}\) воспользуемся их скалярным произведением, из которого следует, что \(\cos \alpha=\mathbf{v a} / v a\). После соответствующих преобразований получим

\[
\cos \alpha=1 / \sqrt{1+(\omega / b)^{2}} .
\]

Отсюда видно, что в данном случае угол \(\alpha\) остается постоянным при движении.

\section*{Основное уравнение динамики}
\section*{§ 2.1. Инерциальные системы отсчета}
\section*{Закон инерции}
В кинематике, где речь идет лишь об описании движений и не затрагивается вопрос о причинах, вызывающих эти движения, никакой принципиальной разницы между различными системами отсчета нет, и все они в этом отношении равноправны. Совершенно иначе обстоит дело в динамике - при изучении законов движения. Здесь обнаруживается существенное различие между разными системами отсчета и преимущества одного класса систем отсчета по сравнению с другими.

В принципе можно взять любую из бесчисленного множества систем отсчета. Однако законы механики в разных системах отсчета имеют, вообще говоря, различный вид и может оказаться, что в произвольной системе отсчета законы даже совсем простых явлений будут весьма сложными. Естественно, возникает задача отыскания такой системы отсчета, в которой законы механики были бы возможно более простыми. Такая система отсчета, очевидно, наиболее удобна для описания механических явлений.

Для решения этой задачи рассмотрим ускорение материальной точки относительно некоторой произвольной системы отсчета. Какова причина этого ускорения? Опыт показывает, что этой причиной могут быть как действие на данную точку ка-ких-то определенных тел, так и свойства самой системы отсчета (действительно, относительно разных систем отсчета ускорение в общем случае будет различным).

Можно, однако, предположить, что существует такая система отсчета, в которой ускорение материальной точки целиком обусловлено только взаимодействием ее с другими телами. Свободная материальная точка, не подверженная действию никаких других тел, движется относительно такой системы отсчета прямолинейно и равномерно, или, как говорят, по инерции. Такую систему отсчета называют инерииальной.

Утверждение, что инерциальные системы отсчета существуют, составляет содержание первого закона механики - закона инерции Галилея-Ньютона.

Существование инерциальных систем отсчета подтверждается опытом. Первоначальными опытами было установлено, что такой системой отсчета является Земля. Последующие более точные опыты (опыт Фуко и все аналогичные ему) показали, что эта система отсчета не совсем инерциальная*, а именно: были обнаружены ускорения, существование которых нельзя объяснить действием каких-либо определенных тел. В то же время наблюдения над ускорениями планет показали инерциальность гелиоцентрической системь отсчета, связанной с центром Солнца и «неподвижными» звездами. В настоящее время инерциальность гелиоцентрической системы отсчета подтверждается всей совокупностью опытов.

Любая другая система отсчета, движущаяся равномерно и прямолинейно относительно гелиоцентрической системы, является также инерциальной. Действительно, если в гелиоцентрической системе отсчета ускорение тела равно нулю, то оно равно нулю и в любой другой из этих систем отсчета.

Таким образом, существует не одна, а бесчисленное множество инерциальных систем отсчета, движущихся относительно друг друга прямолинейно и равномерно. Системы отсчета, движущиеся с ускорением относительно инерциальных систем, называют неинерциальньιми.

О свойствах симметрии пространства и времени. Важной особенностью инерциальных систем отсчета является то, что по отношению к ним пространство и время обладают определенными свойствами симметрии. А именно: опыт убеждает, что в этих системах отсчета пространство однородно и изотропно, а время однородно.

Однородность и изотропность пространства заключаются в том, что свойства пространства одинаковы в различных точках (однородность), а в каждой точке одинаковы во всех направлениях (изотропность).

\footnotetext{\begin{itemize}
  \item Заметим, что во многих случаях систему отсчета, связанную с Землей, можно считать практически инерциальной.
\end{itemize}
}Однородность времени заключается в том, что протекание физических явлений (в одних и тех же условиях) в разное время их наблюдения одинаково. Иначе говоря, различные моменты времени эквивалентны друг другу по своим физическим свойствам.

Заметим, что по отношению к неинерциальным системам отсчета пространство является неоднородным и неизотропным. Это значит, что если какое-либо тело не взаимодействует ни с какими другими телами, то тем не менее его различные положения в пространстве и его различные ориентации в механическом отношении не эквивалентны. То же самое относится в общем случае и ко времени, которое будет неоднородным (в неинерциальных системах), т. е. его различные моменты не эквивалентны. Ясно, что такие свойства пространства и времени вносили бы большие усложнения в описание механических явлений. Так, например, тело, не подверженное действию со стороны других тел, не могло бы покоиться: если его скорость в некоторый момент времени и равна нулю, то уже в следующий момент тело начало бы двигаться в определенном направлении.

\section*{Принцип относительности Галилея}
Для инерциальных систем отсчета справедлив принцип относительности, согласно которому все инерциальные системы по своим механическим свойствам эквивалентны друг другу. Это значит, что никакими механическими опытами, проводимыми «внутри» данной инерциальной системы, нельзя установить, покоится эта система отсчета или движется. Во всех инерциальных системах отсчета свойства пространства и времени одинаковы, одинаковы также и все законы механики.

Данное утверждение составляет содержание принципа относительности Галилея - одного из важнейших принципов ньютоновской механики. Этот принцип является обобщением опыта и подтверждается всем многообразием приложений ньютоновской механики к движению тел, скорости которых значительно меньше скорости света.

Все сказанное достаточно ясно свидетельствует об исключительности свойств инерциальных систем отсчета, в силу которых именно эти системы должны, как правило, использоваться при изучении механических явлений.

\section*{Преобразования Галилея}
Найдем формулы преобразования координат при переходе от одной инерциальной системы отсчета к другой. Пусть инерциальная система \(K^{\prime}\) движется со скоростью \(\mathbf{V}\) относительно другой инерциальной системы \(K\). Выберем оси координат \(X^{\prime}, Y^{\prime}, Z^{\prime} K^{\prime}\)-системы параллельно соответствующим осям \(X, Y, Z \quad K\)-системы так, чтобы оси \(X^{\prime}\) и \(X\) совпадали между собой и были направлены вдоль вектора V (рис. 2.1). Взяв за начало отсчета

\begin{figure}[h]
\begin{center}
  \includegraphics[alt={},max width=\textwidth]{66a35da9-3faa-4823-9ca9-40799f37b3db-040_338_428_300_728}
\captionsetup{labelformat=empty}
\caption{Рис. 2.1}
\end{center}
\end{figure}


\begin{equation*}
\mathbf{r}^{\prime}=\mathbf{r}-\mathbf{V} t \tag{2.1}
\end{equation*}


и, кроме того,


\begin{equation*}
t^{\prime}=t . \tag{2.2}
\end{equation*}


Здесь подразумевается, что длина отрезков и ход времени не зависят от состояния движения и, следовательно, одинаковы в обеих системах отсчета. Предположение об абсолютности пространства и времени лежит в самой основе представлений ньютоновской механики, представлений, основанных на обширном экспериментальном материале, относящемся к изучению движений со скоростями, значительно меньшими скорости света.

Соотношения (2.1) и (2.2) представляют собой преобразования Галилея.

В координатах эти преобразования имеют вид


\begin{equation*}
x^{\prime}=x-V t, \quad y^{\prime}=y, \quad t^{\prime}=t . \tag{2.3}
\end{equation*}


Продифференцировав (2.1) по времени, найдем классический закон преобразования скорости точки при переходе от одной инерциальной системы отсчета к другой:


\begin{equation*}
\mathbf{v}^{\prime}=\mathbf{v}-\mathbf{V} . \tag{2.4}
\end{equation*}


Дифференцируя это выражение по времени с учетом того, что \(\mathbf{V}=\) const, получаем \(\mathbf{a}^{\prime}=\mathbf{a}\), т. е. ускорение точки одинаково во всех инерциальных системах отсчета.

\section*{§ 2.2. Основные законы ньютоновской динамики}
Изучая на опыте различные движения, мы обнаруживаем, что в инерциальных системах отсчета всякое ускорение тела вызывается действием на него каких-либо других тел. При этом степень влияния (действия) каждого из окружающих тел на состояние движения интересующего нас тела \(A\) - это вопрос, на который в каждом конкретном случае может дать ответ только опыт.

Влияние другого тела (или тел), вызывающее ускорение тела \(A\), называют силой. Итак, причиной ускорения тела является действующая на него сила.

Одной из важнейших характеристик силы является ее материальное происхождение. Говоря о силе, мы всегда неявно предполагаем, что в отсутствие посторонних тел сила, действующая на интересующее нас тело, равна нулю. Если же обнаруживается, что сила действует, мы ищем источник в виде того или иного конкретного тела или других тел.

Все силы, с которыми имеет дело механика, обычно подразделяют на силы, возникающие при непосредственном контакте тел (силы давления, трения), и силы, возникающие через посредство создаваемых взаимодействующими телами полей (силы гравитационные, электромагнитные). Заметим, однако, что такое подразделение сил имеет условный характер: в сущности и при непосредственном контакте силы взаимодействия обусловлены также наличием тех или иных полей, создаваемых молекулами или атомами тел. Таким образом, все силы взаимодействия между телами обусловлены в конечном счете полями. Вопрос о природе сил взаимодействия выходит за рамки механики и рассматривается в других разделах физики.

Масса. Опыт показывает, что всякое тело «оказывает сопротивление» при любых попытках изменить его скорость - как по модулю, так и по направлению. Это свойство, выражающее степень неподатливости тела к изменению его скорости, называют инертностью. У различных тел оно проявляется в разной степе-

ни. Мерой инертности служит величина, называемая массой. Тело с большей массой является более инертным, и наоборот.

Введем понятие массы \(m\), определив отношение масс двух различных тел по обратному отношению ускорений, сообщаемых им равными силами:


\begin{equation*}
m_{1} / m_{2}=a_{2} / a_{1} . \tag{2.5}
\end{equation*}


Отметим, что такое определение не требует предварительного измерения самих сил. Достаточно лишь располагать критерием равенства сил. Например, если на два различных тела, лежащих на гладкой горизонтальной плоскости, последовательно подействовать одной и той же пружиной, ориентировав ее горизонтально и растянув на одну и ту же длину, то можно утверждать, что в обоих случаях влияние пружины на каждое тело одинаково, другими словами, одинакова и сила.

Таким образом, сравнение масс двух тел, на которые действует одна и та же сила, сводится к сравнению ускорений этих тел. Взяв некоторое тело за эталон массы, мы имеем возможность сравнить массу любого тела с этим эталоном.

Единицей массы в СИ является, как известно, килограмм (кг).\\
Как показывает опыт, в рамках ньютоновской механики масса обладает следующими двумя важнейшими свойствами:

\begin{enumerate}
  \item масса - величина аддитивная, т. е. масса составного тела равна сумме масс его частей;
  \item масса тела как такового - величина постоянная, не изменяющаяся при его движении.
\end{enumerate}

Сила. Вернемся к опыту по сравнению ускорений двух различных тел под действием одинаково растянутой пружины. Тот факт, что в обоих случаях пружина была растянута одинаково, позволил нам высказать утверждение об одинаковости действия пружины, или силы со стороны пружины.

С другой стороны, сила является причиной ускорения тела. Ускорения же различных тел под действием одной и той же одинаково растянутой пружины разные. Наша задача так определить силу, чтобы, несмотря на различие ускорений разных тел в рассматриваемом опыте, сила была бы одной и той же.

Для этого прежде всего надо выяснить: что является одинаковым в данных опытах? Ответ очевиден: произведение та.

Эту величину и естественно взять за определение силы. Учитывая, что ускорение - вектор, будем считать и силу вектором, совпадающим по направлению с вектором ускорения а.

Итак, в ньютоновской механике сила, действующая на тело массы \(m\), определяется как произведение та. Оправданием именно такого определения силы, кроме соображений наибольшей простоты и удобства, послужила дальнейшая проверка всех вытекающих из него следствий.

\section*{Второй закон Ньютона}
Изучая на опыте взаимодействие различных материальных точек с окружающими телами, мы обнаруживаем, что та зависит от величин, характеризующих как состояние самой материальной точки, так и состояние окружающих тел.

Это является весьма существенным физическим фактом, лежащим в основе одного из наиболее фундаментальных обобщений ньютоновской механики: произведение массы материальной точки на ее ускорение является функцией положения этой точки относительно окружающих тел, а иногда также и функиией ее скорости. Эту функцию обозначают \(\mathbf{F}\) и называют силой. Именно в этом и состоит фактическое содержание второго закона Ньютона, который кратко формулируют обычно так: произведение массы материальной точки на ее ускорение равно действующей на нее силе:


\begin{equation*}
m \mathbf{a}=\mathbf{F} . \tag{2.6}
\end{equation*}


Это уравнение называют уравнением движения материальной точки.

Сразу же подчеркнем, что второй закон Ньютона и уравнение (2.6) получают конкретное содержание только после того, как установлен вид функции \(\mathbf{F}\) - зависимость от определяющих ее величин, или, как говорят, закон силь. Установление вида этой зависимости в каждом конкретном случае является одной из основных задач физической механики.

Определение силы как \(m \mathbf{a}\), лежащее в основе уравнения (2.6), обладает тем исключительным достоинством, что законы сил при этом оказываются очень простыми. Правда, переход к изучению движений с релятивистскими скоростями показал, что законы сил потребовалось

бы модифицировать, сделав их сложным образом зависящими от скорости материальной точки. Теория стала бы громоздкой и запутанной.

Существует, однако, простой выход из этого затруднения, если дать несколько иное определение силы, а именно: сила есть производная импульса р материальной точки по времени, т. е. \(\mathrm{dp} / \mathrm{d} t\), и уравнение (2.6) записывать в виде

\[
\mathrm{d} \mathbf{p} / \mathrm{d} t=\mathbf{F} .
\]

В ньютоновской механике это определение силы тождественно та, так как \(\mathbf{p}=m \mathbf{v}, m=\) const и \(\mathrm{d} \mathbf{p} / \mathrm{d} t=m \mathbf{a}\). В релятивистской же механике импульс, как мы увидим, зависит от скорости материальной точки более сложным образом. Но важно другое. При таком определении силы (как \(\mathrm{d} \mathbf{p} / \mathrm{d} t\) ) законы сил, оказывается, остаются теми же и в релятивистской области. Так что простое выражение данной силы через физическое окружение изменять не потребуется при переходе к релятивистской механике. Это обстоятельство мы учтем в дальнейшем.

Единицей силы в СИ является ньютон (Н). Ньютон - это сила, которая сообщает телу массой 1 кг ускорение \(1 \mathrm{~m} / \mathrm{c}^{2}\).

О сложении сил. На всякую материальную точку в данных конкретных условиях действует, строго говоря, всего только одна сила \(\mathbf{F}\), модуль и направление которой определяются расположением этой точки относительно всех окружающих тел, а иногда также и ее скоростью. И тем не менее часто бывает удобно эту силу \(\mathbf{F}\) представлять как суммарный результат действия отдельных тел, или сил \(\mathbf{F}_{1}, \mathbf{F}_{2} \ldots\) Опыт показывает, что если тела, являющиеся источниками сил, не влияют друг на друга и поэтому не меняют своего состояния от присутствия других тел, то сила

\[
\mathbf{F}=\mathbf{F}_{1}+\mathbf{F}_{2}+\ldots,
\]

где \(\mathbf{F}_{i}\) - сила, с которой действовало бы на данную материальную точку \(i\)-е тело в отсутствие других тел.

Если это так, то говорят, что силы \(\mathbf{F}_{1}, \mathbf{F}_{2} \ldots\) подчиняются приниипу суперпозиции. Такое утверждение надо рассматривать как обобщение опытных фактов.

\section*{Третий закон Ньютона}
Во всех случаях, когда в опыте участвуют только два тела \(A\) и \(B\) и тело \(A\) сообщает ускорение телу \(B\), обнаруживается, что и

тело \(B\) сообщает ускорение телу \(A\). Отсюда мы заключаем, что действия тел друг на друга имеют характер взаимодействия.

Ньютон постулировал следующее общее свойство всех сил взаимодействия - третий закон Ньютона: силь, с которыми две материальнью точки действуют друг на друга, всегда равны по модулю и направлены в противоположные стороны вдоль прямой, соединяющей эти точки:


\begin{equation*}
\mathbf{F}_{12}=-\mathbf{F}_{21} . \tag{2.7}
\end{equation*}


Это значит, что силы взаимодействия всегда появляются парами. Обе силы приложены к разным материальным точкам и, кроме того, являются силами одной природы.

Закон (2.7) распространяется на системы из произвольного числа материальных точек. Мы исходим из представления, что и в этом случае взаимодействие сводится к силам попарного взаимодействия между материальными точками.

В третьем законе Ньютона предполагается, что обе силы равны по модулю в любой момент времени независимо от движения точек. Это утверждение соответствует ньютоновскому представлению о мгновенном распространении взаимодействий - предположению, которое носит название принципа дальнодействия ньютоновской механики. Согласно этому принципу, взаимодействие между телами распространяется в пространстве с бесконечно большой скоростью. Иначе говоря, если изменить положение (состояние) одного тела, то сразу же можно обнаружить хотя бы очень слабое изменение во взаимодействующих с ним телах, как бы далеко они ни находились.

В действительности это не так - существует конечная максимальная скорость распространения взаимодействий, которая равна скорости света в вакууме. Поэтому третий закон Ньютона (а также и второй) имеет определенные пределы применимости. Однако при скоростях тел, значительно меньших скорости света, с которыми имеет дело ньютоновская механика, оба закона выполняются с очень большой точностью. Свидетельством этому являются хотя бы расчеты траекторий планет и искусственных спутников, которые проводятся с «астрономической» точностью именно с помощью законов Ньютона.

Законы Ньютона являются основными законами механики. Они позволяют, по крайней мере в принципе, решить любую механическую задачу; кроме того, из них могут быть выведены и все остальные законы механики.

В соответствии с принципом относительности Галилея законы механики одинаковы во всех инерциальных системах отсчета. Это значит, в частности, что уравнение (2.6) будет иметь один и тот же вид в любой инерциальной системе отсчета. Действительно, масса \(m\) материальной точки как таковой не зависит от скорости, т. е. одинакова во всех системах отсчета. Кроме того, для инерциальных систем отсчета одинаковым является и ускорение a точки. Сила \(\mathbf{F}\) тоже не зависит от выбора системы отсчета, поскольку она определяется только взаимным расположением и скоростью материальной точки относительно окружающих тел, а эти величины, согласно нерелятивистской кинематике, в разных инерциальных системах отсчета одинаковы.

Таким образом, все три величины, \(m\), а и \(\mathbf{F}\), входящие в уравнение (2.6), не меняются при переходе от одной инерциальной системы отсчета к другой, а следовательно, не меняется и само уравнение (2.6). Другими словами, уравнение \(m \mathbf{a}=\mathbf{F}\) инвариантно относительно преобразований Галилея.

\section*{§ 2.3. Силы}
Чтобы свести нахождение закона движения частицы к чисто математической задаче, необходимо прежде всего - в соответствии с уравнением (2.6) - знать действующую на частицу силу, т. е. зависимость силы от определяющих ее величин. Каждая такая зависимость получена в конечном счете на основании обработки результатов опыта и, по существу, всегда опирается на уравнение (2.6), как на определение силы.

Наиболее фундаментальные силы, лежащие в основе всех механических явлений,- это силы гравитационные и электрические. Приведем выражение для этих сил в самом простом виде, когда взаимодействующие массы (заряды) покоятся или движутся с малой (нерелятивистской) скоростью.

Сила гравитационного притяжения, действующая между двумя материальными точками, в соответствии с законом все-

мирного тяготения пропорциональна произведению масс точек \(m_{1}\) и \(m_{2}\), обратно пропорциональна квадрату расстояния \(r\) между ними и направлена по прямой, соединяющей эти точки:


\begin{equation*}
F=\gamma \frac{m_{1} m_{2}}{r^{2}}, \tag{2.8}
\end{equation*}


где \(\gamma\) - гравитационная постоянная.\\
Фигурирующие в этом законе массы называют гравитационными в отличие от инертной массы, входящей во второй закон Ньютона. Из опыта, однако, установлено, что гравитационная и инертная массы любого тела строго пропорциональны друг другу. Поэтому можно считать их равными (т. е. выбрать один и тот же эталон для измерения обеих масс) и говорить просто о массе, которая выступает как мера инертности тела или как мера гравитационного действия.

Кулоновская сила, действующая между двумя точечными зарядами \(q_{1}\) и \(q_{2}\),


\begin{equation*}
F=k \frac{\left|q_{1} q_{2}\right|}{r^{2}}, \tag{2.9}
\end{equation*}


где \(r\) - расстояние между зарядами, \(k\) - коэффициент пропорциональности, зависящий от выбора системы единиц. В отличие от гравитационной силы кулоновская сила может быть как силой притяжения, так и силой отталкивания.

Заметим, что закон Кулона (2.9) перестает выполняться точно, если заряды движутся. Электрическое взаимодействие движущихся зарядов оказывается сложным образом зависящим от их движения. Одну из частей этого взаимодействия, обусловленную движением, называют магнитной силой (отсюда и другое название данного взаимодействия - электромагнитное). При малых (нерелятивистских) скоростях магнитная сила составляет пренебрежимо малую часть электрического взаимодействия и оно с высокой степенью точности описывается законом (2.9).

Несмотря на то, что гравитационные и электрические взаимодействия лежат в основе всего бесчисленного разнообразия механических явлений, анализ явлений, особенно макроскопических, оказался бы весьма сложным, если бы во всех случаях

мы исходили из этих фундаментальных взаимодействий. Поэтому удобно ввести другие, приближенные, силы (которые в принципе могут быть получены из фундаментальных сил). Это необходимо для того, чтобы упростить математически задачу настолько, чтобы ее можно было практически решить.

С этой целью вводят, например, следующие силы.\\
Однородная сила тяжести:


\begin{equation*}
\mathbf{F}=m \mathbf{g} \tag{2.10}
\end{equation*}


где \(m\) - масса тела, \(g\) - ускорение свободного падения*.\\
Упругая сила - сила, пропорциональная смещению материальной точки из положения равновесия и направленная к положению равновесия:


\begin{equation*}
\mathbf{F}=-\varkappa \mathbf{r} \tag{2.11}
\end{equation*}


где \(\mathbf{r}\) - радиус-вектор, характеризующий смещение частицы из положения равновесия; \(\varkappa\) - положительный коэффициент, зависящий от «упругих» свойств той или иной конкретной силы. Примером такой силы является сила упругой деформации при растяжении (сжатии) пружины или стержня; в соответствии с законом \(\Gamma y к а ~ F=x \Delta l\), где \(\Delta l\) - величина упругой деформации.

Сила трения скольжения, возникающая при скольжении данного тела по поверхности другого тела,


\begin{equation*}
F=k R_{n}, \tag{2.12}
\end{equation*}


где \(k\) - коэффициент трения скольжения, зависящий от природы и состояния соприкасающихся поверхностей (в частности, от их шероховатости); \(R_{n}\) - сила нормального давления, прижимающая трущиеся поверхности друг к другу. Сила F направлена в сторону, противоположную направлению движения данного тела относительно другого.

\footnotetext{\begin{itemize}
  \item Заметим, что в отличие от силы тяжести вес \(\mathbf{P}\) - это сила, с которой тело действует на опору (или подвес), неподвижную относительно данного тела. Например, если тело с опорой (подвесом) неподвижны относительно Земли, то вес \(\mathbf{P}\) совпадает с силой тяжести. В противном случае вес \(\mathbf{P}=m(\mathbf{g}-\mathbf{a})\), где \(\mathbf{a}-\) ускорение тела (с опорой) относительно Земли.
\end{itemize}
}Сила сопротивления, действующая на тело при его поступательном движении в газе или жидкости. Эта сила зависит от скорости \(\mathbf{v}\) тела относительно среды, причем направлена противоположно вектору \(\mathbf{v}\) :


\begin{equation*}
\mathbf{F}=-k \mathbf{v}, \tag{2.13}
\end{equation*}


где \(k\) - положительный коэффициент, характерный для данного тела и данной среды. Этот коэффициент зависит, вообще говоря, от скорости \(v\), однако при малых скоростях во многих случаях его можно практически считать постоянным.

\section*{§ 2.4. Основное уравнение динамики}
Основное уравнение динамики материальной точки представляет собой не что иное, как математическое выражение второго закона Ньютона:


\begin{equation*}
m \frac{\mathrm{~d} \mathbf{v}}{\mathrm{~d} t}=\mathbf{F} \tag{2.14}
\end{equation*}


Уравнение (2.14) есть, по существу, дифференциальное уравнение движения точки в векторном виде. Его решение основная задача динамики материальной точки. При этом возможны две противоположные постановки задачи.

\begin{enumerate}
  \item Найти действующую на точку силу \(\mathbf{F}\), если известны масса \(m\) точки и зависимость от времени ее радиуса-вектора \(\mathbf{r}(t)\).
  \item Найти закон движения точки, т. е. зависимость от времени ее радиуса-вектора \(\mathbf{r}(t)\), если известны масса \(m\) точки, действующая на нее сила \(\mathbf{F}\) (или силы \(\mathbf{F}_{i}\) ) и начальные условия - скорость \(\mathbf{v}_{0}\) и положение \(\mathbf{r}_{0}\) точки в начальный момент времени.
\end{enumerate}

В первом случае задача сводится к дифференцированию \(\mathbf{r}(t)\) по времени, во втором - к интегрированию уравнения (2.14). Математическая сторона этого вопроса достаточно подробно была рассмотрена в кинематике точки.

В зависимости от характера и постановки конкретной задачи решение уравнения (2.14) проводят или в векторной форме, или в координатах, или в проекциях на касательную и нормаль к траектории в данной точке. Выясним, как записывают уравнение (2.14) в последних двух случаях.

\section*{В проекциях на оси декартовых координат}
Записывая обе части уравнения (2.14) в проекциях на оси \(X\), \(Y, Z\), получим три дифференциальных уравнения вида


\begin{equation*}
m \frac{\mathrm{~d} v_{x}}{\mathrm{~d} t}=F_{x}, \quad m \frac{\mathrm{~d} v_{y}}{\mathrm{~d} t}=F_{y}, \quad m \frac{\mathrm{~d} v_{z}}{\mathrm{~d} t}=F_{z}, \tag{2.15}
\end{equation*}


где \(F_{x}, F_{y}, F_{z}\) - проекции вектора \(\mathbf{F}\) на оси \(X, Y, Z\). Необходимо помнить, что эти проекции - величины алгебраические: в зависимости от ориентации вектора \(\mathbf{F}\) они могут быть как положительными, так и отрицательными. Знак проекции результирующей силы \(\mathbf{F}\) определяет и знак проекции вектора ускорения.

Проследим на конкретном примере, в чем заключается стандартный подход к решению задач с помощью уравнений (2.15).

Пример. Небольшой брусок массы \(m\) скользит вниз по наклонной плоскости, составляющей угол а с горизонтом. Коэффициент трения равен \(k\). Найдем ускорение бруска относительно плоскости (эта система отсчета предполагается инерциальной).

Прежде всего следует изобразить силы, действующие на брусок. Это сила тяжести \(m g\), нормальная сила реакции \(\mathbf{R}\) со стороны плоскости и сила трения \(\mathbf{F}_{\text {тр }}\) (рис. 2.2), направленная в сторону, противоположную движению бруска.

После этого свяжем с системой отсчета «наклонная плоскость» систему координат \(X, Y, Z\). Вообще говоря, систему координат можно ориентировать как угодно, однако во

\begin{figure}[h]
\begin{center}
  \includegraphics[alt={},max width=\textwidth]{66a35da9-3faa-4823-9ca9-40799f37b3db-050_340_362_980_790}
\captionsetup{labelformat=empty}
\caption{Рис. 2.2}
\end{center}
\end{figure}

многих случаях выбор направления осей диктуется характером движения. В нашем случае, например, заранее известно направление движения бруска, поэтому наиболее целесообразно оси координат расположить так, чтобы одна из них совпадала с направлением движения. Тогда задача сведется к решению только одного уравнения (2.15). Итак, выберем ось \(X\), как показано на рис. 2.2 , обязательно указав при этом ее положительное направление (стрелкой).

И только теперь приступим к составлению уравнений (2.15): слева - произведение массы \(m\) бруска на проекцию его ускорения \(a_{x}\) и справа - проекции всех сил на ось \(X\).

Тогда

\[
m a_{x}=m g_{x}+R_{x}+F_{\text {тр } x} .
\]

В данном случае \(g_{x}=g \sin \alpha, R_{x}=0\) и \(F_{\text {тр } x}=-F_{\text {тр }}\), поэтому

\[
m a_{x}=m g \sin \alpha-F_{\text {тр }} .
\]

Так как брусок движется только вдоль оси \(X\), то это значит, согласно второму закону Ньютона, что сумма проекций всех сил на любое перпендикулярное оси \(X\) направление равна нулю. Взяв в качестве такого направления ось \(Y\) (рис. 2.2), получим

\[
R=m g \cos \alpha, \quad F_{\text {тр }}=k R=k m g \cos \alpha .
\]

В результате

\[
m a_{x}=m g \sin \alpha-k m g \cos \alpha .
\]

Если правая часть этого уравнения окажется положительной, то \(a_{x}>0\), а это значит, что вектор а направлен вниз по наклонной плоскости, и наоборот.

\section*{В проекциях на касательную и нормаль к траектории в данной точке}
Записывая обе части (2.14) в проекциях на подвижные орты

\begin{figure}[h]
\begin{center}
  \includegraphics[alt={},max width=\textwidth]{66a35da9-3faa-4823-9ca9-40799f37b3db-051_236_326_1109_67}
\captionsetup{labelformat=empty}
\caption{Рис. 2.3}
\end{center}
\end{figure}

\(\boldsymbol{\tau}\) и \(\mathbf{n}\) (рис. 2.3) и используя полученные ранее выражения (1.10) для тангенциального и нормального ускорений, получим


\begin{equation*}
m \frac{\mathrm{~d} v_{\tau}}{\mathrm{d} t}=F_{\tau}, \quad m \frac{v^{2}}{\rho}=F_{n} \tag{2.16}
\end{equation*}


где \(F_{\tau}\) и \(F_{n}\) - проекции вектора \(\mathbf{F}\) на орты \(\boldsymbol{\tau}\) и \(\mathbf{n}\). На рис. 2.3 обе проекции положительные. Векторы \(\mathbf{F}_{\tau}\) и \(\mathbf{F}_{n}\) называют тангенциальной и нормальной составляющими силь F.

Напомним, что направление орта \(\boldsymbol{\tau}\) выбирают в сторону возрастания дуговой координаты \(l\), а направление орта \(\mathbf{n}-\kappa\) центру кривизны траектории в данной точке.

Уравнениями (2.16) удобно пользоваться, если заранее известна траектория материальной точки.

Пример. Небольшое тело \(A\) соскальзывает с вершины гладкой сферы радиуса \(r\). Найдем скорость тела в момент отрыва от поверхности сферы, если его начальная скорость пренебрежимо мала.

Изобразим силы, действующие на тело \(A\) (это сила тяжести \(m \mathbf{g}\) и нормальная сила реакции R), и запишем уравнения (2.16) в проекциях на орты \(\tau\) и \(\mathbf{n}\) (рис. 2.4):\\
\(m \frac{\mathrm{~d} v}{\mathrm{~d} t}=m g \sin \vartheta\), \(m \frac{v^{2}}{r}=m g \cos \vartheta-R\).

\begin{figure}[h]
\begin{center}
  \includegraphics[alt={},max width=\textwidth]{66a35da9-3faa-4823-9ca9-40799f37b3db-052_249_366_314_794}
\captionsetup{labelformat=empty}
\caption{Рис. 2.4}
\end{center}
\end{figure}

Здесь индекс \(\tau\) несуществен, поэтому мы его опустили.\\
Преобразуем первое уравнение к виду, удобному для интегрирования. Воспользовавшись тем, что \(\mathrm{d} t=\mathrm{d} l / v=r \mathrm{~d} \vartheta / v\), где \(\mathrm{d} l\) - элементарный путь тела \(A\) за промежуток времени \(\mathrm{d} t\), перепишем первое уравнение в виде

\[
v \mathrm{~d} v=g r \sin \vartheta \mathrm{~d} \vartheta .
\]

Проинтегрировав левую часть этого выражения от 0 до \(v\), правую - от 0 до \(\vartheta\), найдем

\[
v^{2}=2 g r(1-\cos \vartheta) .
\]

В момент отрыва \(R=0\), поэтому второе исходное уравнение принимает вид

\[
v^{2}=g r \cos \vartheta,
\]

где \(v\) и \(\vartheta\) соответствуют точке отрыва. Исключив \(\cos \vartheta\) из последних двух равенств, получим

\[
v=\sqrt{2 g r / 3} .
\]

\section*{§ 2.5. Неинерциальные системы отсчета. Силы инерции}
\section*{Основное уравнение динамики в неинерциальной системе}
Ранее было отмечено, что основное уравнение динамики справедливо только в инерциальных системах отсчета. Между тем имеется много случаев, когда решение интересующей нас задачи необходимо получить в неинерциальных системах (на-

пример, движение математического маятника в ускоренно движущемся вагоне, движение спутника относительно поверхности Земли и др.). Поэтому возникает вопрос: как следует изменить основное уравнение динамики, чтобы оно оказалось справедливым и для неинерциальных систем отсчета?

С этой целью возьмем две системы отсчета: инерциальную \(K\)-систему и неинерциальную \(K^{\prime}\)-систему. Пусть известны масса \(m\) частицы, сила \(\mathbf{F}\), действующая на нее со стороны окружающих тел, и характер движения \(K^{\prime}\)-системы относительно \(K\)-системы. Рассмотрим достаточно общий случай, когда \(K^{\prime}\)-система вращается с постоянной угловой скоростью \(\omega\) вокруг оси, перемещающейся поступательно с ускорением \(\mathbf{a}_{0}\) относительно \(K\)-системы. Воспользуемся формулой преобразования ускорений (1.31). Из нее следует, что ускорение частицы в \(K^{\prime}\)-системе


\begin{equation*}
\mathbf{a}^{\prime}=\mathbf{a}-\mathbf{a}_{0}+\omega^{2} \boldsymbol{\rho}+2\left[\mathbf{v}^{\prime} \boldsymbol{\omega}\right], \tag{2.17}
\end{equation*}


где \(\mathbf{v}^{\prime}\) - скорость частицы относительно \(K^{\prime}\)-системы, \(\rho\) - ради-ус-вектор, перпендикулярный оси вращения и характеризующий положение частицы относительно этой оси.

Умножив обе части (2.17) на массу \(m\) частицы и учтя, что в инерциальной системе отсчета \(m \mathbf{a}=\mathbf{F}\), получим


\begin{equation*}
m \mathbf{a}^{\prime}=\mathbf{F}-m \mathbf{a}_{0}+m \omega^{2} \boldsymbol{\rho}+2 m\left[\mathbf{v}^{\prime} \boldsymbol{\omega}\right] \tag{2.18}
\end{equation*}


Это и есть основное уравнение динамики в неинерциальной системе отсчета, которая вращается с постоянной угловой скоростью \(\omega\) вокруг оси, перемещающейся поступательно с ускорением \(\mathbf{a}_{0}\). Из него видно, что даже при \(\mathbf{F}=0\) частица будет двигаться в этой системе с ускорением, в общем случае отличным от нуля, причем так, как если бы на нее действовали некоторые силы, соответствующие последним трем членам уравнения (2.18). Эти силы назвали силами инериии.

Уравнение (2.18) показывает, что введение сил инерции позволяет сохранить по форме основное уравнение динамики и для неинерциальных систем: слева - произведение массы частицы на ее ускорение (но уже по отношению к неинерциальной системе отсчета), справа - силы. Однако кроме силы \(\mathbf{F}\), обусловленной действием окружающих тел (силы взаимодейст-

вия), необходимо учесть и силы инерции - остальные слагаемые в правой части уравнения (2.18).

\section*{Силы инерции}
Перепишем уравнение (2.18) в таком виде:


\begin{equation*}
m \mathbf{a}^{\prime}=\mathbf{F}+\mathbf{F}_{\text {пси }}+\mathbf{F}_{\text {цб }}+\mathbf{F}_{\text {кор }}, \tag{2.19}
\end{equation*}


где


\begin{equation*}
\mathbf{F}_{\text {пси }}=-m \mathbf{a}_{0} \tag{2.20}
\end{equation*}


\begin{itemize}
  \item поступательная сила инерции, обусловленная поступательным движением неинерциальной системы отсчета;
\end{itemize}


\begin{equation*}
\mathbf{F}_{\text {цб }}=m \omega^{2} \boldsymbol{\rho} \tag{2.21}
\end{equation*}


\begin{itemize}
  \item центробежная сила инерции;
\end{itemize}


\begin{equation*}
\mathbf{F}_{\text {кор }}=2 m\left[\mathbf{v}^{\prime} \boldsymbol{\omega}\right], \tag{2.22}
\end{equation*}


\begin{itemize}
  \item сила Кориолиса, или кориолисова сила инерции. Последние две силы обусловлены вращательным движением системы отсчета.
\end{itemize}

Мы видим, таким образом, что силы инерции зависят от свойств неинерциальной системы отсчета ( \(\mathbf{a}_{0}, \boldsymbol{\omega}_{0}\) ), а также от расстояния \(\rho\) и скорости \(\mathbf{v}^{\prime}\) частицы в этой системе отсчета.

Если, например, неинерциальная система отсчета движется поступательно (по отношению к инерциальной системе отсчета), то в этой системе на свободную частицу действует только одна сила (2.20), направление которой противоположно ускорению \(\mathbf{a}_{0}\) данной системы отсчета. Вспомним, как при резком торможении вагона сила инерции бросает нас вперед, т. е. в сторону, противоположную вектору \(\mathbf{a}_{0}\).

Другой случай: система отсчета вращается с угловой скоростью \(\omega\) вокруг неподвижной оси, и тело \(A\) покоится в этой системе (например, вы сидите на горизонтальном вращающемся круге аттракциона «колесо смеха»). На тело \(A\) кроме силы взаимодействия с окружающими телами действует центробежная сила инерции (2.21), направленная от оси вращения вдоль

радиуса-вектора \(\rho\). Пока тело \(A\) покоится относительно круга \(\left(\mathbf{v}^{\prime}=0\right)\), эта сила компенсирует силу взаимодействия. Но как только тело придет в движение, т. е. появится скорость \(\mathbf{v}^{\prime}\), начнет действовать и сила Кориолиса (2.22), направление которой определяет векторное произведение \(\left[\mathbf{v}^{\prime} \boldsymbol{\omega}\right]\). Заметим, что сила Кориолиса появляется в дополнение к центробежной силе инерции, действующей независимо от того, покоится тело или движется во вращающейся системе отсчета.

Ранее было отмечено, что система отсчета, связанная с земной поверхностью, во многих случаях может считаться практически инерциальной. Однако существует ряд явлений, истолкование которых в этой системе отсчета невозможно без учета ее неинерциальности.

Известно, например, что ускорение свободного падения тел относительно поверхности Земли имеет наибольшее значение у полюсов. Уменьшение этого ускорения по мере приближения к экватору объясняется не только несферичностью Земли, но и возрастающим действием центробежной силы инерции. Или такие явления, как отклонение свободно падающих тел к востоку, размыв правых берегов рек в северном полушарии и левых берегов - в южном, вращение плоскости качания маятника Фуко и др. Подобные явления связаны с движением тел относительно поверхности Земли и могут быть объяснены действием сил Кориолиса.

Пример. Поезд массы \(m\) движется по меридиану на широте \(\varphi\) со скоростью \(\mathbf{v}^{\prime}\). Найти силу \(\mathbf{R}^{\prime}\) бокового давления, с которой поезд действует на рельсы.

\begin{figure}[h]
\begin{center}
  \includegraphics[alt={},max width=\textwidth]{66a35da9-3faa-4823-9ca9-40799f37b3db-055_358_346_1327_185}
\captionsetup{labelformat=empty}
\caption{Рис. 2.5}
\end{center}
\end{figure}

В системе отсчета, связанной с Землей (она вращается с угловой скоростью \(\boldsymbol{\omega}\) ), составляющая ускорения поезда, перпендикулярная плоскости меридиана, равна нулю. Поэтому и сумма проекций сил, действующих на поезд в этом направлении, также равна нулю. А это значит, что сила Кориолиса \(\mathbf{F}_{\text {кор }}\) (рис. 2.5) должна уравновешиваться силой \(\mathbf{R}\) бокового давления, действующей на поезд со стороны правого по ходу движения рельса,\\
т. е. \(\mathbf{F}_{\text {кор }}=-\mathbf{R}\). По третьему закону Ньютона, поезд будет действовать на этот рельс в горизонтальном направлении с силой \(\mathbf{R}^{\prime}=-\mathbf{R}\). Следовательно, \(\mathbf{R}^{\prime}=\mathbf{F}_{\text {кор }}=2 m\left[\mathbf{v}^{\prime} \boldsymbol{\omega}\right]\). Модуль вектора \(\mathbf{R}^{\prime}\) равен \(R^{\prime}=2 m v^{\prime} \omega \sin \varphi\).

Следующий простой пример показывает, как «возникают» силы инерции при переходе от инерциальной системы отсчета к неинерциальной.

Пример. На поверхности стола находится горизонтальный диск \(D\), свободно вращающийся вокруг вертикальной оси с постоянной угловой скоростью ω. Над диском висит шарик массы \(m\), как показано на рис. 2.6, \(a\). Рассмотрим поведение этого шарика в \(K\)-системе отсчета, связанной со столом (она предполагается инерциальной), и в \(K^{\prime}\)-системе, связанной с вращающимся диском.

В инерциальной \(K\)-системе на шарик действуют две силы: сила

\begin{figure}[h]
\begin{center}
  \includegraphics[alt={},max width=\textwidth]{66a35da9-3faa-4823-9ca9-40799f37b3db-056_463_414_508_742}
\captionsetup{labelformat=empty}
\caption{Рис. 2.6}
\end{center}
\end{figure}

тяжести и сила натяжения со стороны нити. Эти силы компенсируют друг друга, и шарик покоится.

В неинерциальной \(K^{\prime}\)-системе шарик движется равномерно по окружности с нормальным ускорением \(\omega^{2} \rho\), где \(\rho\) - расстояние от шарика до оси вращения. Легко убедиться, что это ускорение обусловлено действием сил инерции. В самом деле, в \(K^{\prime}\)-системе помимо указанных выше двух сил, компенсирующих друг друга, действуют еще центробежная сила инерции и сила Кориолиса (рис. 2.6, б). Взяв проекции этих сил на нормаль \(\mathbf{n}\) к траектории в точке нахождения шарика, запишем

\[
m a_{n}^{\prime}=F_{\text {кор }}-F_{\text {цб }}=2 m v^{\prime} \omega-m \omega^{2} \rho=m \omega^{2} \rho,
\]

где учтено, что в данном случае \(v^{\prime}=\omega \rho\). Тогда \(a_{n}^{\prime}=\omega^{2} \rho\).

\section*{Особенности сил инерции}
Подводя итог, перечислим важнейшие особенности сил инерции, отличающие их от сил взаимодействия:

\begin{enumerate}
  \item Силы инерции обусловлены не взаимодействием тел, а свойствами самих неинерциальных систем отсчета. Поэтому на силы инерции третий закон Ньютона не распространяется.
  \item Эти силы существуют только в неинерциальных системах отсчета - это необходимо твердо помнить во избежание недоразумений. В инерциальных системах отсчета сил инерции вообще нет, и понятие сила в этих системах отсчета применяется только в ньютоновском смысле - как мера взаимодействия тел.
  \item Все силы инерции, подобно силам тяготения, пропорциональны массе тела. Поэтому в однородном поле сил инерции, как и в поле сил тяготения, все тела движутся с одним и тем же ускорением независимо от их масс. Это весьма существенный факт с далеко идущими последствиями.
\end{enumerate}

\section*{Принцип эквивалентности}
Тот факт, что силы инерции, как и силы тяготения, пропорциональны массам тел, приводит к следующему важному заключению. Представим себе, что мы находимся в некоторой закрытой лаборатории и не имеем возможности наблюдать внешний мир. Допустим, кроме того, что мы не знаем, где находится лаборатория: в космическом пространстве или, скажем, на Земле. Замечая, что все тела независимо от их массы падают в лаборатории с одинаковым ускорением, мы не можем на основании только этого факта установить, чем вызвано это ускорение - полем тяготения, ускоренным поступательным движением самой лаборатории или, наконец, обеими этими причинами вместе. Никакие опыты по свободному падению тел в такой лаборатории не могут отличить однородное поле тяготения от однородного поля сил инерции.

Эйнштейн высказал предположение, что вообще никакими физическими опытами невозможно отличить однородное поле тяготения от однородного поля сил инерции. Это предположение, возведенное в постулат, и составляет содержание приниипа эквивалентности сил тяготения и сил инерции: все физические явления в однородном поле тяготения происходят совершенно так же, как и в соответствующем однородном поле сил инериии.

Глубокая аналогия между силами инерции и силами тяготения послужила отправным пунктом при построении Эйнштейном общей теории относительности, или релятивистской теории гравитации.

В заключение необходимо отметить, что любую механическую задачу можно решить в инерциальной и неинерциальной системах отсчета. Выбор той или иной системы отсчета обычно диктуется или постановкой вопроса, или стремлением полу-

чить решение возможно более простым путем. При этом часто наиболее удобно пользоваться именно неинерциальными системами отсчета (см. задачи 2.9-2.11).

\section*{Задачи}
2.1. Основное уравнение динамики. Брусок массы \(m_{1}\) находится на доске массы \(m_{2}\), которая лежит на гладкой горизонтальной плоскости (рис. 2.7). Коэффициент трения между бруском и доской равен \(k\). К доске приложили горизонтальную силу \(F\), зависящую от времени по закону \(F=\alpha t\), где \(\alpha\) - постоянная. Найти:

\begin{enumerate}
  \item момент времени \(t_{0}\), когда доска начнет выскальзывать из-под бруска;
  \item ускорения бруска \(a_{1}\) и доски \(a_{2}\) в процессе движения.
\end{enumerate}

Решение. 1. Запишем основное уравнение динамики для бруска и доски, взяв положительное направление оси \(X\), как показано на рисунке:


\begin{equation*}
m_{1} a_{1}=F_{\text {тр }}, \quad m_{2} a_{2}=F-F_{\text {тр }} . \tag{1}
\end{equation*}


По мере возрастания силы \(F\) будет расти и сила трения \(F_{\text {тр }}\) (вначале она является силой трения покоя). Но \(F_{\text {тр }}\) имеет предел \(F_{\text {тр.макс }}=k m_{1} g\). Пока этот предел не достигнут, оба тела будут двигаться как одно целое с одинаковыми ускорениями. Когда же сила \(F_{\text {тр }}\) достигнет предела, до-

\begin{figure}[h]
\begin{center}
  \includegraphics[alt={},max width=\textwidth]{66a35da9-3faa-4823-9ca9-40799f37b3db-058_172_371_922_789}
\captionsetup{labelformat=empty}
\caption{Рис. 2.7}
\end{center}
\end{figure}

ска начнет выскальзывать из-под бруска, т. е.

\[
a_{2} \geqslant a_{1} .
\]

Подставив сюда выражения для \(a_{1}\) и \(a_{2}\) из (1) с учетом того, что \(F_{\text {тр }}=k m_{1} g\), получим

\[
\left(\alpha t-k m_{1} g\right) / m_{2} \geqslant k g,
\]

где знак равенства соответствует моменту \(t=t_{0}\). Отсюда

\[
t_{0}=\left(m_{1}+m_{2}\right) k g / \alpha .
\]

\begin{enumerate}
  \setcounter{enumi}{1}
  \item Если \(t \leqslant t_{0}\), то
\end{enumerate}

\[
a_{1}=a_{2}=\alpha t /\left(m_{1}+m_{2}\right) ;
\]

если же \(t \geqslant t_{0}\), то

\[
a_{1}=k g=\text { const }, \quad a_{2}=\left(\alpha t-k m_{1} g\right) / m_{2} .
\]

\begin{figure}[h]
\begin{center}
  \includegraphics[alt={},max width=\textwidth]{66a35da9-3faa-4823-9ca9-40799f37b3db-059_288_412_132_184}
\captionsetup{labelformat=empty}
\caption{Рис. 2.8}
\end{center}
\end{figure}

\begin{figure}[h]
\begin{center}
  \includegraphics[alt={},max width=\textwidth]{66a35da9-3faa-4823-9ca9-40799f37b3db-059_286_453_130_670}
\captionsetup{labelformat=empty}
\caption{Рис. 2.9}
\end{center}
\end{figure}

Графики зависимостей \(a_{1}\) и \(a_{2}\) от \(t\) показаны на рис. 2.8.\\
2.2. В установке (рис. 2.9) наклонная плоскость составляет угол \(\alpha=30^{\circ}\) с горизонтом. Отношение масс тел \(m_{1} / m_{2}=\eta=2 / 3\). Коэффициент трения между телом \(m_{2}\) и плоскостью \(k=0,10\). Массы блока и нити пренебрежимо малы. Найти модуль и направление ускорения тела \(m_{1}\), если система пришла в движение из состояния покоя.

Решение. Здесь сразу же возникает вопрос, связанный с направлением силы трения, действующей на тело \(m_{2}\). Без ответа на этот вопрос нельзя записать основное уравнение динамики для тела \(m_{2}\) в проекциях и задача становится неопределенной.\\
Будем рассуждать так: в отсутствие силы трения тело начало бы скользить по наклонной плоскости, допустим, вверх. Ясно, что «включение» силы трения не может изменить направления движения, а только уменьшит ускорение. Таким образом, направление силы трения, действующей на тело \(m_{2}\), будет определено, если найти направление ускорения этого тела в отсутствие трения ( \(k=0\) ). С этого мы и начнем.\\
Запишем основное уравнение динамики для обоих тел в проекциях, взяв положительные направления осей \(X_{1}\) и \(X_{2}\), как показано на рис. 2.9:

\[
m_{1} a_{x}=m_{1} g-T, \quad m_{2} a_{x}=T-m_{2} g \sin \alpha
\]

где \(T\) - сила натяжения нити. Сложив почленно левые и правые части этих уравнений, получим

\[
a_{x}=\frac{\eta-\sin \alpha}{\eta+1} g .
\]

Подставив в это выражение \(\eta=\frac{2}{3}\) и \(\alpha=30^{\circ}\), найдем \(a_{x}>0\), т. е. тело \(m_{2}\) начнет двигаться вверх по наклонной плоскости. Следовательно, сила трения, действующая на это тело, направлена в про-

тивоположную сторону. С учетом этого обстоятельства снова запишем уравнения движения:

\[
\begin{gathered}
m_{1} a_{x}^{\prime}=m_{1} g-T^{\prime}, \\
m_{2} a_{x}^{\prime}=T^{\prime}-m_{2} g \sin \alpha-k m_{2} g \cos \alpha,
\end{gathered}
\]

откуда

\[
a_{x}^{\prime}=\frac{\eta-\sin \alpha-k \cos \alpha}{\eta+1} g=0,05 g .
\]

2.3. Через блок (рис. 2.10) перекинута нерастяжимая нить, на концах которой висят грузы 1 и 2 массами \(m_{1}\) и \(m_{2}\) соответственно. Блок начали поднимать вверх с ускорением \(\mathbf{a}_{0}\) относительно поверхности земли. Полагая, что нить скользит по блоку без трения, найти ускорение \(a_{1}\) груза 1 относительно поверхности земли.\\
Решение. Выберем положительное направление оси \(X\) вверх и запишем для обоих грузов основное уравнение динамики в проекциях на эту ось:

\begin{figure}[h]
\begin{center}
  \includegraphics[alt={},max width=\textwidth]{66a35da9-3faa-4823-9ca9-40799f37b3db-060_338_271_596_889}
\captionsetup{labelformat=empty}
\caption{Рис. 2.10}
\end{center}
\end{figure}


\begin{align*}
& m_{1} a_{1 x}=T-m_{1} g  \tag{1}\\
& m_{2} a_{2 x}=T-m_{2} g \tag{2}
\end{align*}


Эти уравнения содержат три неизвестных: \(a_{1 x}, a_{2 x}\) и T. Для составления третьего уравнения воспользуемся кинематической связью между ускорениями:

\[
\mathbf{a}_{1}=\mathbf{a}_{0}+\mathbf{a}^{\prime}, \quad \mathbf{a}_{2}=\mathbf{a}_{0}-\mathbf{a}^{\prime}
\]

где \(\mathbf{a}^{\prime}\) - ускорение груза 1 относительно блока. Сложив почленно левые и правые части этих равенств, получим \(\mathbf{a}_{1}+\mathbf{a}_{2}=2 \mathbf{a}_{0}\), или в проекциях на ось \(X\)


\begin{equation*}
a_{1 x}+a_{2 x}=2 a_{0} \tag{3}
\end{equation*}


Решив совместно уравнения (1), (2) и (3), найдем

\[
a_{1 x}=\frac{2 m_{2} a_{0}+\left(m_{2}-m_{1}\right) g}{m_{1}+m_{2}}
\]

Отсюда видно, что при заданном \(a_{0}\) знак \(a_{1 x}\) зависит от соотношения масс \(m_{1}\) и \(m_{2}\).\\
2.4. Небольшая шайба движется по наклонной плоскости, коэффици-

\begin{figure}[h]
\begin{center}
  \includegraphics[alt={},max width=\textwidth]{66a35da9-3faa-4823-9ca9-40799f37b3db-061_235_395_192_129}
\captionsetup{labelformat=empty}
\caption{Рис. 2.11}
\end{center}
\end{figure}

ент трения которой \(k=\operatorname{tg} \alpha\), где \(\alpha\) угол наклона плоскости к горизонту. Найти зависимость скорости \(v\) шайбы от угла \(\varphi\) между вектором \(\mathbf{v}\) и осью \(X\) (рис. 2.11), если в начальный момент \(v=v_{0}\) и \(\varphi=\pi / 2\).

Решение. Ускорение шайбы вдоль плоскости определяется составляющей силы тяжести на эту плоскость \(F_{x}=m g \sin \alpha\) и силой трения \(F_{\text {тр }}=k m g \cos \alpha\). В нашем случае \(k=\operatorname{tg} \alpha\), поэтому

\[
F_{\text {тр }}=F_{x}=m g \sin \alpha .
\]

Найдем проекции ускорения на направление касательной к траектории и на ось \(X\) :

\[
\begin{aligned}
& m a_{\tau}=F_{x} \cos \varphi-F_{\text {тр }}=m g \sin \alpha(\cos \alpha-1), \\
& m a_{x}=F_{x}-F_{\text {тр }} \cos \varphi=m g \sin \alpha(1-\cos \varphi) .
\end{aligned}
\]

Отсюда видно, что \(a_{\tau}=-a_{x}\), а это значит, что скорость \(v\) и ее проекция \(v_{x}\) различаются лишь на некоторую постоянную \(C\), не зависящую от времени, т. е. \(v=-v_{x}+C\), где \(v_{x}=v \cos \varphi\). Постоянную \(C\) находим из начального условия \(v=v_{0}\), откуда \(C=v_{0}\). В результате получим

\[
v=v_{0} /(1+\cos \varphi) .
\]

С ростом времени \(\varphi \rightarrow 0\) и \(v \rightarrow v_{0} / 2\).\\
2.5. Тонкий однородный упругий шнур массы \(m\) и длины \(l_{0}\) (в нерастя-

\begin{figure}[h]
\begin{center}
  \includegraphics[alt={},max width=\textwidth]{66a35da9-3faa-4823-9ca9-40799f37b3db-061_371_362_1282_135}
\captionsetup{labelformat=empty}
\caption{Рис. 2.12}
\end{center}
\end{figure}

нутом состоянии) имеет коэффициент упругости \(\varkappa\). Склеив торцы, шнур положили на гладкую горизонтальную плоскость, придали ему форму окружности и раскрутили до угловой скорости \(\omega\) вокруг вертикальной оси, проходящей через его центр. Найти силу натяжения шнура в этом состоянии.

Решение. Мысленно выделим малый элемент шнура массы \(\delta m\), как показано на рис. 2.12, \(a\). Этот элемент движется по окружности под действием силы, представляющей собой геометрическую сумму двух векторов, каждый из которых равен по

модулю искомой силе натяжения \(T\) (рис. 2.12, б). Поэтому, согласно основному уравнению динамики,


\begin{equation*}
\delta m \omega^{2} r=T \delta \alpha . \tag{1}
\end{equation*}


Учтем, что \(\delta m=(m / 2 \pi)\) и \(r=l / 2 \pi, l\) — длина шнура во вращающемся состоянии.\\
Тогда (1) примет вид


\begin{equation*}
m \omega^{2} l / 4 \pi^{2}=T . \tag{2}
\end{equation*}


С другой стороны, по закону Гука,


\begin{equation*}
T=\varkappa\left(l-l_{0}\right) . \tag{3}
\end{equation*}


Исключив \(l\) из (2) и (3), получим

\[
T=\frac{\varkappa l_{0}}{4 \pi^{2} \varkappa / m \omega^{2}-1}
\]

Заметим, что в случае нерастяжимого шнура ( \(\varkappa \rightarrow \infty\) )\\
\(T=m \omega^{2} l_{0} / 4 \pi^{2}\).\\
2.6. Интегрирование уравнений движения. Частица массы \(m\) движется под действием силы \(\mathbf{F}\). В момент \(t=0\) известны ее радиус-вектор \(\mathbf{r}(0)\) и скорость \(\mathbf{v}(0)\) - начальные условия. Найти положение частицы, т. е. ее радиус-вектор \(\mathbf{r}\), в зависимости от времени \(t\), если:

\begin{enumerate}
  \item \(\mathbf{F}=\mathbf{F}_{0} \sin \omega t, \quad \mathbf{r}(0)=0, \quad \mathbf{v}(0)=0\);
  \item \(\mathbf{F}=-k \mathbf{v}, \quad \mathbf{r}(0)=0, \quad \mathbf{v}(0)=\mathbf{v}_{0}\).
\end{enumerate}

Здесь \(\mathbf{F}_{0}\) - постоянный вектор, \(\omega\) и \(k\) - положительные постоянные.

Решение. 1. Согласно основному уравнению динамики, в данном случае ускорение

\[
\mathrm{d} \mathbf{v} / \mathrm{d} t=\left(\mathbf{F}_{\mathbf{0}} / m\right) \sin \omega t .
\]

Отсюда находим элементарное приращение вектора скорости \(d \mathbf{v}\) за время \(\mathrm{d} t\) и затем приращение этого вектора за время от 0 до \(t\) :

\[
\mathbf{v}(t)-\mathbf{v}(0)=\left(\mathbf{F}_{0} / m\right) \int_{0}^{t} \sin \omega t \mathrm{~d} t
\]

Учитывая, что \(\mathbf{v}(0)=0\), после интегрирования получим

\[
\mathbf{v}(t)=(1-\cos \omega t) \mathbf{F}_{0} / m \omega .
\]

Теперь найдем \(\mathrm{d} \mathbf{r}\) - элементарное перемещение, или приращение радиуса-вектора \(\mathbf{r}\) частицы за время \(\mathrm{d} t: \mathrm{d} \mathbf{r}=\mathbf{v}(t) \mathrm{d} t\). Приращение же радиуса-вектора за время от 0 до \(t\)

\[
\mathbf{r}(t)-\mathbf{r}(0)=\left(\mathbf{F}_{0} / m \omega\right) \int_{0}^{t}(1-\cos \omega t) \mathrm{d} t
\]

В результате интегрирования находим

\begin{figure}[h]
\begin{center}
  \includegraphics[alt={},max width=\textwidth]{66a35da9-3faa-4823-9ca9-40799f37b3db-063_224_412_489_138}
\captionsetup{labelformat=empty}
\caption{Рис. 2.13}
\end{center}
\end{figure}

где учтено, что \(\mathbf{r}(0)=0\).\\
На рис. 2.13 показаны графики зависимостей \(v_{x}(t)\) и \(x(t)\) - проекций векторов \(\mathbf{v}\) и \(\mathbf{r}\) на ось \(X\), выбранную в направлении движения частицы, т. е. в направлении вектора \(\mathbf{F}_{0}\).\\
2. В этом случае ускорение

\[
\mathrm{d} \mathbf{v} / \mathrm{d} t=-(k / m) \mathbf{v}
\]

Для интегрирования этого уравнения перейдем к скалярной форме - к модулю вектора \(\mathbf{v}\) :

\[
\mathrm{d} v / \mathrm{d} t=-(k / m) v .
\]

Интегрирование этого уравнения (с учетом начальных условий) дает \(\ln \left(v / v_{0}\right)=-(k / m) t\). После потенцирования возвращаемся к векторной форме:

\[
\mathbf{v}=\mathbf{v}_{0} \mathrm{e}^{-k t / m}
\]

Последнее уравнение интегрируем еще раз (и также с учетом начальных условий):

\begin{figure}[h]
\begin{center}
  \includegraphics[alt={},max width=\textwidth]{66a35da9-3faa-4823-9ca9-40799f37b3db-063_283_355_1405_135}
\captionsetup{labelformat=empty}
\caption{Рис. 2.14}
\end{center}
\end{figure}

\[
\mathbf{r}=\int_{0}^{t} \mathbf{v} \mathrm{~d} t=\left(1-\mathrm{e}^{-k t / m}\right) m \mathbf{v}_{0} / k
\]

На рис. 2.14 показаны графики скорости и пройденного пути \(s\) в зависимости от времени \(t\) (в нашем случае \(s=r\) ).\\
2.7. Частица массы \(m\) движется в некоторой плоскости под действием постоянной по модулю силы \(F\), направление которой поворачивается с постоянной угловой скоростью \(\omega\). В момент \(t=0\) скорость частицы равна нулю. Найти модуль скорости частицы как функцию времени \(t\), а также путь, проходимый частицей между двумя последовательными остановками.\\
Решение. Свяжем с данной плоскостью систему координат \(x\), \(y\), взяв ось \(X\) в направлении, которое имел вектор силы в момент \(t=0\) (рис. 2.15). Тогда основное уравнение динамики в проекциях на оси \(X\) и \(Y\) будет иметь вид

\[
\begin{aligned}
& m \mathrm{~d} v_{x} / \mathrm{d} t=F \cos \omega t, \\
& m \mathrm{~d} v_{y} / \mathrm{d} t=F \sin \omega t
\end{aligned}
\]

Проинтегрировав эти уравнения по времени с учетом начального условия \(v(0)=0\), получим

\begin{figure}[h]
\begin{center}
  \includegraphics[alt={},max width=\textwidth]{66a35da9-3faa-4823-9ca9-40799f37b3db-064_303_322_433_838}
\captionsetup{labelformat=empty}
\caption{Рис. 2.15}
\end{center}
\end{figure}

\[
v_{x}=(F / m \omega) \sin \omega t, \quad v_{y}=(F / m \omega)(1-\cos \omega t) .
\]

Модуль вектора скорости частицы

\[
v=\sqrt{v_{x}^{2}+v_{y}^{2}}=2|\sin (\omega t / 2)| F / m \omega
\]

Отсюда видно, что скорость \(v\) обращается в нуль через промежуток времени \(\Delta t\), который можно найти из условия \(\omega \Delta t / 2=\pi\). Поэтому искомый путь

\[
s=\int_{0}^{\Delta t} v \mathrm{~d} t=8 F / m \omega^{2} .
\]

Заметим, что траектория частицы имеет вид циклоиды.\\
2.8. Автомашина движется с постоянным тангенциальным ускорением \(a_{\tau}\) по горизонтальной поверхности, описывая окружность радиуса \(R\). Коэффициент трения между колесами машины и поверхностью равен \(k\). Какой путь \(s\) пройдет машина без скольжения, если начальная скорость ее была равна нулю?\\
Решение. По мере увеличения скорости будет расти как нормальное, так и полное ускорение машины. Движение будет происходить без скольжения, пока необходимое полное ускорение будет обеспечиваться силой трения. Максимально возможное значение этой силы \(F_{\text {макс }}=k m g\), где \(m\) - масса машины. Поэтому максимальное значение полного ускорения (согласно основному уравнению динамики \(m \mathbf{a}=\mathbf{F}\) )


\begin{equation*}
a_{\text {макс }}=k g . \tag{1}
\end{equation*}


С другой стороны,


\begin{equation*}
a_{\text {макс }}=\sqrt{a_{\tau}^{2}+\left(v^{2} / R\right)^{2}}, \tag{2}
\end{equation*}


где \(v\) - скорость машины в момент, когда ее ускорение станет максимально возможным. Эта скорость и искомый путь \(s\) связаны формулой


\begin{equation*}
v^{2}=2 a_{\tau} s . \tag{3}
\end{equation*}


Выразив \(v\) из (1) и (2) и подставив в (3), получим

\[
s=(R / 2) \sqrt{\left(k g / a_{\tau}\right)^{2}-1}
\]

Нетрудно видеть, что решение имеет смысл при \(a_{\tau}<k g\).\\
2.9. Неинерциальные системы отсчета. Спутник движется в экваториальной плоскости Земли с запада на восток по круговой орбите радиуса \(r\). Пренебрегая ускорением, обусловленным движением Земли вокруг Солнца, найти ускорение \(a^{\prime}\) спутника в системе отсчета, связанной с Землей.

\begin{figure}[h]
\begin{center}
  \includegraphics[alt={},max width=\textwidth]{66a35da9-3faa-4823-9ca9-40799f37b3db-065_287_533_844_141}
\captionsetup{labelformat=empty}
\caption{Рис. 2.16}
\end{center}
\end{figure}

Решении е. Пусть \(K\) инерциальная (по условию) система отсчета, в которой ось вращения Земли покоится, а \(K^{\prime}\) - неинерциальная система отсчета, которая связана с Землей и вращается с угловой скоростью \(\omega\) относительно \(K\)-системы.

Нас интересует ускорение \(a^{\prime}\) спутника в \(K^{\prime}\)-системе. Для этого прежде всего изобразим все силы, действующие на спутник в этой системе отсчета: силу тяготения \(\mathbf{F}\), силу Кориолиса \(\mathbf{F}_{\text {кор }}\) и центробежную силу инерции \(\mathbf{F}_{\text {цб }}\) (рис. 2.16, вид со стороны Северного полюса).\\
Теперь воспользуемся уравнением (2.18), считая \(\mathbf{a}_{0}=0\) (по условию). Спутник движется по окружности в \(K^{\prime}\)-системе, поэтому запишем уравнение (2.18) в проекциях на нормаль \(\mathbf{n}\) к траектории:


\begin{equation*}
m a^{\prime}=F-2 m v^{\prime} \omega-m \omega^{2} r, \tag{1}
\end{equation*}


где \(F=\gamma m M / r^{2}, m\) и \(M\) - массы спутника и Земли. Остается найти скорость \(v^{\prime}\) спутника в \(K^{\prime}\)-системе. Для этого воспользуемся формулой (1.24) в скалярном виде


\begin{equation*}
v^{\prime}=v-\omega r, \tag{2}
\end{equation*}


где \(v\) - скорость спутника в \(K\)-системе. Эту скорость можно определить с помощью уравнения движения спутника в \(K\)-системе:


\begin{equation*}
m v^{2} / r=\gamma m M / r^{2} . \tag{3}
\end{equation*}


Решив совместно уравнения (1), (2) и (3), получим

\[
a^{\prime}=\frac{\gamma M}{r^{2}}\left(1-\omega r \sqrt{\frac{r}{\gamma M}}\right)^{2}
\]

В частности, \(a^{\prime}=0\) при \(r=\sqrt[3]{\gamma M / \omega^{2}}=4,2 \cdot 10^{4}\) км. Такой спутник называют стационарным: он неподвижен относительно поверхности Земли.\\
2.10. Небольшая муфта массы \(m\) свободно скользит по гладкому горизонтальному стержню, который вращают с постоянной угловой скоростью \(\omega\) вокруг неподвижной вертикальной оси, проходящей через один из его концов. Найти горизонтальную составляющую силы, действующей на муфту со стороны стержня в момент, когда она находится на расстоянии \(r\) от оси вращения. (В начальный момент муфта находилась непосредственно около оси и имела пренебрежимо малую скорость.)\\
Решение. Рассмотрим движение муфты во вращающейся системе отсчета, жестко связанной со стержнем. В этой системе отсчета муфта движется прямолинейно, а это значит, что искомая сила \(\mathbf{R}\) уравновешивается силой Кориолиса (рис. 2.17, вид сверху):

\begin{figure}[h]
\begin{center}
  \includegraphics[alt={},max width=\textwidth]{66a35da9-3faa-4823-9ca9-40799f37b3db-066_218_370_984_790}
\captionsetup{labelformat=empty}
\caption{Рис. 2.17}
\end{center}
\end{figure}


\begin{equation*}
\mathbf{R}=-\mathbf{F}_{\text {кор }}=2 m\left[\omega \mathbf{v}^{\prime}\right] . \tag{1}
\end{equation*}


Задача, таким образом, сводится к нахождению скорости \(\mathbf{v}^{\prime}\) муфты относительно стержня. Согласно (2.19),

\[
\mathrm{d} v^{\prime} / \mathrm{d} t=F_{\text {цб }} / m=\omega^{2} r .
\]

Преобразуем это уравнение, имея в виду, что \(\mathrm{d} t=\mathrm{d} r / v^{\prime}\). Тогда получим \(v^{\prime} \mathrm{d} v^{\prime}=\omega^{2} r \mathrm{~d} r\). Проинтегрировав последнее уравнение с учетом начальных условий ( \(v^{\prime}=0, r=0\) ), найдем \(v^{\prime}=\omega r\), или в векторном виде


\begin{equation*}
\mathbf{v}^{\prime}=\omega \mathbf{r} . \tag{2}
\end{equation*}


Подстановка (2) в (1) дает

\[
\mathbf{R}=2 m \omega[\omega \mathbf{r}] .
\]

2.11. Устойчивость движения. Небольшая муфта \(M\) может скользить без трения по гладкому проводу, изогнутому в форме полуок-

\begin{figure}[h]
\begin{center}
  \includegraphics[alt={},max width=\textwidth]{66a35da9-3faa-4823-9ca9-40799f37b3db-067_378_215_221_187}
\captionsetup{labelformat=empty}
\caption{Рис. 2.18}
\end{center}
\end{figure}

ружности радиуса \(r\) (рис.2.18). Систему привели во вращение с постоянной угловой скоростью \(\omega\) вокруг вертикальной оси \(O O^{\prime}\). Найти угол \(\vartheta_{0}\), соответствующий устойчивому положению муфты.

Решение. Рассмотрим поведение муфты в системе отсчета, связанной с вращающимся проводом. Движение муфты вдоль провода будет определяться проекцией \(F_{\tau}\) результирующей силы на направление орта \(\boldsymbol{\tau}\) в точке \(M\).\\
Из рис. 2.18 видно, что

\[
F_{\tau}=F_{\text {цб }} \cos \vartheta-m g \sin \vartheta,
\]

где справа записаны проекции центробежной силы инерции и силы тяжести. Учтя, что \(F_{\text {цб }}=m \omega^{2} r \sin \vartheta\), перепишем предыдущее выражение так:


\begin{equation*}
F_{\tau} \circlearrowleft \sin \vartheta\left(\cos \vartheta-g / \omega^{2} r\right) . \tag{1}
\end{equation*}


Из условия равновесия ( \(F_{\tau}=0\) ) найдем два значения угла \(\vartheta_{0}\), при которых оно возможно: \(\sin \vartheta_{0}=0\) и \(\cos \vartheta_{0}=g / \omega^{2} r\). Первое условие может быть осуществлено при любых значениях \(\omega\), второе же - только при \(g / \omega^{2} r<1\). Таким образом, при малых \(\omega\) существует только одно положение равновесия - в нижней точке ( \(\vartheta_{0}=0\) ); при больших же \(\omega(\omega>\sqrt{g / r})\) возможно и другое положение равновесия, определяемое вторым условием.\\
Для устойчивости определенного состояния равновесия необходимо, чтобы сила \(F_{\tau}\) при выведении муфты из положения равновесия (в любую сторону) была направлена обратно - к положению равновесия. Другими словами, знак \(F_{\tau}\) должен быть противоположным знаку отклонения \(\Delta \vartheta\) от равновесного угла \(\vartheta_{0}\).\\
При малом отклонении \(d \vartheta\) от угла \(\vartheta_{0}\) возникающая сила \(\delta F_{\tau}\) может быть найдена как дифференциал выражения (1):

\[
\delta F_{\tau} \sim\left[\cos \vartheta_{0}\left(\cos \vartheta_{0}-g / \omega^{2} r\right)-\sin ^{2} \vartheta_{0}\right] \mathrm{d} \vartheta .
\]

В нижнем положении равновесия ( \(\vartheta_{0}=0\) )


\begin{equation*}
\delta F_{\tau} \circlearrowleft\left(1-g / \omega^{2} r\right) \mathrm{d} \vartheta . \tag{2}
\end{equation*}


Это положение равновесия будет устойчивым, когда выражение, стоящее в скобках, отрицательно, т. е. при \(\omega<\sqrt{g / r}\).\\
В другом положении равновесия ( \(\cos \vartheta_{0}=g / \omega^{2} r\) )

\[
\delta F_{\tau} \circlearrowleft-\sin ^{2} \vartheta_{0} \mathrm{~d} \vartheta .
\]

Видно, что это положение равновесия (если оно существует) всегда устойчиво.

Итак, пока существует лишь нижнее положение равновесия (при \(\omega<\sqrt{g / r}\) ), оно всегда устойчиво. При появлении же другого положения равновесия (когда \(\omega>\sqrt{g / r}\) ) нижнее положение, согласно (2), становится неустойчивым и муфта сразу переходит из нижнего положения в верхнее, которое всегда устойчиво.

\section*{Глава 3}
\section*{Закон сохранения импульса}
\section*{§ 3.1. О законах сохранения}
Любое тело (или совокупность тел) представляет собой, по существу, систему материальных точек, или частиц. Если система с течением времени изменяется, то говорят, что изменяется ее состояние. Состояние системы характеризуется одновременным заданием положений (координат) и скоростей всех ее частиц.

Зная законы действующих на частицы системы сил и состояние системы в некоторый начальный момент времени, можно, как показывает опыт, с помощью уравнений движения предсказать ее дальнейшее поведение, т. е. найти состояние системы в любой момент времени. Так, например, решается задача о движении планет Солнечной системы.

Однако детальное рассмотрение поведения системы с помощью уравнений движения часто бывает настолько затруднительно (например, из-за сложности самой системы), что довести решение до конца представляется практически невозможным. А в тех случаях, когда законы действующих сил вообще неизвестны, такой подход оказывается в принципе неосуществимым. Кроме того, существует ряд задач, в которых детальное рассмотрение движения отдельных частиц просто и не имеет смысла (например, описание движения отдельных молекул газа).

При таком положении естественно возникает вопрос: нет ли каких-либо общих принципов, являющихся следствием законов Ньютона, которые позволили бы иначе подойти к решению задачи и помогли бы в какой-то степени обойти подобные трудности?

Оказывается, такие принципы есть. Это законы сохранения. Как уже было сказано, при движении системы ее состояние изменяется со временем. Существуют, однако, такие величины, которые обладают весьма важным и замечательным свойством сохраняться во времени. Среди этих сохраняющихся величин наиболее важную роль играют энергия, импульс и момент импульса. Эти три величины имеют важное общее свойство аддитивности: их значение для системы, состоящей из частей, взаи-

модействие которых пренебрежимо мало, равно сумме значений для каждой из частей в отдельности (впрочем, для импульса и момента импульса свойство аддитивности выполняется и при наличии взаимодействия). Именно свойство аддитивности и придает этим трем величинам особую роль.

Законы сохранения энергии, импульса и момента импульса имеют, как выяснилось впоследствии, весьма глубокое происхождение, связанное с фундаментальными свойствами времени и пространства - однородностью и изотропностью. А именно: закон сохранения энергии связан с однородностью времени, а законы сохранения импульса и момента импульса - соответственно с однородностью и изотропностью пространства. Сказанное следует понимать в том смысле, что перечисленные законы сохранения можно получить из второго закона Ньютона, если к нему присоединить соответствующие свойства симметрии времени и пространства. Более подробно обсуждать этот вопрос мы, однако, не будем.

Законы сохранения энергии, импульса и момента импульса относятся к числу тех фундаментальных принципов физики, значение которых трудно переоценить. Роль этих законов особенно возросла после того, как выяснилось, что они далеко выходят за рамки механики и представляют собой универсальные законы природы. Во всяком случае, до сих пор не обнаружено ни одного явления, где бы эти законы нарушались. Они безошибочно «действуют» и в области элементарных частиц, и в области космических объектов, в физике атома и физике твердого тела и являются одними из тех немногих наиболее общих законов, которые лежат в основе современной физики.

Открыв возможность иного подхода к рассмотрению различных механических явлений, законы сохранения стали весьма мощным и эффективным инструментом исследования, которым повседневно пользуются физики. Эта важнейшая роль законов сохранения как инструмента исследования обусловлена рядом причин:

\begin{enumerate}
  \item Законы сохранения не зависят ни от траекторий частиц, ни от характера действующих сил. Поэтому они позволяют получить ряд весьма общих и существенных заключений о свойствах различных механических процессов, не вникая в их детальное рассмотрение с помощью уравнений движения. Если, например, выясняется, что такой-то процесс противоречит законам сохранения, то сразу можно утверждать: этот процесс невозможен, и бессмысленно пытаться его осуществить.
  \item Тот факт, что законы сохранения не зависят от характера действующих сил, позволяет использовать их даже тогда, когда силы вообще неизвестны. В этих случаях законы сохранения являются единственным и незаменимым инструментом исследования. Так, например, обстоит дело в физике элементарных частиц.
  \item Даже в тех случаях, когда силы в точности известны, законы сохранения могут оказать существенную помощь при решении многих задач о движении частиц. Хотя все эти задачи могут быть решены с помощью уравнений движения (в этом отношении из законов сохранения мы не получим никакой дополнительной информации), привлечение законов сохранения очень часто позволяет получить решение наиболее простым и изящным путем, избавляя нас от громоздких и утомительных расчетов. Поэтому при решении новых задач обычно принято придерживаться следующего порядка: прежде всего один за другим применяют соответствующие законы сохранения и, только убедившись, что этого недостаточно, переходят затем к решению с помощью уравнения движения.
\end{enumerate}

Изучение законов сохранения начнем с закона сохранения импульса.

\section*{§ 3.2. Импульс системы}
\section*{Импульс частицы}
По определению, импульс частиць*

\[
\mathbf{p}=m \mathbf{v},
\]

где \(m\) и \(\mathbf{v}\) - ее масса и скорость. Воспользовавшись понятием импульса, запишем основное уравнение динамики (2.6) в иной форме:


\begin{equation*}
\mathrm{d} \mathbf{p} / \mathrm{d} t=\mathbf{F}, \tag{3.1}
\end{equation*}


т. е. производная импульса материальной точки по времени равна действующей на нее силе. В частности, если \(\mathbf{F} \equiv 0\), то \(\mathbf{p}=\) const.

\footnotetext{\begin{itemize}
  \item Другое название этой величины - количество движения.
\end{itemize}
}Заметим, что в неинерциальной системе отсчета сила \(\mathbf{F}\) в (3.1) включает в себя не только силы взаимодействия данной частицы с другими телами, но и силы инерции.

Уравнение (3.1) позволяет найти приращение импульса частицы за любой промежуток времени, если известна зависимость силы \(\mathbf{F}\) от времени. Действительно, из (3.1) следует, что элементарное приращение импульса частицы за промежуток времени \(\mathrm{d} t\) есть \(\mathrm{d} \mathbf{p}=\mathbf{F} \mathrm{d} t\). Проинтегрировав это выражение по времени, найдем приращение импульса частицы за конечный промежуток времени \(t\) :


\begin{equation*}
\mathbf{p}_{2}-\mathbf{p}_{1}=\int_{0}^{t} \mathbf{F} \mathrm{~d} t \tag{3.2}
\end{equation*}


Величину, стоящую в правой части этого уравнения, называют импульсом силь. Таким образом, приращение импульса частицы за любой промежуток времени зависит не только от значения силы, но и от продолжительности ее действия, или, другими словами, равно импульсу силы за это время.

В частности, если \(\mathbf{F}=\) const, то вектор \(\mathbf{F}\) можно вынести из-под интеграла и тогда \(\mathbf{p}_{2}-\mathbf{p}_{1}=\mathbf{F} t\).

Рассмотрим пример на использование уравнения (3.2).\\
Пример. На частицу, которая в момент \(t=0\) имела импульс \(\mathbf{p}_{0}\), действует в течение промежутка времени \(\tau\) сила, зависящая от времени \(t\) как \(\mathbf{F}=\mathbf{a} t(1-t / \tau)\), где \(\mathbf{a}\) - постоянный вектор. Найдем импульс p частицы после окончания действия этой силы.\\
Согласно (3.2),

\[
\mathbf{p}=\mathbf{p}_{0}+\int_{0}^{\tau} \mathbf{F} \mathrm{d} t=\mathbf{p}_{0}+\mathbf{a} \tau^{2} / 6 \text { (рис. 3.1). }
\]

\begin{figure}[h]
\begin{center}
  \includegraphics[alt={},max width=\textwidth]{66a35da9-3faa-4823-9ca9-40799f37b3db-072_217_342_1177_822}
\captionsetup{labelformat=empty}
\caption{Рис. 3.1}
\end{center}
\end{figure}

\section*{Импульс системы}
Рассмотрим произвольную систему частиц. В общем случае частицы этой системы могут взаимодействовать как между собой, так и с телами, не входящими в данную систему. В соответствии с этим силы взаимодействия между частицами системы называют внутренними, а силы, обусловленные действием

других тел, не входящих в данную систему, - внешними. Ясно, что такое разделение сил на внутренние и внешние условно - оно целиком зависит от выбора интересующей нас системы частиц. Заметим также, что в неинерциальных системах отсчета к внешним силам относятся и силы инерции.

Теперь введем понятие импульса системь как векторную сумму импульсов ее отдельных частиц:


\begin{equation*}
\mathbf{p}=\sum \mathbf{p}_{i}, \tag{3.3}
\end{equation*}


где \(\mathbf{p}_{i}\) - импульс \(i\)-й частицы. Заметим, что импульс системы - величина аддитивная, т. е. импульс системы равен сумме импульсов ее отдельных частей независимо от того, взаимодействуют они между собой или нет.

Найдем физическую величину, которая определяет изменение импульса системы. Для этого продифференцируем (3.3) по времени:

\[
\mathrm{d} \mathbf{p} / \mathrm{d} t=\sum \mathrm{d} \mathbf{p}_{i} / \mathrm{d} t .
\]

Согласно (3.1),

\[
\mathrm{d} \mathbf{p}_{i} / \mathrm{d} t=\sum_{k} \mathbf{F}_{i k}+\mathbf{F}_{i},
\]

где \(\mathbf{F}_{i k}\) - силы, действующие на \(i\)-ю частицу со стороны других частиц системы (внутренние силы); \(\mathbf{F}_{i}\) - сила, действующая на эту же частицу со стороны других тел, не входящих в рассматриваемую систему (внешние силы). Подставив последнее выражение в предыдущее, получим

\[
\mathrm{d} \mathbf{p} / \mathrm{d} t=\sum_{i} \sum_{k} \mathbf{F}_{i k}+\sum_{i} \mathbf{F}_{i} .
\]

Двойная сумма справа - это сумма всех внутренних сил. В соответствии с третьим законом Ньютона силы взаимодействия между частицами системы попарно одинаковы по модулю и противоположны по направлению. Поэтому результирующая сила в каждой паре взаимодействия равна нулю, а значит, равна нулю и векторная сумма всех внутренних сил. В результате последнее уравнение принимает следующий вид:


\begin{equation*}
\mathrm{d} \mathbf{p} / \mathrm{d} t=\mathbf{F}_{\text {внеш }}, \tag{3.4}
\end{equation*}


где \(\mathbf{F}_{\text {внеш }}\) - результирующая всех внешних сил, \(\mathbf{F}_{\text {внеш }}=\sum \mathbf{F}_{i}\).

Уравнение (3.4) означает: производная импульса системь по времени равна векторной сумме всех внешних сил, действующих на частицы системы.

Как и в случае одной частицы, из уравнения (3.4) следует, что приращение импульса системы за конечный промежуток времени \(t\) есть


\begin{equation*}
\mathbf{p}_{2}-\mathbf{p}_{1}=\int_{0}^{t} \mathbf{F}_{\text {внеш }} \mathrm{d} t, \tag{3.5}
\end{equation*}


т. е. приращение импульса системы равно импульсу результирующей всех внешних сил за соответствующий промежуток времени. И здесь, конечно, \(\mathbf{F}_{\text {внеш }}\) - результирующая всех внешних сил.

Уравнения (3.4) и (3.5) справедливы как в инерциальной, так и в неинерциальной системах отсчета. Следует только иметь в виду, что в неинерциальной системе отсчета необходимо учитывать и действие сил инериии, играющих роль внешних сил, т. е. под \(\mathbf{F}_{\text {внеш }}\) в этих уравнениях надо понимать сумму \(\mathbf{F}_{\text {вз }}+\mathbf{F}_{\text {ин }}\), где \(\mathbf{F}_{\text {вз }}\) - результирующая всех внешних сил взаимодействия, а \(\mathbf{F}_{\text {ин }}\) - результирующая всех сил инерции.

\section*{§ 3.3. Закон сохранения импульса}
Прежде всего введем понятие замкнутой (или изолированной) системь. Так называют систему частиц, на которую не действуют никакие посторонние тела (или их воздействие пренебрежимо мало). Другими словами, система замкнута, если внешние силы отсутствуют. Очевидно, что понятие замкнутой системы имеет смысл только по отношению к инерциальным системам отсчета, поскольку в неинерциальных системах отсчета всегда действуют силы инерции, играющие роль внешних сил. Понятие замкнутой системы является естественным обобщением понятия изолированной материальной точки и играет весьма важную роль в физике.

Согласно уравнению (3.4), импульс системь может изменяться под действием только внешних сил. Внутренние силы не могут изменить импульс системы. Отсюда непосредственно вытекает закон сохранения импульса:

импульс замкнутой системы частиц остается постоянным, т. е. не меняется со временем:


\begin{equation*}
\mathbf{p}=\sum \mathbf{p}_{i}(t)=\text { const } \tag{3.6}
\end{equation*}


При этом импульсы отдельных частиц или частей замкнутой системы могут меняться со временем, что и подчеркнуто в последнем выражении. Однако эти изменения всегда происходят так, что приращение импульса одной части системы равно убыли импульса оставшейся части системы. Другими словами, отдельные часчти замкнутой системы могут только обмениваться импульсами. Обнаружив в некоторой системе приращение импульса, можно утверждать, что это приращение произошло за счет убыли импульса в окружающих телах.

В этом смысле уравнения (3.4) и (3.5) следует рассматривать как более общую формулировку закона сохранения импульса, формулировку, в которой указана причина изменения импульса у незамкнутой системы - действие других тел (внешних сил). Сказанное справедливо, разумеется, только по отношению к инерциальным системам отсчета.

Импульс может сохраняться и у незамкнутой системы при условии, что результирующая всех внешних сил равна нулю. Это непосредственно вытекает из уравнений (3.4) и (3.5). В практическом отношении сохранение импульса в этих случаях представляет особый интерес, ибо дает возможность получать достаточно простым путем ряд сведений о поведении системы, не вникая в детальное рассмотрение процесса.

И еще. У незамкнутой системы может сохраняться не сам импульс \(\mathbf{p}\), а его проекция \(p_{x}\) на некоторое направление \(X\). Это бывает тогда, когда проекция результирующей внешней силы \(\mathbf{F}_{\text {внеш }}\) на направление \(X\) равна нулю, т. е. вектор \(\mathbf{F}_{\text {внеш }}\) перпендикулярен ему. Действительно, спроецировав уравнение (3.4), получим


\begin{equation*}
\mathrm{d} p_{x} / \mathrm{d} t=F_{\text {внеш } x}, \tag{3.7}
\end{equation*}


откуда следует, что если \(F_{\text {внеш } x} \equiv 0\), то \(p_{x}\) = const. Например, при движении системы в однородном поле сил тяжести сохраняется проекция ее импульса на любое горизонтальное направление, что бы в системе ни происходило.

Пример 1. Движущаяся частица распалась на две частицы с импульсами \(\mathbf{p}_{1}\) и \(\mathbf{p}_{2}\), угол между которыми равен \(\theta\). Найдем модуль импульса p распавшейся частицы.

Подобного рода вопросы проще всего решать с помощью треугольника импульсов (рис. 3.2), который выражает закон сохранения импульса: \(\mathbf{p}=\mathbf{p}_{1}+\mathbf{p}_{2}\). Остается воспользоваться теоремой косинусов, и мы сразу можем записать, что

\[
p=\sqrt{p_{1}^{2}+p_{2}^{2}+2 p_{1} p_{2} \cos \theta}
\]

\begin{figure}[h]
\begin{center}
  \includegraphics[alt={},max width=\textwidth]{66a35da9-3faa-4823-9ca9-40799f37b3db-076_265_336_270_818}
\captionsetup{labelformat=empty}
\caption{Рис. 3.2}
\end{center}
\end{figure}

В этих рассуждениях предполагалось, что система замкнута. Если же она находится под действием каких-то внешних сил, то под импульсами \(\mathbf{p}, \mathbf{p}_{1}\) и \(\mathbf{p}_{2}\) надо понимать те значения этих величин, которые они имели непосредственно до и после распада, а сам процесс распада считать протекающим за очень малое время. Последнее необходимо для того, чтобы импульс внешних сил за время распада был пренебрежимо мал.

Пример 2. Человек массы \(m_{1}\) находится на узком плоту массы \(m_{2}\), который покоится на поверхности озера. Человек совершил перемещение \(\Delta \mathbf{r}^{\prime}\) относительно плота и остановился. Сопротивление воды пренебрежимо мало. Найдем соответствующее перемещение \(\Delta \mathbf{r}_{2}\) плота относительно берега.

В данном случае результирующая всех внешних сил, действующих на систему человек - плот, равна нулю, поэтому импульс этой системы меняться не будет, оставаясь равным нулю в процессе движения:

\[
m_{1} \mathbf{v}_{1}+m_{2} \mathbf{v}_{2}=\mathbf{0},
\]

где \(\mathbf{v}_{1}\) и \(\mathbf{v}_{2}\) - скорости человека и плота относительно берега. Но скорость человека относительно берега можно представить в виде \(\mathbf{v}_{1}=\mathbf{v}_{2}+\mathbf{v}^{\prime}\), где \(\mathbf{v}^{\prime}\) - скорость человека относительно плота. Исключив \(\mathbf{v}_{1}\) из этих двух уравнений, получим

\[
\mathbf{v}_{2}=-\frac{m_{1}}{m_{1}+m_{2}} \mathbf{v}^{\prime}
\]

Умножив обе части на \(\mathrm{d} t\), найдем связь между элементарными перемещениями плота \(\mathrm{d} \mathbf{r}_{2}\) и человека \(\mathrm{d} \mathbf{r}^{\prime}\) относитель-

но плота. Такая же связь будет, очевидно, и для конечных перемещений:

\[
\Delta \mathbf{r}_{2}=-\frac{m_{1}}{m_{1}+m_{2}} \Delta \mathbf{r}^{\prime}
\]

Отсюда видно, что перемещение плота \(\Delta \mathbf{r}_{2}\) не зависит от характера движения человека, т. е. не зависит от закона \(\mathbf{v}^{\prime}(t)\).

Подчеркнем еще раз: закон сохранения импульса выполняется только в инерциальных системах. Это, однако, не исключает случаев, когда импульс системы сохранялся бы и в неинерциальных системах отсчета. Для этого достаточно, чтобы в уравнении (3.4), справедливом и в неинерциальных системах отсчета, внешняя сила \(\mathbf{F}_{\text {внеш }}\) (она включает в себя и силы инерции) была равна нулю. Ясно, что такое положение может осуществляться лишь при специальных условиях. Соответствующие случаи довольно редки и имеют частный характер.

Теперь покажем, что если импульс системы сохраняется в одной инерциальной \(K\)-системе отсчета, то он сохраняется и в любой другой инерциальной \(K^{\prime}\)-системе. Пусть в \(K\)-системе

\[
\sum m_{i} \mathbf{v}_{i}=\text { const. }
\]

Если \(K^{\prime}\)-система движется относительно \(K\)-системы со скоростью \(\mathbf{V}\), то скорость \(i\)-й частицы в \(K\)-системе можно представить как \(\mathbf{v}_{i}=\mathbf{v}_{i}^{\prime}+\mathbf{V}\), где \(\mathbf{v}_{i}^{\prime}\) - скорость этой частицы в \(K^{\prime}\)-системе. Тогда выражение для импульса системы можно преобразовать к следующему виду:

\[
\sum m_{i} \mathbf{v}_{i}^{\prime}+\sum m_{i} \mathbf{V}=\text { const. }
\]

Вторая сумма в этом равенстве не зависит от времени. А это значит, что и первая сумма - импульс системы в \(K^{\prime}\)-системе отсчета тоже не зависит от времени, т. е.

\[
\sum m_{i} \mathbf{v}_{i}^{\prime}=\text { const. }
\]

Полученный результат полностью соответствует принципу относительности Галилея, согласно которому законы механики одинаковы во всех инерциальных системах отсчета.

Рассуждения, которые привели нас к закону сохранения импульса, целиком опирались на справедливость законов Ньютона. В частности, предполагалось, что материальные точки замкнутой системы взаимодействуют между собой попарно и это взаимодействие подчиняется третьему закону Ньютона. А как

обстоит дело в случае систем, не подчиняющихся законам Ньютона, например в системах с электромагнитным излучением?

Ответ на этот вопрос дает опыт, который со всей убедительностью показывает, что закон сохранения импульса оказывается справедливым и для таких систем. Однако в этих случаях в общем балансе импульса необходимо учитывать не только импульсы частиц, но и импульс, которым обладает, как выясняется в электродинамике, само электромагнитное поле.

Таким образом, опыт показывает, что закон сохранения импульса, надлежащим образом, обобщенный, представляет собой фундаментальный закон природы, не знающий никаких исключений. Но в таком широком понимании он уже не является следствием законов Ньютона, а должен рассматриваться как самостоятельный общий принцип, являющийся обобщением опытных фактов.

\section*{§ 3.4. Центр масс. Ц-система}
\section*{Центр масс}
В любой системе частиц имеется одна замечательная точка \(C\), называемая центром масс, которая обладает рядом интересных и важных свойств. Ее положение относительно начала \(O\) данной системы отсчета характеризуется радиусом-вектором


\begin{equation*}
\mathbf{r}_{C}=\frac{1}{m} \sum m_{i} \mathbf{r}_{i}, \tag{3.8}
\end{equation*}


где \(m_{i}\) и \(\mathbf{r}_{i}\) - масса и радиус-вектор \(i\)-й частицы, \(m\) - масса всей системы (рис. 3.3).

\begin{figure}[h]
\begin{center}
  \includegraphics[alt={},max width=\textwidth]{66a35da9-3faa-4823-9ca9-40799f37b3db-078_336_370_1171_790}
\captionsetup{labelformat=empty}
\caption{Рис. 3.3}
\end{center}
\end{figure}

Пример. Покажем, что центр масс системы из двух частиц с массами \(m_{1}\) и \(m_{2}\) находится на прямой, их соединяющей, в точке \(C\), которая делит расстояние между этими частицами в отношении \(l_{1}: l_{2}=m_{2}: m_{1}\).

Пусть \(\mathbf{r}_{1}, \mathbf{r}_{2}, \mathbf{r}_{C}\) - радиусы-векторы частиц 1,2 и точки \(C\) (рис. 3.4). Тогда положения этих частиц относительно точки \(C\) характеризуются радиусами-векторами

\[
\begin{aligned}
& \mathbf{r}_{1}^{\prime}=\mathbf{r}_{1}-\mathbf{r}_{C}, \\
& \mathbf{r}_{2}^{\prime}=\mathbf{r}_{2}-\mathbf{r}_{C} .
\end{aligned}
\]

\begin{figure}[h]
\begin{center}
  \includegraphics[alt={},max width=\textwidth]{66a35da9-3faa-4823-9ca9-40799f37b3db-079_301_418_137_746}
\captionsetup{labelformat=empty}
\caption{Рис. 3.4}
\end{center}
\end{figure}

После подстановки в эти равенства согласно (3.8) выражения \(\mathbf{r}_{C}=\left(m_{1} \mathbf{r}_{1}+m_{2} \mathbf{r}_{2}\right) /\left(m_{1}+m_{2}\right)\) получим

\[
\mathbf{r}_{1}^{\prime}=\frac{m_{2}}{m_{1}+m_{2}}\left(\mathbf{r}_{1}-\mathbf{r}_{2}\right), \quad \mathbf{r}_{2}^{\prime}=\frac{m_{1}}{m_{1}+m_{2}}\left(\mathbf{r}_{2}-\mathbf{r}_{1}\right) .
\]

Отсюда следует, что векторы \(\mathbf{r}_{1}^{\prime}\) и \(\mathbf{r}_{2}^{\prime}\) коллинеарны (причем \(\mathbf{r}_{1}^{\prime} \uparrow \downarrow \mathbf{r}_{2}^{\prime}\) ), значит точка \(C\) лежит на прямой, проходящей через частицы (рис. 3.4). Кроме того, модули этих векторов, т. е. расстояния \(l_{1}\) и \(l_{2}\), обратно пропорциональны массам частиц.\\
Следует заметить, что центр масс системы совпадает с ее центром тяжести. Впрочем, это утверждение справедливо лишь в том случае, когда поле сил тяжести в пределах данной системы можно считать однородным.

Теперь найдем скорость \(\mathbf{V}_{C}\) иентра масс системы. Продифференцировав (3.8) по времени, получим


\begin{equation*}
\mathbf{V}_{C}=\frac{1}{m} \sum m_{i} \mathbf{v}_{i} . \tag{3.9}
\end{equation*}


Если скорость центра масс равна нулю, то система как целое покоится. Это вполне естественное обобщение понятия покоя отдельной материальной точки. Скорость же \(\mathbf{V}_{C}\) приобретает смысл скорости движения всей системы как целого.

Из последней формулы с учетом (3.3) следует, что


\begin{equation*}
\mathbf{p}=m \mathbf{V}_{C}, \tag{3.10}
\end{equation*}


т. е. импульс системь равен произведению массы системы на скорость ее центра масс.

\section*{Уравнение движения центра масс}
Понятие центра масс позволяет придать уравнению (3.4) иную форму, которая часто бывает более удобной. Для этого достаточно (3.10) подставить в (3.4) и учесть, что масса системы как таковой есть величина постоянная. Тогда получим


\begin{equation*}
m \frac{\mathrm{~d} \mathbf{V}_{C}}{\mathrm{~d} t}=\mathbf{F}_{\text {внеш }}, \tag{3.11}
\end{equation*}


где \(\mathbf{F}_{\text {внеш }}\) - результирующая всех внешних сил, действующих на систему. Это и есть уравнение движения центра масс системы - одно из важнейших уравнений механики. Согласно этому уравнению, центр масс любой системь частиц движется так, как если бы вся масса системы была сосредоточена в этой точке и к ней были бы приложены все внешние силы.

При этом ускорение центра масс совершенно не зависит от точек приложения внешних сил.

Далее, из уравнения (3.11) следует, что если \(\mathbf{F}_{\text {внеш }} \equiv 0\), то \(\mathrm{d} \mathbf{V}_{C} / \mathrm{d} t \equiv 0\), а значит, \(\mathbf{V}_{C}=\) const. Таков, в частности, случай замкнутой системы (в инерциальной системе отсчета). Кроме того, если \(\mathbf{V}_{C}=\) const, то, согласно (3.11), и импульс системы \(\mathbf{p}=\) const.

Таким образом, если центр масс системь движется равномерно и прямолинейно, то это означает, что ее импульс сохраняется в процессе движения. Разумеется, справедливо и обратное утверждение.

Уравнение (3.11) по форме совпадает с основным уравнением динамики материальной точки и является его естественным обобщением на систему частиц: ускорение системы как целого пропорционально результирующей всех внешних сил и обратно пропорционально суммарной массе системы. Напомним, что в неинерциальных системах отсчета результирующая всех внешних сил включает в себя как силы взаимодействия с окружающими телами, так и силы инерции.\\
Пример 1. Покажем, как можно иначе решить задачу с человеком на плоту (см. пример 2 на с. 75 ), если воспользоваться понятием центра масс.

Так как сопротивление воды пренебрежимо мало, то результирующая всех внешних сил, действующих на систему человек - плот, равна нулю. А это значит, что положение центра масс данной системы в процессе движения человека (и плота) меняться не будет, т. е.

\[
m_{1} \mathbf{r}_{1}+m_{2} \mathbf{r}_{2}=\text { const },
\]

где \(\mathbf{r}_{1}\) и \(\mathbf{r}_{2}\) - радиусы-векторы, характеризующие положения центров масс человека и плота относительно некоторой точки берега. Из этого равенства найдем связь между приращениями векторов \(\mathbf{r}_{1}\) и \(\mathbf{r}_{2}\) :

\[
m_{1} \Delta \mathbf{r}_{1}+m_{2} \Delta \mathbf{r}_{2}=0 .
\]

Имея в виду, что приращения \(\Delta \mathbf{r}_{1}\) и \(\Delta \mathbf{r}_{2}\) представляют собой перемещения человека и плота относительно берега, причем \(\Delta \mathbf{r}_{1}=\Delta \mathbf{r}_{2}+\Delta \mathbf{r}^{\prime}\), найдем перемещение плота:

\[
\Delta \mathbf{r}_{2}=-\frac{m_{1}}{m_{1}+m_{2}} \Delta \mathbf{r}^{\prime}
\]

Пример 2. Человек прыгает с вышки в воду. Движение прыгуна в общем случае имеет весьма сложный характер. Однако если сопротивление воздуха пренебрежимо мало, то можно сразу утверждать, что центр масс прыгуна движется по параболе, как материальная точка, на которую действует постоянная сила \(m \mathrm{~g}\), где \(m\) - масса человека.\\
Пример 3. Замкнутая цепочка, соединенная нитью с концом оси цент-

\begin{figure}[h]
\begin{center}
  \includegraphics[alt={},max width=\textwidth]{66a35da9-3faa-4823-9ca9-40799f37b3db-081_376_317_1212_246}
\captionsetup{labelformat=empty}
\caption{Рис. 3.5}
\end{center}
\end{figure}

робежной машины, равномерно вращается вокруг вертикальной оси с угловой скоростью \(\omega\) (рис. 3.5). При этом нить образует угол \(\vartheta\) с вертикалью. Как ведет себя центр масс цепочки?

Прежде всего ясно, что при равномерном вращении центр масс цепочки не движется в вертикальном направлении. Отсюда следует, что вертикальная составляющая силы натяжения \(T\) нити компенсирует силу тяжести \(m g\) (рис. 3.5, справа). Горизонтальная составляющая силы натяжения постоянна по модулю и все время направлена к оси вращения. Отсюда следует, что центр масс цепочки (точка \(C\) ) движется

по горизонтальной окружности, радиус которой \(\rho\) легко найти с помощью уравнения (3.11), записав его в виде

\[
m \omega^{2} \rho=m g \operatorname{tg} \vartheta,
\]

где \(m\) - масса цепочки. При этом точка \(C\) все время находится между осью вращения и нитью, как показано на рисунке.

\section*{Ц-система}
В тех часто встречающихся случаях, когда нас интересует лишь относительное движение частиц внутри системы, а не ее движение как целого, наиболее целесообразно пользоваться системой отсчета, в которой центр масс покоится. Это позволяет значительно упростить и анализ явления, и соответствующие расчеты.

Систему отсчета, жестко связанную с центром масс и перемещающуюся поступательно по отношению к инерциальным системам, называют системой центра масс или, кратко, Ц-системой.

Отличительной особенностью Ц-системы является то, что полный импульс системь частиц в ней всегда равен нулю это непосредственно следует из формулы (3.10), ибо в Ц-системе \(\mathbf{V}_{C}=0\). Другими словами, любая система частиц как целое покоится в своей Ц-системе.

Для замкнутой системы частиц ее Ц-система является инерциальной, для незамкнутой - в общем случае неинерциальной.

Необходимо отметить, что Ц-система играет существенную роль в физике. Это обусловлено рядом несомненных преимуществ, которые дает ее применение во многих ситуациях. В дальнейшем мы будем обращаться к этой системе отсчета неоднократно (в теории столкновений частиц, в динамике твердого тела и др.).

Система из двух частиц. Пусть массы частиц равны \(m_{1}\) и \(m_{2}\), а их скорости в исходной \(K\)-системе отсчета - \(\mathbf{v}_{1}\) и \(\mathbf{v}_{2}\) соответственно. Найдем импульсы этих частиц в Ц-системе.

Будем помечать все величины в Ц-системе сверху значком \~{} (тильда). Тогда искомые импульсы можно записать так:

\[
\tilde{\mathbf{p}}_{1}=m_{1} \tilde{\mathbf{v}}_{1}=m_{1}\left(\mathbf{v}_{1}-\mathbf{V}_{C}\right), \quad \tilde{\mathbf{p}}_{2}=m_{2} \tilde{\mathbf{v}}_{2}=m_{2}\left(\mathbf{v}_{2}-\mathbf{V}_{C}\right),
\]

где \(\mathbf{V}_{C}\) - скорость Ц-системы относительно \(K\)-системы отсчета. После подстановки в эти формулы выражения для \(\mathbf{V}_{C}\), \(\mathbf{V}_{C}=\left(m_{1} \mathbf{v}_{1}+m_{2} \mathbf{v}_{2}\right) /\left(m_{1}+m_{2}\right)\), получим


\begin{equation*}
\tilde{\mathbf{p}}_{1}=\frac{m_{1} m_{2}}{m_{1}+m_{2}}\left(\mathbf{v}_{1}-\mathbf{v}_{2}\right), \quad \tilde{\mathbf{p}}_{2}=\frac{m_{1} m_{2}}{m_{1}+m_{2}}\left(\mathbf{v}_{2}-\mathbf{v}_{1}\right) \tag{3.12}
\end{equation*}


Видно, что импульсы обеих частиц в Ц-системе одинаковы по модулю и противоположны по направлению: \(\tilde{\mathbf{p}}_{1}=-\tilde{\mathbf{p}}_{2}\). Это так и должно быть, поскольку суммарный импульс частиц в Ц-системе всегда равен нулю.

Полученные результаты справедливы независимо от того, замкнута эта система или нет, а также независимо от наличия взаимодействия между частицами.

\section*{§ 3.5. Движение тела переменной массы}
Имеется много случаев, когда масса тела изменяется в процессе движения за счет непрерывного отделения или присоединения вещества (ракета, реактивный самолет, платформа, нагружаемая на ходу, и др.).

Наша задача: найти уравнение движения такого тела.\\
Рассмотрим решение этого вопроса для материальной точки, называя ее для краткости телом. Пусть в некоторый момент \(t\) масса движущегося тела \(A\) равна \(m\), а присоединяемое (или отделяемое) вещество имеет скорость \(\mathbf{u}\) относительно данного тела.

Введем вспомогательную инерциальную \(K\)-систему отсчета, скорость которой такова же, как и скорость тела \(A\) в данный момент \(t\). Это значит, что в момент \(t\) тело \(A\) покоится в \(K\)-системе.

Пусть далее за промежуток времени от \(t\) до \(t+\mathrm{d} t\) тело \(A\) приобретает в \(K\)-системе импульс \(m \mathrm{dv}\). Этот импульс тело \(A\) получит, во-первых, вследствие присоединения (отделения) массы \(\delta m\), которая приносит (уносит) импульс \(\delta m \cdot \mathbf{u}\), и, во-вторых, вследствие действия силы \(\mathbf{F}\) со стороны окружающих тел или силового поля. Таким образом,

\[
m \mathrm{~d} \mathbf{v}=\mathbf{F} \mathrm{d} t \pm \delta m \cdot \mathbf{u}
\]

где знак плюс соответствует присоединению массы, а знак минус - отделению. Оба эти случая можно объединить, представив \(\pm \delta m\) в виде приращения \(\mathrm{d} m\) массы тела \(A\) (действительно,

в случае присоединения массы \(\mathrm{d} m=+\delta m\), а в случае отделения \(\mathrm{d} m=-\delta m\) ). Тогда предыдущее уравнение примет вид

\[
m \mathrm{~d} \mathbf{v}=\mathbf{F} \mathrm{d} t+\mathrm{d} m \cdot \mathbf{u} .
\]

Разделив это выражение на \(\mathrm{d} t\), получим


\begin{equation*}
m \frac{\mathrm{~d} \mathbf{v}}{\mathrm{~d} t}=\mathbf{F}+\frac{\mathrm{d} m}{\mathrm{~d} t} \mathbf{u}, \tag{3.13}
\end{equation*}


где \(\mathbf{u}\) - скорость присоединяемого (или отделяемого) вещества относительно рассматриваемого тела.

Это уравнение является основным уравнением динамики точки переменной массь. Его называют уравнением Мещерского. Будучи полученным в одной инерциальной системе отсчета, это уравнение в силу принципа относительности справедливо и в любой другой инерциальной системе. Заметим, что если система отсчета неинерциальна, то под силой \(\mathbf{F}\) следует понимать результирующую силу как сил взаимодействия данного тела с окружающими телами, так и сил инерции.

Последнее слагаемое уравнения (3.13) названо реактивной силой: \(\mathbf{R}=(\mathrm{d} m / \mathrm{d} t) \mathbf{u}\). Эта сила возникает в результате действия на данное тело присоединяемой (или отделяемой) массы. Если масса присоединяется, то \(\mathrm{d} m / \mathrm{d} t>0\) и вектор \(\mathbf{R}\) совпадает по направлению с вектором \(\mathbf{u}\); если же масса отделяется, то \(\mathrm{d} m / \mathrm{d} t<0\) и вектор \(\mathbf{R}\) противоположен вектору \(\mathbf{u}\).

Уравнение Мещерского по своей форме совпадает с основным уравнением динамики материальной точки постоянной массы: слева - произведение массы тела на ускорение, справа - действующие на него силы, включая реактивную силу. Однако в случае переменной массы нельзя внести массу \(m\) под знак дифференцирования и представить левую часть уравнения как производную по времени от импульса, ибо \(m \mathrm{~d} \mathbf{v} / \mathrm{d} t \neq \mathrm{d}(m \mathbf{v}) / \mathrm{d} t\).

Обратим внимание на два частных случая:

\begin{enumerate}
  \item Если \(\mathbf{u}=0\), т. е. масса присоединяется или отделяется без скорости относительно тела, то \(\mathbf{R}=0\) и уравнение (3.13) принимает вид
\end{enumerate}


\begin{equation*}
m(t) \frac{\mathrm{d} \mathbf{v}}{\mathrm{~d} t}=\mathbf{F}, \tag{3.14}
\end{equation*}


где \(m(t)\) - масса тела в данный момент времени. Это уравнение определяет, например, движение платформы, из которой свободно высыпается песок (см. задачу 3.7, п. 1).\\
2. Если \(\mathbf{u}=-\mathbf{v}\), т. е. присоединяемая масса неподвижна в выбранной системе отсчета или отделяемая масса становится неподвижной в этой системе, то уравнение (3.13) принимает другой вид: \(m(\mathrm{~d} \mathbf{v} / \mathrm{d} t)+(\mathrm{d} m / \mathrm{d} t) \mathbf{v}=\mathbf{F}\), или


\begin{equation*}
\frac{\mathrm{d}}{\mathrm{~d} t}(m \mathbf{v})=\mathbf{F} . \tag{3.15}
\end{equation*}


Иначе говоря, в этом частном случае - и только этом действие силы \(\mathbf{F}\) определяет изменение импульса тела с переменной массой. Данный случай реализуется, например, при движении платформы, нагружаемой сыпучим веществом из неподвижного бункера (см. задачу 3.7, п. 2).

Рассмотрим пример на применение уравнения Мещерского.\\
Пример. Ракета движется в инерциальной \(K\)-системе отсчета в отсутствие внешнего силового поля, причем так, что газовая струя испускается с постоянной относительно ракеты скоростью u. Найдем зависимость скорости \(\mathbf{v}\) ракеты от ее массы \(m\), если в момент старта ее масса была равна \(m_{0}\).

В данном случае \(\mathbf{F}=0\) и из уравнения (3.13) следует

\[
\mathrm{d} \mathbf{v}=\mathbf{u} \mathrm{d} m / m .
\]

Проинтегрировав это выражение с учетом начальных условий, получим


\begin{equation*}
\mathbf{v}=-\mathbf{u} \ln \left(m_{0} / m\right), \tag{1}
\end{equation*}


где знак минус показывает, что вектор \(\mathbf{v}\) (скорость ракеты) противоположен по направлению вектору \(\mathbf{u}\). Отсюда видно, что скорость ракеты в данном случае ( \(\mathbf{u}=\) const) не зависит от времени сгорания топлива: \(\mathbf{v}\) определяется только отношением начальной массы \(m_{0}\) ракеты к оставшейся массе \(m\).

Заметим, что если бы вся масса горючего была одновременно выброшена со скоростью \(\mathbf{u}\) относительно ракеты, то скорость последней оказалась бы иной. Действительно, если ракета вначале покоилась в выбранной инерциальной системе отсчета, а после одновременного выброса всего горючего приобрела

скорость \(\mathbf{v}\), то из закона сохранения импульса для системы ракета-горючее следует

\[
\mathbf{0}=m \mathbf{v}+\left(m_{0}-m\right)(\mathbf{u}+\mathbf{v})
\]

где \((\mathbf{u}+\mathbf{v})\) - скорость горючего относительно данной системы отсчета. Отсюда


\begin{equation*}
\mathbf{v}=-\mathbf{u}\left(1-m / m_{0}\right) . \tag{2}
\end{equation*}


Скорость \(\mathbf{v}\) ракеты в этом случае оказывается меньше, чем в предыдущем (при одинаковых значениях отношения \(m_{0} / m\) ). В этом нетрудно убедиться, сравнив характер зависимости \(\mathbf{v}\) от \(m_{0} / m\) в обоих случаях. С ростом \(m_{0} / m\) в первом случае (когда вещество отделяется непрерывно) скорость \(\mathbf{v}\) ракеты, согласно (1), растет неограниченно, во втором же (когда вещество отделяется одновременно) скорость \(v\), согласно (2), стремится к пределу, равному - u.

\section*{Задачи}
3.1. Частица с импульсом \(\mathbf{p}(t)\) движется под действием силы \(\mathbf{F}(t)\). Пусть \(\mathbf{a}\) и \(\mathbf{b}\) - постоянные векторы, причем \(\mathbf{a} \perp \mathbf{b}\). Полагая, что:

\begin{enumerate}
  \item \(\mathbf{p}(t)=\mathbf{a}+t(1-\alpha t) \mathbf{b}\), где \(\alpha\) - положительная постоянная, найти вектор \(\mathbf{F}\) в те моменты времени, когда \(\mathbf{F} \perp \mathbf{p}\);
  \item \(\mathbf{F}(t)=\mathbf{a}+2 t \mathbf{b}\) и \(\mathbf{p}(0)=\mathbf{p}_{0}\), где \(\mathbf{p}_{0}\) - вектор, противоположный по направлению вектору \(\mathbf{a}\), найти вектор \(\mathbf{p}\) в момент \(t_{0}\), когда он окажется повернутым на \(90^{\circ}\) по отношению к вектору \(\mathbf{p}_{0}\).\\
Р е шение. 1. Сила \(\mathbf{F}=\mathrm{d} \mathbf{p} / \mathrm{d} t=(1-2 \alpha t) \mathbf{b}\), т. е. вектор \(\mathbf{F}\) все время перпендикулярен вектору \(\mathbf{a}\). Следовательно, вектор \(\mathbf{F}\) будет перпендикулярен вектору \(\mathbf{p}\) в те моменты, когда коэффициент при b в выражении для \(\mathbf{p}(t)\) обращается в нуль. Отсюда \(t_{1}=0\) и \(t_{2}=1 / \alpha\). Соответствующие значения вектора \(\mathbf{F}\) равны:
\end{enumerate}

\[
\mathbf{F}_{1}=\mathbf{b}, \quad \mathbf{F}_{2}=-\mathbf{b} .
\]

\begin{enumerate}
  \setcounter{enumi}{1}
  \item Приращение вектора \(\mathbf{p}\) за промежуток времени \(\mathrm{d} t\) есть \(\mathrm{d} \mathbf{p}=\mathbf{F} \mathrm{d} t\). Интегрируя это уравнение с учетом начальных условий, находим
\end{enumerate}

\[
\mathbf{p}-\mathbf{p}_{0}=\int_{0}^{t} \mathbf{F} \mathrm{~d} t=\mathbf{a} t+\mathbf{b} t^{2}
\]

где, по условию, \(\mathbf{p}_{0}\) противоположен вектору \(\mathbf{a}\). Вектор \(\mathbf{p}\) окажется перпендикулярным вектору \(\mathbf{p}_{0}\) в момент \(t_{0}\), когда \(a t_{0}=p_{0}\). В этот момент \(\mathbf{p}=\left(p_{0} / a\right)^{2} \mathbf{b}\).\\
3.2. Орудие массы \(m\) соскальзывает по гладкой наклонной плоскости, составляющей угол α с горизонтом. В момент, когда скорость орудия оказалось равной \(\mathbf{v}\), произвели выстрел, в результате которого орудие остановилось, а вылетевший в горизонтальном направлении снаряд «унес» импульс p. Пусть продолжительность выстрела равна \(\tau\). Найти среднее за это время значение силы реакции \(\mathbf{R}\) со стороны наклонной плоскости.

Решение. Здесь система орудие - снаряд незамкнутая. За время \(\tau\) эта система получает приращение импульса, равное \(\mathbf{p}-m \mathbf{v}\). Изменение импульса системы обусловлено действием двух внешних сил: силы реакции \(\mathbf{R}\) (она перпендикулярна наклонной плоскости) и силы тяжести \(m g\). Поэтому

\[
\mathbf{p}=\mathrm{m} \mathbf{v}=\langle\mathbf{R}\rangle \tau+m \mathbf{g} \tau,
\]

\begin{figure}[h]
\begin{center}
  \includegraphics[alt={},max width=\textwidth]{66a35da9-3faa-4823-9ca9-40799f37b3db-087_225_339_713_137}
\captionsetup{labelformat=empty}
\caption{Рис. 3.6}
\end{center}
\end{figure}

где \(\langle\mathbf{R}\rangle\) - среднее за время \(\tau\) значение вектора R. Это соотношение очень полезно представить графически (рис. 3.6). Из данного рисунка сразу видно, что искомое значение модуля \(\langle\mathbf{R}\rangle\) определяется формулой

\[
\langle R\rangle=(p \sin \alpha+m g \tau \cos \alpha) / \tau .
\]

3.3. Закон сохранения импульса. Две тележки, каждая массы \(M\), движутся друг за другом по инерции (без трения) с одинаковой скоростью \(\mathbf{v}_{0}\). На задней тележке находится человек массы \(m\). В некоторый момент человек прыгнул в переднюю тележку со скоростью \(\mathbf{u}\) относительно своей тележки. Какой стала скорость передней тележки?

Решение. Импульс всей системы в результате того, что человек перепрыгнул из задней тележки 2 в переднюю 1 , не изменится, поэтому

\[
(2 M+m) \mathbf{v}_{0}=M \mathbf{v}_{2}^{\prime}+(M+m) \mathbf{v}_{1}^{\prime},
\]

где \(\mathbf{v}_{1}^{\prime}\) и \(\mathbf{v}_{2}^{\prime}\) - конечные скорости тележек.\\
Аналогично запишем баланс импульсов для задней тележки с человеком (до и после перепрыгивания):

\[
(M+m) \mathbf{v}_{0}=M \mathbf{v}_{2}^{\prime}+m\left(\mathbf{v}_{2}^{\prime}+\mathbf{u}\right),
\]

где \(\left(\mathbf{v}_{2}^{\prime}+\mathbf{u}\right)\) - скорость спрыгнувшего человека относительно полотна дороги.

Из этих двух уравнений следует, что

\[
\mathbf{v}_{1}^{\prime}=\mathbf{v}_{0}+\frac{m M}{(m+M)^{2}} \mathbf{u}
\]

3.4. На краю покоящейся тележки массы \(M\) стоят два человека, каждый массы \(m\). Пренебрегая трением, найти скорость тележки после того, как оба человека спрыгнут с одной и той же горизонтальной скоростью \(\mathbf{u}\) относительно тележки: 1) одновременно; 2) друг за другом. В каком случае скорость тележки будет больше и во сколько раз?

Решение. 1. Согласно закону сохранения импульса,

\[
M \mathbf{v}^{\prime}+2 m\left(\mathbf{v}^{\prime}+\mathbf{u}\right)=0,
\]

где \(\mathbf{v}^{\prime}\) - скорость тележки, ( \(\mathbf{v}^{\prime}+\mathbf{u}\) ) - скорость человека (обе скорости относительно полотна дороги). Отсюда

\[
\mathbf{v}^{\prime}=-\frac{2 m}{M+2 m} \mathbf{u}
\]

\begin{enumerate}
  \setcounter{enumi}{1}
  \item В этом случае необходимо записать два уравнения. Когда спрыгнул один человек, то
\end{enumerate}

\[
(M+m) \mathbf{v}^{\prime}+m\left(\mathbf{v}^{\prime}+\mathbf{u}\right)=0
\]

где \(\mathbf{v}^{\prime}\) - скорость тележки с оставшимся вторым человеком. Когда же спрыгнул другой человек, то

\[
(M+m) \mathbf{v}^{\prime}=M \mathbf{v}^{\prime \prime}+m\left(\mathbf{v}^{\prime \prime}+\mathbf{u}\right),
\]

где \(\mathbf{v}^{\prime \prime}\) - скорость пустой тележки.\\
Исключив из последних двух уравнений \(\mathbf{v}^{\prime}\), найдем

\[
\mathbf{v}^{\prime \prime}=-\frac{(2 M+3 m) m}{(M+m)(M+2 m)} \mathbf{u} .
\]

Отношение скорости тележки \(v^{\prime \prime}\) в случае 2) к скорости \(v^{\prime}\) в случае 1) равно

\[
\frac{v^{\prime \prime}}{v^{\prime}}=1+\frac{m}{2(M+m)}>1 .
\]

3.5. Центр масс. Через блок перекинут шнур, на одном конце которого находится лестница с человеком, а на другом конце - уравновешивающий груз массы \(M\) (рис. 3.7). Человек, масса которого \(m\),

совершил вверх перемещение \(\Delta \mathbf{r}^{\prime}\) относительно лестницы и остановился. Пренебрегая массами блока и шнура, найти перемещение центра масс этой системы.

Решение. В системе отсчета, связанной с осью блока, положение центра масс данной системы характеризуется радиусом-вектором

\[
\mathbf{r}_{C}=\left[M \mathbf{r}_{1}+(M-m) \mathbf{r}_{2}+m \mathbf{r}_{3}\right] / 2 M,
\]

где \(\mathbf{r}_{1}, \mathbf{r}_{2}\) и \(\mathbf{r}_{3}\) - радиусы-векторы центров масс уравновешиваю-

\begin{figure}[h]
\begin{center}
  \includegraphics[alt={},max width=\textwidth]{66a35da9-3faa-4823-9ca9-40799f37b3db-089_435_222_510_155}
\captionsetup{labelformat=empty}
\caption{Рис. 3.7}
\end{center}
\end{figure}

щего груза, лестницы и человека - все относительно некоторой точки \(O\) выбранной системы отсчета. Отсюда перемещение центра масс системы

\[
\Delta \mathbf{r}_{C}=\left[M \Delta \mathbf{r}_{1}+(M-m) \Delta \mathbf{r}_{2}+m \Delta \mathbf{r}_{3}\right] / 2 M,
\]

где \(\Delta \mathbf{r}_{1}, \Delta \mathbf{r}_{2}\) и \(\Delta \mathbf{r}_{3}\) - перемещения уравновешивающего груза, лестницы и человека. Имея в виду, что

\[
\Delta \mathbf{r}_{1}=-\Delta \mathbf{r}_{2}, \quad \Delta \mathbf{r}_{3}=\Delta \mathbf{r}_{2}+\Delta \mathbf{r}^{\prime}
\]

получим в результате

\[
\Delta \mathbf{r}_{C}=(m / 2 M) \Delta \mathbf{r}^{\prime} .
\]

Таким образом, перемещение центра масс всей системы совпадает по направлению с перемещением человека относительно лестницы, и полученный результат не зависит от характера движения человека.

Замечание. На первый взгляд может показаться, что данная система «замкнута», т. е. результирующая всех внешних сил равна нулю, и центр масс системы не должен переместиться. Однако это не так. Когда человек начинает подниматься, он действует на лестницу с дополнительной силой, направленной вниз. В результате натяжение шнура возрастает и внешняя сила, действующая на систему со стороны подвеса, окажется больше суммарной силы тяжести. Поэтому результирующая всех внешних сил будет направлена вверх - она и обусловливает перемещение вверх центра масс всей системы.\\
3.6. Ц-система. Две небольшие шайбы, массы которых \(m_{1}\) и \(m_{2}\), связаны между собой нитью длины \(l\) и движутся по гладкой горизонтальной плоскости. В некоторый момент скорость одной шайбы равна нулю,

а другой - \(v\), причем ее направление перпендикулярно нити (рис. 3.8, a). Найти силу натяжения нити в процессе движения.

\begin{figure}[h]
\begin{center}
  \includegraphics[alt={},max width=\textwidth]{66a35da9-3faa-4823-9ca9-40799f37b3db-090_361_812_241_207}
\captionsetup{labelformat=empty}
\caption{Рис. 3.8}
\end{center}
\end{figure}

Решение. Перейдем в систему центра масс - Ц-систему. В этой системе отсчета шайбы движутся по окружностям вокруг центра масс \(C\) (рис. 3.8, б), поэтому искомая сила


\begin{equation*}
F=m_{1} \tilde{v}_{1}^{2} / l_{1}, \tag{1}
\end{equation*}


где \(\tilde{v}_{1}\) - скорость шайбы массы \(m_{1}, l_{1}\) - радиус окружности, по которой она движется. Подобное выражение можно было бы, конечно, записать и для другой шайбы - это несущественно.

Найдем значения \(l_{1}\) и \(\tilde{v}_{1}\). В примере на с. \(77-78\) было показано, что отношение \(l_{1} / l_{2}=m_{2} / m_{1}\). Кроме того, \(l_{1}+l_{2}=l\). Из этих двух соотношений следует


\begin{equation*}
l_{1}=l m_{2} /\left(m_{1}+m_{2}\right) . \tag{2}
\end{equation*}


Далее, скорость \(\tilde{\mathbf{v}}^{\prime}=\mathbf{v}_{1}-\mathbf{V}_{C}\). В нашем случае \(\mathbf{v}_{1}=0\) и \(\mathbf{V}_{C}=m_{2} \mathbf{v} /\left(m_{1}+m_{2}\right)\). Поэтому модуль вектора \(\tilde{\mathbf{v}}_{1}\)


\begin{equation*}
\tilde{v}_{1}=m_{2} v /\left(m_{1}+m_{2}\right) . \tag{3}
\end{equation*}


Эта величина в процессе движения остается постоянной.\\
После подстановки (2) и (3) в (1) получим

\[
F=\mu v^{2} / l, \quad \mu=m_{1} m_{2} /\left(m_{1}+m_{2}\right) .
\]

3.7. Движение тела переменной массы. Железнодорожная платформа в момент \(t=0\) начинает двигаться под действием постоянной силы тяги F. Пренебрегая трением в осях, найти зависимость от времени скорости платформы \(\mathbf{v}(t)\), если:

\begin{enumerate}
  \item платформа нагружена песком, который высыпается через отверстие в ее дне с постоянной скоростью \(\mu(к г / с)\), а в момент \(t=0\) масса платформы с песком равна \(m_{0}\);
  \item на платформу, масса которой \(m_{0}\), в момент \(t=0\) начинает высыпаться песок из неподвижного бункера так, что скорость погрузки постоянна и равна \(\mu\) кг/с.\\
Решение.1. В этом случае реактивная сила равна нулю и уравнение (3.13) имеет вид \(\left(m_{0}-\mu t\right) \mathrm{d} \mathbf{v} / \mathrm{d} t=\mathbf{F}\), откуда
\end{enumerate}

\[
\mathrm{d} \mathbf{v}=\mathbf{F} \mathrm{d} t /\left(m_{0}-\mu t\right)
\]

Проинтегрировав это уравнение с учетом начальных условий, получим

\[
\mathbf{v}=\frac{\mathbf{F}}{\mu} \ln \frac{m_{0}}{m_{0}-\mu t}
\]

\begin{enumerate}
  \setcounter{enumi}{1}
  \item Здесь горизонтальная составляющая реактивной силы (а только эта составляющая нас и интересует) \(\mathbf{R}=\mu(-\mathbf{v})\), где \(\mathbf{v}\) - скорость платформы. Поэтому уравнение (3.13) приводится к виду (3.15), или
\end{enumerate}

\[
\mathrm{d}(m \mathbf{v})=\mathbf{F} \mathrm{d} t .
\]

Интегрирование с учетом начальных условий дает

\[
m \mathbf{v}=\mathbf{F} t,
\]

где \(m=m_{0}+\mu t\). Отсюда

\[
\mathbf{v}=\mathbf{F} t /\left(m_{0}+\mu t\right)
\]

Полученные в обоих случаях выражения справедливы, разумеется, лишь в процессе разгрузки (или погрузки) платформы.\\
3.8. Ракета поддерживается в воздухе на постоянной высоте, выбрасывая вертикально вниз струю газа со скоростью u. Найти:

\begin{enumerate}
  \item сколько времени ракета сможет оставаться на этой высоте, если начальная масса топлива составляет \(\eta\)-ю часть ее массы (без топлива);
  \item какую массу \(\mu(t)\) газов должна ежесекундно выбрасывать ракета, чтобы оставаться на постоянной высоте, если начальная масса ракеты (с топливом) равна \(m_{0}\).\\
Решение.1. В данном случае \(\mathrm{dv} / \mathrm{d} t=0\) и уравнение (3.13) примет вид
\end{enumerate}

\[
m \mathbf{g}+(\mathrm{d} m / \mathrm{d} t) \mathbf{u}=0
\]

или после разделения переменных


\begin{equation*}
\mathrm{d} m / m=-(g / u) / \mathrm{d} t . \tag{1}
\end{equation*}


Интегрирование этого уравнения дает


\begin{equation*}
\ln \left(m / m_{0}\right)=-(g / u) t . \tag{2}
\end{equation*}


Отсюда

\[
t=(u / g) \ln \left(m_{0} / m\right)=(u / g) \ln (1+\eta),
\]

где учтено, что \(\eta=\left(m_{0}-m\right) / m\).\\
2. Из уравнения (1) предыдущего пункта следует, что

\[
\mu=-\mathrm{d} m / \mathrm{d} t=(g / u) m,
\]

где \(m\) находим из (2): \(m=m_{0} \mathrm{e}^{-g t / u}\). В результате

\[
\mu=(g / u) m_{0} \mathrm{e}^{-g t / u} .
\]

По такому закону \(\mu\) меняется со временем в течение промежутка времени, найденного в п. 1.\\
3.9. Ракета поднимается с нулевой начальной скоростью вертикально вверх в однородном поле тяжести. Первоначальная масса ракеты (с топливом) равна \(m_{0}\). Скорость газовой струи постоянна и равна \(u\) относительно ракеты. Пренебрегая сопротивлением воздуха, найти скорость \(v\) ракеты в зависимости от ее массы \(m\) и времени подъема \(t\).

Решение. Запишем уравнение движения ракеты - уравнение (3.13) - в проекциях на вертикальную ось с положительным направлением вверх:

\[
m \frac{\mathrm{~d} v}{\mathrm{~d} t}=-m g-u \frac{\mathrm{~d} m}{\mathrm{~d} t} .
\]

Перепишем это уравнение так:

\[
m \frac{\mathrm{~d}}{\mathrm{~d} t}(v+g t)=-u \frac{\mathrm{~d} m}{\mathrm{~d} t},
\]

откуда

\[
\mathrm{d}(v+g t)=-u \mathrm{~d} m / m .
\]

Проинтегрировав с учетом начальных условий последнее уравнение, получим

\[
v+g t=-u \ln \left(m / m_{0}\right)
\]

Искомая скорость ракеты

\[
v=u \ln \left(m_{0} / m\right)-g t .
\]

3.10. Космический корабль массы \(m_{0}\) движется в отсутствие внешнего силового поля с постоянной скоростью \(\mathbf{v}_{0}\). Для изменения направления движения был включен реактивный двигатель, который стал выбрасывать струю газа с постоянной относительно корабля скоростью \(\mathbf{u}\), причем вектор \(\mathbf{u}\) все время перепендикулярен направлению движения корабля. В конце работы двигателя масса корабля стала равна \(m\). На какой угол изменилось направление движения корабля за время работы двигателя?

Решение. Найдем приращение вектора скорости корабля за промежуток времени \(\mathrm{d} t\). Умножив обе части уравнения (3.13) на \(\mathrm{d} t\) и учитывая, что \(\mathbf{F}=0\), получим

\[
\mathrm{d} \mathbf{v}=\mathbf{u} \mathrm{d} m / m .
\]

Здесь \(\mathrm{d} m<0\). Так как вектор \(\mathbf{u}\) все время перпендикулярен вектору \(\mathbf{v}\) (скорости корабля), то модуль вектора \(\mathbf{v}\) не изменяется и остается равным своему первоначальному значению: \(|\mathbf{v}|=v_{0}\). Отсюда следует, что угол поворота \(\mathrm{d} \alpha\) вектора \(\mathbf{v}\) за время \(\mathrm{d} t\) определяется как

\[
\mathrm{d} \alpha=|\mathrm{d} \mathbf{v}| / v_{0}=\left(u / v_{0}\right)|\mathrm{d} m / m| .
\]

Проинтегрировав это уравнение, найдем

\[
\alpha=\left(u / v_{0}\right) \ln \left(m_{0} / m\right) .
\]

\section*{Закон сохранения энергии}
\section*{§ 4.1. Работа и мощность}
\section*{Работа}
Пусть частица под действием силы \(\mathbf{F}\) совершает перемещение по некоторой траектории \(1-2\) (рис. 4.1). В общем случае сила \(\mathbf{F}\) в процессе движения частицы может изменяться как по модулю, так и по направлению. Рассмотрим элементарное перемещение dr, в пределах которого силу \(\mathbf{F}\) можно считать постоянной.

Действие силы \(\mathbf{F}\) на перемещении dr характеризуют величиной, равной скалярному произведению \(\mathbf{F d r}\), которую называют элементарной работой силы \(\mathbf{F}\) на перемещении dr. Ее можно представить и в другом виде:

\[
\mathbf{F} \mathrm{d} \mathbf{r}=F \cos \alpha \mathrm{~d} s=F_{s} \mathrm{~d} s
\]

где \(\alpha\) - угол между векторами \(\mathbf{F}\) и dr , \(\mathrm{d} s=|\mathrm{d} \mathbf{r}|\) - элементарный путь, \(F_{s}\) - проекция вектора \(\mathbf{F}\) на вектор dr (рис. 4.1).

\begin{figure}[h]
\begin{center}
  \includegraphics[alt={},max width=\textwidth]{66a35da9-3faa-4823-9ca9-40799f37b3db-094_342_350_800_806}
\captionsetup{labelformat=empty}
\caption{Рис. 4.1}
\end{center}
\end{figure}

Итак, элементарная работа силы \(\mathbf{F}\) на перемещении dr


\begin{equation*}
\delta A=\mathbf{F} \mathrm{d} \mathbf{r}=F_{s} \mathrm{~d} s . \tag{4.1}
\end{equation*}


Величина \(\delta A\) - алгебраическая: в зависимости от угла между векторами \(\mathbf{F}\) и \(\mathrm{d} \mathbf{r}\) (или от знака проекции \(F_{s}\) вектора \(\mathbf{F}\) на вектор dr) она может быть как положительной, так и отрицательной и, в частности, равной нулю (если \(\mathbf{F} \perp \mathrm{d} \mathbf{r}\), т. е. \(F_{s}=0\) ).

Суммируя (интегрируя) выражение (4.1) по всем элементарным участкам пути от точки 1 до точки 2 , находим paботy силь \(\mathbf{F}\) на данном пути:


\begin{equation*}
A=\int_{1}^{2} \mathbf{F} \mathrm{~d} \mathbf{r}=\int_{1}^{2} F_{s} \mathrm{~d} s . \tag{4.2}
\end{equation*}


Отметим следующее важное обстоятельство: формула (4.2) справедлива не только для частиць, но и вообще для любого тела (или системь тел). Надо только иметь в виду, что под dr (или \(\mathrm{d} s\) ) следует понимать перемещение точки приложения силы F. Игнорирование этого обстоятельства зачастую приводит к ошибочным результатам.

Выражению (4.2) можно придать наглядный геометрический смысл. Изобразим график \(F_{s}\) как функцию положения частицы на траектории. Пусть, например, этот график имеет вид, показанный на рис. 4.2. Из рисунка видно, что элемен-

\begin{figure}[h]
\begin{center}
  \includegraphics[alt={},max width=\textwidth]{66a35da9-3faa-4823-9ca9-40799f37b3db-095_278_425_592_68}
\captionsetup{labelformat=empty}
\caption{Рис. 4.2}
\end{center}
\end{figure}

\section*{Работа упругой силы}
Работа упругой силы \(\mathbf{F}=-\varkappa \mathbf{r}\), где \(\mathbf{r}\) - радиус-вектор частицы \(M\) относительно точки \(O\) (рис. 4.3, a). Переместим частицу \(M\), на которую действует эта сила, по произвольному пути из точки 1 в точку 2 . Найдем сначала элементарную работу силы \(\mathbf{F}\) на элементарном перемещении dr:

\[
\delta A=\mathbf{F} \mathrm{d} \mathbf{r}=-\varkappa \mathbf{r} \mathrm{d} \mathbf{r} .
\]

Скалярное произведение \(\mathbf{r d} \mathbf{r}=r(\mathrm{~d} \mathbf{r})_{r}\), где \((\mathrm{d} \mathbf{r})_{r}\) - проекция \(\mathrm{d} \mathbf{r}\) на вектор \(\mathbf{r}\). Эта проекция равна \(\mathrm{d} r\) - приращению модуля вектора \(\mathbf{r}\). Поэтому \(\mathbf{r} \mathrm{d} \mathbf{r}=r \mathrm{~d} r\) и

\[
\delta A=-\varkappa \mathbf{r} \mathrm{d} \mathbf{r}=-\mathrm{d}\left(\varkappa r^{2} / 2\right)
\]

Теперь вычислим работу данной силы на всем пути, т. е. проинтегрируем последнее выражение от точки 1 до точки 2 :


\begin{equation*}
A=-\int_{1}^{2} \mathrm{~d}\left(\frac{\varkappa r^{2}}{2}\right)=\frac{\varkappa r_{1}^{2}}{2}-\frac{\varkappa r_{2}^{2}}{2} . \tag{4.3}
\end{equation*}


\section*{Работа гравитационной (или кулоновской) силы}
Пусть в точке \(O\) (рис. 4.3, б) находится неподвижный силовой центр - материальная точка, действующая на частицу \(M\) с силой \(\mathbf{F}\), которая как для гравитационного, так и для кулоновского взаимодействий может быть представлена в виде

\[
\mathbf{F}=\left(\alpha / r^{2}\right) \mathbf{e}_{r},
\]

где \(\alpha\) - соответствующая постоянная ( \(-\gamma m_{1} m_{2}\) или \(k q_{1} q_{2}\) ), \(r\) расстояние от точки \(O\) до частицы \(M, \mathbf{e}_{r}\) - орт радиуса-вектора \(\mathbf{r}\).

\begin{figure}[h]
\begin{center}
  \includegraphics[alt={},max width=\textwidth]{66a35da9-3faa-4823-9ca9-40799f37b3db-096_384_804_756_210}
\captionsetup{labelformat=empty}
\caption{Рис. 4.3}
\end{center}
\end{figure}

Элементарная работа этой силы на перемещении dr

\[
\delta A=\mathbf{F} \mathrm{d} \mathbf{r}=\left(\alpha / r^{2}\right) \mathbf{e}_{r} \mathrm{~d} \mathbf{r} .
\]

Скалярное произведение \(\mathbf{e}_{r} \mathrm{~d} \mathbf{r}=\mathrm{d} r\), т. е. равно приращению модуля вектора \(\mathbf{r}\), поэтому

\[
\delta A=\alpha \mathrm{d} r / r^{2}=-\mathrm{d}(\alpha / r) .
\]

Работа же этой силы на всем пути от точки 1 до точки 2


\begin{equation*}
A=-\int_{1}^{2} \mathrm{~d}\left(\frac{\alpha}{r}\right)=\frac{\alpha}{r_{1}}-\frac{\alpha}{r_{2}} . \tag{4.4}
\end{equation*}


\section*{Работа однородной силы тяжести}
Запишем эту силу в виде \(\mathbf{F}=-m g \mathbf{k}\), где \(\mathbf{k}\) - орт вертикальной оси \(Z\), положительное направление которой выбрано вверх (рис. 4.4). Элементарная работа силы тяжести на перемещении dr

\[
\delta A=\mathbf{F} \mathrm{d} \mathbf{r}=-m g \mathbf{k} \mathrm{~d} \mathbf{r} .
\]

Скалярное произведение \(\mathbf{k d} \mathbf{r}=(\mathrm{d} \mathbf{r})_{\mathrm{k}}\), где \((\mathrm{d} \mathbf{r})_{\mathrm{k}}\) - проекция \(\mathrm{d} \mathbf{r}\) на орт \(\mathbf{k}\), равная \(\mathrm{d} z\) - приращению координаты \(z\). Поэтому \(\mathbf{k} \mathrm{d} \mathbf{r}=\mathrm{d} z\) и

\[
\delta A=-m g \mathrm{~d} z=-\mathrm{d}(m g z)
\]

Работа данной силы на всем пути от точки 1 до точки 2


\begin{equation*}
A=-\int_{1}^{2} \mathrm{~d}(m g z)=m g\left(z_{1}-z_{2}\right) \tag{4.5}
\end{equation*}


Рассмотренные силы интересны в том отношении, что их ра-

\begin{figure}[h]
\begin{center}
  \includegraphics[alt={},max width=\textwidth]{66a35da9-3faa-4823-9ca9-40799f37b3db-097_386_395_974_100}
\captionsetup{labelformat=empty}
\caption{Рис. 4.4}
\end{center}
\end{figure}

бота, как видно из (4.3)-(4.5), не зависит от формы пути между точками 1 и 2, а зависит только от положения этих точек. Эта весьма важная особенность данных сил присуща, однако, не всем силам. Например, сила трения этим свойством не обладает: работа этой силы зависит не только от положения начальной и конечной точек, но и от формы пути между ними.\\
До сих пор речь шла о работе одной силы. Если же на частицу в процессе движения действуют несколько сил, результирующая которых \(\mathbf{F}=\mathbf{F}_{1}+\mathbf{F}_{2}+\ldots\), то нетрудно показать, что работа результирующей силы \(\mathbf{F}\) на некотором перемещении равна алгебраической сумме работ, совершаемых каждой из сил в отдельности на том же перемещении. Действительно,


\begin{equation*}
A=\int\left(\mathbf{F}_{1}+\mathbf{F}_{2}+\ldots\right) \mathrm{d} \mathbf{r}=\int \mathbf{F}_{1} \mathrm{~d} \mathbf{r}+\int \mathbf{F}_{2} \mathrm{~d} \mathbf{r}+\ldots=A_{1}+A_{2}+\ldots \tag{4.6}
\end{equation*}


Единицей работы в СИ является джоуль (Дж). Джоуль - это работа силы в 1 H на пути 1 m (при условии, что направление силы совпадает по направлению с перемещением), или 1 Дж \(= =1 \mathrm{H} \cdot \mathrm{m}\).

\section*{Мощность}
Для характеристики скорости, с которой совершается работа, вводят величину, называемую мощностью. Мощность, по определению, - это работа, совершаемая силой за единицу времени. Если за промежуток времени \(\mathrm{d} t\) сила \(\mathbf{F}\) совершает работу \(\mathbf{F d r}\), то мощность, развиваемая этой силой в данный момент времени, есть \(P=\mathbf{F} \mathrm{d} \mathbf{r} / \mathrm{d} t\). Учитывая, что \(\mathrm{d} \mathbf{r} / \mathrm{d} t=\mathbf{v}\), получаем


\begin{equation*}
P=\mathbf{F v .} \tag{4.7}
\end{equation*}


Таким образом, мощность, развиваемая силой \(\mathbf{F}\), равна скалярному произведению вектора силы на вектор скорости, с которой движется точка приложения данной силы. Как и работа, мощность - величина алгебраическая.

Зная мощность силы \(\mathbf{F}\), можно найти и работу, которую совершает эта сила за промежуток времени \(t\). В самом деле, представив подынтегральное выражение в формуле (4.2) в виде \(\mathbf{F} \mathrm{d} \mathbf{r}=\mathbf{F} \mathbf{v} \mathrm{d} t=P \mathrm{~d} t\), получим

\[
A=\int_{0}^{t} P \mathrm{~d} t
\]

Единицей мощности в СИ является ватт (Вт), равный джоулю в секунду (Дж/с).

В заключение обратим внимание на одно весьма существенное обстоятельство. Когда говорят о работе (или мощности), то необходимо в каждом конкретном случае четко указывать или представлять себе, работа какой именно силь (или сил) имеется в виду. В противном случае, как правило, неизбежны недоразумения.

\section*{§ 4.2. Консервативные силы. Потенциальная энергия}
\section*{Консервативные силы}
Если в каждой точке пространства на помещенную туда частицу действует сила, то частица находится в поле сил. Так, например, частица может находиться в поле сил тяжести, в поле упругих сил, в поле сил сопротивления (в потоке жидкости, газа) и т. д.

Поле, остающееся постоянным во времени, называют стаиионарным. Стационарное поле в одной системе отсчета может оказаться нестационарным в другой системе отсчета. В стационарном силовом поле сила, действующая на частицу, зависит только от ее положения.

Работа, которую совершают силы поля при перемещении частицы из точки 1 в точку 2 , зависит, вообще говоря, от пути между этими точками. Вместе с тем имеются стационарные силовые поля, в которых работа, совершаемая над частицей силами поля, не зависит от пути между точками 1 и 2. Силы, обладающие таким свойством, называют консервативными*.

Это свойство консервативных сил можно сформулировать и иначе:

\begin{figure}[h]
\begin{center}
  \includegraphics[alt={},max width=\textwidth]{66a35da9-3faa-4823-9ca9-40799f37b3db-099_239_378_1089_68}
\captionsetup{labelformat=empty}
\caption{Рис. 4.5}
\end{center}
\end{figure}

силы поля являются консервативными, если в стаиионарном случае их работа на любом замкнутом пути равна нулю. Чтобы убедиться в этом, разобьем произвольный замкнутый контур на две части: \(1 a 2\) и \(2 b 1\) (рис. 4.5). Тогда работа \(A\) на замкнутом пути

\[
A=A_{1 a 2}+A_{2 b 1} .
\]

Нетрудно сообразить, что \(A_{2 b 1}=-A_{1 b 2}\), поэтому

\[
A=A_{1 a 2}-A_{1 b 2} .
\]

А так как в нашем случае работа не зависит от пути, т. е. \(A_{1 a 2}=A_{1 b 2}\), то в результате и оказывается, что работа на произвольном замкнутом пути действительно равна нулю: \(A=0\).

\footnotetext{\begin{itemize}
  \item Их называют также потенциальными.
\end{itemize}
}Все силы, не являющиеся консервативными, называют неконсервативными. К числу неконсервативных сил относятся, например, силы трения и сопротивления. Работа этих сил зависит, вообще говоря, от пути между начальным и конечным положениями частицы (и не равна нулю на любом замкнутом пути).

\section*{Поле центральных сил}
Всякое силовое поле вызывается действием определенных тел. Сила, действующая на частицу \(M\) в таком поле, обусловлена взаимодействием этой частицы с данными телами. Силы, зависящие только от расстояния между взаимодействующими частицами и направленные по прямой, проходящей через эти частицы, называют центральньлми. Примером последних являются силы гравитационные, кулоновские и упругие.

Центральную силу, действующую на частицу \(M\) со стороны частицы \(O\), можно представить в виде


\begin{equation*}
\mathbf{F}=f(r) \mathbf{e}_{r}, \tag{4.8}
\end{equation*}


где \(f(r)\) - фукнция, зависящая при данном характере взаимодействия только от \(r\) - расстояния между частицами; \(\mathbf{e}_{r}\) - единичный вектор, задающий направление радиуса-вектора частицы \(M\) относительно частицы \(O\) (рис. 4.6).

Оказывается, центральные сильл являются консервативными. Для доказательства этого утверждения найдем сначала работу центральной силы в

\begin{figure}[h]
\begin{center}
  \includegraphics[alt={},max width=\textwidth]{66a35da9-3faa-4823-9ca9-40799f37b3db-100_368_370_984_780}
\captionsetup{labelformat=empty}
\caption{Рис. 4.6}
\end{center}
\end{figure}

случае, когда силовое поле вызвано наличием одной неподвижной частицы \(O\). Элементарная работа силы (4.8) на перемещении \(\mathrm{d} \mathbf{r}\) есть \(\delta A=\mathbf{F} \mathrm{d} \mathbf{r}=f(r) \mathbf{e}_{r} \mathrm{~d} \mathbf{r}\). Так как \(\mathbf{e}_{r} \mathrm{~d} \mathbf{r}=\mathrm{d} r\) - проекция вектора \(\mathrm{d} \mathbf{r}\) на вектор \(\mathbf{e}_{r}\) или на соответствующий радиус-вектор \(\mathbf{r}\) (см. рис. 4.6), то \(\delta A=f(r) \mathrm{d} r\). Работа этой силы на произвольном пути от точки 1 до точки 2

\[
A_{12}=\int_{1}^{2} f(r) \mathrm{d} r .
\]

Полученное выражение зависит только от вида функции \(f(r)\), т. е. от характера взаимодействия и от значений \(r_{1}\) и \(r_{2}\) — начального и конечного расстояний между частицами \(M\) и \(O\). От пути оно никак не зависит.

Обобщим полученный результат на стационарное силовое поле, вызванное наличием совокупности неподвижных частиц, действующих на частицу \(M\) с силами \(\mathbf{F}_{1}, \mathbf{F}_{2}, \ldots\), каждая из которых является центральной. В этом случае работа результирующей силы при перемещении частицы \(M\) из одной точки в другую равна алгебраической сумме работ отдельных сил. А так как работа каждой из этих сил не зависит от пути, то и работа результирующей силы также не зависит от пути.

Вывод: поскольку центральные силы обладают таким свойством, они являются консервативными.

\section*{Потенциальная энергия частицы в поле}
То обстоятельство, что работа консервативных сил в случае стационарного поля зависит только от начального и конечного положений частицы, дает возможность ввести чрезвычайно важное понятие потенциальной энергии.

Представим себе стационарное поле консервативных сил, в котором мы перемещаем частицу из разных точек \(P_{i}\) в некоторую фиксированную точку \(O\). Так как работа сил поля не зависит от пути, то остается зависимость ее только от положения точки \(P\) (при фиксированной точке \(O\) ). А это значит, что данная работа будет некоторой функцией радиуса-вектора \(\mathbf{r}\) точки \(P\). Обозначив эту функцию \(U(\mathbf{r})\), запишем


\begin{equation*}
A_{P O}=\int_{P}^{O} \mathbf{F} \mathrm{~d} \mathbf{r}=U(\mathbf{r}) . \tag{4.9}
\end{equation*}


\begin{figure}[h]
\begin{center}
  \includegraphics[alt={},max width=\textwidth]{66a35da9-3faa-4823-9ca9-40799f37b3db-101_281_308_1484_76}
\captionsetup{labelformat=empty}
\caption{Рис. 4.7}
\end{center}
\end{figure}

Функцию \(U(\mathbf{r})\) называют потенциальной энергией частицы в данном поле.

Найдем работу сил поля при перемещении частицы из точки 1 в точку 2 (рис. 4.7). Так как работа не зависит от пути, выберем путь, проходящий через точку \(O\).

Тогда работа на пути 102 может быть представлена в виде

\[
A_{12}=A_{1 O}+A_{O 2}=A_{1 O}-A_{2 O},
\]

или с учетом (4.9)


\begin{equation*}
A_{12}=\int_{1}^{2} \mathbf{F} \mathrm{~d} \mathbf{r}=U_{1}-U_{2} \tag{4.10}
\end{equation*}


Выражение, стоящее справа, есть убыль* потенциальной энергии, т. е. разность значений потенциальной энергии частицы в начальной и конечной точках пути.

Таким образом, работа сил поля на пути \(1-2\) равна убыли потенциальной энергии частицы в данном поле.

Очевидно, частице, находящейся в точке \(O\) поля, всегда можно приписать любое наперед выбранное значение потенциальной энергии. Это соответствует тому обстоятельству, что работа сил поля определяет лишь разность потенциальных энергий в двух точках, но не их абсолютное значение. Однако как только фиксирована потенциальная энергия в какой-либо точке, значения ее во всех остальных точках поля однозначно определяются формулой (4.10).

Формула (4.10) позволяет найти выражение \(U(\mathbf{r})\) для любого стационарного поля консервативных сил. Для этого достаточно вычислить работу, совершаемую силами поля на любом пути между двумя точками, и представить ее в виде убыли некоторой функции, которая и есть потенциальная энергия \(U(\mathbf{r})\).

\footnotetext{\begin{itemize}
  \item Изменение какой-либо величины \(X\) можно характеризовать либо ее приращением, либо убылью. Приращением величины \(X\) называют разность конечного \(\left(X_{2}\right)\) и начального ( \(X_{1}\) ) значений этой величины:
\end{itemize}

\[
\text { приращение } \Delta X=X_{2}-X_{1} \text {. }
\]

Убылью величины \(X\) называют разность ее начального ( \(X_{1}\) ) и конечного ( \(X_{2}\) ) значений:

\[
\text { убыль } X_{1}-X_{2}=-\Delta X \text {, }
\]

т. е. убыль величины \(X\) равна ее приращению, взятому с обратным знаком.

Приращение и убыль - величины алгебраические: если, например, \(X_{2}<X_{1}\), то приращение отрицательно, а убыль положительна.
}Именно так и было сделано при вычислении работы в полях упругой и гравитационной (кулоновской) сил, а такдже в однородном поле сил тяжести [см. формулы (4.3)-(4.5)]. Из этих формул сразу видно, что потенциальная энергия частицы в данных силовых полях имеет следующий вид:

\begin{enumerate}
  \item в поле упругой силы
\end{enumerate}


\begin{equation*}
U(r)=x r^{2} / 2 ; \tag{4.11}
\end{equation*}


\begin{enumerate}
  \setcounter{enumi}{1}
  \item в гравитационном (кулоновском) поле материальной точки
\end{enumerate}


\begin{equation*}
U(r)=\alpha / r ; \tag{4.12}
\end{equation*}


\begin{enumerate}
  \setcounter{enumi}{2}
  \item в однородном поле сил тяжести
\end{enumerate}


\begin{equation*}
U(z)=m g z . \tag{4.13}
\end{equation*}


Еще раз отметим, что потенциальная энергия \(U\) - функция, которая определяется с точностью до прибавления некоторой произвольной постоянной. Это обстоятельство, однако, совершенно несущественно, так как в формулы входит только разность значений \(U\) в двух положениях частицы. Поэтому произвольная постоянная, одинаковая для всех точек поля, выпадает. В связи с этим ее обычно опускают, что и сделано в трех предыдущих выражениях.

И еще одно важное обстоятельство. Потенциальную энергию следует относить не к частице, а к системе взаимодействующих между собой частицы и тел, вызывающих силовое поле. При данном характере взаимодействия потенциальная энергия взаимодействия частицы с данными телами зависит только от положения частицы относительно этих тел.

\section*{Потенциальная энергия и сила поля}
Взаимодействие частицы с окружающими телами можно описывать двумя способами: с помощью сил или с помощью потенциальной энергии. В ньютоновской механике оба способа используют одинаково широко. Однако первый способ обладает несколько большей общностью, так как он применим и к таким силам, для которых нельзя ввести потенциальную энергию (например, к силам трения). Второй способ применим только в случае консервативных сил.

Наша задача - установить связь между потенциальной энергией и силой поля, точнее, определить поле сил \(\mathbf{F}(\mathbf{r})\) по заданной потенциальной энергии \(U(\mathbf{r})\) как функции положения частицы в поле.

Мы уже знаем, что при перемещении частицы из одной точки стационарного поля консервативных сил в другую работа, которую производят силы поля, может быть представлена как убыль потенциальной энергии частицы в данном поле, т. е. \(A_{12}=U_{1}-U_{2}=-\Delta U\). Это относится и к элементарному перемещению \(\mathrm{d} \mathbf{r}\), а именно: \(\delta A=-\mathrm{d} U\), или


\begin{equation*}
\mathbf{F} \mathrm{d} \mathbf{r}=-\mathrm{d} U \tag{4.14}
\end{equation*}


Имея в виду, что \(\mathbf{F} \mathrm{d} \mathbf{r}=F_{s} \mathrm{~d} s\), где \(\mathrm{d} s=|\mathrm{d} \mathbf{r}|\) - элементарный путь, \(F_{s}\) - проекция вектора \(\mathbf{F}\) на перемещение \(\mathrm{d} \mathbf{r}\), перепишем уравнение (4.14) в форме

\[
F_{s} \mathrm{~d} s=-\mathrm{d} U
\]

где \(-\mathrm{d} U\) - убыль потенциальной энергии в направлении перемещения dr. Отсюда


\begin{equation*}
F_{s}=-\partial U / \partial s, \tag{4.15}
\end{equation*}


т. е. проекция силы поля - вектора \(\mathbf{F}\) - в данной точке на направление перемещения dr равна с обратным знаком производной потенциальной энергии \(U\) по данному направлению. Символ \(\partial / \partial s\) - частной производной - подчеркивает, что производная берется по определенному направлению.

Перемещение dr можно взять в любом направлении, в частности вдоль координатных осей \(X, Y, Z\). Если перемещение \(\mathrm{d} \mathbf{r}\), например, параллельно оси \(X\), то его можно представить так: \(\mathrm{d} \mathbf{r}=\mathbf{i} \mathrm{d} x\), где \(\mathbf{i}\) - орт оси \(X, \mathrm{~d} x\) - приращение координаты \(x\). Тогда работа силы \(\mathbf{F}\) на перемещении dr , параллельном оси \(X\),

\[
\mathbf{F} \mathrm{d} \mathbf{r}=\mathbf{F i} \mathrm{d} x=F_{x} \mathrm{~d} x
\]

где \(F_{x}\) - проекция вектора \(\mathbf{F}\) на орт \(\mathbf{i}\) (а не на перемещение \(\mathrm{d} \mathbf{r}\), как в случае \(F_{s}\) ). Подставив последнее выражение в уравнение (4.14), получим

\[
F_{x}=-\partial U / \partial x
\]

где символ частной производной означает, что \(U(x, y, z)\) при дифференцировании должна рассматриваться как функция одного аргумента \(x\), остальные же аргументы должны оставаться при этом постоянными. Ясно, что для проекций \(F_{y}\) и \(F_{z}\) уравнения будут аналогичны уравнению для \(F_{x}\).

Итак, взяв с обратными знаками частные производные функции \(U\) по \(x, y\) и \(z\), мы найдем проекции \(F_{x}\), \(F_{y}\) и \(F_{z}\) вектора \(\mathbf{F}\) на орты i, j и k. Отсюда легко найти и сам вектор: \(\mathbf{F}=F_{x} \mathbf{i}+F_{y} \mathbf{j}+F_{z} \mathbf{k}\), или


\begin{equation*}
\mathbf{F}=-\left(\frac{\partial U}{\partial x} \mathbf{i}+\frac{\partial U}{\partial y} \mathbf{j}+\frac{\partial U}{\partial z} \mathbf{k}\right) . \tag{4.16}
\end{equation*}


Величину, стоящую в скобках, называют градиентом скалярной фукниии \(U\) и обозначают \(\operatorname{grad} U\) или \(\nabla U\). Мы будем пользоваться вторым, более удобным, обозначением, где \(\nabla\) («набла») означает символический вектор или оператор


\begin{equation*}
\nabla=\mathbf{i} \frac{\partial}{\partial x}+\mathbf{j} \frac{\partial}{\partial y}+\mathbf{k} \frac{\partial}{\partial z} . \tag{4.17}
\end{equation*}


Поэтому \(\nabla U\) формально можно рассматривать как произведение символического вектора \(\nabla\) на скаляр \(U\).

Таким образом, связь между силой поля и потенциальной энергией как функцией координат можно представить в следующем компактном виде:


\begin{equation*}
\mathbf{F}=-\nabla U \tag{4.18}
\end{equation*}


т. е. сила поля \(\mathbf{F}\) равна со знаком минус градиенту потенциальной энергии частицы в данной точке поля. Формула (4.18) позволяет, зная функцию \(U(\mathbf{r})\), восстановить поле сил \(\mathbf{F}(\mathbf{r})\).\\
Пример. Потенциальная энергия частицы в некотором поле имеет вид:\\
a) \(U(x, y)=-\alpha x y\), где \(\alpha\) - постоянная;\\
б) \(U(\mathbf{r})=\mathbf{a r}\), где \(\mathbf{a}\) - постоянный вектор, \(\mathbf{r}\) - радиус-вектор точки поля.\\
Найдем соответствующее каждому случаю поле сил:\\
а) \(\mathbf{F}=-\left(\frac{\partial U}{\partial x} \mathbf{i}+\frac{\partial U}{\partial y} \mathbf{j}\right)=\alpha(y \mathbf{i}+x \mathbf{j})\);\\
б) представим функцию \(U\) в виде \(U=a_{x} x+a_{y} y+a_{z} z\); тогда

\[
\mathbf{F}=-\left(\frac{\partial U}{\partial x} \mathbf{i}+\frac{\partial U}{\partial y} \mathbf{j}+\frac{\partial U}{\partial z} \mathbf{k}\right)=-\left(a_{x} \mathbf{i}+a_{y} \mathbf{j}+a_{z} \mathbf{k}\right)=-\mathbf{a} .
\]

Смысл градиента станет нагляднее и яснее, если ввести понятие эквипотенииальной поверхности - поверхности, во всех точках которой потенциальная энергия \(U\) имеет одно и то же значение. Ясно, что каждому значению \(U\) соответствует своя эквипотенциальная поверхность.

Из формулы (4.15) следует, что проекция вектора \(\mathbf{F}\) на любое направление, касательное к эквипотенциальной поверхности в данной точке, равна нулю. Это значит, что вектор \(\mathbf{F}\) нормален эквипотенциальной поверхности в данной точке. Далее, возьмем перемещение \(\partial s\) по нормали к эквипотенциальной поверхности в сторону уменьшения \(U\), тогда \(\partial U<0\) и, согласно (4.15), \(F_{s}>0\), т. е. вектор \(\mathbf{F}\) направлен в сторону уменьшения \(U\). А так как \(\mathbf{F}\) противоположен по направлению вектору \(\nabla U\), то мы приходим к выводу, что градиент \(U\) - это вектор, направленный по нормали к эквипотенциальной поверхности в сторону возрастанаия потенииальной энергии \(U\).

Сказанное поясняет рис. 4.8, относящийся к двумерному случаю. На нем изображены система эквипотенциалей ( \(U_{1}<U_{2}<U_{3}<U_{4}\) ), а также градиент потенциальной энергии \(\nabla U\) и соответствующий вектор силы \(\mathbf{F}\) в точке \(A\) поля. Полезно подумать, какими будут векторы этих двух величин, например в точке \(B\) данного поля.

В заключение заметим, что можно

\begin{figure}[h]
\begin{center}
  \includegraphics[alt={},max width=\textwidth]{66a35da9-3faa-4823-9ca9-40799f37b3db-106_350_394_1024_760}
\captionsetup{labelformat=empty}
\caption{Рис. 4.8}
\end{center}
\end{figure}

говорить о градиенте не только функции \(U\), но и любой другой скалярной функции координат. Понятие градиента широко используется в самых различных разделах физики.

\section*{Понятие поля}
Опыт показывает, что в случае гравитационных и электростатических взаимодействий сила \(\mathbf{F}\), действующая на интересующую нас частицу со стороны окружающих тел, пропорциона-

льна массе (или заряду) частицы, причем сила \(\mathbf{F}\) может быть представлена в виде произведения двух величин, например в случае тяготения


\begin{equation*}
\mathbf{F}=m \mathbf{G} \tag{4.19}
\end{equation*}


где \(m\) - масса частицы, \(\mathbf{G}\) - некоторый вектор, зависящий как от положения частицы, так и от свойств окружающих тел.

Это открывает возможность иной физической интерпретации взаимодействия, связанной с понятием поля. А именно: говорят, что интересующая нас частица находится в поле, создаваемом окружающими ее телами и характеризуемом вектором \(\mathbf{G}(\mathbf{r})\). Или, иначе, считают, что в каждой точке пространства вокруг этих тел (источников поля) создаются такие условия (вектор G), при которых частица, помещенная в эти точки, испытывает действие силы (4.19), причем считают, что поле, характеризуемое \(\mathbf{G ( r )}\), существует безотносительно к тому, есть в нем частица или нет*.

Вектор \(\mathbf{G}\) называют напряженностью поля. Напряженность электрического поля обозначают \(\mathbf{E}\). Сила \(\mathbf{F}\), действующая на точечный заряд \(q\) в электростатическом поле, имеет вид, аналогичный (4.19), т. е. \(\mathbf{F}=q \mathbf{E}\).

Далее в этом параграфе мы будем пользоваться величинами \(m\) и \(\mathbf{G}\), т. е. рассматривать гравитационное поле. Чтобы получить соответствующие соотношения для электростатического поля, достаточно заменить в формулах \(m\) и \(\mathbf{G}\) на \(q\) и \(\mathbf{E}\).

Одно из важнейших свойств полей заключается в том, что поле, образованное несколькими источниками, равно сумме полей, созданных каждым из них. Точнее, напряженность \(\mathbf{G}\) peзультирующего поля в произвольной точке


\begin{equation*}
\mathbf{G}=\sum \mathbf{G}_{i}, \tag{4.20}
\end{equation*}


где \(\mathbf{G}_{i}\) - напряженность поля \(i\)-го источника в этой же точке. Эта формула выражает принцип суперпозиции (или наложения) полей.

\footnotetext{\begin{itemize}
  \item Пока мы остаемся в рамках статики, понятие поля может рассматриваться как чисто условное (формальное), введенное лишь для удобства описания явлений. Однако при переходе к переменным полям выясняется, что понятие поля имеет глубокий физический смысл: поле есть физическая реальность.
\end{itemize}
}Обратимся к потенциальной эрегргии частицы. Согласно (4.19), формулу (4.14) можно записать так: \(m \mathbf{G d} \mathbf{r}=-\mathrm{d} U\). Разделив обе части на \(m\) и обозначив отношение \(U / m\) через \(\varphi\), получим


\begin{equation*}
\mathbf{G} \mathrm{d} \mathbf{r}=-\mathrm{d} \varphi, \tag{4.21}
\end{equation*}


или


\begin{equation*}
\int_{1}^{2} \mathbf{G} \mathrm{~d} \mathbf{r}=\varphi_{1}-\varphi_{2} . \tag{4.22}
\end{equation*}


Функцию \(\varphi(\mathbf{r})\) называют потенииалом поля в точке с радиу-сом-вектором r.

Формула (4.22) позволяет найти потенциал любого гравитационного и электростатического поля. Для этого достаточно вычислить интеграл \(\int \mathbf{G} \mathrm{d} \mathbf{r}\) по произвольному пути между точками 1 и 2 и представить полученное выражение в виде убыли некоторой функции, которая и есть потенциал \(\varphi(\mathbf{r})\). Так, потенциалы гравитационного поля точечной массы \(m\) и кулоновского поля точечного заряда \(q\) определяются, согласно (4.12), формулами


\begin{equation*}
\varphi_{\text {гр }}=-\gamma m / r, \quad \varphi_{\text {кул }}=k q / r . \tag{4.23}
\end{equation*}


Заметим, что потенциал φ, как и потенциальная энергия, может быть определен только с точностью до некоторой произвольной постоянной, также совершенно несущественной. Поэтому ее обычно опускают.

Итак, поле можно описывать или в векторном виде G(r), или в скалярном φ (r). Оба способа адекватны. Практически же оказывается, что второй способ описания поля (с помощью потенциала φ) в большинстве случаев значительно удобнее, и вот почему.

\begin{enumerate}
  \item Зная φ (r), можно немедленно вычислить потенциальную энергию \(U\) и работу сил поля \(A\) :
\end{enumerate}


\begin{equation*}
U=m \varphi, \quad A_{12}=m\left(\varphi_{1}-\varphi_{2}\right) . \tag{4.24}
\end{equation*}


\begin{enumerate}
  \setcounter{enumi}{1}
  \item Вместо трех компонент векторной фукнции G(r) проще задавать скалярную функцию \(\varphi(\mathbf{r})\).
  \item Когда поле создается многими источниками, потенциал φ рассчитывать легче, чем вектор G: потенциалы - скаляры, их
\end{enumerate}

можно просто складывать, не заботясь о направлении сил. Действительно, согласно (4.20) и (4.21),

\[
\mathbf{G} \mathrm{d} \mathbf{r}=\sum \mathbf{G}_{i} \mathrm{~d} \mathbf{r}=-\sum \mathrm{d} \varphi_{i}=-\mathrm{d} \sum \varphi_{i}=-\mathrm{d} \varphi .
\]

Таким образом


\begin{equation*}
\varphi(\mathbf{r})=\sum \varphi_{i}(\mathbf{r}), \tag{4.25}
\end{equation*}


где \(\varphi_{i}\) - потенциал, создаваемый \(i\)-й частицей в данной точке поля.\\
4. И наконец, зная функцию \(\varphi(\mathbf{r})\), можно легко восстановить поле \(\mathbf{G ( r )}\) как


\begin{equation*}
\mathbf{G}=-\nabla \varphi . \tag{4.26}
\end{equation*}


Эта формула непосредственно следует из (4.18)

\section*{§ 4.3. Механическая энергия частицы в поле}
Кинетическая энергия\\
Пусть частица массы \(m\) движется под действием некоторой силы \(\mathbf{F}\) (в общем случае сила \(\mathbf{F}\) может быть результирующей нескольких сил). Найдем элементарную работу, которую совершает эта сила \(\mathbf{F}\) на элементарном перемещении dr. Имея в виду, что \(\mathbf{F}=m \mathrm{~d} \mathbf{v} / \mathrm{d} t\) и \(\mathrm{d} \mathbf{r}=\mathbf{v} \mathrm{d} t\), запишем

\[
\delta A=\mathbf{F} \mathrm{d} \mathbf{r}=m \mathbf{v} \mathrm{~d} \mathbf{v} .
\]

Скалярное произведение \(\mathbf{v d} \mathbf{v}=v(\mathbf{d v})_{\mathbf{v}}\), где \((\mathbf{d v})_{\mathbf{v}}\) - проекция вектора \(\mathrm{d} \mathbf{v}\) на направление вектора \(\mathbf{v}\). Эта проекция равна \(\mathrm{d} v-\) приращению модуля вектора скорости. Поэтому \(\mathbf{v} \mathrm{d} \mathbf{v}=v \mathrm{~d} v\) и элементарная работа

\[
\delta A=m v \mathrm{~d} v=\mathrm{d}\left(m v^{2} / 2\right) .
\]

Отсюда видно, что работа результирующей силы \(\mathbf{F}\) идет на приращение некоторой величины (стоящей в скобках), которую называют кинетической энергией:


\begin{equation*}
K=m v^{2} / 2 \text {. } \tag{4.27}
\end{equation*}


Таким образом, приращение кинетической энергии частицы при элементарном перемещении равно


\begin{equation*}
\mathrm{d} K=\delta A, \tag{4.28}
\end{equation*}


а при конечном перемещении из точки 1 в точку 2


\begin{equation*}
K_{2}-K_{1}=A_{12}, \tag{4.29}
\end{equation*}


т. е. приращение кинетической энергии частицы на некотором перемещении равно алгебраической сумме работ всех сил, действующих на частицу на том же перемещении. Если \(A_{12}>0\), то \(K_{2}>K_{1}\), т. е. кинетическая энергия частицы увеличивается; если \(A_{12}<0\), то кинетическая энергия уменьшается.

Уравнения (4.28) и (4.29) справедливы в инерциальных и неинерциальных системах отсчета. В последних кроме сил, действующих на рассматриваемую частицу со стороны ка-ких-то тел (сил взаимодействия), необходимо учитывать и силы инерции. Поэтому под работой в этих уравнениях надо понимать алгебраическую сумму работ как сил взаимодействия, так и сил инерции.

\section*{Полная механическая энергия частицы}
Согласно (4.28), приращение кинетической энергии частицы равно элементарной работе результирующей \(\mathbf{F}\) всех сил, действующих на частицу. Что это за силы? Если частица находится в интересующем нас стационарном поле консервативных сил, то на нее действует консервативная сила \(\mathbf{F}_{\text {п }}\) со стороны этого поля. Кроме того, на частицу могут действовать и другие силы, не имеющие отношения к данному силовому полю. Назовем их сторонними силами \(\mathbf{F}_{\text {стор }}\). Отметим, что сторонние силы могут быть и консервативными, и неконсервативными. Существенно, повторяем, только одно - чтобы они не являлись силами интересующего нас силового поля.

Таким образом, результирующая \(\mathbf{F}\) всех сил, действующих на частицу, может быть представлена как сумма \(\mathbf{F}=\mathbf{F}_{\text {п }}+\mathbf{F}_{\text {стор }}\). Работа этих сил идет на приращение кинетической энергии частицы:

\[
\Delta K=A_{\text {сп }}+A_{\text {стор }},
\]

где \(A_{\text {сп }}\) - работа сил поля, \(A_{\text {стор }}\) - работа сторонних сил. Cогласно (4.10), работа сил поля равна убыли потенциальной энергии частицы: \(A_{\text {сп }}=-\Delta U\). Подставив это выражение в предыдущее и перенеся \(\Delta U\) влево, получим

\[
\Delta K+\Delta U=\Delta(K+U)=A_{\text {стор }} .
\]

Отсюда видно, что работа сторонних сил идет на приращение \(K+U\). Эту величину - сумму кинетической и потенциальной энергий - называют полной механической энергией частиць в поле и обозначают \(E\) :


\begin{equation*}
E=K+U . \tag{4.30}
\end{equation*}


Заметим, что полная механическая энергия \(E\), как и потенциальная \(U\), определяется с точностью до произвольной постоянной.

Итак, из предыдущих двух уравнений следует, что приращение полной механической энергии частицы в стационарном поле консервативных сил при перемещении ее из точки 1 в точку 2 можно записать в виде


\begin{equation*}
E_{2}-E_{1}=A_{\text {стор }} \tag{4.31}
\end{equation*}


т. е. приращение полной механической энергии частицы на некотором пути равно алгебраической сумме работ всех сторонних сил, действующих на частицу на том же пути. Если \(A_{\text {стор }}>0\), то полная механическая энергия частицы увеличивается, если \(A_{\text {стор }}<0\), то уменьшается.\\
Пример. Тело массы \(m\) бросили со скоростью \(v_{0}\) с обрыва высотой \(h\) над поверхностью воды. Найдем работу, которую совершила сила сопротивления со стороны воздуха, при условии, что тело упало на поверхность воды со скоростью \(v\).\\
Если рассматривать движение тела в поле сил тяжести, то сила сопротивления со стороны воздуха будет сторонней и, согласно уравнению (4.31), искомая работа \(A_{\text {стор }}=E_{2}-E_{1}= =m v^{2} / 2-\left(m v_{0}^{2} / 2+m g h\right)\) или

\[
A_{\text {стор }}=m\left(v^{2}-v_{0}^{2}\right) / 2-m g h .
\]

Интересно, что полученная величина может оказаться не только отрицательной, но и положительной (это зависит, например, от характера ветра в процессе падения тела).

Итак, полная механическая энергия частицы может измениться под действием только сторонних сил. Отсюда непосредственно вытекает закон сохранения механической энергии частицы: если сторонние силь отсутствуют или таковы, что не совершают работь в течение интересующего нас времени, то полная механическая энергия частиць в стационарном поле консервативных сил остается постоянной за это время:


\begin{equation*}
E=K+U=\text { const. } \tag{4.32}
\end{equation*}


Уже в такой простейшей форме закон сохранения энергии позволяет достаточно легко получать ответы на ряд важных вопросов без привлечения уравнений движения, что часто сопряжено с проведением громоздких и утомительных расчетов. Именно это обстоятельство и превращает законы сохранения в весьма действенный инструмент исследования.

Проиллюстрируем возможности и преимущества, которые дает применение закона сохранения (4.32), на следующем примере.\\
Пример. Пусть частица движется в одномерном стационарном поле, где ее потенциальная энергия \(U(x)\) имеет вид, как на рис. 4.9. Если сторонние силы отсутствуют, то полная механическая энергия частицы в данном поле, т. е. \(E\), не изменяется в процессе движения и можно просто ответить, например, на следующие вопросы.

\begin{enumerate}
  \item Определить, не решая основного
\end{enumerate}

\begin{figure}[h]
\begin{center}
  \includegraphics[alt={},max width=\textwidth]{66a35da9-3faa-4823-9ca9-40799f37b3db-112_303_378_970_778}
\captionsetup{labelformat=empty}
\caption{Рис. 4.9}
\end{center}
\end{figure}

Для этого достаточно знать, согласно уравнению (4.32), конкретный вид потенциальной кривой \(U(x)\) и значение полной энергии \(E\).\\
2. Установить область изменения координаты \(x\) частицы, в которой она может находиться при данном значении полной энергии \(E\).

Ясно, что в область, где \(U>E\), частица попасть не может, поскольку потенциальная энергия \(U\) частицы не должна превышать ее полную энергию. Отсюда сразу следует, что при \(E=E_{1}\) (рис. 4.9) частица будет двигаться или в области между коор-

динатами \(x_{1}\) и \(x_{2}\) (совершает колебания), или правее координаты \(x_{3}\). Перейти из первой области во вторую (или обратно) частица не может: этому препятствует потенииальный барьер, разделяющий обе области. Заметим, что когда частица движется в ограниченной области поля, то говорят, что она заперта в потенциальной яме (в нашем случае - между \(x_{1}\) и \(x_{2}\) ).\\
Иначе ведет себя частица при \(E=E_{2}\) (рис. 4.9): для нее доступна вся область правее координаты \(x_{0}\). Если в начальный момент частица находилась в точке \(x_{0}\), то в дальнейшем она будет двигаться вправо. Полезно самостоятельно проследить, как будет меняться при этом кинетическая энергия частицы в зависимости от ее координаты \(x\).

\section*{§ 4.4. Потенциальная энергия системы}
\section*{Собственная потенциальная энергия системы}
До сих пор мы ограничивались рассмотрением поведения одной частицы с энергетической точки зрения. Теперь перейдем к системе частиц. Это может быть любое тело, газ, какой-то механизм, Солнечная система и т. д. Рассмотрим систему, между частицами которой действуют одни лишь центральные силы, т. е. силы, зависящие при данном характере взаимодействия только от расстояния между частицами и направленные по прямой, проходящей через эти частицы.

Покажем, что независимо от системы отсчета работа всех этих внутренних сил при переходе системы частиц из одного положения в другое может быть представлена как убыль некоторой функции, зависящей при данном характере взаимодействия только от относительного расположения частиц системы, т. е. от ее конфигурации. Эту функцию называют собственной потенциальной энергией системы (в отличие от внешней потенциальной энергии, характеризующей взаимодействие данной системы с другими телами).

Сначала возьмем систему из двух частиц 1 и 2. Определим алгебраическую сумму элементарных работ сил \(\mathbf{F}_{1}\) и \(\mathbf{F}_{2}\), с которыми эти частицы взаимодействуют. Пусть в произвольной \(K\)-системе отсчета за время \(\mathrm{d} t\) частицы совершили перемещения \(\mathrm{d} \mathbf{r}_{1}\) и \(\mathrm{d} \mathbf{r}_{2}\). Тогда соответствующая сумма работ этих сил

\[
\delta A_{1,2}=\mathbf{F}_{1} \mathrm{~d} \mathbf{r}_{1}+\mathbf{F}_{2} \mathrm{~d} \mathbf{r}_{2} .
\]

Учитывая, что \(\mathbf{F}_{2}=-\mathbf{F}_{1}\) (согласно третьему закону Ньютона), перепишем предыдущее уравнение:

\[
\delta A_{1,2}=\mathbf{F}_{1}\left(\mathrm{~d} \mathbf{r}_{1}-\mathrm{d} \mathbf{r}_{2}\right) .
\]

Величина, стоящая в скобках, представляет собой не что иное, как перемещение частицы 1 относительно частицы 2 , точнее, перемещение частицы 1 в \(K^{\prime}\)-системе отсчета, жестко связанной с частицей 2 и перемещающейся вместе с ней поступательно относительно исходной \(K\)-системы отсчета. Действительно, перемещение \(\mathrm{d} \mathbf{r}_{1}\) частицы 1 в \(K\)-системе отсчета может быть представлено как перемещение \(\mathrm{d} \mathbf{r}_{2} K^{\prime}\)-системы отсчета (связанной с частицей 2) плюс перемещение \(d \mathbf{r}_{1}^{\prime}\) частицы 1 относительно этой \(K^{\prime}\)-системы, т. е. \(\mathrm{d} \mathbf{r}_{1}=\mathrm{d} \mathbf{r}_{2}+\mathrm{d} \mathbf{r}_{1}^{\prime}\). Отсюда \(\mathrm{d} \mathbf{r}_{1}-\mathrm{d} \mathbf{r}_{2}=\mathrm{d} \mathbf{r}_{1}^{\prime}\) и

\[
\delta A_{1,2}=\mathbf{F}_{1} \mathrm{~d} \mathbf{r}_{1}^{\prime}
\]

Полученный таким образом результат весьма замечателен: алгебраическая сумма элементарных работ пары сил взаимодействия в произольной \(K\)-системе отсчета оказывается всегда равной элементарной работе, которую совершает сила, действующая на одну частицу, в системе отсчета, где другая частица покоится. Иначе говоря, работа \(\delta A_{1,2}\) не зависит от выбора исходной \(K\)-системы отсчета.

Сила \(\mathbf{F}_{1}\), действующая на частицу 1 со стороны частицы 2 , является центральной, а значит и консервативной. Поэтому работа данной силы на перемещении \(d \mathbf{r}_{1}^{\prime}\) может быть представлена, согласно (4.10), как убыль потенциальной энергии частицы 1 в поле частицы 2 или как убыль потенциальной энергии взаимодействия рассматриваемой пары частиц:

\[
\delta A_{1,2}=-\mathrm{d} U_{12}
\]

где \(U_{12}\) - функция, зависящая только от расстояния между этими частицами. При конечном же перемещении

\[
A_{1,2}=-\Delta U_{12} .
\]

Рассмотрим теперь систему из трех частиц. (Полученный в этом случае результат легко обобщить и на систему из произвольного числа частиц.) Работа, которую совершают все силы

взаимодействия при перемещении всех частиц, может быть представлена как алгебраическая сумма работ всех трех пар сил взаимодействий: \(A=A_{1,2}+A_{1,3}+A_{2,3}\). Но для каждой пары этих сил \(A_{i k}=-\Delta U_{i k}\), поэтому

\[
A=-\Delta\left(U_{12}+U_{13}+U_{23}\right)=-\Delta U_{\text {соб }},
\]

где \(U_{\text {соб }}\) - собственная потенциальная энергия данной системы частиц:

\[
U_{\text {соб }}=U_{12}+U_{13}+U_{23} .
\]

Так как каждое слагаемое этой суммы зависит от расстояния между соответствующими частицами, то очевидно, что собственная потенциальная энергия данной системы зависит от относительного расположения частиц (в один и тот же момент), или, другими словами, от конфигурации системы.

Подобные рассуждения справедливы и для системы из любого числа частиц. Поэтому каждой конфигурации системь частии присуще свое значение собственной потенциальной энергии и работа всех внутренних центральных (консервативных) сил при изменении этой конфигурации равна убыли собственной потенциальной энергии системь:


\begin{equation*}
A_{\text {внутр }}=U_{1 \text { соб }}-U_{2 \text { соб }}=-\Delta U_{\text {соб }}, \tag{4.33}
\end{equation*}


где \(U_{1 \text { соб }} U_{2 \text { соб }}\) - собственная потенциальная энергия системы в начальном и конечном состояниях.

Таким образом, суммарная работа внутренних центральных сил не зависит от того, как конкретно система переходит от конфигурации 1 к конфигурации 2. Данная работа определяется исключительно самими конфигурациями системы. Все это позволяет дать более общее определение консервативных сил: консервативными называют сильι, зависящие только от конфигурации системь и суммарная работа которых не зависит от «пути» перехода, а определяется только начальной и конечной конфигурациями системь.

Собственная потенциальная энергия системы - величина не аддитивная, т. е. она не равна в общем случае сумме собственных потенциальных энергий ее частей. Необходимо учесть еще

потенциальную энергию взаимодействия \(U_{\text {вз }}\) отдельных частей системы:


\begin{equation*}
U_{\text {соб }}=\sum U_{n}+U_{\text {вз }}, \tag{4.34}
\end{equation*}


где \(U_{n}\) - собственная потенциальная энергия \(n\)-й части системы.

Следует также иметь в виду, что собственная потенциальная энергия системы, как и потенциальная энергия взаимодействия каждой пары частиц, определяется с точностью до произвольной постоянной.

В заключение приведем полезные формулы для расчета собственной потенциальной энергии системы. Прежде всего покажем, что эта энергия может быть представлена в виде


\begin{equation*}
U_{\text {соб }}=1 / 2 \sum U_{i}, \tag{4.35}
\end{equation*}


где \(U_{i}\) - потенциальная энергия взаимодействия \(i\)-й частицы со всеми остальными частицами системы. Здесь сумма берется по всем частицам системы.

Убедимся в справедливости этой формулы сначала для системы из трех частиц. Выше было показано, что собственная потенциальная энергия данной системы \(U_{\text {соб }}=U_{12}+U_{13}+U_{23}\). Преобразуем эту сумму следующим образом. Представим каждое слагаемое \(U_{i k}\) в симметричном виде: \(U_{i k}=\left(U_{i k}+U_{k i}\right) / 2\), так как ясно, что \(U_{i k}=U_{k i}\). Тогда

\[
U_{\text {соб }}=1 / 2\left(U_{12}+U_{21}+U_{13}+U_{31}+U_{23}+U_{32}\right) .
\]

Сгруппируем члены с одинаковым первым индексом:

\[
U_{\text {соб }}=1 / 2\left[\left(U_{12}+U_{13}\right)+\left(U_{21}+U_{23}\right)+\left(U_{31}+U_{32}\right)\right] .
\]

Каждая сумма в круглых скобках представляет собой потенциальную энергию \(U_{i}\) взаимодействия \(i\)-й частицы с остальными двумя. Поэтому последнее выражение можно переписать так:

\[
U_{\text {соб }}=1 / 2\left(U_{1}+U_{2}+U_{3}\right)=1 / 2 \sum_{i=1}^{3} U_{i},
\]

что полностью соответствует формуле (4.35).\\
Обобщение полученного результата на произвольную систему очевидно, так как подобные рассуждения не зависят от числа частиц, составляющих систему.

Для системы, взаимодействие между частицами которой имеет гравитационный или кулоновский характер, формулу (4.35) можно преобразовать и к другому виду, воспользовавшись понятием потенциала. Заменим в (4.35) потенциальную энергию \(i\)-й частицы выражением \(U_{i}=m_{i} \varphi_{i}\), где \(m_{i}\) - масса (заряд) \(i\)-й частицы, а \(\varphi_{i}\) - потенциал, создаваемый всеми остальными частицами системы в точке нахождения \(i\)-й частицы. Тогда


\begin{equation*}
U_{\text {соб }}=1 / 2 \sum m_{i} \varphi_{i} . \tag{4.36}
\end{equation*}


Если массы (заряды) распределены в системе непрерывно, то суммирование сводится к интегрированию:


\begin{equation*}
U_{\text {coб }}=1 / 2 \int \varphi \mathrm{~d} m=1 / 2 \int \varphi \rho \mathrm{~d} V, \tag{4.37}
\end{equation*}


где \(\rho\) - объемная плотность массы (заряда), \(\mathrm{d} V\) - элемент объема. Здесь интегрирование проводится по всему объему, занимаемому массами (зарядами).

\section*{«Внешняя» потенциальная энергия системы}
Рассмотрим случай, когда система находится во внешнем стационарном поле консервативных сил. В этом случае каждая частица системы будет характеризоваться своим значением потенциальной энергии \(U_{i}\) в данном поле, а вся система - величиной


\begin{equation*}
U_{\text {внеш }}=\sum U_{i} . \tag{4.38}
\end{equation*}


Эту величину мы и будем называть «внешней» потеницальной энергией системы в отличие от \(U_{\text {соб }}\) - собственной потенциальной энергии, зависящей только от взаимодействия частиц системы между собой.

Согласно (4.10), убыль потенциальной энергии каждой частицы во внешнем поле равна работе силы данного поля на соответствующем перемещении, поэтому убыль \(U_{\text {внеш }}\) всей системы равна \(A_{\text {внеш }}\) - алгебраической сумме работ всех сил внешнего поля, действующих на все частицы системы:


\begin{equation*}
A_{\text {внеш }}=-\Delta U_{\text {внеш }} . \tag{4.39}
\end{equation*}


Получим полезную формулу для вычисления внешней потенциальной энергии системы, находящейся в однородном силовом поле. Пусть, например, это поле тяжести, где на \(i\)-ю частицу системы действует сила \(m_{i} g\). В этом случае потенциальная энергия данной частицы, согласно (4.13), есть \(m_{i} g z_{i}\), где \(z_{i}\) - вертикальная координата частицы, отсчитанная от некоторого произвольного уровня \(O\). Тогда потенциальная энергия всей системы во внешнем однородном поле (собственная потенциальная энергия нас сейчас не интересует) может быть записана так:

\[
U_{\text {внеш }}=\sum m_{i} g z_{i}=\left(\sum m_{i} z_{i}\right) g .
\]

Сумма, стоящая в скобках, в соответствии с (3.8) есть не что иное, как произведение массы \(m\) всей системы на вертикальную координату \(z_{C}\) центра масс данной системы, т. е. \(\sum m_{i} z_{i}=m z_{C}\). Поэтому выражение для \(U_{\text {внеш }}\) можно переписать в окончательном виде:


\begin{equation*}
U_{\text {внеш }}=m g z_{C} \text {, } \tag{4.40}
\end{equation*}


т. е. потенциальная энергия системы во внешнем однородном поле тяжести равна произведению массы \(m\) системы на \(g\) и на вертикальную координату \(z_{C}\) ее центра масс.

Приращение величины \(U_{\text {внеш }}\) при перемещении системы равно


\begin{equation*}
\Delta U_{\text {внеш }}=m g \Delta z_{C}, \tag{4.41}
\end{equation*}


где \(\Delta z_{C}\) - приращение вертикальной координаты центра масс данной системы частиц.

\section*{§ 4.5. Закон сохранения механической энергии системы}
\section*{Диссипативные силы}
Помимо разделения всех сил на внешние и внутренние (в зависимости от выбора системы частиц), силы, как мы уже знаем, принято подразделять на консервативные и неконсервативные (в зависимости от их природы).\\
\(К\) неконсервативным силам относятся диссипативные сильι - это силы трения и сопротивления. Любая диссипативная сила может быть представлена в виде


\begin{equation*}
\mathbf{F}=-k(v) \mathbf{v}, \tag{4.42}
\end{equation*}


где \(\mathbf{v}\) - скорость данного тела относительно другого тела (или среды), с которым оно взаимодействует; \(k(v)\) - положительный коэффициент, зависящий в общем случае от скорости \(v\). Сила \(\mathbf{F}\) всегда направлена противоположно вектору \(\mathbf{v}\).

В зависимости от выбора системы отсчета работа этой силы может быть как положительной, так и отрицательной. Однако, как мы сейчас покажем и что будет важно для дальнейшего, суммарная работа всех внутренних диссипативных сил в системе - величина всегда отрицательная независимо от системы отсчета:


\begin{equation*}
A_{\text {внутр }}^{\text {дис }}<0 . \tag{4.43}
\end{equation*}


Переходя к доказательству, отметим прежде всего, что внутренние диссипативные силы в данной системе встречаются попарно, причем в каждой паре, согласно третьему закону Ньютона, они одинаковы по модулю и противоположны по направлению. Найдем элементарную работу произвольной пары диссипативных сил взаимодействия между телами 1 и 2 в системе отсчета, где скорости этих тел в данный момент равны \(\mathbf{v}_{1}\) и \(\mathbf{v}_{2}\) :

\[
\delta A^{\text {дис }}=\mathbf{F}_{1} \mathbf{v}_{1} \mathrm{~d} t+\mathbf{F}_{2} \mathbf{v}_{2} \mathrm{~d} t .
\]

Учтем, что \(\mathbf{F}_{2}=-\mathbf{F}_{1}, \mathbf{v}_{1}-\mathbf{v}_{2}=\mathbf{v}\) - скорость тела 1 относительно тела 2 , а также то, что \(\mathbf{F}_{1}=-k \mathbf{v}\). Тогда выражение для работы преобразуется:

\[
б A^{\text {дис }}=\mathbf{F}_{1}\left(\mathbf{v}_{1}-\mathbf{v}_{2}\right) \mathrm{d} t=-k \mathbf{v v} \mathrm{~d} t=-k v^{2} \mathrm{~d} t .
\]

Отсюда видно, что работа произвольной пары внутренних диссипативных сил взаимодействия всегда отрицательна, а значит, и суммарная работа всех пар внутренних диссипативных сил также всегда отрицательна, причем в любой системе отсчета.

\section*{Кинетическая энергия системы}
Согласно (4.28), приращение кинетической энергии каждой частицы равно работе всех сил, действующих на частицу: \(\Delta K_{i}=A_{i}\). Поэтому работу \(A\), которую совершают все силы, действующие на все частицы системы, при изменении ее состояния, можно записать так: \(A=\sum A_{i}=\sum \Delta K_{i}=\Delta \sum K_{i}\), или


\begin{equation*}
A=\Delta K ; \quad K=\sum K_{i}, \tag{4.44}
\end{equation*}


где \(K\) - суммарная кинетическая энергия системы.\\
Итак, приращение кинетической энергии системы равно работе, которую совершают все силь, действующие на все частицы системы:


\begin{equation*}
\Delta K=A . \tag{4.45}
\end{equation*}


Заметим, что кинетическая энергия системы - величина \(a \partial\) дитивная: она равна сумме кинетических энергий отдельных частей системы независимо от того, взаимодействуют они между собой или нет.

Уравнение (4.45) справедливо как в инерциальных, так и в неинерциальных системах отсчета. Следует только помнить, что в неинерциальных системах отсчета кроме работ сил взаимодействия необходимо учитывать и работу сил инерции.

\section*{Собственная механическая энергия системы}
Только что было показано, что приращение \(\Delta K\) кинетической энергии системы равно работе, которую совершают все силы, действующие на все частицы системы. Разделим эти силы на внешние и внутренние, а внутренние, в свою очередь, на консервативные и диссипативные. Тогда предыдущее утверждение можно записать так:

\[
\Delta K=A_{\text {внеш }}+A_{\text {внутр }}=A_{\text {внеш }}+A_{\text {внутр }}^{\text {конс }}+A_{\text {внутр }}^{\text {дис }} .
\]

Учтем, что работа внутренних консервативных сил равна, согласно (4.33), убыли собственной потенциальной энергии системы: \(A_{\text {внутр }}^{\text {конс }}=-\Delta U_{\text {соб }}\). Тогда


\begin{equation*}
\Delta K+\Delta U_{\text {соб }}=\Delta K\left(K+U_{\text {соб }}\right)=A_{\text {внеш }}+A_{\text {внутр }}^{\text {дис }} . \tag{4.46}
\end{equation*}


Введем понятие собственной механической энергии системы, или, короче, механической энергии*, как суммы кинетической и собственной потенциальной энергии системы:


\begin{equation*}
E_{\text {соб }}=K+U_{\text {соб }} . \tag{4.47}
\end{equation*}


Очевидно, энергия \(E_{\text {соб }}\) зависит от скоростей частиц системы, характера взаимодействия между ними и конфигурации системы. Кроме того, энергия \(E_{\text {соб }}\), как и потенциальная энергия \(U_{\text {соб }}\), определяется с точностью до прибавления несущественной произвольной постоянной и является величиной неаддитивной, т. е. энергия \(E_{\text {соб }}\) не равна в общем случае сумме энергий ее отдельных частей. В соответствии с (4.47)


\begin{equation*}
E_{\text {соб }}=\sum E_{n}+U_{\text {вз }}, \tag{4.48}
\end{equation*}


где \(E_{n}\) - собственная механическая энергия \(n\)-й части системы, \(U_{\text {вз }}\) - потенциальная энергия взаимодействия ее отдельных частей.

Вернемся к уравнению (4.46). Перепишем его с учетом (4.47) в виде


\begin{equation*}
\Delta E_{\text {соб }}=A_{\text {внеш }}+A_{\text {внутр }}^{\text {Дис }} \tag{4.49}
\end{equation*}


\begin{itemize}
  \item приращение собственной механической энергии системь равно алгебраической сумме работ всех внешних сил и всех внутренних диссипативных сил.
\end{itemize}

Уравнение (4.49) справедливо как в инерциальной, так и в неинерциальной системах отсчета. Следует только иметь в виду, что в неинерциальной системе отсчета необходимо учитывать и работу сил инерции, играющих роль внешних сил, т. е. под \(A_{\text {внеш }}\) надо понимать алгебраическую сумму работ внешних сил взаимодействия и работу сил инерции.

\section*{Закон сохранения механической энергии}
Этот закон непосредственно вытекает из уравнения (4.49) и формулируется так: механическая энергия замкнутой систе-

\footnotetext{\begin{itemize}
  \item В отличие от полной механической энергии, о которой речь пойдет ниже.
\end{itemize}
}мы частиц, в которой нет диссипативных сил, сохраняется в процессе движения, т. е.


\begin{equation*}
E_{\text {соб }}=K+U_{\text {соб }}=\text { const. } \tag{4.50}
\end{equation*}


Такую систему называют консервативной*. При движении замкнутой консервативной системы сохраняется именно механическая энергия \(E_{\text {соб }}\), кинетическая же и потенциальная в общем случае изменяются. Однако эти изменения происходят всегда так, что приращение одной из них в точности равно убыли другой: \(\Delta K=-\Delta U_{\text {соб }}\). Это положение справедливо только в инерциальных системах отсчета.

Из уравнения (4.49) следует, что если замкнутая система не консервативна, т. е. в ней имеются диссипативные силы, то механическая энергия такой системы, согласно (4.43), убывает:


\begin{equation*}
\Delta E_{\text {соб }}=A_{\text {внутр }}^{\text {дис }}<0 . \tag{4.51}
\end{equation*}


Уменьшение механической энергии обусловлено тем, что она расходуется на работу против диссипативных сил, действующих в системе. Однако такое объяснение является формальным, поскольку оно не раскрывает физической природы диссипативных сил.

Более глубокое осмысливание этого вопроса привело к фундаментальному выводу о существовании в природе универсального закона сохранения энергии: энергия никогда не создается и не уничтожается, она может только переходить из одной формы в другую или обмениваться между отдельными частями материи. При этом понятие энергии пришлось расширить введением понятий о новых формах ее (помимо механической) - энергия электромагнитного поля, химическая энергия, ядерная и др.

Универсальный закон сохранения энергии охватывает, таким образом, и те физические явления, на которые законы Ньютона не распространяются. Поэтому он не может быть выведен из этих законов, а должен рассматриваться как самостоя-

\footnotetext{\begin{itemize}
  \item С достаточно хорошим приближением замкнутой консервативной системой можно считать Солнечную систему.
\end{itemize}
}тельный закон, представляющий собой одно из наиболее широких обобщений опытных фактов.

При уменьшении механической энергии замкнутой системы всегда возникает эквивалентное количество энергии других видов, не связанных с видимым движением. В этом смысле уравнение (4.49) можно рассматривать как более общую формулировку закона сохранения энергии, в которой указана причина изменения механической энергии системы.

В частности, механическая энергия может сохраняться у незамкнутых систем, но это происходит лишь в тех случаях, когда, согласно уравнению (4.49), уменьшение этой энергии за счет работы против внутренних диссипативных сил компенсируется поступлением энергии за счет работы внешних сил.

\section*{Полная механическая энергия системы во внешнем поле}
Если интересующая нас система частиц находится во внешнем стационарном поле консервативных сил, то часто бывает удобно пользоваться другим выражением для механической энергии этой системы, отличным от (4.47).

В данном случае внешние силы, действующие на частицы системы, можно разделить на силы со стороны внешнего поля (внешние силь поля) и все остальные внешние силы, не относящиеся к данному внешнему полю (внешние сторонние силь). Соответственно работа \(A_{\text {внеш }}\) внешних сил может быть представлена как алгебраическая сумма работ внешних сил поля и внешних сторонних сил:

\[
A_{\text {внеш }}=A_{\text {внеш }}^{\text {сп }}+A_{\text {внеш }}^{\text {стор }} .
\]

Но работа внешних сил поля, в свою очередь, может быть представлена, согласно (4.39), как убыль внешней потенциальной энергии, а именно \(A_{\text {внеш }}^{\text {сп }}=-\Delta U_{\text {внеш }}\). Тогда

\[
A_{\text {внеш }}=-\Delta U_{\text {внеш }}+A_{\text {внеш }}^{\text {стор }} .
\]

Подставив это выражение в формулу (4.49), получим


\begin{equation*}
\Delta\left(E_{\text {соб }}+U_{\text {внеш }}\right)=A_{\text {внеш }}^{\text {стор }}+A_{\text {внутр }}^{\text {дис }} . \tag{4.52}
\end{equation*}


Величину, стоящую слева в скобках, называют полной механической энергией \(E\) системьь во внешнем стационарном поле консервативных сил:


\begin{equation*}
E=E_{\text {соб }}+U_{\text {внеш }} \tag{4.53}
\end{equation*}


В отличие от собственной механической энергии (4.47) полная механическая энергия включает в себя помимо суммарной кинетической и собственной потенциальной энергии еще и потенциальную энергию системы во внешнем поле \(U_{\text {внеш }}\).

С учетом (4.53) уравнение (4.52) можно переписать так:


\begin{equation*}
\Delta E=A_{\text {внеш }}^{\text {стор }}+A_{\text {внутр }}^{\text {дис }} . \tag{4.54}
\end{equation*}


Из уравнения (4.54) вытекает закон сохранения полной механической энергии системы, находящейся во внешнем стационарном поле консервативных сил: если на систему частии не действуют внешние сторонние силь и нет внутренних диссипативных сил, то полная механическая энергия системы остается постоянной:


\begin{equation*}
E=E_{\text {соб }}+U_{\text {внеш }}=\text { const. } \tag{4.55}
\end{equation*}


Простейшим примером подобной системы могут служить два небольших тела, соединенные друг с другом пружинкой (упругая гантель). Если эта система движется в поле тяжести в отсутствие сопротивления воздуха (т. е. нет внешних сторонних сил), то меняются ее кинетическая энергия \(K\), собственная потенциальная энергия \(U_{\text {соб }}\) и внешняя потенциальная энергия \(U_{\text {внеш }}\), однако алгебраическая сумма этих трех величин будет оставаться постоянной.

Другой пример - это система Земля-Луна в поле тяготения Солнца. В процессе движения этой системы также изменяются \(K, U_{\text {соб }}\) и \(U_{\text {внеш }}\), но их алгебраическая сумма сохраняется постоянной.

В заключение остается отметить, что уравнение (4.54) выполняется как в инерциальной, так и в неинерциальной системах отсчета, закон же сохранения полной механической энергии (4.55) - только в инерциальной.

\section*{Еще о роли новых понятий}
Для правильного понимания вопросов, связанных с изменением и сохранением механической энергии, необходимо еще раз обратить внимание на особую роль таких новых понятий как сторонние силь и собственная механическая энергия системы.

\begin{enumerate}
  \item Во многих случаях без введения понятия сторонних сил в принципе невозможно рассмотрение поведения частицы или системы с энергетической точки зрения. В каждой конкретной задаче следует четко уяснить себе какие силы являются сторонними, ибо работа именно сторонних сил определяет приращение механической энергии частицы в поле (4.31) и приращение полной механической энергии системы в поле (4.54), совмстно с работой внутренних диссипативных сил.
  \item Говоря о механической энергии системы, необходимо в каждом конкретном случае четко различать - о какой именно энергии идет речь: о собственной механической энергии \(E_{\text {соб }}\) или о полной механической энергии \(E\) во внешнем поле. Их приращения определяются разными формулами: (4.49) и (4.54) соответственно. Это разные энергии. Во втором случае (система во внешнем поле) энергия \(E\) включает в себя внешнюю потенциальную энергию в интересующем нас поле (4.53).
\end{enumerate}

К сожалению, собственную и полную энергию обычно не различают или путают, что, естественно, приводит к досадным недоразумениям и грубым ошибкам (даже в формулировке законов изменения и сохранения механической энергии). Приняв во внимание указанные предостережения, мы тем самым уже гарантируем себе корректный подход к решению соответствующих вопросов.

\section*{Связь между энергиями в \(\boldsymbol{K}\) - и Ц-системах отсчета}
Прежде всего установим эту связь для кинетических энергий системы. Пусть в \(K\)-системе отсчета кинетическая энергия интересующей нас системы частиц равна \(K\). Скорость \(i\)-й частицы можно представить как \(\mathbf{v}_{i}=\tilde{\mathbf{v}}_{i}+\mathbf{V}_{C}\), где \(\tilde{\mathbf{v}}_{i}\) - скорость этой частицы в Ц-системе, а \(\mathbf{V}_{C}\) - скорость Ц-системы относительно \(K\)-системы отсчета. Тогда кинетическая энергия системы

\[
K=\sum \frac{m_{i} v_{i}^{2}}{2}=\sum \frac{m_{i}\left(\tilde{\mathbf{v}}_{i}+\mathbf{V}_{C}\right)^{2}}{2}=\sum \frac{m_{i} \tilde{v}_{i}^{2}}{2}+\mathbf{V}_{C} \sum m_{i} \tilde{\mathbf{v}}_{i}+\sum \frac{m_{i} V_{C}^{2}}{2} .
\]

Так как в Ц-системе центр масс покоится, значит, согласно (3.9), \(\sum m_{i} \tilde{\mathbf{v}}_{i}=0\) и предыдущее выражение примет вид


\begin{equation*}
K=\tilde{K}+1 / 2 m V_{C}^{2}, \tag{4.56}
\end{equation*}


где \(\tilde{K}=\frac{1}{2} \sum m_{i} \tilde{v}_{i}^{2}\) — суммарная кинетическая энергия частиц в Ц-системе, \(m\) - масса всей системы.

Равенство (4.56) выражает теорему Кёнига: кинетическая энергия системь частиц складьвается из суммарной кинетической энергии \(\tilde{K}\) в Ц-системе и кинетической энергии, связанной с движением системь частии как целого. Это важный вывод, и он неоднократно будет использоваться в дальнейшем (в частности, при изучении динамики твердого тела).

Из формулы (4.56) следует, что кинетическая энергия системь частиц минимальна в Ц-системе. В этом еще одна особенность Ц-системы. Действительно, в Ц-системе \(\mathbf{V}_{C}=0\), поэтому в (4.56) остается только \(\tilde{K}\).

Перейдем к механической энергии \(E_{\text {соб }}\) системы. Так как собственная потенциальная энергия системы \(U_{\text {соб }}\) зависит только от конфигурации системы, то значение \(U_{\text {соб }}\) одинаково во всех системах отсчета. Добавив \(U_{\text {соб }}\) в левую и правую части равенства (4.56), получим формулу преобразования собственной механической энергии \(E_{\text {соб }}\) при переходе от \(K\) - к Ц-системе:


\begin{equation*}
E_{\text {соб }}=\tilde{E}+\frac{1}{2} m V_{C}^{2}, \tag{4.57}
\end{equation*}


где \(\tilde{E}=\tilde{K}+U_{\text {соб }}\). Эту энергию называют внутренней механической энергией системы.\\
Пример. На гладкой горизонтальной плоскости лежат две небольшие шайбы, каждая массы \(m\), которые соединены между собой невесомой пружинкой. Одной из шайб сообщили начальную скорость \(v_{0}\) (рис. 4.10, вид сверху). Найти внутреннюю механическую энергию \(\widetilde{E}\) этой системы в процессе движения.\\
Так как плоскость гладкая, то система в про-

\begin{figure}[h]
\begin{center}
  \includegraphics[alt={},max width=\textwidth]{66a35da9-3faa-4823-9ca9-40799f37b3db-126_289_203_1361_947}
\captionsetup{labelformat=empty}
\caption{Рис. 4.10}
\end{center}
\end{figure}

цессе движения будет вести себя как замкнутая. Поэтому ее механическая энергия \(E_{\text {соб }}\) и скорость \(\mathbf{V}_{C}\) центра масс будут сохраняться, оставаясь равными тем значе-

ниям, которые они имели в начальный момент: \(E_{\text {соб }}=m v_{0}^{2} / 2\) и \(V_{C}=v_{0} / 2\). Подставив эти значения в формулу (4.57), получим

\[
\tilde{E}=E_{\text {соб }}-2 m V_{C}^{2} / 2=m v_{0}^{2} / 4,
\]

где учтено, что масса системы равна \(2 m\). Внутренняя энергия \(\tilde{E}\) связана с вращением и колебанием данной системы, причем в начальный момент \(\tilde{E}\) была равна только энергии вращательного движения.

Если система частиц замкнута и в ней происходят процессы, связанные с изменением механической энергии, то из \((4.57)\) следует, что \(\Delta E_{\text {соб }}=\Delta \tilde{E}\), т. е. приращение собственной механической энергии относительно произвольной инерциальной системы отсчета равно приращению внутренней механической энергии. При этом кинетическая энергия, обусловленная движением системы частиц как целого, не меняется, так как для замкнутой системы \(\mathbf{V}_{C}=\) const.

В частности, если замкнутая система консервативна, то ее механическая энергия сохраняется во всех инерциальных системах отсчета. Этот вывод находится в полном соответствии с принципом относительности Галилея.

\section*{§ 4.6. Столкновение двух частиц}
\section*{Предварительные сведения}
Рассмотрим различные случаи столкновения двух частиц, используя в качестве инструмента исследования только законы сохранения импульса и энергии. При этом мы увидим, что законы сохранения позволяют сделать ряд весьма общих и существенных заключений о свойствах данного процесса вне какой-либо зависимости от конкретного закона взаимодействия частиц.

Попутно покажем, какие преимущества дает Ц-система, использование которой, как будет видно, значительно упрощает анализ процесса и многие расчеты.

Хотя мы будем говорить о столкновении частиц, необходимо сразу же оговорить, что все последующие рассуждения и выводы в равной степени относятся и к столкновению любых тел. Надо только иметь в виду, что вместо скорости частицы следует брать скорость центра масс каждого тела, а вместо ки-

нетической энергии частицы - ту часть кинетической энергии каждого тела, которая характеризует его движение как целого.

Прежде чем перейти к рассмотрению теории столкновений, приведем несколько важных и полезных соотношений для системы из двух частиц в ее Ц-системе отсчета.

В конце § 3.4 были получены выражения (3.12) для импульса каждой частицы в Ц-системе. Запишем эти выражения в такой форме:


\begin{equation*}
\tilde{\mathbf{p}}_{1}=\mu\left(\mathbf{v}_{1}-\mathbf{v}_{2}\right), \quad \tilde{\mathbf{p}}_{2}=\mu\left(\mathbf{v}_{2}-\mathbf{v}_{1}\right), \tag{4.58}
\end{equation*}


где \(\mathbf{v}_{1}\) и \(\mathbf{v}_{2}\) - скорости частиц в исходной системе отсчета, \(\mu\) приведенная масса системь,


\begin{equation*}
\mu=\frac{m_{1} m_{2}}{m_{1}+m_{2}}, \tag{4.59}
\end{equation*}


где \(m_{1}\) и \(m_{2}\) - массы частиц.\\
Из формул (4.58) видно, что импульсы обеих частиц в Ц-системе одинаковы по модулю и противоположны по направлению, причем модуль импульса каждой частицы


\begin{equation*}
\tilde{p}=\mu v_{\mathrm{oTH}}, \tag{4.60}
\end{equation*}


где \(v_{\text {отн }}=\left|\mathbf{v}_{1}-\mathbf{v}_{2}\right|-\) скорость одной частицы «относительно другой».

Теперь обратимся к кинетической энергии. Суммарная кинетическая энергия обеих частиц в Ц-системе

\[
\tilde{K}=\tilde{K}_{1}+\tilde{K}_{2}=\tilde{p}^{2} / 2 m_{1}+\tilde{p}^{2} / 2 m_{2} .
\]

Так как, согласно (4.59), \(1 / m_{1}+1 / m_{2}=1 / \mu\), то выражение для \(\tilde{K}\) примет следующий вид:


\begin{equation*}
\tilde{K}=\frac{\tilde{p}^{2}}{2 \mu}=\frac{\mu v_{\mathrm{OTH}}^{2}}{2} . \tag{4.61}
\end{equation*}


Если частицы взаимодействуют друг с другом, то механическая энергия частиц в Ц-системе

\[
\tilde{E}=\tilde{K}+U,
\]

где \(U\) - потенциальная энергия взаимодействия данных частиц.

В дальнейшем при рассмотрении столкновений частиц будем считать:

\begin{enumerate}
  \item исходная \(K\)-система отсчета инерциальная,
  \item система из двух частиц замкнутая,
  \item импульсы (и скорости) частиц до и после столкновения соответствуют достаточно большим расстояниям между ними; при этом потенциальной энергией взаимодействия можно просто пренебречь.
\end{enumerate}

Кроме того, величины, относящиеся к системе после столкновения, будем отмечать штрихом, а величины в Ц-системе значком \~{} (тильда) сверху.

Перейдем к существу вопроса. Различают три типа столкновения частиц: абсолютно неупругое, абсолютно упругое и промежуточный случай - неупругое.

\section*{Абсолютно неупругое столкновение}
Это такое столкновение, в результате которого обе частицы «слипаются» и далее движутся как единое целое. Пусть две частицы, массы которых \(m_{1}\) и \(m_{2}\), имеют до столкновения скорости \(\mathbf{v}_{1}\) и \(\mathbf{v}_{2}\) (в \(K\)-системе). После столкновения образуется частица с массой \(m_{1}+m_{2}\), что следует из аддитивности массы в ньютоновской механике. Скорость \(\mathbf{v}^{\prime}\) образовавшейся частицы можно найти из закона сохранения импульса:

\[
\left(m_{1}+m_{2}\right) \mathbf{v}^{\prime}=m_{1} \mathbf{v}_{1}+m_{2} \mathbf{v}_{2} .
\]

Ясно, что скорость \(\mathbf{v}^{\prime}\) равна скорости центра масс системы.\\
В Ц-системе этот процесс выглядит наиболее просто: до столкновения обе частицы движутся навстречу друг другу с одинаковыми импульсами \(\tilde{p}\), а после столкновения образовавшаяся частица оказывается неподвижной. При этом суммарная кинетическая энергия \(\tilde{K}\) частиц целиком переходит во внутреннюю энергию \(Q\) образовавшейся частицы, т. е. \(\tilde{K}=Q\). Отсюда с учетом формулы (4.61) найдем


\begin{equation*}
Q=\frac{\mu v_{\mathrm{oTH}}^{2}}{2}=\frac{1}{2} \frac{m_{1} m_{2}}{m_{1}+m_{2}}\left(v_{1}-v_{2}\right)^{2} . \tag{4.62}
\end{equation*}


Таким образом, величина \(Q\) для данной пары частиц зависит только от их относительной скорости.

\section*{Абсолютно упругое столкновение}
Это такое столкновение, в результате которого внутренняя энергия частиц не меняется, а поэтому не меняется и кинетическая энергия системы. Рассмотрим два частных случая: лобовое и нелобовое упругие столкновения.

\begin{enumerate}
  \item Лобовое столкновение - обе частицы до и после столкновения движутся по одной и той же прямой. Пусть до столкновения скорости частиц в \(K\)-системе отсчета равны \(\mathbf{v}_{1}\) и \(\mathbf{v}_{2}\) (частицы движутся или навстречу друг другу, или одна частица догоняет другую). Каковы скорости этих частиц после столкновения?
\end{enumerate}

Рассмотрим этот процесс сначала в Ц-системе, где до и после столкновения обе частицы имеют одинаковые по модулю и противоположные по направлению импульсы (рис. 4.11). Более

\begin{figure}[h]
\begin{center}
  \includegraphics[alt={},max width=\textwidth]{66a35da9-3faa-4823-9ca9-40799f37b3db-130_158_511_760_653}
\captionsetup{labelformat=empty}
\caption{Рис. 4.11}
\end{center}
\end{figure}

того, так как суммарная кинетическая энергия частиц до и после столкновения одинакова, как и их приведенная масса, то, согласно (4.61), импульс каждой частицы в результате столкновения изменит только направление на противоположное, не меняясь при этом по модулю, т. е. \(\tilde{\mathbf{p}}_{i}^{\prime}=-\tilde{\mathbf{p}}_{i}\), где \(i=1,2\). Последнее относится и к скорости каждой частицы в Ц-системе:

\[
\tilde{\mathbf{v}}_{i}^{\prime}=-\tilde{\mathbf{v}}_{i} .
\]

Теперь найдем скорость каждой частицы после столкновения в \(K\)-системе отсчета. Для этого используем формулы преобразования скоростей при переходе от Ц- \(\kappa K\)-системе, а также предыдущее равенство. Тогда

\[
\mathbf{v}_{i}^{\prime}=\mathbf{V}_{C}+\tilde{\mathbf{v}}_{i}^{\prime}=V_{C}-\tilde{\mathbf{v}}_{i}=\mathbf{V}_{C}-\left(\mathbf{v}_{i}-\mathbf{V}_{C}\right)=2 \mathbf{V}_{C}-\mathbf{v}_{i},
\]

где \(\mathbf{V}_{c}\) - скорость центра масс ( Ц-системы) в \(K\)-системе отсчета; эта скорость определяется формулой (3.9). Итак, скорость \(i\)-й частицы в \(K\)-системе после столкновения


\begin{equation*}
\mathbf{v}_{i}^{\prime}=2 \mathbf{V}_{C}-\mathbf{v}_{i}, \tag{4.63}
\end{equation*}


где \(i=1\), 2. В проекциях на произвольную ось \(x\) это равенство имеет вид


\begin{equation*}
v_{i x}^{\prime}=2 V_{C x}-v_{i x} . \tag{4.64}
\end{equation*}


В частности, если массы частиц одинаковы, то легко убедиться, что частицы в результате столкновения просто обмениваются скоростями, т. е. \(\mathbf{v}_{1}^{\prime}=\mathbf{v}_{2}\) и \(\mathbf{v}_{2}^{\prime}=\mathbf{v}_{1}\).\\
2. Нелобовое столкновение. Ограничимся случаем, когда одна из частиц покоится до столкновения. Пусть в \(K\)-системе отсчета частица массы \(m_{1}\) с импульсом \(\mathbf{p}_{1}\) испытала упругое нелобовое столкновение с покоившейся частицей массы \(m_{2}\). Каковы возможные импульсы этих частиц после столкновения?

Рассмотрим этот процесс также сначала в Ц-системе. Здесь, как и в предыдущем случае, обе частицы в любой момент времени до и после столкновения имеют одинаковые по модулю и противоположные по направлению импульсы. Кроме того, импульс каждой частицы не изменится по модулю в результате столкновения:

\[
\tilde{p}^{\prime}=\tilde{p} .
\]

Однако направление разлета ча-

\begin{figure}[h]
\begin{center}
  \includegraphics[alt={},max width=\textwidth]{66a35da9-3faa-4823-9ca9-40799f37b3db-131_217_469_1072_66}
\captionsetup{labelformat=empty}
\caption{Рис. 4.12}
\end{center}
\end{figure}

стиц теперь будет иным. Оно будет составлять с первоначальным направлением движения некоторый угол э (рис. 4.12), зависящий от закона взаимодействия частиц и их взаимного расположения в процессе столкновения.

Найдем импульс каждой частицы в \(K\)-системе отсчета после столкновения. С помощью формул преобразования скоростей при переходе от Ц- к \(K\)-системе получим:


\begin{align*}
& \mathbf{p}_{1}^{\prime}=m_{1} \mathbf{v}_{1}^{\prime}=m_{1}\left(\mathbf{V}_{C}+\tilde{\mathbf{v}}_{1}^{\prime}\right)=m_{1} \mathbf{V}_{C}+\tilde{\mathbf{p}}_{1}^{\prime}, \\
& \mathbf{p}_{2}^{\prime}=m_{2} \mathbf{v}_{2}^{\prime}=m_{2}\left(\mathbf{V}_{C}+\tilde{\mathbf{v}}_{2}^{\prime}\right)=m_{2} \mathbf{V}_{C}+\tilde{\mathbf{p}}_{2}^{\prime}, \tag{4.65}
\end{align*}


где \(\mathbf{V}_{C}\) - скорость Ц-системы относительно \(K\)-системы отсчета.

Сложив отдельно левые и правые части этих равенств с учетом того, что \(\tilde{\mathbf{p}}_{1}^{\prime}=-\tilde{\mathbf{p}}_{2}^{\prime}\), получим

\[
\mathbf{p}_{1}^{\prime}+\mathbf{p}_{2}^{\prime}=\left(m_{1}+m_{2}\right) \mathbf{V}_{C}=\mathbf{p}_{1},
\]

как и должно быть в соответствии с законом сохранения импульса.\\
Построим теперь векторную диаграмму импульсов. Сначала изобразим вектор \(\mathbf{p}_{1}\) отрезком \(A B\) (рис. 4.13), затем векторы \(\mathbf{p}_{1}^{\prime}\) и \(\mathbf{p}_{2}^{\prime}\), каждый из которых представляет собой, согласно (4.65), сумму двух векторов.

Заметим, что это построение справедливо вне зависимости от угла Э. Отсюда следует, что точка \(C\) (рис. 4.13) может находиться только на окружности радиуса \(\tilde{p}\) с центром в точке \(O\), которая делит отрезок \(A B\) на две части в отношении \(A O: O B=m_{1}: m_{2}\). Более того, в рассматриваемом случае (частица массы

\begin{figure}[h]
\begin{center}
  \includegraphics[alt={},max width=\textwidth]{66a35da9-3faa-4823-9ca9-40799f37b3db-132_274_416_584_738}
\captionsetup{labelformat=empty}
\caption{Рис. 4.13}
\end{center}
\end{figure}

\(m_{2}\) покоится до столкновения) эта окружность проходит через точку \(B\) - конец вектора \(\mathbf{p}_{1}\), так как \(O B=\tilde{p}\). Действительно,

\[
O B=m_{2} V_{C}=m_{2} \frac{m_{1} v_{1}}{m_{1}+m_{2}},
\]

где \(v_{1}\) - скорость налетающей частицы. А так как в нашем случае \(v_{1}=v_{\text {отн }}\), то, согласно (4.59) и (4.60),

\[
O B=\mu v_{\text {отн }}=\tilde{p}
\]

Таким образом, для построения векторной диаграммы импульсов, соответствующей упругому столкновению двух частиц (одна из которых первоначально покоилась), необходимо:

\begin{enumerate}
  \item изобразить отрезок \(A B\), равный импульсу \(p_{1}\) налетающей частицы;
  \item через точку \(B\) - конец вектора \(\mathbf{p}_{1}\) - провести окружность радиуса
\end{enumerate}


\begin{equation*}
\tilde{p}=\mu v_{\text {отн }}=\frac{m_{2}}{m_{1}+m_{2}} p_{1}, \tag{4.66}
\end{equation*}


центр которой - точка \(O\) - делит отрезок \(A B\) на две части в отношении \(A O: O B=m_{1}: m_{2}\).

Эта окружность есть геометрическое место точек всех возможных положений вершины \(C\) треугольника импульсов \(A B C\), стороны \(A C\) и \(C B\) которого и представляют собой возможные импульсы частиц после столкновения (в \(K\)-системе отсчета).

В зависимости от соотношения масс частиц точка \(A\) - начало вектора \(\mathbf{p}_{1}\) - может находиться внутри данной окружности, на ней или снаружи (рис. 4.14, \(a\), б, в). При этом во всех трех случаях угол \(\tilde{\vartheta}\) может принимать все значения от 0 до \(\pi\). Возможные значения угла рассеяния налетающей частицы \(\vartheta_{1}\) и угла разлета частиц \(\theta\) будут такими:\\
a) \(m_{1}<m_{2} \quad 0<\vartheta_{1} \leqslant \pi \quad \theta>\pi / 2\),\\
б) \(m_{1}=m_{2} \quad 0<\vartheta_{1} \leqslant \pi / 2 \quad \theta=\pi / 2\),\\
в) \(m_{1}>m_{2} \quad 0<\vartheta_{1} \leqslant \vartheta_{1 \text { макс }} \theta<\pi / 2\), где \(\vartheta_{1 \text { макс }}\) - предельный угол. Он определяется формулой


\begin{equation*}
\sin \vartheta_{1 \text { макс }}=m_{2} / m_{1}, \tag{4.67}
\end{equation*}


так как (см. рис. 4.14, в)

\[
\sin \vartheta_{1 \text { макс }}=O C^{\prime} / A O=O B / A O=m_{2} / m_{1} .
\]

Кроме того, обнаруживается еще один интересный факт. В последнем случае ( \(m_{1}>m_{2}\) ) под одним и тем же углом \(\vartheta_{1}\) возможно рассеяние частицы \(m_{1}\) как с импульсом \(A C\), так и с импульсом \(A D\) (рис. 4.14, в), т. е. в этом случае решение неоднозначно. Аналогично обстоит дело и с частицей \(m_{2}\).

И наконец, из той же векторной диаграммы импульсов можно найти связь между углами \(\vartheta_{1}\) и \(\tilde{\vartheta}\).

Этим исчерпываются сведения, которые можно получить о данном процессе, исходя из одних только законов сохранения импульса и энергии.

\begin{figure}[h]
\begin{center}
  \includegraphics[alt={},max width=\textwidth]{66a35da9-3faa-4823-9ca9-40799f37b3db-133_293_1116_1472_58}
\captionsetup{labelformat=empty}
\caption{Рис. 4.14}
\end{center}
\end{figure}

Мы видим, таким образом, что уже сами по себе законы сохранения импульса и энергии действительно позволяют сделать ряд важных заключений о свойствах рассматриваемого процесса. При этом особенно существен тот факт, что эти свойства имеют общий характер, т. е. не зависят от рода взаимодействия частиц.

Следует, однако, обратить внимание на одно принципиальное обстоятельство. Векторная диаграмма импульсов, в основе которой лежат законы сохранения импульса и энергии, давая нам полную картину всех возможных случаев разлета частиц после столкновения - результат сам по себе весьма существенный, - совершенно не говорит о том, какой из этих возможных случаев реализуется конкретно. Для установления этого необходимо обратиться к более детальному рассмотрению процесса столкновения с помощью уравнений движения. При этом выясняется, например, что угол рассеяния \(\vartheta_{1}\) налетающей частицы зависит от характера взаимодействия сталкивающихся частиц и от прицельного параметра*, неоднозначность же решения в случае \(m_{1}>m_{2}\) объясняется тем, что один и тот же угол рассеяния \(\vartheta_{1}\) может реализоваться при двух значениях прицельного параметра, причем независимо от закона взаимодействия частиц.

Указанное обстоятельство представляет собой очень характерную и принципиальную черту законов сохранения вообще. Законы сохранения никогда не дают и не могут дать однозначного ответа на вопрос о том, чтó произойдет. Но если, исходя из каких-либо других соображений, можно указать, что́ именно должно произойти, то законы сохранения дают ответ на вопрос, как это должно произойти.

\section*{Неупругое столкновение}
Это такое столкновение, в результате которого внутренняя энергия разлетающихся частиц (или одной из них) изменяется, а следовательно, изменяется и суммарная кинетическая энергия системы. Соответствующее приращение кинетической энергии системы принято обозначать \(Q\). В зависимости от знака \(Q\) неупругое столкновение называ-

\footnotetext{\begin{itemize}
  \item Прицельный параметр - это расстояние между прямой, вдоль которой направлен импульс налетающей частицы, и частицей, с которой происходит «столкновение».
\end{itemize}
}ют экзоэнергетическим ( \(Q>0\) ) или эндоэнергетическим ( \(Q<0\) ). В первом случае кинетическая энергия системы увеличивается, во втором - уменьшается. При упругом столкновении, разумеется, \(Q=0\).

Наша задача: найти возможные импульсы частиц после неупругого столкновения.

Этот вопрос наиболее просто решается в Ц-системе. Согласно условию, приращение суммарной кинетической энергии системы в данном процессе


\begin{equation*}
\tilde{K}^{\prime}-\tilde{K}=Q . \tag{4.68}
\end{equation*}


Так как \(\tilde{K}^{\prime} \neq \tilde{K}\), то, согласно (4.61), импульсы частиц после столкновения изменятся по модулю. Импульс каждой частицы после столкновения \(\tilde{p}^{\prime}\) легко найти, заменив \(\tilde{K}^{\prime}\) в (4.68) его выражением \(\tilde{K}^{\prime}=\tilde{p}^{\prime 2} / 2 \mu\). В результате получим


\begin{equation*}
\tilde{p}^{\prime}=\sqrt{2 \mu(\tilde{K}+Q)} . \tag{4.69}
\end{equation*}


\begin{figure}[h]
\begin{center}
  \includegraphics[alt={},max width=\textwidth]{66a35da9-3faa-4823-9ca9-40799f37b3db-135_212_411_843_68}
\captionsetup{labelformat=empty}
\caption{Рис. 4.15}
\end{center}
\end{figure}

Теперь рассмотрим тот же вопрос в \(K\)-системе отсчета, где частица массы \(m_{1}\) с импульсом \(\mathbf{p}_{1}\) испытывает столкновение с покоящейся частицей массы \(m_{2}\). Для определения возможных случаев разлета частиц после столкновения здесь также полезно воспользоваться векторной диаграммой импульсов. Ее построение аналогично тому, как это было сделано для упругого столкновения. Импульс налетающей частицы \(p_{1}=A B\) (рис.4.15) делят точкой \(O\) на две части, пропорциональные массам частиц ( \(A O: O B= m_{1}: m_{2}\) ). Затем из точки \(O\) проводят окружность радиуса \(\tilde{p}^{\prime}\) [см. (4.69)]. Эта окружность является геометрическим местом точек возможных положений вершины \(C\) треугольника импульсов \(A B C\), стороны \(A C\) и \(C B\) которого равны импульсам соответствующих частиц после столкновения.

Отметим, что теперь в отличие от упругого столкновения точка \(B\) (конец вектора \(\mathbf{p}_{1}\) ) не лежит на окружности, а именно: при \(Q>0\) эта точка находится внутри окружности, а при \(Q<0\) - вне ее. Рис. 4.15 соответствует последнему случаю - эндоэнергетическому столкновению.

Порог. Есть много неупругих столкновений, в которых внутренняя энергия частиц способна изменяться только на совершенно определенную величину, зависящую от свойств самих частиц (таковы, например, неупругие столкновения атомов и молекул). Несмотря на это, экзоэнергетические столкновения ( \(Q>0\) ) могут происходить при сколь

угодно малой кинетической энергии налетающей частицы. Эндоэнергетические же процессы ( \(Q<0\) ) в таких случаях обладают порогом. Порогом называют минимальную кинетическую энергию налетающей частицы, начиная с которой данный процесс становится энергетически возможным.

Итак, пусть нам необходимо осуществить такое эндоэнергетическое столкновение, в котором внутренняя энергия частиц способна получить приращение не меньше некоторого значения \(|Q|\). При каком условии такой процесс возможен?

Этот вопрос наиболее просто решается также в Ц-системе, где ясно, что суммарная кинетическая энергия \(\tilde{K}\) частиц до столкновения во всяком случае должна быть не меньше \(|Q|\), т. е. \(\tilde{K} \geqslant|Q|\). Отсюда следует, что существует минимальное значение \(\tilde{K}_{\text {мин }}=|Q|\), при котором кинетическая энергия системы целиком пойдет на увеличение внутренней энергии частиц, и частицы после столкновения остановятся в Ц-системе.

Рассмотрим этот же процесс в \(K\)-системе отсчета, где частица массы \(m_{1}\) налетает на покоящуюся частицу массы \(m_{2}\). Так как в Ц-системе при \(\tilde{K}_{\text {мин }}\) частицы после столкновения останавливаются, то это значит, что в \(K\)-системе при соответствующей пороговой кинетической энергии \(K_{1 п о р}\) налетающей частицы обе частицы после столкновения будут двигаться как единое целое, причем с суммарным импульсом, равным импульсу \(\mathbf{p}_{1}\) налетающей частицы, и кинетической энергией \(p_{1}^{2} / 2\left(m_{1}+m_{2}\right)\). Поэтому

\[
K_{1 \text { пор }}=|Q|+\frac{p_{1}^{2}}{2\left(m_{1}+m_{2}\right)} .
\]

А так как \(K_{1 \text { пор }}=p_{1}^{2} / 2 m_{1}\), то, исключив \(p_{1}^{2}\) из этих двух уравнений, получим


\begin{equation*}
K_{1 \text { пор }}=\frac{m_{1}+m_{2}}{m_{2}}|Q| . \tag{4.70}
\end{equation*}


Это и есть та пороговая кинетическая энергия налетающей частицы, начиная с которой данный эндоэнергетический процесс становится энергетически возможным.

Заметим, что формула (4.70) играет большую роль особенно в атомной и ядерной физике. С ее помощью определяют как порог различных эндоэнергетических процессов, так и соответствующую им энергию \(|Q|\).

\section*{§ 4.7. Механика несжимаемой жидкости}
В этом разделе механики изучают законы движения жидкости как сплошной (непрерывной) среды. Плотность жидкости практически не зависит от давления, поэтому жидкость будем считать несжимаемой средой, плотность которой везде одинакова.

Для кинематического описания движения (течения) жидкости обычно используют метод Эйлера: в интересующей нас системе отсчета задается поле скоростей жидкости, т. е. зависимость скорости \(\mathbf{v}\) каждой точки жидкости от ее радиуса-вектора \(\mathbf{r}\) и времени \(t\).

Во многих случаях, когда сила трения между отдельными слоями текущей жидкости пренебрежимо мала, жидкость можно считать идеальной (без внутреннего трения).

\section*{Линии и трубки тока}
Мысленно проведем в движущейся жидкости линии так, чтобы касательные к ним в каждой точке совпадали по направлению с вектором \(\mathbf{v}\). Эти линии называют линиями тока. Их проводят так, чтобы густота, т. е. число линий, пронизывающих единичную площадку, перпендикулярную линиям в данной точке, была пропорциональна модулю вектора \(\mathbf{v}\). Кроме того, этим линиям приписывают направление, совпадающее с направлением вектора \(\mathbf{v}\). Полученная картина дает наглядное представление о направлении и модуле вектора \(\mathbf{v}\) в разных точках жидкости, т. е. о характере ее движения. Там, где скорость больше, линии тока гуще, и наоборот (это доказывается ниже).

При стационарном течении, когда \(\mathbf{v}\) не зависит от \(t\), картина линий тока остается неизменной и линии тока совпадают с траекториями частиц жидкости.

Поверхность, образованная линиями тока, которые проведены через все точки замкнутого контура, называют трубкой тока. При стационарном течении жидкости ее частицы при своем движении не пересекают трубку тока.

\section*{Уравнение неразрывности струи}
Рассмотрим жидкость, текущую внутри некоторой трубки тока - такой, что скорость движения частиц жидкости одинакова во всех точках произвольного сечения данной трубки. Тог-

да за промежуток времени \(\Delta t\) сквозь сечение площади \(S\) пройдет объем жидкости \(S v \Delta t\). Поскольку жидкость несжимаема, масса жидкости между сечениями \(\quad S_{1}\) и \(\quad S_{2}\) трубки тока (рис. 4.16) будет оставаться неизменной. Значит, объем жидкости, протекающей сквозь сечения \(S_{1}\) и \(S_{2}\) за

\begin{figure}[h]
\begin{center}
  \includegraphics[alt={},max width=\textwidth]{66a35da9-3faa-4823-9ca9-40799f37b3db-138_285_432_145_728}
\captionsetup{labelformat=empty}
\caption{Рис. 4.16}
\end{center}
\end{figure}

время \(\Delta t\), должен быть одинаковым. Отсюда следует, что \(S_{1} v_{1}=S_{2} v_{2}\). Другими словами, для несжимаемой жидкости величина \(S v\) в любом сечении одной и той же трубки тока одинакова:


\begin{equation*}
S v=\text { const. } \tag{4.71}
\end{equation*}


Это соотношение называют уравнением неразрььности струи.

\section*{Уравнение Бернулли}
Рассмотрим стационарное течение идеальной жидкости в однородном поле сил тяжести. Покажем, что с помощью энергетического подхода можно получить весьма важное соотношение между некоторыми параметрами текущей жидкости.

С этой целью выделим мысленно часть жидкости, которая в момент \(t\) заполняет объем узкой трубки тока между нормальными сечениями 1 и 2 (рис. 4.17). \(К\) моменту \(t+\Delta t\) эта часть жидкости переместится вдоль трубки тока в направлении сечения, показанном на рисунке двойной стрелкой, и окажется между сечениями \(1^{\prime}\) и \(2^{\prime}\).

\begin{figure}[h]
\begin{center}
  \includegraphics[alt={},max width=\textwidth]{66a35da9-3faa-4823-9ca9-40799f37b3db-138_312_483_1139_669}
\captionsetup{labelformat=empty}
\caption{Рис. 4.17}
\end{center}
\end{figure}

Согласно уравнению (4.54), приращение полной механической энергии этой части жидкости за время \(\Delta t\)


\begin{equation*}
\Delta E=A_{\text {внеш }}^{\text {стор }}, \tag{4.72}
\end{equation*}


где \(A_{\text {внеш }}^{\text {стор }}\) - работа, которую совершают силы давления (они и есть в данном случае сторонние). При этом силы давления, пер-

пендикулярные выделенной трубке, работы не совершают. Работу будут совершать только силы давления, действующие в сечениях 1 и 2. Эта работа равна

\[
A_{\text {внеш }}^{\text {стор }}=p_{1} S_{1} \Delta l_{1}-p_{2} S_{2} \Delta l_{2} .
\]

В силу неразрывности струи \(S_{1} \Delta l_{1}=S_{2} \Delta l_{2}=\Delta V\), т. е. объемы между сечениями \(1-1^{\prime}\) и \(2-2^{\prime}\) одинаковы и


\begin{equation*}
A_{\text {внеш }}^{\text {стор }}=\left(p_{1}-p_{2}\right) \Delta V . \tag{4.73}
\end{equation*}


Течение жидкости стационарно, поэтому полная механическая энергия части жидкости трубки тока между сечениями \(1^{\prime}\) и 2 не меняется. Значит, приращение энергии \(\Delta E\) рассматриваемой части жидкости можно представить как разность энергий элементов \(2-2^{\prime}\) и \(1-1^{\prime}\) :


\begin{equation*}
\Delta E=\left(1 / 2 \rho \Delta V v_{2}^{2}+\rho \Delta V g h_{2}\right)-\left(1 / 2 \rho \Delta V v_{1}^{2}+\rho \Delta V g h_{1}\right), \tag{4.74}
\end{equation*}


где \(\rho\) - плотность жидкости.\\
Приравняв, согласно (4.72), выражения (4.73) и (4.74), получим после сокращения на \(\Delta V\) и перегруппировки слагаемых следующее уравнение:

\[
\rho v_{1}^{2} / 2+p_{1}+\rho g h_{1}=\rho v_{2}^{2} / 2+p_{2}+\rho g h_{2} .
\]

Так как сечения 1 и 2 взяты произвольно, то


\begin{equation*}
\rho v^{2} / 2+p+\rho g h=\text { const } \tag{4.75}
\end{equation*}


где все величины относятся, по существу, к одной и той же линии тока. Для разных линий тока эта константа, вообще говоря, будет своей.

Уравнение (4.75) называют уравнением Бернулли. Несмотря на то, что это уравнение получено для идеальной жидкости, оно достаточно хорошо выполняется и для реальных жидкостей, внутреннее трение которых (вязкость) сравнительно мало.

Рассмотрим два примера на применение уравнения Бернулли.

Пример 1. В случае горизонтальной линии тока величина \(h\) одинакова и уравнение Бернулли примет вид


\begin{equation*}
\rho v^{2} / 2+p=\text { const }, \tag{4.76}
\end{equation*}


т. е. давление больше в тех точках, где скорость меньше. Скажем, в горизонтальной трубке с изменяющимся вдоль оси сечением (рис. 4.18) скорость \(v\) течения, согласно \((4.71)\), в сечении \(S_{2}\), будет больше, чем в сечении \(S_{1}\), а значит, давление \(p\) слева больше. Именно перепад давления и будет создавать ускорение а жидкости, которое направлено вправо. Заметим, что ход графиков \(v, p\) и направление вектора а не зависят от направления движения жидкости.

\begin{figure}[h]
\begin{center}
  \includegraphics[alt={},max width=\textwidth]{66a35da9-3faa-4823-9ca9-40799f37b3db-140_306_332_639_298}
\captionsetup{labelformat=empty}
\caption{Рис. 4.18}
\end{center}
\end{figure}

\begin{figure}[h]
\begin{center}
  \includegraphics[alt={},max width=\textwidth]{66a35da9-3faa-4823-9ca9-40799f37b3db-140_311_372_636_739}
\captionsetup{labelformat=empty}
\caption{Рис. 4.19}
\end{center}
\end{figure}

Пример 2. Рассмотрим истечение идеальной жидкости через малое отверстие 2 в боковой стенке или дне широкого открытого сосуда (рис. 4.19). Все линии тока проходят через отверстие, начинаясь у свободной поверхности жидкости, где скорость пренебережимо мала (сосуд широкий). Поэтому постоянная в законе Бернулли будет одна и та же у всех линий тока. Применим уравнение (4.75), например, к линии тока \(1-2\) и будем отсчитывать \(h\) от уровня, на котором находится малое отверстие 2. Тогда

\[
\rho g h+p_{0}=\rho v^{2} / 2+p_{0},
\]

где \(p_{0}\) - атмосферное давление, \(v\) - скорость течения в точке 2. Отсюда


\begin{equation*}
v=\sqrt{2 g h} . \tag{4.77}
\end{equation*}


Это формула Торричелли. Видно, что при истечении через малое отверстие жидкость имеет скорость, которую приобретает тело, свободно падающее с высоты \(h\). Для реальных жидкостей скорость истечения будет меньше из-за вязкости жидкости.

\section*{Вязкость}
Всем реальным жидкостям в той или иной степени присуще внутреннее трение, или вязкость. Рассмотрим движение жидкости, скорость \(v\) отдельных слоев которой зависит только от поперечной координаты \(z\) (рис. 4.20). Обобщение результатов опыта приводит к выводу, что сила трения \(F_{\text {тр }}\), действующая между слоями движущейся жидкости, может быть представлена следующей формулой:


\begin{equation*}
F_{\mathrm{Tp}}=\eta|\partial v / \partial z| S, \tag{4.78}
\end{equation*}


где \(\eta\) - коэффициент внутреннего трения, или вязкость, - величина, зависящая от природы и состояния жидкости (например, от температуры); \(\partial v / \partial z\) - градиент модуля скорости (он характеризует крутизну графика зависимости \(v\) от \(z\) ); \(S\) - площадь интересующей нас поверхности раздела между слоями (эта поверхность перпендикулярна оси \(Z\) ).

Формула (4.78) означает, что жидкость, находящаяся над поверхностью \(S\) (рис. 4.20), действу-

\begin{figure}[h]
\begin{center}
  \includegraphics[alt={},max width=\textwidth]{66a35da9-3faa-4823-9ca9-40799f37b3db-141_270_429_947_68}
\captionsetup{labelformat=empty}
\caption{Рис. 4.20}
\end{center}
\end{figure}

ет на жидкость под поверхностью \(S\) с силой \(F_{\text {тр }}\) (вправо), а нижняя часть жидкости действует на верхнюю часть с той же по модулю силой \(F_{\text {тр }}\), но влево.

Из рис. 4.20 видно, что сила вязкости вдоль поверхности \(S\), проведенной через точку 1 , будет меньше, чем через точку 2 , поскольку крутизна графика \(v(z)\), а значит, и градиент \(\partial v / \partial z\) в точке 2 больше.

Приведем значения вязкости \(\eta\) для некоторых жидкостей при комнатной температуре ( \(20^{\circ} \mathrm{C}\) ), мПа•с:

\begin{center}
\begin{tabular}{ll}
вода . . . . . . 1, 0 & масло касторовое. . . . 1, \(0 \cdot 10^{3}\) \\
глицерин . . . \(8,5 \cdot 10^{2}\) & ртуть . . . . . . . . 1,6 \\
\end{tabular}
\end{center}

Ламинарное и турбулентное течения. Особенностью ламинарного (слоистого) течения является его регулярность. Например, при ламинарном течении в прямолинейной трубе частицы жидкости движутся вдоль прямолинейных траекторий, параллельных оси трубы. Однако при достаточно больших скоростях ламинарное течение становится неустойчивым и переходит в

турбулентное, при котором частицы жидкости совершают нерегулярные беспорядочные движения, что приводит к интенсивному перемешиванию между слоями движущейся жидкости.

Такие быстрые и нерегулярные изменения происходят вследствие неустойчивости ламинарных течений при определенных условиях. Характер течения зависит от значения безразмерной величины - числа Рейнольдса:


\begin{equation*}
\operatorname{Re}=\rho v l / \eta, \tag{4.79}
\end{equation*}


где \(\rho\) - плотность жидкости, \(v\) - характерная скорость потока, \(l\) - характерный размер, \(\eta\) - вязкость.

При малых значениях числа Re наблюдается ламинарное течение. Но, начиная с некоторого критического значения Re , ламинарное течение переходит в турбулентное. Если в качестве характерного размера, например для круглой трубы, взять ее радиус, то критическое значение \(\operatorname{Re} \approx 1000\) (для воды).

\section*{Течение жидкости в трубе круглого сечения}
Пусть вязкая жидкость течет в прямой трубе радиуса \(R\). Течение стационарное, линии тока параллельны оси трубы. Скорость жидкости равна нулю у стенок трубы и максимальна на ее оси. Найдем сначала зависимость скорости жидкости от расстояния до оси, т. е. \(v(r)\).

Выделим мысленно цилиндрический объем жидкости радиуса \(r\) и длины \(l\) (рис. 4.21). Поскольку все элементы жидкости движутся без ускорения, сумма всех внешних сил, приложенных к любому объему жидкости, равна нулю. На боковую по-

\begin{figure}[h]
\begin{center}
  \includegraphics[alt={},max width=\textwidth]{66a35da9-3faa-4823-9ca9-40799f37b3db-142_170_416_1198_744}
\captionsetup{labelformat=empty}
\caption{Рис. 4.21}
\end{center}
\end{figure}

верхность выделенного цилиндра действует сила трения, равная \(\eta|\mathrm{d} v / \mathrm{d} r| 2 \pi r l\). На основания этого цилиндра - силы давления, алгебраическая сумма которых равна \(\left(p_{1}-p_{2}\right) \pi r^{2}\).

Течение стационарное, поэтому эти две взаимно противоположные силы должны быть равны по модулю друг другу:


\begin{equation*}
\eta|\mathrm{d} v / \mathrm{d} r| 2 \pi r l=\left(p_{1}-p_{2}\right) \pi r^{2} . \tag{4.80}
\end{equation*}


Скорость \(v\) уменьшается с ростом расстояния \(r\) от оси трубы, поэтому \(\mathrm{d} v / \mathrm{d} r<0\) и \(|\mathrm{d} v / \mathrm{d} r|=-\mathrm{d} v / \mathrm{d} r\). Тогда уравнение (4.80) можно переписать так:

\[
-\mathrm{d} v=\frac{p_{1}-p_{2}}{2 \eta l} r \mathrm{~d} r .
\]

Интегрируем его с учетом того, что у стенок \((r=R) v=0\) :

\[
-\int_{v}^{0} \mathrm{~d} v=\frac{p_{1}-p_{2}}{2 \eta l} \int_{r}^{R} r \mathrm{~d} r
\]

В результате получим:


\begin{equation*}
v=\frac{p_{1}-p_{2}}{4 \eta l}\left(R^{2}-r^{2}\right) . \tag{4.81}
\end{equation*}


График этой зависимости \(v(r)\) показан на рис. 4.22 , где

\[
v_{0}=\frac{p_{1}-p_{2}}{4 \eta l} R^{2}
\]

\begin{figure}[h]
\begin{center}
  \includegraphics[alt={},max width=\textwidth]{66a35da9-3faa-4823-9ca9-40799f37b3db-143_217_527_1060_68}
\captionsetup{labelformat=empty}
\caption{Рис. 4.22}
\end{center}
\end{figure}

Теперь найдем поток \(Q\) жидкости, т. е. объем жидкости, протекающей через поперечное сечение трубы за единицу времени ( \(\mathrm{m}^{3} / \mathrm{c}\) ). Объем жидкости, ежесекундно протекающей через элементарное кольцо радиуса \((r, r+\mathrm{d} r), \quad \mathrm{d} Q=v \cdot 2 \pi r \mathrm{~d} r\). Подставив сюда выражение \(v(r)\) из формулы (4.81) и проинтегрировав по \(r\), найдем


\begin{equation*}
Q=\frac{p_{1}-p_{2}}{2 \eta l} \pi R^{4} \tag{4.82}
\end{equation*}


Это формула Пуазейля. Видно, что поток \(Q\) особенно сильно зависит от радиуса трубы - в четвертой степени. На этой формуле, в частности, основан один из экспериментальных методов определения вязкости \(\eta\) жидкостей.

В заключение определим мощность сил вязкости при стационарном течении жидкости в данной трубе. Воспользуемся следующими соображениями. Кинетическая энергия текущей в трубе жидкости остается неизменной, поэтому алгебраическая сумма мощностей сил давления и сил вязкости должна быть равна нулю: \(P_{\text {давл }}+P_{\text {вязк }}=0\). Отсюда \(P_{\text {вязк }}=-P_{\text {давл }}\) и задача сводится к вычислению \(P_{\text {давл }}=\int v \mathrm{~d} F\), где \(\mathrm{d} F\) - сила, создаваемая перепадом давлений на торцах цилиндрического слоя сечением \(2 \pi r \mathrm{~d} r\). Значит,

\[
P_{\text {давл }}=\int_{0}^{R} v\left(p_{1}-p_{2}\right) 2 \pi r \mathrm{~d} r .
\]

После подстановки выражения (4.81) и интегрирования с учетом (4.82) получим

\[
P_{\text {давл }}=\left(p_{1}-p_{2}\right) Q .
\]

Таким образом, искомая мощность сил вязкости


\begin{equation*}
P_{\text {вязк }}=-\left(p_{1}-p_{2}\right) Q, \tag{4.83}
\end{equation*}


где, напомним, \(p_{1}>p_{2}\).\\
Пример. При перепаде давлений \(p_{1}-p_{2}=10\) атм \(=10^{6}\) Па и потоке жидкости \(Q=1,0 \mathrm{~m}^{3} / \mathrm{c}\) модуль мощности сил вязкости, согласно (4.83), равен \(10^{6} \cdot 1,0=1 \mathrm{MBr}\). Это дает представление о той мощности, которую приходится затрачивать при создании в трубопроводах стационарного течения жидкости.

\section*{Задачи}
4.1. Работа и мощность. Камень массы \(m\) бросили с поверхности земли под углом \(\alpha\) к горизонту с начальной скоростью \(\mathbf{v}_{0}\). Пренебрегая сопротивлением воздуха, найти мощность силы тяжести через \(t\) секунд после начала движения, а также работу этой силы за первые \(t\) секунд движения.

Решение. Скорость камня через \(t\) секунд после начала движения \(\mathbf{v}=\mathbf{v}_{0}+\mathbf{g} t\). Мощность, развиваемая силой тяжести в этот момент,

\[
P=m \mathrm{gv}=m\left(\mathrm{gv}_{0}+g^{2} t\right) .
\]

В нашем случае \(\mathbf{g v}_{0}=g v_{0} \cos (\pi / 2+\alpha)=-g v_{0} \sin \alpha\), поэтому

\[
P=m g\left(g t-v_{0} \sin \alpha\right) .
\]

Отсюда видно, что при \(t<t_{0}=\left(v_{0} / g\right) \sin \alpha\) мощность \(P<0\), а при

\begin{figure}[h]
\begin{center}
  \includegraphics[alt={},max width=\textwidth]{66a35da9-3faa-4823-9ca9-40799f37b3db-145_258_423_319_141}
\captionsetup{labelformat=empty}
\caption{Рис. 4.23}
\end{center}
\end{figure}

Работа силы тяжести за первые \(t\) секунд

\[
A=\int_{0}^{t} P \mathrm{~d} t=m g t\left(g t / 2-v_{0} \sin \alpha\right) .
\]

Графики зависимостей \(P(t)\) и \(A(t)\) показаны на рис. 4.23.\\
4.2. Консервативность сил поля. Имеются два стационарных силовых поля: 1) \(\mathbf{F}=a y \mathbf{i}\); 2) \(\mathbf{F}=a x \mathbf{i}+b y \mathbf{j}\), где \(\mathbf{i}, \mathbf{j}\) - орты осей \(X\) и \(Y ; a\) и \(b\) - постоянные.

Консервативны ли силы этих полей?\\
Решение. Найдем работу силы каждого поля на пути от некоторой точки \(1\left(x_{1}, y_{1}\right)\) до некоторой точки \(2\left(x_{2}, y_{2}\right)\) :

\begin{enumerate}
  \item \(\delta A=\mathbf{F} \mathrm{d} \mathbf{r}=a y \mathbf{i} \mathrm{~d} \mathbf{r}=a y \mathrm{~d} x, \quad A=a \int_{x_{1}}^{x_{2}} y \mathrm{~d} x ;\)
  \item \(\delta A=(a x \mathbf{i}+b y \mathbf{j}) \mathrm{d} \mathbf{r}=a x \mathrm{~d} x+b y \mathrm{~d} y, \quad A=a \int_{x_{1}}^{x_{2}} x \mathrm{~d} x+b \int_{y_{1}}^{y_{2}} y \mathrm{~d} y\).
\end{enumerate}

В первом случае интеграл зависит от вида функции \(y(x)\), т. е. от пути, поэтому первая сила неконсервативная. Во втором случае оба интеграла не зависят от пути, они зависят только от координат начальной и конечной точек пути. Следовательно, вторая сила консервативная.\\
4.3. Потенциальная энергия частицы в поле. Сила, действующая на частицу в некотором поле консервативных сил, имеет вид \(\mathbf{F}=a(y \mathbf{i}+x \mathbf{j})\), где \(a\) - постоянная, \(\mathbf{i}\) и \(\mathbf{j}\) - орты осей \(X\) и \(Y\). Найти потенциальную энергию \(U(x, y)\) частицы в этом поле.\\
P е шен и е. Вычислим элементарную работу силы \(\mathbf{F}\) на перемещении dr и представим ее, согласно (4.14), в виде убыли некоторой функции \(U\). Эта функция и есть, по определению, потенциальная энергия частицы в данном поле. Итак,

\[
\delta A=\mathbf{F} \mathrm{d} \mathbf{r}=a(y \mathrm{~d} x+x \mathrm{~d} y)=-\mathrm{d}(-a x y) .
\]

Отсюда \(U(x, y)=-a x y+\) const .\\
4.4. О разных подходах к решению. Шарик массы \(m\) подвесили на упругой невесомой нити, жесткость которой \(\varkappa\). Затем шарик подняли так, чтобы нить оказалась в недеформированном состоянии, и без толчка отпустили. Найти максимальное удлинение \(x_{m}\) нити в процессе движения шарика.

Решение. Рассмотрим три эквивалентных способа решения, основанных на энергетических соображениях.

\begin{enumerate}
  \item Исходим из уравнения (4.29): приращение кинетической энергии шарика должно быть равно алгебраической сумме работ всех сил, действующих на него. В нашем случае это сила тяжести \(m g\) и упругая сила со стороны нити \(F_{\text {упр }}=\varkappa x\), где \(x\) - удлинение нити. В начальном и конечном положениях шарика его кинетическая энергия равна нулю (ясно, что при максимальном растяжении нити шарик остановится), поэтому, согласно (4.29), сумма работ \(A_{\text {тяж }}+A_{\text {упр }}=0\), или
\end{enumerate}

\[
m g x_{m}+\int_{0}^{x_{m}}(-\varkappa x) \mathrm{d} x=m g x_{m}-\varkappa x_{m}^{2} / 2=0
\]

Отсюда \(x_{m}=2 m g / \varkappa\).\\
2. Можно рассматривать шарик в поле тяжести Земли. При таком подходе следует говорить о полной механической энергии шарика в поле тяжести Земли. Приращение этой энергии, согласно (4.31), равно работе сторонних сил. В данном случае сторонней силой надо считать силу упругости, приращение же полной механической энергии шарика равно приращению только его потенциальной энергии в поле тяжести Земли. Поэтому

\[
\Delta E=0-m g x_{m}=\int_{0}^{x_{m}}(-\varkappa x) \mathrm{d} x=-\varkappa x_{m}^{2} / 2 .
\]

Отсюда тот же результат для \(x_{m}\).\\
Заметим, что можно было бы поступить и наоборот, т. е. рассматривать шарик в поле упругой силы. Тогда роль сторонней силы играла бы сила тяжести. Полезно убедиться, что и в этом случае результат будет тем же.\\
3. И наконец, можно рассматривать шарик в поле, образованном совместным действием и силы тяжести, и упругой силы. Тогда сторонние силы отсутствуют и полная механическая энергия шарика в таком поле остается постоянной в процессе движения, т. е. \(\Delta E=\Delta K+\Delta U=0\). При переходе шарика из начального положе-

ния в конечное (нижнее) \(\Delta K=0\), а следовательно, и \(\Delta U=\Delta U_{\text {тяж }}+\Delta U_{\text {упр }}=0\), или

\[
-m g x_{m}+\varkappa x_{m}^{2} / 2=0 .
\]

Результат опять тот же.\\
4.5. Небольшое тело массы \(m\) поднимается с нулевой начальной скоростью с поверхности земли под действием двух сил: силы \(\mathbf{F}\), меняющейся с высотой подъема \(y\) по закону \(\mathbf{F}=-2 m \mathbf{g}(1-a y)\), где \(a-\) положительная постоянная, и силы тяжести \(m g\). Найти работу силы \(\mathbf{F}\) на первой половине пути подъема и соответствующее приращение потенциальной энергии тела в поле тяжести Земли. Поле тяжести предполагается однородным.

Решение. Сначала найдем весь путь подъема. В начале и конце пути скорость тела равна нулю, поэтому равно нулю и приращение кинетической энергии тела. Согласно (4.29), \(\Delta K\) равно алгебраической сумме работ \(A\) силы \(\mathbf{F}\) и силы тяжести. А так как \(\Delta K=0\), то и \(A=0\). Учитывая, что положительное направление оси \(Y\) взято вверх, запишем

\[
A=\int_{0}^{h}\left(F_{y}-m g\right) \mathrm{d} y=m g \int_{0}^{h}(1-2 a y) \mathrm{d} y=m g h(1-a h)=0 .
\]

Отсюда \(h=1 / a\).\\
Работа силы \(\mathbf{F}\) на первой половине пути подъема

\[
A_{F}=\int_{0}^{h / 2} F_{y} \mathrm{~d} y=2 m g \int_{0}^{h / 2}(1-a y) \mathrm{d} y=3 m g / 4 a .
\]

Соответствующее приращение потенциальной энергии

\[
\Delta U=m g h / 2=m g / 2 a .
\]

4.6. Потенциал и напряженность поля. Найти потенциал и напряженность гравитационного поля, созданного однородным шаром массы \(M\) и радиуса \(R\), в зависимости от расстояния \(r\) до его центра.\\
Решен и е. Сначала определим потенциал поля, создаваемого тонким однородным сферическим слоем вещества массы \(m\) и радиуса \(a\). Для этого найдем потенциал \(\mathrm{d} \varphi\) в точке \(P(r>a)\), который создает элементарный пояс \(\mathrm{d} S\) данного слоя (рис. 4.24, a). Если масса этого пояса \(\mathrm{d} m\) и его точки находятся на расстоянии \(x\)

\begin{figure}[h]
\begin{center}
  \includegraphics[alt={},max width=\textwidth]{66a35da9-3faa-4823-9ca9-40799f37b3db-148_292_759_130_272}
\captionsetup{labelformat=empty}
\caption{Рис. 4.24}
\end{center}
\end{figure}

от точки \(P\), то \(\mathrm{d} \varphi=-\gamma \mathrm{d} m / x\). Учитывая, что \(\mathrm{d} m=\frac{1}{2} m \sin \vartheta \mathrm{~d} \vartheta\), получаем


\begin{equation*}
\mathrm{d} \varphi=-\frac{\gamma m}{2 x} \sin \vartheta \mathrm{~d} \vartheta . \tag{1}
\end{equation*}


Далее, из теоремы косинусов (для \(\triangle O A P\) ) следует, что \(x^{2}=a^{2}+r^{2}-2 a r \cos \vartheta\). Взяв дифференциал этого выражения, найдем


\begin{equation*}
x \mathrm{~d} x=a r \sin \vartheta \mathrm{~d} \vartheta . \tag{2}
\end{equation*}


Преобразуем (1) с помощью (2) к виду \(\mathrm{d} \varphi=\frac{1}{2}(\gamma m / a r) \mathrm{d} x\) и проинтегрируем это уравнение по всему слою. Тогда


\begin{equation*}
\varphi_{\text {Вне }}=-\frac{\gamma m}{2 a r} \int_{r-a}^{r+a} \mathrm{~d} x=-\frac{\gamma m}{r} . \tag{3}
\end{equation*}


Таким образом, потенииал в точке \(P\) вне тонкого однородного сферического слоя таков, как если бы вся масса этого слоя была сосредоточена в его центре. Если же точка \(P\) находится внутри слоя ( \(r<a\) ), то предыдущие расчеты остаются в силе вплоть до интегрирования. Теперь пределы интегрирования по \(x\) будут от \(a-r\) до \(a+r\). В результате


\begin{equation*}
\varphi_{\text {внутри }}=-\gamma m / a, \tag{4}
\end{equation*}


т. е. потенциал внутри слоя не зависит от положения точки \(P\), а следовательно, он одинаков во всех точках внутри слоя.

Напряженность поля в точке \(P\), согласно (4.26),

\[
G_{r}=-\frac{\partial \varphi}{\partial r}=\left\{\begin{array}{cc}
-\gamma m / r^{2} & \text { вне слоя, } \\
0 & \text { внутри слоя. }
\end{array}\right.
\]

Графики зависимостей \(\varphi(r)\) и \(G(r)\) для тонкого однородного сферического слоя показаны на рис. 4.24 , б.

Обобщим полученные результаты на однородный шар массы \(M\) и радиуса \(R\). Если точка \(P\) находится вне шара ( \(r>R\) ), то из формулы (3) следует


\begin{equation*}
\varphi_{\text {вне }}=-\gamma M / r . \tag{5}
\end{equation*}


Если точка \(P\) находится внутри шара ( \(r<R\) ), то потенциал в этой точке

\[
\varphi_{\text {внутри }}=\varphi_{1}+\varphi_{2},
\]

где \(\varphi_{1}\) - потенциал от шара радиуса \(r, \varphi_{2}\) - потенциал от слоя с радиусами от \(r\) до \(R\). Согласно (5),

\[
\varphi_{1}=-\gamma \frac{M(r / R)^{3}}{r}=-\gamma \frac{M}{R^{3}} r^{2} .
\]

Потенциал \(\varphi_{2}\), создаваемый слоем, одинаков во всех точках внутри этого слоя. Проще всего \(\varphi_{2}\) вычислить для точки, находящейся в центре слоя:

\[
\varphi_{2}=-\gamma \int_{r}^{R} \frac{\mathrm{~d} M}{r}=-\frac{3}{2} \frac{\gamma M}{R^{3}}\left(R^{2}-r^{2}\right),
\]

где \(\mathrm{d} M=3\left(M / R^{3}\right) r^{2} \mathrm{~d} r\) - масса тонкого слоя между радиусами \(r\) и \(r+\mathrm{d} r\). В результате


\begin{equation*}
\varphi_{\text {внутри }}=\varphi_{1}+\varphi_{2}=-(\gamma M / 2 R)\left(3-r^{2} / R^{2}\right) . \tag{6}
\end{equation*}


Напряженность поля в точке \(P\), как следует из (5) и (6),

\begin{figure}[h]
\begin{center}
  \includegraphics[alt={},max width=\textwidth]{66a35da9-3faa-4823-9ca9-40799f37b3db-149_322_421_1452_86}
\captionsetup{labelformat=empty}
\caption{Рис. 4.25}
\end{center}
\end{figure}

\[
G_{r}=-\frac{\partial \varphi}{\partial r}= \begin{cases}-\gamma M / r^{2} & \text { при } r \geqslant R, \\ -\gamma M / R^{3} & \text { при } r \leqslant R .\end{cases}
\]

Графики зависимостей \(\varphi(r)\) и \(G(r)\) для однородного шара радиуса \(R\) показаны на рис. 4.25.\\
4.7. Космические скорости. Показать, что кинетическая энергия \(K_{2}\), которую необходимо сообщить телу для удаления его за пределы земного тяготения, в два раза превышает кинетическую энергию \(K_{1}\), необходимую для выведения этого тела на круговую орбиту искусственного спутника Земли (вблизи ее поверхности). Сопротивлением воздуха и вращением Земли пренебречь.

Решение. Найдем скорость \(v_{1}\) тела, движущегося по круговой орбите. Согласно основному уравнению динамики,

\[
m v_{1}^{2} / R=m g,
\]

где \(m\) - масса тела, \(R\) - радиус орбиты, приблизительно равный радиусу Земли. Отсюда

\[
v_{1}=\sqrt{g R}=7,9 \quad к м / с .
\]

Это первая космическая скорость.\\
Для того чтобы тело могло преодолеть поле тяготения Земли, ему необходимо сообщить вторую космическую скорость \(v_{2}\). Ee можно найти из закона сохранения энергии: кинетическая энергия тела вблизи поверхности Земли должна быть равна глубине потенциальной ямы в этом месте. Последняя равна приращению потенциальной энергии тела в поле тяготения Земли между точками \(r_{1}=R\) и \(r_{2}=\infty\). Таким образом,

\[
m v_{2}^{2} / 2=\gamma m M / R,
\]

где \(M\) - масса Земли. Отсюда

\[
v_{2}=\sqrt{2 \gamma M / R}=\sqrt{2 g R}=11 \mathrm{~km} / \mathrm{c} .
\]

Следовательно, \(v_{2}=v_{1} \sqrt{2}\) и \(K_{2}=2 K_{1}\).\\
4.8. Решение в неинерциальной системе отсчета. Плоскую жесткую спираль из гладкой проволоки, расположенную в горизонтальной плоскости, вращают с постоянной угловой скоростью \(\omega\) вокруг неподвижной вертикальной оси \(O\) (рис. 4.26). По спирали скользит небольшая муфта \(M\). Найти ее скорость \(v^{\prime}\) относительно спирали как функцию расстояния \(\rho\) от оси вращения \(O\), если муфта начала двигаться от этой оси со скоростью \(v_{0}^{\prime}\).

\begin{figure}[h]
\begin{center}
  \includegraphics[alt={},max width=\textwidth]{66a35da9-3faa-4823-9ca9-40799f37b3db-150_196_297_1460_863}
\captionsetup{labelformat=empty}
\caption{Рис. 4.26}
\end{center}
\end{figure}

Решение. Этот вопрос наиболее целесообразно решать в системе отсчета, связанной со спиралью. Известно, что приращение кинетической энергии тела равно алгебраической сумме работ всех сил, действующих на тело. В нашем случае из всех сил работу будет совершать только центробежная сила инерции. Все остальные силы (сила тяжести, сила реакции со стороны спирали и сила Кориолиса) перпендикулярны скорости \(\mathbf{v}^{\prime}\) муфты, поэтому работы не совершают.

Согласно уравнению (4.29),

\[
m\left(v^{\prime 2}-v_{0}^{\prime 2}\right) / 2=\int m \omega^{2} \rho \mathrm{~d} \mathbf{r},
\]

где \(m\) - масса муфты, dr - ее элементарное перемещение относительно данной спирали. Так как скалярное произведение \(\rho \mathrm{d} \mathbf{r}=\rho(\mathrm{d} \mathbf{r})_{\rho}=\rho \mathrm{d} \rho\), то интеграл равен \(m \omega^{2} \rho^{2} / 2\), откуда

\[
v^{\prime}=\sqrt{v_{0}^{\prime 2}+\omega^{2} \rho^{2}} .
\]

4.9. Система частиц. Три одинаковые заряженные частицы, каждая массы \(m\) и с зарядом \(q\), поместили в вершины углов равностороннего треугольника со стороной \(a\). Затем частицы одновременно освободили и они стали симметрично разлетаться под действием кулоновских сил отталкивания. Найти: 1) скорость каждой частицы в зависимости от расстояния \(r\) между ними; 2) работу \(A_{1}\), которую совершили кулоновские силы, действующие на каждую частицу при разлете их на очень большое расстояние друг от друга.

Решение. 1. Данная система замкнутая, поэтому для нее приращение кинетической энергии равно убыли потенциальной, т. е.

\[
3 m v^{2} / 2=3 k q^{2}(1 / a-1 / r),
\]

откуда

\[
v=\sqrt{2 k q^{2}(r-a) / m r a} .
\]

Видно, что при \(r \rightarrow \infty\) скорость каждой частицы стремится к предельному значению

\[
v_{\text {макс }}=\sqrt{2 k q^{2} / m a} .
\]

\begin{enumerate}
  \setcounter{enumi}{1}
  \item Работа всех сил взаимодействия при изменении конфигурации этой системы равна убыли потенциальной энергии системы:
\end{enumerate}

\[
A=U_{1}-U_{2}=3 k q^{2} / a,
\]

где учтено, что в конечном положении \(U_{2}=0\). Тогда искомая работа


\begin{equation*}
A_{1}=A / 3=k q^{2} / a . \tag{*}
\end{equation*}


Замечание. Следует обратить внимание на одну часто встречающуюся ошибку при решении подобного рода задач. Рассуждают так: в начальном положении потенциальная энергия каждой частицы в поле других двух равна \(2 k q^{2} / a\), а в конечном нуль. Отсюда искомая работа \(A_{1}=2 k q^{2} / a\). Как видно, полученный результат отличается вдвое от (*). Почему?\\
Ошибка здесь в том, что поле, в котором перемещается каждая частица, нестационарное (ведь другие две частицы тоже перемещаются друг относительно друга), поэтому работа сил такого поля не может быть представлена как убыль потенциальной энергии частицы в поле.\\
4.10. На гладкой горизонтальной плоскости находятся два бруска с массами \(m_{1}\) и \(m_{2}\), соединенные между собой пружинкой жесткости \(\varkappa\) (рис. 4.27). Брусок 2 переместили влево на небольшое расстояние \(x\) и отпустили. Найти скорость \(v_{c}\) центра масс системы после отрыва бруска 1 от стенки.

\begin{figure}[h]
\begin{center}
  \includegraphics[alt={},max width=\textwidth]{66a35da9-3faa-4823-9ca9-40799f37b3db-152_124_332_895_832}
\captionsetup{labelformat=empty}
\caption{Рис. 4.27}
\end{center}
\end{figure}

Решение. При распрямлении пружинки на систему будет действовать некоторая горизонтальная сила \(F\) со стороны стенки. Импульс этой силы приводит к возникновению импульса системы. После отрыва бруска 1 от стенки импульс системы меняться не будет, и мы можем записать:


\begin{equation*}
m_{2} v_{20}=\left(m_{1}+m_{2}\right) v_{c}, \tag{1}
\end{equation*}


где учтено, что в момент отрыва бруска 1 его скорость \(v_{10}=0\).\\
Поскольку точка приложения силы \(F\) неподвижна (в процессе ее действия), эта сила работы не совершает. Значит, механическая энергия системы в процессе движения будет оставаться неизменной и равной своей первоначальной величине. В момент отрыва бруска 1 потенциальная энергия первоначально сжатой пружины целиком переходит в кинетическую энергию бруска 2:


\begin{equation*}
\varkappa x^{2} / 2=m_{2} v_{20}^{2} / 2 \tag{2}
\end{equation*}


Исключив \(v_{20}\) из равенств (1) и (2), получим

\[
v_{c}=x \sqrt{\varkappa m_{2}} /\left(m_{1}+m_{2}\right) .
\]

Заметим, что нечто подобное происходит при начале движения, например, автомашины: импульс сил трения сообщает импульс автомашине, а внутренний источник энергии (топливо) - кинетическую энергию.\\
4.11. Внутренняя механическая энергия системы. Система состоит из двух шариков с массами \(m_{1}\) и \(m_{2}\), которые соединены между собой невесомой пружинкой. Шарикам сообщили скорости \(\mathbf{v}_{1}\) и \(\mathbf{v}_{2}\) соответственно, после чего система начала двигаться в однородном поле сил тяжести Земли. Пренебрегая сопротивлением воздуха и считая, что в начальный момент пружинка не деформирована и \(\mathbf{v}_{1} \perp \mathbf{v}_{2}\), найти внутреннюю механическую энергию данной системы в процессе движения.\\
Решение. Внутренняя механическая энергия системы - это ее энергия \(\tilde{E}\) в Ц-системе. Здесь Ц-система движется с ускорением \(\mathbf{g}\), поэтому в этой системе отсчета на каждый шарик действуют две внешние силы: сила тяжести \(m_{i} \mathbf{g}\) и сила инерции \(-m_{i} \mathbf{g}\). Суммарная работа этих внешних сил равна нулю (в Ц-системе), а следовательно, энергия \(\tilde{E}\) изменяться не будет. Чтобы ее найти, достаточно рассмотреть начальный момент, когда пружинка еще не деформирована и энергия \(\widetilde{E}\) равна только суммарной кинетической энергии \(\tilde{K}_{0}\) в Ц-системе. Воспользовавшись формулой (4.61), получим

\[
\tilde{E}=\tilde{K}_{0}=\frac{\mu}{2}\left(\mathbf{v}_{1}-\mathbf{v}_{2}\right)^{2}=\frac{m_{1} m_{2}}{2\left(m_{1}+m_{2}\right)}\left(v_{1}^{2}+v_{2}^{2}\right)
\]

4.12. Столкновение частиц. В \(K\)-системе отсчета частица 1 массы \(m_{1}\) налетает на покоящуюся частицу 2 массы \(m_{2}\). Заряд каждой частицы равен \(q\). Найти минимальное расстояние, на которое они сблизятся при лобовом «соударении», если кинетическая энергия частицы 1 вдали от частицы 2 равна \(K_{1}\).\\
Решение. Рассмотрим процесс столкновения отдельно в \(K\) - и Ц-системах отсчета.

\begin{enumerate}
  \item В \(K\)-системе в момент наибольшего сближения обе частицы будут двигаться как единое целое со скоростью \(v\), которую можно определить из закона сохранения импульса:
\end{enumerate}

\[
p_{1}=\left(m_{1}+m_{2}\right) v
\]

где \(p_{1}\) - импульс налетающей частицы: \(p_{1}=\sqrt{2 m_{1} K_{1}}\). Из закона сохранения энергии следует, что

\[
K_{1}=\left(m_{1}+m_{2}\right) v^{2} / 2+\Delta U
\]

где приращение потенциальной энергии системы \(\Delta U=k q^{2} / r_{\text {мин }}\). Исключив \(v\) из этих двух уравнений, найдем

\[
r_{\text {мин }}=\left(1+m_{1} / m_{2}\right) k q^{2} / K_{1} .
\]

\begin{enumerate}
  \setcounter{enumi}{1}
  \item В Ц-системе решение наиболее просто: здесь суммарная кинетическая энергия частиц идет целиком на приращение потенциальной энергии системы частиц в момент их наибольшего сближения:
\end{enumerate}

\[
\tilde{K}=\Delta U
\]

где, согласно (4.61), \(\tilde{K}=\mu v_{1}^{2} / 2=K_{1} m_{2} /\left(m_{1}+m_{2}\right), \Delta U=k q^{2} / r_{\text {мин }}\). Отсюда легко найти \(r_{\text {мин }}\).\\
4.13. Частица массы \(m_{1}\) с импульсом \(p_{1}\) испытала упругое столкновение с покоившейся частицей массы \(m_{2}\). Найти импульс \(p_{1}^{\prime}\) первой частицы после столкновения, в результате которого она рассеялась под углом Э к первоначальному направлению движения.\\
Решение. Из закона сохранения импульса (рис. 4.28) находим

\begin{figure}[h]
\begin{center}
  \includegraphics[alt={},max width=\textwidth]{66a35da9-3faa-4823-9ca9-40799f37b3db-154_204_361_936_801}
\captionsetup{labelformat=empty}
\caption{Рис. 4.28}
\end{center}
\end{figure}


\begin{equation*}
p_{2}^{\prime 2}=p_{1}^{2}+p_{1}^{\prime 2}-2 p_{1} p_{1}^{\prime} \cos \vartheta, \tag{1}
\end{equation*}


где \(p_{2}^{\prime}\) - импульс покоившейся частицы после столкновения. Из закона сохранения энергии следует, что \(K_{1}=K_{1}^{\prime}+K_{2}^{\prime}\), где \(K_{1}^{\prime}\) и \(K_{2}^{\prime}\) - кинетические энергии первой и второй частиц после столкновения. Преобразуем это равенство с помощью соотношения \(K=p^{2} / 2 m\) к виду


\begin{equation*}
p_{2}^{\prime 2}=\left(p_{1}^{2}-p_{1}^{\prime 2}\right) m_{2} / m_{1} . \tag{2}
\end{equation*}


Исключив \(p_{2}^{\prime 2}\) из (1) и (2), получим

\[
p_{1}^{\prime}=p_{1} \frac{\cos \vartheta \pm \sqrt{\cos ^{2} \vartheta+\left(m_{2}^{2} / m_{1}^{2}-1\right)}}{1+m_{2} / m_{1}} .
\]

Если \(m_{1}<m_{2}\), то физический смысл имеет только знак плюс перед корнем. Это следует из того, что при этом условии корень будет больше, чем \(\cos \vartheta\), а так как \(p_{1}^{\prime}\) - это модуль вектора, то он не может быть отрицательным.\\
Если \(m_{1}>m_{2}\), то физический смысл имеют оба знака перед корнем - ответ в этом случае неоднозначен: под углом \(\vartheta\) импульс рассеянной частицы может иметь одно из двух значений (это зависит от относительного расположения частиц в момент соударения). Последний случай соответствует векторной диаграмме, показанной на рис. 4.14, в.\\
4.14. Какую часть \(\eta\) своей кинетической энергии теряет частица массы \(m_{1}\) при упругом рассеянии под предельным углом на покоящейся частице массы \(m_{2}\) ? Здесь \(m_{1}>m_{2}\).\\
Решение. Пусть \(K_{1}, p_{1}\) и \(K_{1}^{\prime}, p_{1}^{\prime}\) - кинетическая энергия и импульс налетающей частицы до и после рассеяния. Тогда


\begin{equation*}
\eta=\left(K_{1}-K_{1}^{\prime}\right) / K_{1}=1-K_{1}^{\prime} / K_{1}=1-\left(p_{1}^{\prime} / p_{1}\right)^{2}, \tag{1}
\end{equation*}


\begin{figure}[h]
\begin{center}
  \includegraphics[alt={},max width=\textwidth]{66a35da9-3faa-4823-9ca9-40799f37b3db-155_172_372_870_141}
\captionsetup{labelformat=empty}
\caption{Рис. 4.29}
\end{center}
\end{figure}

т. е. задача сводится к нахождению отношения \(p_{1}^{\prime} / p_{1}\).\\
Воспользуемся векторной диаграммой импульсов, соответствующей предельному углу \(\vartheta_{1 \text { пр }}\) (рис. 4.29). Из прямоугольного треугольника \(A C O\) следует, что

\[
p_{1}^{\prime 2}=\left(p_{1}-\tilde{p}\right)^{2}-\tilde{p}^{2}=p_{1}^{2}-2 p_{1} \tilde{p},
\]

откуда


\begin{equation*}
\left(p_{1}^{\prime} / p_{1}\right)^{2}=1-2 \tilde{p} / p_{1}=1-2 m_{2} /\left(m_{1}+m_{2}\right) . \tag{2}
\end{equation*}


После подстановки (2) в (1) получим

\[
\eta=2 m_{2} /\left(m_{1}+m_{2}\right) .
\]

4.15. Атом массы \(m_{1}\) испытал неупругое столкновение с покоившейся молекулой массы \(m_{2}\). После соударения обе частицы разлетелись под углом \(\vartheta\) друг к другу с кинетическими энергиями \(K_{1}^{\prime}\) и \(K_{2}^{\prime}\) соответственно, причем молекула оказалась в возбужденном состоянии - ее внутренняя энергия увеличилась на определенную величину \(Q\). Найти \(Q\), а также пороговую кинетическую энергию атома, при которой возможен переход молекулы в данное возбужденное состояние.

Решение. Из законов сохранения энергии и импульса в этом процессе следует:

\[
\begin{gathered}
K_{1}=K_{1}^{\prime}+K_{2}^{\prime}+Q, \\
p_{1}^{2}=p_{1}^{\prime 2}+p_{2}^{\prime 2}+2 p_{1}^{\prime} p_{2}^{\prime} \cos \vartheta,
\end{gathered}
\]

где штрихами отмечены величины после соударения (второе соотношение сразу следует из треугольника импульсов согласно теореме косинусов). Воспользовавшись формулой \(p^{2}=2 m K\), исключим \(K_{1}\) из этих уравнений. В результате получим

\[
\begin{gathered}
Q=\left(m_{2} / m_{1}-1\right) K_{2}^{\prime}+2 \sqrt{\left(m_{2} / m_{1}\right) K_{1}^{\prime} K_{2}^{\prime} \cos \vartheta}, \\
K_{1 \text { пор }}=|Q|\left(m_{1}+m_{2}\right) / m_{2} .
\end{gathered}
\]

4.16. Распад частицы. Частица с импульсом \(\mathbf{p}_{0}\) (в \(K\)-системе) распалась на лету на две частицы с массами \(m_{1}\) и \(m_{2}\). При этом выделилась энергия \(Q\) - энергия распада (она перешла в кинетическую энергию). Построить векторную диаграмму импульсов для этого процесса и найти с помощью нее возможные импульсы \(\mathbf{p}_{1}\) и \(\mathbf{p}_{2}\) возникших частиц.\\
Решение. Наиболее просто этот процесс выглядит в Ц-системе: здесь распадающаяся частица покоится, а частицы распада разлетаются в противоположные стороны с одинаковыми по модулю импульсами \(\tilde{p}_{1}=\tilde{p}_{2}=\tilde{p}\). Энергия распада \(Q\) целиком переходит в суммарную кинетическую энергию \(\widetilde{K}\) возникающих частиц. Поэтому

\[
\tilde{p}=\sqrt{2 \mu \tilde{K}}=\sqrt{2 \mu Q},
\]

где \(\mu\) - приведенная масса системы возникших частиц.\\
Найдем импульсы возникших частиц в \(K\)-системе. Воспользовавшись формулой преобразования скоростей при переходе от \(Ц\) - к \(K\)-системе, запишем:

\[
\begin{aligned}
& \mathbf{p}_{1}=m_{1} \mathbf{v}_{1}=m_{1}\left(\mathbf{V}_{C}+\tilde{\mathbf{v}}_{1}\right)=m_{1} \mathbf{V}_{C}+\tilde{\mathbf{p}}_{1}, \\
& \mathbf{p}_{2}=m_{2} \mathbf{v}_{2}=m_{2}\left(\mathbf{V}_{C}+\tilde{\mathbf{v}}_{2}\right)=m_{2} \mathbf{V}_{C}+\tilde{\mathbf{p}}_{2},
\end{aligned}
\]

причем, согласно закону сохранения импульса, \(\mathbf{p}_{1}+\mathbf{p}_{2}=\mathbf{p}_{0}\).\\
С помощью этих формул построим векторную диаграмму импульсов (рис. 4.30). Изобразим сначала отрезок \(A B\), равный импульсу \(p_{0}\). Затем

\begin{figure}[h]
\begin{center}
  \includegraphics[alt={},max width=\textwidth]{66a35da9-3faa-4823-9ca9-40799f37b3db-156_201_400_1553_762}
\captionsetup{labelformat=empty}
\caption{Рис. 4.30}
\end{center}
\end{figure}

радиусом \(\tilde{p}\) проведем окружность с центром в точке \(O\), которая делит отрезок \(A B\) на две части в отношении \(m_{1}: m_{2}\). Эта окружность и есть геометрическое место точек всех возможных положений вершины \(C\) треугольника импульсов \(A B C\).\\
4.17. Открытый сверху цилиндрический сосуд высотой \(h\) заполнен доверху идеальной жидкостью. В дне сосуда открыли малое отверстие, площадь которого в \(\eta\) раз меньше площади сечения сосуда. Считая \(\eta \gg 1\), найти, через какое время вся жидкость вытечет из сосуда.\\
Решение. Скорость \(v_{1}\), с которой будет опускаться уровень жидкости в сосуде, не постоянна. Поэтому сначала найдем время \(\mathrm{d} t\), за которое убыль высоты уровня равна \(-\mathrm{d} x\) :


\begin{equation*}
\mathrm{d} t=-\mathrm{d} x / v_{1} . \tag{1}
\end{equation*}


Теперь найдем связь между \(v_{1}\) и высотой уровня \(x\). Для этого воспользуемся уравнением неразрывности струи (4.71) и уравнением Бернулли (4.75). Запишем последнее для двух сечений: 1 - когда уровень находится на высоте \(x\), и 2 - на выходе из отверстия. Тогда оба уравнения будут иметь вид


\begin{equation*}
\eta v_{1}=v_{2}, \quad v_{1}^{2} / 2+g x=v_{2}^{2} / 2 . \tag{2}
\end{equation*}


Здесь учтено, что давление \(p\) в обоих сечениях одинаково (атмосферное). Из уравнений (2), имея в виду, что \(\eta \gg 1\), получим


\begin{equation*}
v_{1}=\sqrt{2 g x} / \eta . \tag{3}
\end{equation*}


Подставив (3) в (1) и проинегрировав, найдем

\[
t=-\eta \int_{h}^{0} \frac{\mathrm{~d} x}{\sqrt{2 g x}}=\eta \sqrt{2 h / g} .
\]

4.18. Вязкость. Вычислить силу трения, испытываемую прямой трубой длины \(l=100\) м с радиусом круглого сечения \(R=25\) см при стационарном течении воды, поток которой \(Q=5,0 \mathrm{~m}^{3} / \mathrm{c}\). Вязкость воды \(\eta=1,0\) мПа•с.\\
Решение. Искомую силу трения найдем с помощью формулы (4.78). Для этого надо сначала вычислить производную \(\mathrm{d} v / \mathrm{d} r\) выражения (4.81) при \(r=R\). В результате получим

\[
F_{\text {тр }}=\eta\left|\frac{\mathrm{d} v}{\mathrm{~d} r}\right| 2 \pi R l=\eta \frac{\left(p_{1}-p_{2}\right) 2 R}{4 \eta l} 2 \pi R l,
\]

или с учетом формулы (4.82)

\[
F_{\text {тр }}=8 \eta l Q / R^{2}=64 \mathrm{H} .
\]

\section*{Закон сохранения момента импульса}
\section*{§ 5.1. Момент импульса частицы. Момент силы}
Анализ поведения систем показывает, что кроме энергии и импульса существует еще одна механическая величина, с которой также связан закон сохранения, - это момент импульса*. Что это за величина и каковы ее свойства?

Сначала возьмем одну частицу. Пусть \(\mathbf{r}\) - радиус-вектор, характеризующий ее положение относительно некоторой точки \(O\) выбранной системы отсчета, а p - ее импульс в этой системе.

Моментом импульса частицы \(A\) относительно точки \(O\) (рис. 5.1) называют вектор М, равный векторному произве-

\begin{figure}[h]
\begin{center}
  \includegraphics[alt={},max width=\textwidth]{66a35da9-3faa-4823-9ca9-40799f37b3db-158_267_386_623_778}
\captionsetup{labelformat=empty}
\caption{Рис. 5.1}
\end{center}
\end{figure}

дению векторов \(\mathbf{r}\) и \(\mathbf{p}\) :


\begin{equation*}
\mathbf{M}=[\mathbf{r p}] \tag{5.1}
\end{equation*}


Из этого определения следует, что \(\mathbf{M}\) является аксиальным вектором. Его направление выбрано так, что вращение вокруг точки \(O\) в направлении вектора \(\mathbf{p}\) и вектор \(\mathbf{M}\) образуют правовинтовую систему. Модуль вектора М равен


\begin{equation*}
M=r p \sin \alpha=l p \tag{5.2}
\end{equation*}


где \(\alpha\) - угол между \(\mathbf{r}\) и \(\mathbf{p}, l=r \sin \alpha\) - плечо вектора \(\mathbf{p}\) относительно точки \(O\) (рис. 5.1).

\section*{Уравнение моментов}
Выясним, какая механическая величина ответственна за изменение вектора М в данной системе отсчета. Для этого продифференцируем (5.1) по времени:

\[
\mathrm{d} \mathbf{M} / \mathrm{d} t=[\mathrm{d} \mathbf{r} / \mathrm{d} t, \mathbf{p}]+[\mathbf{r}, \mathrm{d} \mathbf{p} / \mathrm{d} t] .
\]

\footnotetext{\begin{itemize}
  \item Используют также названия момент количества движения, механический момент, угловой момент или просто момент.
\end{itemize}
}Так как точка \(O\) неподвижна, то вектор \(\mathrm{d} \mathbf{r} / \mathrm{d} t\) равен скорости \(\mathbf{v}\) частицы, т. е. совпадает по направлению с вектором \(\mathbf{p}\), поэтому

\[
[\mathrm{d} \mathbf{r} / \mathrm{d} t, \mathbf{p}]=0 .
\]

Согласно второму закону Ньютона, \(\mathrm{d} \mathbf{p} / \mathrm{d} t=\mathbf{F}\), где \(\mathbf{F}\) - равнодействующая всех сил, приложенных к частице. Следовательно,

\begin{figure}[h]
\begin{center}
  \includegraphics[alt={},max width=\textwidth]{66a35da9-3faa-4823-9ca9-40799f37b3db-159_270_390_451_72}
\captionsetup{labelformat=empty}
\caption{Рис. 5.2}
\end{center}
\end{figure}

\[
\mathrm{d} \mathbf{M} / \mathrm{d} t=[\mathbf{r F}] .
\]

Величину, стоящую в правой части этого уравнения, называют моментом силь \(\mathbf{F}\) относительно точки \(O\) (рис. 5.2). Обозначив ее буквой \(\mathbf{N}\), запишем


\begin{equation*}
\mathbf{N}=[\mathbf{r F}] . \tag{5.3}
\end{equation*}


Вектор N, как и M, является аксиальным. Модуль этого вектора, аналогично (5.2), равен


\begin{equation*}
N=l F, \tag{5.4}
\end{equation*}


где \(l\) - плечо вектора \(\mathbf{F}\) относительно точки \(O\) (рис. 5.2).\\
Итак, производная по времени от момента импульса \(\mathbf{M}\) частицы относительно некоторой точки \(O\) выбранной системы отсчета равна моменту \(\mathbf{N}\) равнодействующей силы \(\mathbf{F}\) относительно той же точки \(O\) :


\begin{equation*}
\mathrm{d} \mathbf{M} / \mathrm{d} t=\mathbf{N} . \tag{5.5}
\end{equation*}


Это уравнение моментов. Заметим, что если система отсчета является неинерциальной, то момент силы \(\mathbf{N}\) включает в себя как момент сил взаимодействия, так и момент сил инерции (относительно той же точки \(O\) ).

Из уравнения моментов (5.5), в частности, следует, что если \(\mathbf{N} \equiv \mathbf{0}\), то \(\mathbf{M}=\) const. Другими словами, если относительно некоторой точки \(O\) выбранной системы отсчета момент всех сил, действующих на частицу, равен нулю в течение интересующего нас промежутка времени, то относительно этой точки момент импульса частицы остается постоянным в течение этого времени.

Пример 1. Некоторая планета \(A\) движется в поле тяготения Солнца \(C\) (рис. 5.3). Относительно какой точки гелиоцентрической системы отсчета момент импульса данной планеты будет сохраняться во времени?\\
Для ответа на этот вопрос прежде всего необходимо установить, какие силы действуют на планету \(A\). В данном случае это только сила тяготения \(\mathbf{F}\) со стороны Солнца. Так как при движении планеты направление этой силы все время проходит через центр Солнца, то последний и является той точкой,

\begin{figure}[h]
\begin{center}
  \includegraphics[alt={},max width=\textwidth]{66a35da9-3faa-4823-9ca9-40799f37b3db-160_275_259_308_851}
\captionsetup{labelformat=empty}
\caption{Рис. 5.3}
\end{center}
\end{figure}

относительно которой момент силы \(\mathbf{F}\) все время равен нулю, и момент импульса планеты будет оставаться постоянным. Импульс р планеты при этом будет меняться.

Пример 2. Шайба \(A\), двигаясь по гладкой горизонтальной плоскости, упруго отскакивает от гладкой вертикальной стенки (рис. 5.4, вид сверху). Найдем точку, относительно которой момент импульса шайбы будет оставаться постоянным в этом процессе.\\
На шайбу действуют сила тяжести, сила реакции со стороны горизонтальной плоскости и сила реакции \(\mathbf{R}\) со стороны стенки в момент удара о нее. Первые две силы уравновешивают друг друга, остается сила R. Ее момент равен нулю относительно любой точки, лежащей на линии действия вектора \(\mathbf{R}\), а значит, отно-

\begin{figure}[h]
\begin{center}
  \includegraphics[alt={},max width=\textwidth]{66a35da9-3faa-4823-9ca9-40799f37b3db-160_285_336_996_824}
\captionsetup{labelformat=empty}
\caption{Рис. 5.4}
\end{center}
\end{figure}

сительно любой из этих точек момент импульса шайбы будет оставаться постоянным в данном процессе.

Пример 3. На горизонтальной гладкой плоскости находятся неподвижный вертикальный цилиндр и шайба \(A\), соединенная с цилиндром горизонтальной нитью \(A B\) (рис. 5.5, вид сверху). Шайбе сообщили начальную скорость \(\mathbf{v}\), как показано на рисунке.

\begin{figure}[h]
\begin{center}
  \includegraphics[alt={},max width=\textwidth]{66a35da9-3faa-4823-9ca9-40799f37b3db-160_154_332_1490_828}
\captionsetup{labelformat=empty}
\caption{Рис. 5.5}
\end{center}
\end{figure}

Есть ли здесь точка, относительно которой момент импульса шайбы будет оставаться постоянным в процессе движения?

\begin{displayquote}
В данном случае единственная некомпенсированная сила, действующая на шайбу \(A\), - это сила натяжения \(\mathbf{F}\) со стороны нити. Нетрудно видеть, что точки, относительно которой момент силы \(\mathbf{F}\) в процессе движения был бы все время равен нулю, здесь нет. А следовательно, нет и точки, относительно которой момент импульса шайбы оставался бы постоянным.
\end{displayquote}

\begin{displayquote}
Этот пример показывает, что не всегда существует точка, относительно которой момент импульса частицы оставался бы постоянным.
\end{displayquote}

Уравнение моментов (5.5) позволяет получить ответ на два вопроса:

\begin{enumerate}
  \item найти момент силы \(\mathbf{N}\) относительно интересующей нас точки \(O\) в любой момент времени \(t\), если известна зависимость от времени момента импульса \(\mathbf{M}(t)\) частицы, относительно той же точки;
  \item определить приращение момента импульса частицы относительно точки \(O\) за любой промежуток времени, если известна зависимость от времени момента силы \(\mathbf{N}(t)\), действующего на эту частицу (относительно той же точки \(O\) ).
\end{enumerate}

Решение первого вопроса сводится к нахождению производной по времени от момента импульса; т. е. \(\mathrm{dM} / \mathrm{d} t\), которая и равна, согласно (5.5), искомому моменту силы N. Решение второго вопроса сводится к интегрированию уравнения (5.5). Умножив обе части этого уравнения на \(\mathrm{d} t\), получим \(\mathrm{d} \mathbf{M}=\mathbf{N} \mathrm{d} t\) выражение, которое определяет элементарное приращение вектора М. Проинтегрировав это выражение по времени, найдем приращение вектора М за конечный промежуток времени \(t\) :


\begin{equation*}
\mathbf{M}_{2}-\mathbf{M}_{1}=\int_{0}^{t} \mathbf{N} \mathrm{~d} t . \tag{5.6}
\end{equation*}


Величину, стоящую в правой части этого уравнения, называют импульсом момента силь. Таким образом, приращение момента импульса частицы за любой промежуток времени равно импульсу момента силы за это же время.

Рассмотрим два примера.

Пример 1. Момент импульса частицы относительно некоторой точки меняется со временем \(t\) по закону \(\mathbf{M}(t)=\mathbf{a}+\mathbf{b} t^{2}\), где \(\mathbf{a}\) и \(\mathbf{b}\) - некоторые постоянные векторы, причем \(\mathbf{a} \perp \mathbf{b}\). Найдем момент силы N, действующий на частицу, когда угол между векторами \(\mathbf{N}\) и \(\mathbf{M}\) окажется равным \(45^{\circ}\).

Согласно (5.5), \(\mathbf{N}=\mathrm{d} \mathbf{M} / \mathrm{d} t=2 \mathbf{b} t\), т. е. вектор \(\mathbf{N}\) все время совпадает по направлению с вектором b. Изобразим векторы N и \(\mathbf{M}\) в некоторый момент \(t\) (рис. 5.6).

Из этого рисунка видно, что угол \(\alpha=45^{\circ}\) в момент \(t_{0}\), когда \(a=b t_{0}^{2}\). Отсюда \(t_{0}=\sqrt{a / b}\) и \(\mathbf{N}=2 \sqrt{a / b} \cdot \mathbf{b}\).

\begin{figure}[h]
\begin{center}
  \includegraphics[alt={},max width=\textwidth]{66a35da9-3faa-4823-9ca9-40799f37b3db-162_283_306_359_826}
\captionsetup{labelformat=empty}
\caption{Рис. 5.6}
\end{center}
\end{figure}

Пример 2. Камень \(A\) массы \(m\) бросили под углом к горизонту с начальной скоростью \(\mathbf{v}_{0}\). Пренебрегая сопротивлением воздуха, найдем зависимость от времени момента импульса камня \(\mathbf{M}(t)\) относительно точки бросания \(O\) (рис. 5.7).

За промежуток времени \(\mathrm{d} t\) момент импульса камня относительно точки \(O\) получит приращение \(\mathrm{dM}=\mathbf{N} \mathrm{d} t=[\mathbf{r}, m \mathbf{g}] \mathrm{d} t\). Так как \(\mathbf{r}=\mathbf{v}_{0} t+\mathbf{g} t^{2} / 2\) (см. стр. 11), то \(\mathrm{d} \mathbf{M}=\left[\mathbf{v}_{0}, m \mathbf{g}\right] t \mathrm{~d} t\). Проинтегрировав это выражение с учетом того, что в момент \(\quad t=0 \quad \mathbf{M}(0)=0\), получим \(\mathbf{M}(t)=\left[\mathbf{v}_{0}, m \mathbf{g}\right] t^{2} / 2\). Отсюда видно, что направление вектора \(\mathbf{M}\) остается неизменным в процессе движения (вектор M направлен за плоскость рис. 5.7).

\begin{figure}[h]
\begin{center}
  \includegraphics[alt={},max width=\textwidth]{66a35da9-3faa-4823-9ca9-40799f37b3db-162_211_382_1066_778}
\captionsetup{labelformat=empty}
\caption{Рис. 5.7}
\end{center}
\end{figure}

\section*{Момент импульса и момент силы относительно оси}
Возьмем в интересующей нас системе отсчета произвольную неподвижную ось \(Z\). Пусть относительно некоторой точки \(O\) на оси \(Z\) момент импульса частицы \(A\) равен \(\mathbf{M}\), а момент силы, действующий на частицу, - N.

Моментом импульса относительно оси \(Z\) называют проекцию на эту ось вектора М, определенного относительно произвольной точки \(O\) данной оси (рис. 5.8). Аналогично вводят понятие момента силы относительно

\begin{figure}[h]
\begin{center}
  \includegraphics[alt={},max width=\textwidth]{66a35da9-3faa-4823-9ca9-40799f37b3db-162_233_225_1544_899}
\captionsetup{labelformat=empty}
\caption{Рис. 5.8}
\end{center}
\end{figure}

оси. Их обозначают соответственно \(M_{z}\) и \(N_{z}\). Далее мы увидим, что \(M_{z}\) и \(N_{z}\) не зависят от выбора точки \(O\) на оси \(Z\).

Выясним свойства этих величин. Записав уравнение (5.5) в проекциях на ось \(Z\), получим


\begin{equation*}
\mathrm{d} M_{z} / \mathrm{d} t=N_{z}, \tag{5.7}
\end{equation*}


т. е. производная по времени от момента импульса частицы относительно оси \(Z\) равна моменту силы относительно этой оси. В частности, если \(N_{z} \equiv 0\), то \(M_{z}=\) const. Другими словами, если момент силы относительно некоторой неподвижной оси \(Z\) равен нулю, то момент импульса частицы относительно этой оси остается постоянным. При этом вектор М может и меняться.

Пример. Небольшое тело массы \(m\), подвешенное на нити, равномерно

\begin{figure}[h]
\begin{center}
  \includegraphics[alt={},max width=\textwidth]{66a35da9-3faa-4823-9ca9-40799f37b3db-163_400_251_728_219}
\captionsetup{labelformat=empty}
\caption{Рис. 5.9}
\end{center}
\end{figure}

движется по горизонтальной окружности (рис. 5.9) под действием силы тяжести \(m g\) и силы натяжения T со стороны нити. Относительно точки \(O\) момент импульса тела - вектор М - находится в одной плоскости с осью \(Z\) и нитью, и при движении тела вектор \(\mathbf{M}\) под действием момента \(\mathbf{N}\) силы тяжести все время поворачивается, т.е. меняется. Проекция же \(M_{z}\) остается при этом постоянной, так как вектор N перпендикулярен оси \(Z\) и \(N_{z}=0\).

Найдем теперь аналитические выражения для \(M_{z}\) и \(N_{z}\). Heтрудно видеть, что эта задача сводится к нахождению проекций на ось \(Z\) векторных произведений [rp] и [rF].

\begin{figure}[h]
\begin{center}
  \includegraphics[alt={},max width=\textwidth]{66a35da9-3faa-4823-9ca9-40799f37b3db-163_395_281_1323_80}
\captionsetup{labelformat=empty}
\caption{Рис. 5.10}
\end{center}
\end{figure}

Воспользуемся цилиндрической системой координат \(\rho, \varphi, z\), связав с частицей \(A\) (рис. 5.10) орты \(\mathbf{e}_{\rho}, \mathbf{e}_{\varphi}, \mathbf{e}_{z}\), направленные в сторону возрастания соответствующих координат. В этой системе координат радиус-вектор \(\mathbf{r}\) и импульс \(\mathbf{p}\) частицы записывают так:

\[
\mathbf{r}=\rho \mathbf{e}_{\rho}+z \mathbf{e}_{z}, \quad \mathbf{p}=p_{\rho} \mathbf{e}_{\rho}+p_{\varphi} \mathbf{e}_{\varphi}+p_{z} \mathbf{e}_{z},
\]

где \(p_{\rho}, p_{\varphi}, p_{z}\) - проекции вектора \(\mathbf{p}\) на соответствующие орты. Из векторной алгебры

известно, что векторное произведение [rp] можно представить определителем

\[
\mathbf{M}=[\mathbf{r p}]=\left|\begin{array}{ccc}
\mathbf{e}_{\rho} & \mathbf{e}_{\varphi} & \mathbf{e}_{z} \\
\rho & 0 & z \\
p_{\rho} & p_{\varphi} & p_{z}
\end{array}\right|,
\]

откуда сразу видно, что момент импульса частицы относительно оси \(Z\)


\begin{equation*}
M_{z}=\rho p_{\varphi} \tag{5.8}
\end{equation*}


где \(\rho\) - расстояние частицы от оси \(Z\). Преобразуем это выражение к виду, более удобному для практических применений. Имея в виду, что \(p_{\varphi}=m v_{\varphi}=m \rho \omega_{z}\), получаем


\begin{equation*}
M_{z}=m \rho^{2} \omega_{z}, \tag{5.9}
\end{equation*}


где \(\omega_{z}\) - проекция угловой скорости \(\omega\), с которой поворачивается радиус-вектор частицы.

Аналогично (5.8) записывается и момент силы относительно оси \(Z\) :


\begin{equation*}
N_{z}=\rho F_{\varphi} \tag{5.10}
\end{equation*}


где \(F_{\varphi}\) - проекция вектора силы \(\mathbf{F}\) на орт \(\mathbf{e}_{\varphi}\).\\
Обратим внимание, что проекции \(M_{z}\) и \(N_{z}\) действительно не зависят от выбора точки \(O\) на оси \(Z\), относительно которой определены векторы М и N. Кроме того, видно, что \(M_{z}\) и \(N_{z}\) величины алгебраические, их знаки соответствуют знакам проекций \(p_{\varphi}\) и \(F_{\varphi}\).

\section*{§ 5.2. Закон сохранения момента импульса}
Выберем произвольную систему частиц. Введем понятие момента импульса данной системы как векторную сумму моментов импульсов ее отдельных частиц:


\begin{equation*}
\mathbf{M}=\sum \mathbf{M}_{i} \tag{5.11}
\end{equation*}


где все векторы определены относительно одной и той же точки \(O\) заданной системы отсчета. Заметим, что момент импульса

системы - величина аддитивная: момент импульса системы равен сумме моментов импульсов ее отдельных частей независимо от того, взаимодействуют они между собой или нет.

Выясним, какая величина определяет изменение момента импульса системы. Для этого продифференцируем (5.11) по времени: \(\mathrm{dM} / \mathrm{d} t=\sum \mathrm{d} \mathbf{M}_{i} / \mathrm{d} t\). В предыдущем параграфе было показано, что производная \(\mathrm{d} \mathbf{M}_{i} / \mathrm{d} t\) равна моменту всех сил, действующих на \(i\)-ю частицу. Представим этот момент в виде суммы моментов внутренних и внешних сил, т. е. \(\mathbf{N}_{i}^{\prime}+\mathbf{N}_{i}\). Тогда

\[
\mathrm{d} \mathbf{M} / \mathrm{d} t=\sum \mathbf{N}_{i}^{\prime}+\sum \mathbf{N}_{i} .
\]

Здесь первая сумма - это суммарный момент всех внутренних сил относительно точки \(O\), вторая сумма - суммарный момент всех внешних сил относительно той же точки \(O\).

Покажем, что суммарный момент всех внутренних сил относительно любой точки равен нулю. Действительно, внутренние силы - это силы взаимодействия между частицами данной системы. По третьему закону Ньютона, эти силы попарно одинаковы по модулю, противоположны по направлению и лежат на одной прямой, т. е. имеют одинаковое плечо. Поэтому моменты сил каждой пары взаимодействия равны по модулю и противоположны по направлению, т. е. уравновешивают друг друга, и значит, суммарный момент всех внутренних сил всегда равен нулю.

В результате последнее уравнение принимает вид


\begin{equation*}
\mathrm{d} \mathbf{M} / \mathrm{d} t=\mathbf{N}_{\text {внеш }}, \tag{5.12}
\end{equation*}


где \(\mathbf{N}_{\text {внеш }}\) - суммарный момент всех внешних сил, \(\mathbf{N}_{\text {внеш }}=\sum \mathbf{N}_{i}\).\\
Уравнение (5.12) утверждает: производная момента импульса системь по времени равна суммарному моменту всех внешних сил. Разумеется, оба момента, \(\mathbf{M}\) и \(\mathbf{N}\), здесь определены относительно одной и той же точки \(O\) заданной системы отсчета.

Как и в случае одной частицы, из уравнения (5.12) следует, что приращение момента импульса системы за конечный промежуток времени \(t\)


\begin{equation*}
\mathbf{M}_{2}-\mathbf{M}_{1}=\int_{0}^{t} \mathbf{N}_{\text {внеш }} \mathrm{d} t \tag{5.13}
\end{equation*}


т. е. приращение момента импульса системы равно импульсу суммарного момента всех внешних сил за соответствующий промежуток времени. И здесь, конечно, оба момента, М и \(\mathbf{N}_{\text {внеш }}\), определены относительно одной и той же точки \(O\) выбранной системы отсчета.

Уравнения (5.12) и (5.13) справедливы как в инерциальной, так и в неинерциальной системах отсчета. Только в неинерциальной системе отсчета нужно учитывать и действие сил инерции, играющих роль внешних сил, т. е. под \(\mathbf{N}_{\text {внеш }}\) в этих уравнениях следует понимать сумму \(\mathbf{N}_{\text {вз }}+\mathbf{N}_{\text {ин }}\), где \(\mathbf{N}_{\text {вз }}\) - суммарный момент внешних сил взаимодействия, \(\mathbf{N}_{\text {ин }}\) - суммарный момент сил инерции (относительно одной и той же точки \(O\) системы отсчета).

Итак, мы пришли к важному выводу: согласно уравнению (5.12), момент импульса системь может изменяться под действием только суммарного момента всех внешних сил. Отсюда непосредственно вытекает и другой важный вывод - закон сохранения момента импульса: момент импульса замкнутой системьь частиц остается постоянным, т. е. не меняется со временем, причем это справедливо для момента импульса, взятого относительно любой точки инерциальной системы отсчета.

Таким образом, в инерциальной системе отсчета момент импульса замкнутой системы частиц


\begin{equation*}
\mathbf{M}=\sum \mathbf{M}_{i}(t)=\text { const } \tag{5.14}
\end{equation*}


При этом моменты импульса отдельных частей или частиц замкнутой системы могут изменяться со временем, что и подчеркнуто в последнем выражении. Однако эти изменения всегда происходят так, что приращение момента импульса одной части системы равно убыли момента импульса ее другой части (конечно, относительно одной и той же точки системы отсчета).

В этом смысле уравнения (5.12) и (5.13) можно рассматривать как более общую формулировку закона сохранения момента импульса, формулировку, в которой указана и причина изменения момента импульса интересующей нас системы - действие других тел (через момент внешних сил взаимодействия). Сказанное, разумеется, справедливо только по отношению к инерциальным системам отсчета.

Подчеркнем еще раз: закон сохранения момента импульса имеет место только по отношению к инерциальным системам отсчета. Однако это не исключает случаев, когда момент импульса системы сохраняется и в неинерциальных системах отсчета. Для этого достаточно, чтобы согласно уравнению (5.12) - а оно справедливо и в неинерциальных системах отсчета - суммарный момент всех внешних сил (включая и силы инерции) был равен нулю. Такие ситуации реализуются довольно редко и соответствующие случаи имеют весьма частный характер.

Закон сохранения момента импульса играет такую же важную роль, как и законы сохранения энергии и импульса. Уже сам по себе он позволяет сделать во многих случаях ряд существенных заключений о свойствах тех или иных процессов, совершенно не вникая в их детальное рассмотрение. Проиллюстрируем сказанное на таком примере.\\
Пример. Два одинаковых шара насажены на гладкий горизонтальный

\begin{figure}[h]
\begin{center}
  \includegraphics[alt={},max width=\textwidth]{66a35da9-3faa-4823-9ca9-40799f37b3db-167_242_480_850_199}
\captionsetup{labelformat=empty}
\caption{Рис. 5.11}
\end{center}
\end{figure}

стержень, по которому они могут скользить (рис. 5.11). Шары сближают и соединяют нитью. Затем всю установку приводят во вращение вокруг вертикальной оси, предоставляют ее самой себе и пережигают нить. Шары, естественно, разлетаются к концам стержня. Угловая скорость установки при этом резко уменьшается.

Наблюдаемый эффект является прямым следствием закона сохранения момента импульса. Данная установка ведет себя, по существу, как замкнутая: внешние силы компенсируют друг друга, силы трения в оси предполагаются пренебрежимо малыми. Для количественной оценки изменения угловой скорости будем считать, что масса всей установки практически сосредоточена в шарах, а их размеры достаточно малы. Тогда из равенства моментов импульса шаров относительно точки \(C\) в начальном и конечном состояниях системы, а именно \(2 m\left[\mathbf{r}_{1} \mathbf{v}_{1}\right]=2 m\left[\mathbf{r}_{2} \mathbf{v}_{2}\right]\), следует

\[
r_{1}^{2} \omega_{1}=r_{2}^{2} \omega_{2} .
\]

Отсюда видно, что с увеличением расстояния \(r\) шаров от оси вращения угловая скорость установки уменьшается (как\\
\(1 / r^{2}\) ). И наоборот, если бы расстояние между шарами уменьшалось (под действием каких-либо внутренних сил), угловая скорость установки увеличивалась бы. Этот эффект имеет общий характер, и его широко используют, например, фигуристы и гимнасты.

Обратим внимание на тот факт, что конечный результат совершенно не зависит от характера внутренних сил (здесь это силы трения между шарами и стержнем).

Особый интерес представляют случаи, когда момент импульса М сохраняется для незамкнутых систем, у которых, как известно, импульс р меняется со временем. Если относительно некоторой точки \(O\) выбранной системы отсчета суммарный момент внешних сил \(\mathbf{N}_{\text {внеш }} \equiv 0\) в течение интересующего нас промежутка времени, то, согласно (5.12), момент импульса системы относительно точки \(O\) сохраняется за это время. В незамкнутых системах такой точки, вообще говоря, может и не быть, что следует прежде всего выяснить для каждого конкретного случая.

Пример 1. Система Земля-Луна, движущаяся в поле тяготения Солнца, является незамкнутой. Ее импульс все время меняется под действием сил тяготения со стороны Солнца. Здесь, однако, имеется одна точка, относительно которой момент сил тяготения, действующих на данную систему, все время равен нулю, - это центр Солнца. Поэтому можно сразу утверждать, что момент импульса системы Земля-Луна относительно центра Солнца остается постоянным.\\
Пример 2. На гладкой горизонтальной плоскости лежит стержень \(O B\), который может свободно вращаться вокруг неподвижной вертикальной оси, проходящей через его конец \(O\) (рис. 5.12). В конец \(B\) стержня попадает, застревая, шайба \(A\), скользившая по плоскости, и вся система начинает вращаться как единое целое вокруг

\begin{figure}[h]
\begin{center}
  \includegraphics[alt={},max width=\textwidth]{66a35da9-3faa-4823-9ca9-40799f37b3db-168_239_374_1308_790}
\captionsetup{labelformat=empty}
\caption{Рис. 5.12}
\end{center}
\end{figure}

точки \(O\).

Ясно, что система шайба-стержень незамкнутая: кроме сил, уравновешивающих друг друга в вертикальном направлении, со стороны оси в процессе удара будет действовать горизонтальная сила, а после того, как стержень нач-

нет вращаться, возникает еще одна сила со стороны оси, благодаря которой центр масс системы будет двигаться по окружности. Но обе силы проходят через точку \(O\), а следовательно, момент этих внешних сил относительно точки \(O\) все время равен нулю. Отсюда вывод: момент импульса данной системы будет оставаться постоянным относительно точки \(O\).

В более ограниченном случае у незамкнутых систем может сохраняться не сам момент импульса \(\mathbf{M}\), а его проекция на некоторую неподвижную ось \(Z\). Это бывает тогда, когда проекция суммарного момента \(\mathbf{N}_{\text {внеш }}\) всех внешних сил на эту ось \(Z\) равна нулю. В самом деле, записав уравнение (5.12) в проекциях на ось \(Z\), получим


\begin{equation*}
\mathrm{d} M_{z} / \mathrm{d} t=N_{\text {внеш } z} . \tag{5.15}
\end{equation*}


Здесь \(M_{z}\) и \(N_{\text {внеш } z}\) - момент импульса и суммарный момент внешних сил относительно оси \(Z\) :


\begin{equation*}
M_{z}=\sum M_{i z}, \quad N_{\text {внеш } z}=\sum N_{i z}, \tag{5.16}
\end{equation*}


где \(M_{i z}\) и \(N_{i z}\) - момент импульса и момент внешних сил относительно оси \(Z\) для \(i\)-й частицы системы.

Из уравнения (5.15) следует, что если относительно некоторой неподвижной в данной системе отсчета оси \(Z\) проекция \(N_{\text {внеш } z} \equiv 0\), то момент импульса системы относительно этой оси сохраняется:


\begin{equation*}
M_{z}=\sum M_{i z}(t)=\text { const } . \tag{5.17}
\end{equation*}


При этом вектор М, определенный относительно произвольной точки \(O\) на этой оси, может меняться. Например, если система движется в однородном поле тяжести, то суммарный момент всех сил тяжести относительно любой неподвижной точки \(O\) перпендикулярен вертикали, а значит, относительно любой вертикальной оси \(N_{\text {внеш } z} \equiv 0\) и \(M_{z}=\) const, чего нельзя сказать о векторе М.

Рассуждения, которые приводят к закону сохранения момента импульса, целиком опираются на справедливость законов Ньютона. А как обстоит дело в системах, не подчиняющих-

ся этим законам, например в системах с электромагнитным излучением, в атомах, ядрах и др.?

Учитывая громадную роль, которую играет закон сохранения момента импульса, в физике понятие момента импульса расширяют на немеханические системы (которые не подчиняются законам Ньютона) и постулируют закон сохранения момента импульса для всех физических процессов.

Такой расширенный закон сохранения момента импульса уже не является следствием законов Ньютона, а представляет собой самостоятельный общий принцип, являющийся обобщением опьтных фактов. Наряду с законами сохранения энергии и импульса закон сохранения момента импульса является одним из фундаментальных законов природь.

\section*{§ 5.3. Собственный момент импульса}
В § 5.2 было установлено, что момент импульса М системы изменяется только под действием суммарного момента \(\mathbf{N}\) всех внешних сил; именно этот вектор \(\mathbf{N}\) определяет поведение вектора М. Теперь рассмотрим некоторые наиболее существенные свойства этих величин и те важные выводы, которые из них вытекают.

\section*{Суммарный момент внешних сил}
Как и момент каждой силы, суммарный момент сил зависит, вообще говоря, от выбора точки, относительно которой его определяют. Пусть N - суммарный момент сил относительно точки \(O\), а \(\mathbf{N}^{\prime}\) - относительно точки \(O^{\prime}\), ра-диус-вектор которой \(\mathbf{r}_{0}\) (рис. 5.13). Найдем связь между \(\mathbf{N}\) и \(\mathbf{N}^{\prime}\).

Радиусы-векторы \(\mathbf{r}_{i}\) и \(\mathbf{r}_{i}^{\prime}\) точки прило-

\begin{figure}[h]
\begin{center}
  \includegraphics[alt={},max width=\textwidth]{66a35da9-3faa-4823-9ca9-40799f37b3db-170_229_320_1181_834}
\captionsetup{labelformat=empty}
\caption{Рис. 5.13}
\end{center}
\end{figure}

жения силы \(\mathbf{F}_{i}\) связаны соотношением \(\mathbf{r}_{i}=\mathbf{r}_{i}^{\prime}+\mathbf{r}_{0}\) (рис. 5.13). Поэтому выражение для \(\mathbf{N}\) можно записать в виде

\[
\mathbf{N}=\sum\left[\mathbf{r}_{i} \mathbf{F}_{i}\right]=\sum\left[\mathbf{r}_{i}^{\prime} \mathbf{F}_{i}\right]+\sum\left[\mathbf{r}_{0} \mathbf{F}_{i}\right],
\]

или


\begin{equation*}
\mathbf{N}=\mathbf{N}^{\prime}+\left[\mathbf{r}_{0} \mathbf{F}\right], \tag{5.18}
\end{equation*}


где \(\mathbf{F}=\sum \mathbf{F}_{i}\) - результирующая всех внешних сил.

Из формулы (5.18) видно, что если \(\mathbf{F}=0\), то суммарный момент внешних сил не зависит от выбора точки, относительно которой его определяют. Таков, в частности, случай, когда к системе приложена пара сил.\\
Пример. К телу в точках 1 и 2 приложены две одинаковые по модулю и противоположно направленные силы \(\mathbf{F}_{1}\) и \(\mathbf{F}_{2}\), не действующие вдоль одной прямой (пара сил). Пусть \(\mathbf{r}_{12}\) - радиус-вектор, проведенный из точки 1 в точку 2 . Найдем суммарный момент N этой пары сил.\\
Здесь результирующая сила \(\mathbf{F}=\mathbf{F}_{1}+\mathbf{F}_{2}=0\), поэтому согласно (5.18) момент N этой пары сил не должен зависеть от выбора точки \(O\), относительно которой его определяют. Воспользовавшись этим, выберем в качестве точки \(O\) точку 1 (относительно нее момент силы \(\mathbf{F}_{1}\) равен нулю), тогда

\[
\mathbf{N}=\left[\mathbf{r}_{12} \mathbf{F}_{2}\right] .
\]

Модуль вектора \(\mathbf{N}\) равен, как нетрудно сообразить, \(N=l F\), где \(l\) - плечо пары, т. е. расстояние между прямыми, вдоль которых действуют силы, а \(F\) - модуль каждой силы.\\
Интересной и важной особенностью в этом отношении обладает Ц-система (напомним, что эта система отсчета жестко связана с центром масс системы частиц и перемещается поступательно по отношению к инерциальным системам). Так как в общем случае Ц-система является неинерциальной, то результирующая всех внешних сил должна включать в себя кроме внешних сил взаимодействия \(\mathbf{F}_{\text {вз }}\) и силы инерции \(\mathbf{F}_{\text {ин }}\). C другой стороны, в Ц-системе система частиц как целое покоится, а это значит, согласно (3.11), что \(\mathbf{F}=\mathbf{F}_{\text {вз }}+\mathbf{F}_{\text {ин }}=0\). Имея в виду (5.18), мы приходим к следующему важному выводу: в Ц-системе суммарный момент всех внешних сил, включая силь инерции, не зависит от выбора точки О.

И другой важный вывод: в Ц-системе суммарный момент сил инерции относительно центра масс всегда равен нулю:


\begin{equation*}
\mathbf{N}_{C}^{\text {ин }}=\mathbf{0} . \tag{5.19}
\end{equation*}


В самом деле, сила инерции, действующая на каждую частицу системы, \(\mathbf{F}_{i}=-m_{i} \mathbf{a}_{0}\), где \(\mathbf{a}_{0}\) - ускорение Ц-системы. Поэтому суммарный момент всех этих сил относительно центра масс \(C\)

\[
\mathbf{N}_{C}^{\mathrm{HH}}=\sum\left[\mathbf{r}_{i},-m_{i} \mathbf{a}_{0}\right]=-\left[\left(\sum m_{i} \mathbf{r}_{i}\right), \mathbf{a}_{0}\right] .
\]

Согласно (3.8), \(\sum m_{i} \mathbf{r}_{i}=m \mathbf{r}_{C}\), а так как в нашем случае \(\mathbf{r}_{C}=0\), то и \(\mathbf{N}_{C}^{\text {ин }}=0\).

\section*{Собственный момент импульса}
Как и момент сил, момент импульса системы зависит, вообще говоря, от выбора точки \(O\), относительно которой его определяют. При переносе этой точки на расстояние \(\mathbf{r}_{0}\) (рис. 5.13) новые радиусы-векторы частиц \(\mathbf{r}_{i}^{\prime}\) связаны со старыми \(\mathbf{r}_{i}\) формулой \(\mathbf{r}_{i}=\mathbf{r}_{i}^{\prime}+\mathbf{r}_{0}\). Поэтому момент импульса системы относительно точки \(O\) можно представить так:

\[
\mathbf{M}=\sum\left[\mathbf{r}_{i} \mathbf{p}_{i}\right]=\sum\left[\mathbf{r}_{i}^{\prime} \mathbf{p}_{i}\right]+\sum\left[\mathbf{r}_{0} \mathbf{p}_{i}\right],
\]

или


\begin{equation*}
\mathbf{M}=\mathbf{M}^{\prime}+\left[\mathbf{r}_{0} \mathbf{p}\right] \tag{5.20}
\end{equation*}


где \(\mathbf{M}^{\prime}\) - момент импульса системы относительно точки \(O^{\prime}\), а \(\mathbf{p}=\sum \mathbf{p}_{i}\) - полный импульс системы.

Из формулы (5.20) следует, что если полный импульс системы \(\mathbf{p}=0\), то ее момент импульса не зависит от выбора точки \(O\). А этим как раз и отличается Ц-система, в которой система частиц как целое покоится. Отсюда мы приходим к третьему важному выводу: в Ц-системе момент импульса системь частиц не зависит от выбора точки, относительно которой его определяют.

Этот момент называют собственным моментом импульса системы и обозначают \(\tilde{\mathbf{M}}\).

\section*{Связь между М и \(\tilde{\mathbf{M}}\)}
Пусть М - момент импульса системы частиц относительно точки \(O K\)-системы отсчета. Так как собственный момент импульса \(\tilde{\mathbf{M}}\) в Ц-системе не зависит от выбора точки \(O^{\prime}\), возьмем точку \(O^{\prime}\) совпадающей в данный момент с точкой \(O K\)-системы. Тогда радиусы-векторы каждой частицы в обеих системах отсчета будут одинаковы в этот момент ( \(\mathbf{r}_{i}^{\prime}=\mathbf{r}_{i}\) ), скорости частиц связаны формулой


\begin{equation*}
\mathbf{v}_{i}=\tilde{\mathbf{v}}_{i}+\mathbf{V}_{C}, \tag{5.21}
\end{equation*}


где \(\mathbf{V}_{C}\) - скорость Ц-системы относительно \(K\)-системы. Поэтому


\begin{equation*}
\mathbf{M}=\sum m_{i}\left[\mathbf{r}_{i} \mathbf{v}_{i}\right]=\sum m_{i}\left[\mathbf{r}_{i} \tilde{\mathbf{v}}_{i}\right]+\sum m_{i}\left[\mathbf{r}_{i} \mathbf{V}_{C}\right] . \tag{5.22}
\end{equation*}


Первая сумма в правой части этого равенства есть собственный момент импульса \(\tilde{\mathbf{M}}\). Вторую сумму в соответствии с формулой (3.8) представим как \(m\left[\mathbf{r}_{C} \mathbf{V}_{C}\right]\), или \(\left[\mathbf{r}_{C} \mathbf{p}\right]\), где \(m\) - масса всей системы, \(\mathbf{r}_{C}\) - радиус-вектор ее центра масс в \(K\)-системе, p - суммарный импульс системы частиц. В результате


\begin{equation*}
\mathbf{M}=\tilde{\mathbf{M}}+\left[\mathbf{r}_{C} \mathbf{p}\right] \tag{5.23}
\end{equation*}


т. е. момент импульса \(\mathbf{M}\) системы частии складывается из ее собственного момента импульса \(\tilde{\mathbf{M}}\) и момента \(\left[\mathbf{r}_{C} \mathbf{p}\right]\), обусловленного движением системы частиц как целого.

Возьмем, например, однородный шар, скатывающийся по наклонной плоскости. Его момент импульса относительно некоторой точки этой плоскости складывается из момента импульса, связанного с движением центра масс шара, и собственного момента импульса, обусловленного вращением шара вокруг собстенной оси.

Из формулы (5.23), в частности, следует, что если центр масс системы покоится (импульс системы \(\mathbf{p}=0\) ), то ее момент импульса \(\mathbf{M}\) - это собственный момент импульса. С этим случаем мы уже знакомы. В другом крайнем случае, когда \(\tilde{\mathbf{M}}=0\), момент импульса системы относительно некоторой точки определяется только моментом, связанным с движением системы как целого, т. е. вторым слагаемым (5.23). Так, например, ведет себя момент импульса любого твердого тела, совершающего поступательное движение.

\section*{Уравнение моментов в Ц-системе}
В § 5.2 было отмечено, что уравнение (5.12) справедливо в любой системе отсчета. Значит, оно справедливо и в Ц-системе. Поэтому \(\mathrm{d} \tilde{\mathbf{M}} / \mathrm{d} t=\tilde{\mathbf{N}}\), где \(\tilde{\mathbf{N}}\) - суммарный момент внешних сил в Ц-системе.

Так как Ц-система в общем случае неинерциальная, то в Ñ входит помимо моментов внешних сил взаимодействия и мо-

мент сил инерции. С другой стороны, в начале этого параграфа (см. с. 175) было показано, что момент сил \(\tilde{\mathbf{N}}\) в Ц-системе не зависит от выбора точки, относительно которой его определяют. Обычно в качестве такой точки берут точку \(C\) - центр масс системы. Целесообразность выбора именно этой точки в том, что относительно ее суммарный момент сил инерции равен нулю, поэтому следует учитывать только суммарный момент внешних сил взаимодействия \(\mathbf{N}_{C}\). Итак,


\begin{equation*}
\mathrm{d} \tilde{\mathbf{M}} / \mathrm{d} t=\mathbf{N}_{C}, \tag{5.24}
\end{equation*}


т. е. производная по времени от собственного момента импульса системы равна суммарному моменту всех внешних сил взаимодействия относительно центра масс данной системы.

В частности, если \(\mathbf{N}_{C} \equiv 0\), то \(\tilde{\mathbf{M}}=\mathrm{const}\), т. е. собственный момент импульса системь сохраняется.

В проекциях на ось \(Z\), проходящую через центр масс системы, уравнение (5.24) имеет вид


\begin{equation*}
\mathrm{d} \tilde{M}_{z} / \mathrm{d} t=N_{C z}, \tag{5.25}
\end{equation*}


где \(N_{C z}\) - суммарный момент внешних сил взаимодействия относительно неподвижной в Ц-системе оси \(Z\), проходящей через центр масс. И здесь если \(N_{C z} \equiv 0\), то \(\tilde{M}_{z}=\) const.

\section*{§ 5.4. Динамика твердого тела}
Движение твердого тела в общем случае определяется двумя векторными уравнениями. Одно из них - уравнение движения центра масс (3.11), другое - уравнение моментов в Ц-системе (5.24):


\begin{equation*}
m \mathrm{~d} \mathbf{V}_{C} / \mathrm{d} t=\mathbf{F} ; \quad \mathrm{d} \tilde{\mathbf{M}} / \mathrm{d} t=\mathbf{N}_{C} . \tag{5.26}
\end{equation*}


Зная законы действующих внешних сил, точки их приложения и начальные условия, можно с помощью этих уравнений найти как скорость, так и положение каждой точки твердого тела в любой момент времени, т. е. полностью решить задачу о движении тела. Однако несмотря на кажущуюся простоту уравнений (5.26), решение их в общем случае представляет со-

бой весьма трудную задачу. И прежде всего это обусловлено тем обстоятельством, что связь между собственным моментом импульса \(\tilde{\mathbf{M}}\) и скоростями отдельных точек твердого тела в Ц-системе оказывается сложной, за исключением немногих частных случаев. Мы не будем рассматривать эту задачу в общем виде (она решается в общей теории) и ограничимся в дальнейшем только отдельными частными случаями.

Но прежде приведем некоторые соображения, прямо вытекающие из вида самих уравнений (5.26). Если мы будем переносить силы вдоль направления их действия, то ясно, что не изменяется ни их результирующая \(\mathbf{F}\), ни их суммарный момент \(\mathbf{N}_{C}\). При этом уравнения (5.26) тоже не изменятся, а следовательно, не изменится и движение твердого тела. Поэтому точки приложения внешних сил можно переносить вдоль направления действия сил - прием, которым постоянно пользуются.

\section*{Равнодействующая сила}
В тех случаях, когда суммарный момент всех внешних сил оказывается перпендикулярным резуль-

\begin{figure}[h]
\begin{center}
  \includegraphics[alt={},max width=\textwidth]{66a35da9-3faa-4823-9ca9-40799f37b3db-175_265_380_1004_70}
\captionsetup{labelformat=empty}
\caption{Рис. 5.14}
\end{center}
\end{figure}

тирующей силе, т. е. \(\mathbf{N} \perp \mathbf{F}\), все внешние силы могут быть сведены к одной силе \(\mathbf{F}\), действующей вдоль определенной прямой. В самом деле, если относительно некоторой точки \(O\) суммарный момент \(\mathbf{N} \perp \mathbf{F}\), то всегда можно найти такой вектор \(\mathbf{r}_{0} \perp \mathbf{N}\) (рис. 5.14), что при заданных N и F

\[
\mathbf{N}=\left[\mathbf{r}_{0} \mathbf{F}\right] .
\]

При этом выбор \(\mathbf{r}_{0}\) неоднозначен: прибавление к нему любого вектора \(\mathbf{r}\), параллельного \(\mathbf{F}\), не изменит последнего равенства. А это означает, что данное равенство определяет не точку «приложения» силы \(\mathbf{F}\), а линию ее действия. Зная модули \(N\) и \(F\) соответствующих векторов, можно найти плечо \(l\) силы \(\mathbf{F}\) (рис. 5.14): \(l=N / F\).

Таким образом, если \(\mathbf{N} \perp \mathbf{F}\), систему сил, действующих на отдельные точки твердого тела, можно заменить одной равно-

действующей силой - силой, которая равна результирующей \(\mathbf{F}\) и создает момент, равный суммарному моменту N всех внешних сил.

Таков, в частности, случай однородного силового поля, например поля тяжести, в котором действующая на каждую частицу сила имеет вид \(\mathbf{F}_{i}=m_{i} \mathbf{g}\). В этом случае суммарный момент сил тяжести относительно любой точки \(O\) равен

\[
\mathbf{N}=\sum\left[\mathbf{r}_{i}, m_{i} \mathbf{g}\right]=\left[\left(\sum m_{i} \mathbf{r}_{i}\right) \mathbf{g}\right] .
\]

Стоящая в круглых скобках сумма, согласно (3.8), равна \(m \mathbf{r}_{C}\), где \(m\) - масса тела, \(\mathbf{r}_{C}\) - радиус-вектор его центра масс относительно точки \(O\). Поэтому

\[
\mathbf{N}=\left[m \mathbf{r}_{C}, \mathbf{g}\right]=\left[\mathbf{r}_{C}, m \mathbf{g}\right] .
\]

Это значит, что равнодействующая \(m g\) сил тяжести проходит через центр масс тела. Обычно говорят, что равнодействующая сил тяжести «приложена» к центру масс тела или к его центру тяжести. Ясно, что момент этой силы относительно центра масс тела равен нулю.

\section*{Условия равновесия твердого тела}
Тело будет оставаться в состоянии покоя, если нет причин, вызывающих его движение. Согласно уравнениям (5.26), для этого необходимо и достаточно выполнение двух условий:

\begin{enumerate}
  \item результирующая всех внешних сил, приложенных к телу, должна быть равной нулю, т. е.
\end{enumerate}

\[
\mathbf{F}=\sum \mathbf{F}_{i \text { внеш }}=0 ;
\]

\begin{enumerate}
  \setcounter{enumi}{1}
  \item суммарный момент внешних сил относительно любой точки тоже должен быть равен нулю, т. е.
\end{enumerate}

\[
\mathbf{N}=\sum \mathbf{N}_{i \text { внеш }}=0 .
\]

Эти условия должны быть выполнены в той системе отсчета, где тело покоится. Если система отсчета неинерциальная, то кроме внешних сил взаимодействия надо учитывать и силы инерции. Это же относится и к моментам сил.

Теперь перейдем к рассмотрению четырех частных случаев движения твердого тела: 1) вращение вокруг неподвижной оси, 2) плоское движение, 3) вращение вокруг свободных осей, 4) особый случай движения тела с одной неподвижной точкой (гироскопы).

\section*{1. Вращение вокруг неподвижной оси}
Найдем сначала выражение для момента импульса твердого тела относительно оси вращения \(O O^{\prime}\) (рис. 5.15). Воспользовавшись формулой (5.9), запишем

\[
M_{z}=\sum M_{i z}=\left(\sum m_{i} \rho_{i}^{2}\right) \omega_{z},
\]

\begin{figure}[h]
\begin{center}
  \includegraphics[alt={},max width=\textwidth]{66a35da9-3faa-4823-9ca9-40799f37b3db-177_285_259_627_86}
\captionsetup{labelformat=empty}
\caption{Рис. 5.15}
\end{center}
\end{figure}

где \(m_{i}\) и \(\rho_{i}\) - масса и расстояние от оси вращения \(i\)-й частицы твердого тела, \(\omega_{z}\) - его угловая скорость. Обозначив величину, стоящую в круглых скобках, через \(I\), получим


\begin{equation*}
M_{z}=I \omega_{z} \tag{5.27}
\end{equation*}


где \(I\) - момент инерции твердого тела относительно оси \(O O^{\prime}\) :


\begin{equation*}
I=\sum m_{i} \rho_{i}^{2} . \tag{5.28}
\end{equation*}


Момент инерции твердого тела зависит, как нетрудно видеть, от распределения масс относительно интересующей нас оси и является величиной аддитивной.

Вычисление момента инерции тела проводится по формуле

\[
I=\int r^{2} \mathrm{~d} m=\int \rho r^{2} \mathrm{~d} V
\]

где \(\mathrm{d} m\) и \(\mathrm{d} V\) - масса и объем элемента тела, находящегося на расстоянии \(r\) от интересующей нас оси \(Z ; \rho\) - плотность тела в данной точке.

Моменты инерции некоторых однородных твердых тел относительно оси \(Z_{C}\), проходящей через центр масс тела, приведены в следующей таблице (здесь \(m\) - масса тела):

\begin{center}
\begin{tabular}{|l|l|l|}
\hline
Твердое тело & Ось \(Z_{C}\) & Момент инерции \\
\hline
Тонкий стержень длины \(l\) & Перпендикулярна стержню & \(\frac{1}{12} m l^{2}\) \\
\hline
Сплошной цилиндр радиуса \(R\) & Совпадает с осью цилиндра & \(\frac{1}{2} m R^{2}\) \\
\hline
Тонкий диск радиуса \(R\) & Совпадает с диаметром диска & \(\frac{1}{2} m R^{2}\) \\
\hline
Шар радиуса \(R\) & Проходит через центр шара & \(\frac{2}{5} m R^{2}\) \\
\hline
\end{tabular}
\end{center}

Вычисление момента инерции твердого тела произвольной формы относительно той или иной оси представляет собой, вообще говоря, довольно кропотливую в математическом отношении задачу. Однако в некоторых случаях нахождение момента инерции значительно упрощается, если воспользоваться теоремой Штейнера: момент инерции I относительно произвольной оси \(Z\) равен моменту инериии \(I_{C}\) относительно оси \(Z_{C}\), параллельной данной и проходящей через центр масс С тела, плюс произведение массь \(m\) тела на квадрат расстояния а между осями:


\begin{equation*}
I=I_{C}+m a^{2} . \tag{5.29}
\end{equation*}


Доказательство этой теоремы дано в приложении 3.\\
Таким образом, если известен момент инерции \(I_{C}\), то нахождение момента инерции \(I\) элементарно. Например, момент инерции тонкого стержня (массы \(m\) и длины \(l\) ) относительно оси, перпендикулярной стержню и проходящей через его конец, равен

\[
I=\frac{1}{12} m l^{2}+m\left(\frac{l}{2}\right)^{2}=\frac{1}{3} m l^{2} .
\]

Уравнение динамики вращения твердого тела (ось вращения неподвижна). Это уравнение легко получить как следствие (5.15), если продифференцировать (5.27) по времени, тогда


\begin{equation*}
I \beta_{z}=N_{z} \tag{5.30}
\end{equation*}


где \(N_{z}\) - суммарный момент всех внешних сил относительно оси вращения. Из этого уравнения, в частности, видно, что момент инерции \(I\) определяет инерционные свойства твердого тела при вращении: при одном и том же значении момента сил \(N_{z}\) тело с бульшим моментом инерции приобретает меньшее угловое ускорение \(\beta_{z}\).

Напомним, что моменты сил относительно оси - величины

\begin{figure}[h]
\begin{center}
  \includegraphics[alt={},max width=\textwidth]{66a35da9-3faa-4823-9ca9-40799f37b3db-179_187_217_471_68}
\captionsetup{labelformat=empty}
\caption{Рис. 5.16}
\end{center}
\end{figure}

алгебраические: их знаки зависят как от выбора положительного направления оси \(Z\) (совпадающей с осью вращения), так и от направления «вращения» соответствующего момента силы. Например, выбрав положительное направление оси \(Z\) (рис. 5.16), мы задаем и положительное направление отсчета угла φ (оба эти направления связаны правилом правого винта). Если некоторый момент \(N_{i z}\) «вращает» в положительном направлении угла \(\varphi\), то этот момент считается положительным, и наоборот. А знак суммарного момента \(N_{z}\) в свою очередь определяет знак \(\beta_{z}\) - проекции вектора углового ускорения на ось \(Z\).

Интегрирование уравнения (5.30) с учетом начальных условий - значений \(\omega_{0 z}\) и \(\varphi_{0}\) в начальный момент времени - позволяет полностью решить задачу о вращении твердого тела вокруг неподвижной оси, т. е. найти зависимость от времени угловой скорости и угла поворота, \(\omega_{z}(t)\) и \(\varphi(t)\).

Заметим, что уравнение (5.30) справедливо в любой системе отсчета, жестко связанной с осью вращения. Однако если система отсчета неинерциальная, то момент сил \(N_{z}\) включает в себя не только моменты сил взаимодействия с другими телами, но и моменты сил инерции.

Кинетическая энергия вращающегося твердого тела (ось вращения неподвижна). Имея в виду, что скорость \(i\)-й частицы вращающегося твердого тела \(v_{i}=\rho_{i} \omega\), запишем

\[
K=\sum m_{i} v_{i}^{2} / 2=\left(\sum m_{i} \rho_{i}^{2}\right) \omega^{2} / 2,
\]

или, короче,


\begin{equation*}
K=\frac{1}{2} I \omega^{2}, \tag{5.31}
\end{equation*}


где \(I\) - момент инерции тела относительно оси вращения, \(\omega\) его угловая скорость.\\
Пример. Диск 1 (рис. 5.17) вращается вокруг гладкой вертикальной оси с угловой скоростью \(\omega_{1}\). На него падает диск 2, вращающийся с угловой скоростью \(\boldsymbol{\omega}_{2}\). Вследствие трения между ними оба диска через некоторое время начинают вращаться как единое целое. Найдем приращение кинетической энергии вращения этой системы, если моменты инерции дисков относительно оси вращения равны соответственно \(I_{1}\) и \(I_{2}\).

\begin{figure}[h]
\begin{center}
  \includegraphics[alt={},max width=\textwidth]{66a35da9-3faa-4823-9ca9-40799f37b3db-180_273_247_278_913}
\captionsetup{labelformat=empty}
\caption{Рис. 5.17}
\end{center}
\end{figure}

Сначала найдем установившуюся угловую скорость вращения. Из закона сохранения момента импульса системы относительно оси \(z\) следует, что \(I_{1} \omega_{1 z}+I_{2} \omega_{2 z}=\left(I_{1}+I_{2}\right) \omega_{z}\), откуда


\begin{equation*}
\omega_{z}=\frac{I_{1} \omega_{1 z}+I_{2} \omega_{2 z}}{I_{1}+I_{2}} \tag{1}
\end{equation*}


Заметим, что \(\omega_{1 z}, \omega_{2 z}\) и \(\omega_{z}\) - величины алгебраические. Если окажется, что \(\omega_{z}>0\), то это значит, что соответствующий вектор \(\omega\) совпадает с положительным направлением оси \(z\), и наоборот.\\
Приращение кинетической энергии вращения этой системы

\[
\Delta K=1 / 2\left(I_{1}+I_{2}\right) \omega_{z}^{2}-1 / 2\left(I_{1} \omega_{1 z}^{2}+I_{2} \omega_{2 z}^{2}\right) .
\]

Заменив \(\omega_{z}\) его выражением (1), получим


\begin{equation*}
\Delta K=-\frac{I_{1} I_{2}}{2\left(I_{1}+I_{2}\right)}\left(\omega_{1 z}-\omega_{2 z}\right)^{2} \tag{2}
\end{equation*}


Знак минус показывает, что кинетическая энергия системы уменьшается.\\
Обратим внимание на то, что полученные результаты (1) и (2) весьма похожи по форме и по смыслу на случай абсолютно неупругого столкновения (см. с. 128).\\
Работа внешних сил при вращении твердого тела вокруг неподвижной оси. В соответствии с уравнением (4.49) работа всех внешних сил, действующих на твердое тело, равна приращению только кинетической энергии тела, так как его собственная потенциальная энергия при этом не меняется. Таким обра-

зом, \(\delta A=\mathrm{d} K\) или, согласно (5.31), \(\delta A=\mathrm{d}\left(I \omega^{2} / 2\right)\). Так как ось \(Z\) совпадает с осью вращения, то \(\omega^{2}=\omega_{z}^{2}\) и

\[
\delta A=I \omega_{z} \mathrm{~d} \omega_{z} .
\]

Но согласно (5.30), \(I \mathrm{~d} \omega_{z}=N_{z} \mathrm{~d} t\). Подставляя это выражение в последнее уравнение для \(\delta A\) и учитывая, что \(\omega_{z} \mathrm{~d} t=\mathrm{d} \varphi\), получаем


\begin{equation*}
\delta A=N_{z} \mathrm{~d} \varphi . \tag{5.32}
\end{equation*}


Работа \(\delta A\) - величина алгебраическая: если \(N_{z}\) и \(\mathrm{d} \varphi\) имеют одинаковые знаки, то \(\delta A>0\); если же их знаки противоположны, то \(\delta A<0\).

Работа внешних сил при повороте твердого тела на конечный угол φ равна


\begin{equation*}
A=\int_{0}^{\varphi} N_{z} \mathrm{~d} \varphi . \tag{5.33}
\end{equation*}


В случае, если \(N_{z}=\) const, выражение (5.33) упрощается: \(A=N_{z} \varphi\).

Таким образом, работа внешних сил при вращении твердого тела вокруг неподвижной оси определяется действием момента \(N_{z}\) этих сил относительно данной оси. Если силы таковы, что их момент \(N_{z}=0\), то работы они не производят.

\section*{2. Плоское движение твердого тела}
При плоском движении центр масс \(C\) твердого тела движется в определенной плоскости, неподвижной в данной \(K\)-системе отсчета, а вектор его угловой скорости ω все время остается перпендикулярным этой плоскости. Последнее означает, что в Ц-системе твердое тело совершает чисто вращательное движение вокруг неподвижной в этой системе оси, проходящей через центр масс тела. Вращательное же движение твердого тела определяется уравнением (5.30), которое, как было отмечено, справедливо в любой системе отсчета.

Таким образом, мы имеем следующие два уравнения, описывающие плоское движение твердого тела:


\begin{equation*}
m \mathbf{a}_{C}=\mathbf{F} ; \quad I_{C} \beta_{z}=N_{C z}, \tag{5.34}
\end{equation*}


где \(m\) - масса тела, \(\mathbf{F}\) - результирующая всех внешних сил, \(I_{C}\) и \(N_{C z}\) - момент инерции и суммарный момент всех внешних сил - оба относительно оси, проходящей через центр масс тела.

При этом следует помнить, что момент \(N_{C z}\) включает в себя только внешние силы взаимодействия, несмотря на то что Ц-система в общем случае является неинерциальной. Это связано с тем, что суммарный момент сил инерции равен нулю как относительно центра масс, так и относительно любой оси, проходящей через эту точку. Поэтому его можно просто не учитывать (см. с. 173).

Заметим также, что угловое ускорение \(\beta_{z}\), а следовательно, \(\omega_{z}\) и φ одинаковы в обеих системах отсчета, так как Ц-система движется поступательно относительно инерциальной \(K\)-системы отсчета.

Интегрируя уравнения (5.34) с учетом начальных условий, можно найти зависимости \(\mathbf{r}_{C}(t)\) и \(\varphi(t)\), определяющие положение твердого тела в любой момент \(t\).

При решении задачи о движении несвободного твердого тела необходимо использовать еще одно, дополнительное, условие, определяющее ограничения движения имеющимися связями. Оно дает кинематическую связь между линейным и угловым ускорениями.\\
Пример. Однородный цилиндр массы \(m\) и радиуса \(r\) скатывается без скольжения по наклонной плоскости, составляющей угол а с горизонтом (рис. 5.18). Найдем уравнения движения цилиндра. Стандартный подход к решению подобных задач состоит в следующем. Прежде всего устанавливают силы, действующие на данное тело, и точки их приложения (в нашем случае это

\begin{figure}[h]
\begin{center}
  \includegraphics[alt={},max width=\textwidth]{66a35da9-3faa-4823-9ca9-40799f37b3db-182_325_394_1206_748}
\captionsetup{labelformat=empty}
\caption{Рис. 5.18}
\end{center}
\end{figure}

\(m \mathbf{g}\) - сила тяжести, \(\mathbf{R}\) - нормальная составляющая силы реакции со стороны наклонной плоскости и \(\mathbf{F}_{\text {тр }}\) - сила трения покоя). Затем выбирают положительные направления оси \(x\) и угла поворота \(\varphi\) (лучше всего эти направления взять сразу согласованными, так чтобы зна-

ки ускорений \(a_{C x}\) и \(\beta_{z}\) были одинаковы), например, как показано на рис. 5.18 справа. И только после этого записывают уравнения движения (5.34) в проекциях на выбранные таким образом положительные направления \(x\) и \(\varphi\) :

\[
m a_{C x}=m g \sin \alpha-F_{\text {тр }}, \quad I_{C} \beta_{z}=r F_{\text {тр }} .
\]

Кроме того, условие отсутствия скольжения определяет еще кинематическую связь между ускорениями:

\[
a_{C x}=r \beta_{z} .
\]

Совместное решение этих трех уравнений дает возможность найти ускорения \(\mathbf{a}_{C}\) и \(\boldsymbol{\beta}\), а также силу \(\mathbf{F}_{\text {тр }}\).\\
Часто уравнение моментов записывают относительно мгновенной оси \(O\) (см. рис. 5.18). Правомерность такого подхода заранее не очевидна и требует доказательства в связи с особыми свойствами мгновенной оси. Доказательство основано на использовании соотношений (5.12) и (5.23). С помощью них мы действительно приходим к уравнению, которое формально можно было записать и сразу:

\[
I_{O} \beta_{z}=r m g \sin \alpha,
\]

где \(I_{O}=I_{C}+m r^{2}\), согласно теореме Штейнера.\\
Как это ни удивительно, приведенное уравнение справедливо не только относительно мгновенной оси, но и любой другой оси, параллельной мгновенной и лежащей на наклонной плоскости. Причем во всех случаях, т. е. независимо от положения оси, \(r\) - это именно радиус (цилиндра, шара) и ничто другое.

\section*{Кинетическая энергия при плоском движении твердого тела}
Пусть тело совершает плоское движение в некоторой инерциальной \(K\)-системе отсчета. Чтобы найти его кинетическую энергию в этой системе, воспользуемся формулой (4.56). Входящая в эту формулу величина \(\tilde{K}\) в данном случае представляет собой кинетическую энергию вращения тела в Ц-системе вокруг оси, проходящей через центр масс тела. Согласно (5.31), \(\tilde{K}=I_{C} \omega^{2} / 2\), поэтому сразу можно записать


\begin{equation*}
K=\frac{I_{C} \omega^{2}}{2}+\frac{m V_{C}^{2}}{2}, \tag{5.35}
\end{equation*}


где \(I_{C}\) - момент инерции тела относительно оси вращения, проходящей через его центр масс, \(\omega\) - угловая скорость тела, \(m\) - его масса, \(V_{C}\) - скорость центра масс тела в \(K\)-системе отсчета. Таким образом, кинетическая энергия твердого тела при плоском движении складывается из энергии вращения в Ц-системе и энергии, связанной с движением центра масс.

\section*{3. Свободные оси. Главные оси тела}
Если твердое тело привести во вращение и затем предоставить самому себе, то направление оси вращения в пространстве, вообще говоря, будет меняться. Для того чтобы произвольная ось вращения тела сохраняла свое направление неизменным, к ней необходимо приложить определенные силы.

Рассмотрим этот вопрос более подробно на следующем примере. Пусть середина \(C\) однородного стержня жестко скреплена с осью вращения так, что угол между стержнем и осью равен Э (рис. 5.19). Найдем момент \(\mathbf{N}\) внешних сил, которые необходимо приложить к оси вращения, чтобы при вращении стержня с угловой скоростью \(\omega\) ее направление не менялось. Согласно (5.12), этот момент \(\mathbf{N}=\mathrm{d} \mathbf{M} / \mathrm{d} t\). Таким образом, чтобы определить \(\mathbf{N}\), сна-

\begin{figure}[h]
\begin{center}
  \includegraphics[alt={},max width=\textwidth]{66a35da9-3faa-4823-9ca9-40799f37b3db-184_388_354_728_806}
\captionsetup{labelformat=empty}
\caption{Рис. 5.19}
\end{center}
\end{figure}

чала надо найти момент импульса стержня M, а затем его производную по времени.

Момент импульса М проще всего определить относительно точки \(C\). Мысленно выделим элемент стержня массы \(\delta m\), находящейся на расстоянии \(\mathbf{r}\) от точки \(C\). Его момент импульса относительно этой точки \(\delta \mathbf{M}=[\mathbf{r}, \delta m \mathbf{v}]\), где \(\mathbf{v}\) - скорость элемента. Легко видеть, что вектор \(\delta \mathbf{M}\) направлен перпендикулярно стержню (рис. 5.19), причем его направление не зависит от выбора элемента \(\delta m\). Поэтому суммарный момент импульса M стержня совпадает по направлению с вектором \(\delta \mathbf{M}\).

Заметим, что в данном случае вектор M не совпадает по направлению с вектором \(\omega\) !

При вращении стержня вектор \(\mathbf{M}\) будет также вращаться с угловой скоростью \(\omega\). За промежуток времени \(\mathrm{d} t\) вектор \(\mathbf{M}\) по-

лучает приращение dM, модуль которого, как видно из рис. 5.19, равен

\[
|\mathrm{d} \mathbf{M}|=M \sin (\pi / 2-\vartheta) \omega \mathrm{d} t,
\]

или в векторном виде \(\mathrm{dM}=[\omega \mathbf{M}] \mathrm{d} t\). Поделив обе части последнего выражения на \(\mathrm{d} t\), получим

\[
\mathrm{N}=[\omega \mathrm{M}]
\]

Таким образом, действительно, для удержания оси вращения в неизменном направлении к ней необходимо в данном случае приложить момент \(\mathbf{N}\) некоторых внешних сил \(\mathbf{F}\) (они показаны на рис. 5.19). Однако нетрудно видеть, что если \(\vartheta=\pi / 2\), то вектор \(\mathbf{M}\) совпадает по направлению с вектором \(\boldsymbol{\omega}\), и в этом случае \(\mathbf{N} \equiv 0\), т. е. направление оси вращения будет оставаться неизменным без внешнего воздействия.

Ось вращения, направление которой в пространстве остается неизменным без действия на нее каких-либо сил извне, называют свободной осью тела.

В общей теории доказывается, что для любого твердого тела существуют три взаимно перепендикулярные оси, проходящие через его центр масс, которые могут служить свободными осями. Их называют главными осями тела.

Нахождение главных осей тела произвольной формы - в математическом отношении сложная задача. Однако она очень упрощается для тел, обладающих той или иной симметрией, ибо положение центра масс и направление главных осей обладают в этом случае той же симметрией.

Например, у однородного прямоугольного параллелепипеда главные оси проходят через центры противоположных граней. Если тело обладает осью симметрии (однородный цилиндр, конус и др.), одной из его главных осей является ось симметрии, в качестве же остальных осей могут служить две любые взаимно перпендикулярные оси, проходящие через центр масс тела и перпендикулярные его оси симметрии. Таким образом, у тела у осевой симметрией фиксирована только одна из главных осей. У тела же с центральной симметрией (например, у однородного шара) главными осями являются три любые взаимно перпендикулярные оси, проходящие через центр тела, - ни одна из главных осей не фиксирована относительно тела.

Важной особенностью главных осей является то, что при вращении тела вокруг любой из них его момент импульса M совпадает по направлению с угловой скоростью \(\omega\) и определяется как


\begin{equation*}
\mathbf{M}=I \omega, \tag{5.36}
\end{equation*}


где \(I\) - момент инерции тела относительно данной главной оси*. Причем М не зависит от выбора точки, относительно которой его определяют (здесь предполагается, что ось вращения неподвижна).

Наиболее просто убедиться в справедливости (5.36) можно для случая однородного тела с осевой симметрией. Действительно, согласно (5.27), момент импульса твердого тела относительно оси вращения \(M_{z}=I \omega_{z}\) (напомним, что \(M_{z}\) - это проекция вектора \(\mathbf{M}\), определенного относительно любой точки на этой оси). Но если тело симметрично относительно оси вращения, то из соображения симметрии сразу следует, что вектор M совпадает по направлению с вектором \(\omega\) и, значит, \(\mathbf{M}=I \omega\).

Еще раз отметим, что в общем случае (ось вращения не совпадает ни с одной из главных осей, хотя и проходит через центр масс тела) направление вектора \(\mathbf{M}\) не совпадает с вектором ω и связь между этими векторами носит сложный характер. Это обстоятельство является причиной сложного поведения вращающихся твердых тел.

\section*{4. Гироскопы}
Гироскопом называют массивное симметричное тело, вращающееся с большой угловой скоростью вокруг своей оси симметрии. Рассмотрим поведение гироскопа на примере волчка. Опыт показывает, что если ось вращающегося волчка наклонена к вертикали, то волчок не падает, а совершает так называемое прецессионное движение (прецессию) - его ось описывает конус вокруг вертикали с некоторой угловой скоростью \(\boldsymbol{\omega}^{\prime}\), причем оказывается: чем больше угловая скорость \(\omega\) вращения волчка, тем меньше угловая скорость прецессии \(\omega^{\prime}\).

\footnotetext{\begin{itemize}
  \item Заметим, что соотношение (5.36) справедливо и относительно осей, параллельных главным осям тела и не проходящих через его центр масс.
\end{itemize}
}Такое поведение волчка-гироскопа можно легко объяснить с помощью уравнения моментов (5.12), если только принять, что \(\omega \gg \omega^{\prime}\) (это условие, кстати, поясняет, что́ имеется в виду под большой угловой скоростью гироскопа). Действительно, момент импульса М прецессирующего волчка относительно точки опоры \(O\) (рис. 5.20) можно представить в виде суммы момента импульса \(\mathbf{M}_{\omega}\), обусловленного вращением волчка вокруг своей оси, и некоторого добавочного момента импульса \(\mathbf{M}^{\prime}\), вызванного прецессией волчка вокруг вертикальной оси, т. е.

\[
\mathbf{M}=\mathbf{M}_{\omega}+\mathbf{M}^{\prime} .
\]

Поскольку ось волчка совпадает с одной из его главных осей, то, согласно (5.36), \(\mathbf{M}_{\omega}=I \omega\), где \(I\) - момент инерции волчка относительно этой оси. Кроме того, ясно, что чем меньше угловая скорость прецессии, тем меньше и соответствующий момент \(\mathbf{M}^{\prime}\). При \(\omega \gg \omega^{\prime}\) во всех практически интересных случаях \(M_{\omega} \gg M^{\prime}\), поэтому результирующий момент импульса \(\mathbf{M}\) почти совпадает с \(\mathbf{M}_{\omega}\) как по модулю, так и по направлению, можно считать, что

\[
\mathbf{M}=I \omega .
\]

Зная поведение вектора \(\mathbf{M}\), мы найдем и характер движения оси волчка-гироскопа.

Но поведением вектора М управляет уравнение моментов (5.12). Согласно ему, момент импульса \(\mathbf{M}\) относительно точки \(O\) (рис. 5.20) получает за время \(\mathrm{d} t\) приращение


\begin{equation*}
\mathrm{d} \mathbf{M}=\mathbf{N} \mathrm{d} t, \tag{5.37}
\end{equation*}


\begin{figure}[h]
\begin{center}
  \includegraphics[alt={},max width=\textwidth]{66a35da9-3faa-4823-9ca9-40799f37b3db-187_426_340_1327_70}
\captionsetup{labelformat=empty}
\caption{Рис. 5.20}
\end{center}
\end{figure}

совпадающее по направлению с вектором N - моментом внешних сил относительно той же точки \(O\) (в данном случае это момент силы тяжести \(m g\) ). Из рис. 5.20 видно, что \(\mathrm{dM} \perp \mathbf{M}\). В результате вектор \(\mathbf{M}\) (а следовательно, и ось волчка) будет поворачиваться вместе с вектором \(\mathbf{N}\) вокруг вертикали, описывая круговой конус с углом полураствора Э. Волчок-гироскоп будет прецессировать вокруг вертикальной оси с некоторой угловой скоростью \(\boldsymbol{\omega}^{\prime}\).

Найдем связь между векторами \(\mathbf{N}, \mathbf{M}\) и \(\boldsymbol{\omega}^{\prime}\). Согласно рисунку, модуль приращения вектора \(\mathbf{M}\) за время \(\mathrm{d} t\) есть \(|\mathrm{d} \mathbf{M}|=M \sin \vartheta \cdot \omega^{\prime} \mathrm{d} t\), или в векторном виде \(\mathrm{d} \mathbf{M}=\left[\boldsymbol{\omega}^{\prime} \mathbf{M}\right] \mathrm{d} t\). После подстановки этого выражения в (5.37) получим


\begin{equation*}
\left[\omega^{\prime} \mathbf{M}\right]=\mathbf{N} . \tag{5.38}
\end{equation*}


Из этого уравнения видно, что момент силы \(\mathbf{N}\) определяет угловую скорость прецессии \(\omega^{\prime}\) (а не ускорение!). Поэтому мгновенное устранение момента \(\mathbf{N}\) приводит к мгновенному исчезновению и прецессии. В этом отношении можно сказать, что прецессия не обладает инерцией.

Заметим, что момент сил N, действующий на гироскоп, может иметь любую природу. Для обеспечения регулярной прецессии (постоянной угловой скорости \(\boldsymbol{\omega}\) ) важно только, чтобы вектор N, не меняясь по модулю, поворачивался вместе с осью гироскопа.

Пример. Найдем угловую скорость прецессии наклонного волчка массы \(m\), вращающегося с большой угловой скоростью \(\omega\) вокруг своей оси симметрии, относительно которой момент инерции волчка равен \(I\). Центр масс волчка находится на расстоянии \(l\) от точки опоры.\\
Согласно (5.38), \(\omega^{\prime} I \omega \sin \vartheta=m g l \sin \vartheta\), где \(\vartheta\) - угол между вертикалью и осью волчка (рис. 5.20). Отсюда

\[
\omega^{\prime}=m g l / I \omega .
\]

Интересно, что величина \(\omega^{\prime}\) не зависит от угла наклона \(\vartheta\) оси волчка. Кроме того, полученный результат показывает, что \(\omega^{\prime}\) обратно пропорциональна \(\omega\), т. е., действительно, чем больше угловая скорость волчка, тем меньше угловая скорость его прецессии.\\
Гироскопический момент. Рассмотрим эффект, возникающий при вынужденном вращении оси гироскопа. Пусть, например, ось гироскопа укреплена в U-образной подставке, которую мы будем поворачивать вокруг оси \(O O^{\prime}\) (рис. 5.21). Если момент импульса \(\mathbf{M}\) гироскопа направлен вправо, то при таком повороте за время \(\mathrm{d} t\) вектор \(\mathbf{M}\) получит при-

\begin{figure}[h]
\begin{center}
  \includegraphics[alt={},max width=\textwidth]{66a35da9-3faa-4823-9ca9-40799f37b3db-188_306_374_1423_786}
\captionsetup{labelformat=empty}
\caption{Рис. 5.21}
\end{center}
\end{figure}

ращение dM - вектор, направленный за плоскость рисунка. Согласно (5.37), это означает, что на гироскоп действует момент сил N, совпадающий по направлению с вектором dM. Момент \(\mathbf{N}\) обусловлен возникновением пары сил \(F\), действующих на ось гироскопа со стороны подставки. Ось же гироскопа в соответствии с третьим законом Ньютона будет действовать на подставку с силами \(F^{\prime}\) (рис. 5.21). Эти силы называют гироскопическими; они создают гироскопический момент \(\mathbf{N}^{\prime}=-\mathbf{N}\). Заметим, что в данном случае гироскоп не обладает способностью противодействовать изменению направления его оси вращения.

Появление гироскопических сил называют гироскопическим эффектом. Подобный гироскопический эффект, связанный с возникновением гироскопического давления на подшипники, наблюдается, например, у роторов турбин на кораблях при поворотах и качке, у винтовых самолетов при виражах и т. п.

Проследим действие гироско-

\begin{figure}[h]
\begin{center}
  \includegraphics[alt={},max width=\textwidth]{66a35da9-3faa-4823-9ca9-40799f37b3db-189_449_475_834_68}
\captionsetup{labelformat=empty}
\caption{Рис. 5.22}
\end{center}
\end{figure}

пического момента на примере гироскопа, ось которого вместе с рамкой (рис. 5.22) может свободно поворачиваться вокруг горизонтальной оси \(O O^{\prime}\) U-образной подставки. Если подставке сообщить вынужденное вращение вокруг вертикальной оси, как показано на рисунке вектором \(\boldsymbol{\omega}^{\prime}\), то момент импульса \(\mathbf{M}\) гироскопа получит за время \(\mathrm{d} t\) приращение \(\mathrm{d} \mathbf{M}_{1}\) - вектор, направленный за рисунок. Это приращение обусловлено моментом \(\mathbf{N}_{1}\) пары сил, действующих на ось гироскопа со стороны рамки. Гироскопические силы, действующие со стороны оси гироскопа на рамку, вызовут поворот последней вокруг горизонтальной оси \(O O^{\prime}\). При этом вектор М получит дополнительное приращение \(\mathrm{dM}_{2}\), которое, в свою очередь, обусловлено моментом \(\mathbf{N}_{2}\) пары сил, действующих на ось гироскопа со стороны рамки. В результате ось гироскопа будет поворачиваться так, что вектор М будет стремиться совпасть по направлению с вектором \(\boldsymbol{\omega}^{\prime}\).

Таким образом, за промежуток времени \(\mathrm{d} t\) момент импульса \(\mathbf{M}\) гироскопа получает приращение \(\mathrm{dM}=\mathrm{dM}_{1}+\mathrm{dM}_{2}=\left(\mathrm{N}_{1}+\mathrm{N}_{2}\right) \mathrm{d} t\). При этом на рамку действует гироскопический момент

\[
\mathbf{N}^{\prime}=-\left(\mathbf{N}_{1}+\mathbf{N}_{2}\right) .
\]

Составляющая этого момента \(\mathbf{N}_{1}^{\prime}=-\mathbf{N}_{1}\) вызывает поворот рамки вокруг горизонтальной оси \(O O^{\prime}\), другая составляющая \(\mathbf{N}_{2}^{\prime}=-\mathbf{N}_{2}\) противодействует повороту всей системы вокруг вертикальной оси (в отличие от предыдущего случая).

Гироскопический эффект лежит в основе разнообразных применений гироскопов: гирокомпас, гироскопический успокоитель качки корабля, гироскопический стабилизатор и др.

\section*{Задачи}
5.1. Законы сохранения момента импульса и энергии. Доказать, что полная механическая энергия \(E\) планеты, движущейся вокруг Солнца по эллипсу, зависит только от его большой полуоси \(a\). Найти выражение для \(E\), если известны массы планеты и Солнца ( \(m\) и \(M\) ), а также большая полуось \(a\) эллипса.\\
Решение. Воспользуемся законами сохранения момента импульса и энергии. Точка, относительно которой момент импульса планеты сохраняется, - это центр Солнца. Поэтому для положений 1 и 2 планеты (рис. 5.23), в которых вектор скорости перпендикулярен радиусу-вектору, можно записать

\begin{figure}[h]
\begin{center}
  \includegraphics[alt={},max width=\textwidth]{66a35da9-3faa-4823-9ca9-40799f37b3db-190_209_346_1004_818}
\captionsetup{labelformat=empty}
\caption{Рис. 5.23}
\end{center}
\end{figure}


\begin{equation*}
m r_{1} v_{1}=m r_{2} v_{2} . \tag{1}
\end{equation*}


Из закона сохранения полной механической энергии \(E\) следует, что для тех же положений планеты


\begin{equation*}
\frac{m v_{1}^{2}}{2}-\gamma \frac{m M}{r_{1}}=\frac{m v_{2}^{2}}{2}-\gamma \frac{m M}{r_{2}} . \tag{2}
\end{equation*}


Решив совместно уравнения (1) и (2), выразим, например, \(v_{1}\) через \(r_{1}\) и \(r_{2}\) :

\[
v_{1}^{2}=\frac{2 \gamma M}{r_{1}+r_{2}} \frac{r_{2}}{r_{1}} .
\]

И наконец, находим формулу для полной энергии \(E\) как

\[
E=K\left(v_{1}\right)+U\left(r_{1}\right)=-\gamma m M /\left(r_{1}+r_{2}\right) .
\]

Учитывая, что \(r_{1}+r_{2}=2 a\), получим окончательно

\[
E=-\gamma m M / 2 a .
\]

5.2. Частица 1 массы \(m_{1}\) налетает на покоящуюся частицу 2 массы \(m_{2}\), имея вдали от частицы 2 кинетическую энергию \(K_{0}\) и прицельный параметр \(l\) - плечо вектора импульса относительно частицы 2 (рис. 5.24). Заряд каждой частицы равен \(q\). Найти наименьшее расстояние, на которое сблизятся частицы, если:

\begin{enumerate}
  \item \(m_{1} \ll m_{2}\); 2) \(m_{1}\) сравнимо с \(m_{2}\).
\end{enumerate}

Решение. 1. Условие \(m_{1} \ll m_{2}\) означает, что частица 2 в процессе взаимодействия будет практически оставаться в покое. Вектор силы, действующей на частицу 1 относительно покоящейся частицы 2 , сохраняется. Тогда\\
\includegraphics[max width=\textwidth, alt={}, center]{66a35da9-3faa-4823-9ca9-40799f37b3db-191_206_393_861_123}

\[
l p_{0}=r_{\text {мин }} p \text {, }
\]

где слева - момент импульса частицы 1 вдали от частицы 2 , а справа - в момент наибольшего сближения, когда \(\mathbf{r} \perp \mathbf{p}\) (рис. 5.24).

Рис. 5.24 Из закона сохранения энергии следует

\[
K_{0}=K+k q^{2} / r_{\text {мин }} \text {, }
\]

где \(K\) - кинетическая энергия частицы 1 в момент наибольшего сближения. Решив эти два уравнения (с учетом связи \(p\) и \(K\) ), получим


\begin{equation*}
r_{\text {мин }}=\frac{k q^{2}}{2 K_{0}}\left(1+\sqrt{1+4\left(\frac{l K_{0}}{k q^{2}}\right)^{2}}\right) . \tag{1}
\end{equation*}


\begin{enumerate}
  \setcounter{enumi}{1}
  \item В данном случае уже нельзя считать, что частица 2 покоится в процессе взаимодействия. Решение наиболее целесообразно провести в Ц-системе, где картина «соударения» выглядит так, как показано на рис. 5.25. Система из двух частиц предполагается замкнутой, поэтому ее собственный момент импульса сохраняется:
\end{enumerate}


\begin{equation*}
l \tilde{p}_{10}=r_{\text {мин }} \tilde{p}_{1}, \tag{2}
\end{equation*}


где учтено, что в момент наибольшего сближения \(\mathbf{r}_{12} \perp \mathbf{p}\). Кроме того, на основании закона сохранения энергии


\begin{equation*}
\tilde{K}_{0}=\tilde{K}+k q^{2} / r_{\text {мин }} \text {, } \tag{3}
\end{equation*}


где \(\tilde{K}_{0}\) и \(\tilde{K}\) - суммарные кинетические энергии частиц в Ц-системе, когда частицы находятся далеко друг

\begin{figure}[h]
\begin{center}
  \includegraphics[alt={},max width=\textwidth]{66a35da9-3faa-4823-9ca9-40799f37b3db-192_263_416_149_744}
\captionsetup{labelformat=empty}
\caption{Рис. 5.25}
\end{center}
\end{figure}

от друга и в момент наибольшего сближения. Из уравнений (2) и (3) получим то же выражение (1), только в нем вместо \(K_{0}\) будет стоять \(\tilde{K}_{0}\), причем в данном случае (частица 2 первоначально покоилась), согласно (4.61),

\[
\tilde{K}_{0}=\frac{m_{2}}{m_{1}+m_{2}} K_{0} .
\]

Заметим, что при \(m_{1} \ll m_{2}\) величина \(\tilde{K}_{0} \approx K_{0}\) и выражение для \(r_{\text {мин }}\) будет полностью совпадать с (1).\\
5.3. Небольшой шарик подвесили к точке \(O\) на легкой нерастяжимой нити длины \(l\). Затем шарик отвели в сторону так, что нить отклонилась на угол \(\vartheta\) от вертикали, и сообщили ему начальную скорость \(v_{0}\) перпендикулярно вертикальной плоскости, в которой расположена нить. При каком значении \(v_{0}\) максимальный угол отклонения нити от вертикали окажется равным \(90^{\circ}\) ?

Решение. На шарик в процессе движения действуют две силы: сила тяжести и сила натяжения нити. Нетрудно видеть, что относительно вертикальной оси \(Z\), проходящей через точку \(O\), момент этих сил \(M_{z} \equiv 0\). Следовательно, относительно данной оси момент импульса шарика \(L_{z}=\) const, или


\begin{equation*}
l \sin \vartheta \cdot m v_{0}=l m v, \tag{1}
\end{equation*}


где \(m\) - масса шарика, \(v\) - его скорость в положении, при котором нить составляет прямой угол с вертикалью.\\
Шарик движется в поле тяжести Земли под действием сторонней силы - силы натяжения со стороны нити. Эта сила все время перпендикулярна вектору скорости шарика и поэтому работы не совершает. Отсюда следует, что, согласно уравнению (4.31), механическая энергия шарика в поле тяжести Земли сохраняется:


\begin{equation*}
m v_{0}^{2} / 2=m v^{2} / 2+m g l \cos \vartheta, \tag{2}
\end{equation*}


где правая часть равенства соответствует горизонтальному положению нити.

Решив совместно уравнения (1) и (2), получим

\[
v_{0}=\sqrt{2 g l / \cos \vartheta} .
\]

5.4. На жестком проволочном полукольце радиуса \(r_{0}\), которое может

\begin{figure}[h]
\begin{center}
  \includegraphics[alt={},max width=\textwidth]{66a35da9-3faa-4823-9ca9-40799f37b3db-193_407_362_404_138}
\captionsetup{labelformat=empty}
\caption{Рис. 5.26}
\end{center}
\end{figure}

свободно вращаться вокруг вертикальной оси \(A B\) (рис. 5.26), находятся две одинаковые небольшие муфточки. Их соединили нитью и установили в положение 1-1. Затем всей установке сообщили угловую скорость \(\omega_{0}\) и, предоставив ее самой себе, пережгли нить в точке \(A\). Считая, что масса установки практически сосредоточена в муфточках, найти ее угловую скорость в момент, когда муфточки соскользнут (без трения) в крайнее нижнее положение 2-2.

Решение. Пусть в нижнем положении расстояние муфточек от оси вращения \(r\) и угловая скорость установки \(\omega\). Тогда из законов сохранения энергии и момента импульса относительно оси вращения следует, что

\[
r^{2} \omega^{2}-r_{0}^{2} \omega_{0}^{2}=2 g h, \quad r^{2} \omega=r_{0}^{2} \omega_{0},
\]

где \(h\) - разность высот верхнего и нижнего положений муфточек. Здесь учтено, что в нижнем положении, как и в верхнем, скорость муфточек относительно проволочного полукольца равна нулю. Кроме того, из рис. 5.26 видно, что

\[
r_{0}^{2}=r^{2}+h^{2} .
\]

Решив совместно эти три уравнения, получим

\[
\omega=\frac{\omega_{0}}{2}\left(1+\sqrt{1+\left(\frac{4 g}{r_{0} \omega_{0}^{2}}\right)^{2}}\right) .
\]

5.5. Гладкий стержень свободно вращается в горизонтальной плоскости с угловой скоростью \(\omega_{0}\) вокруг неподвижной вертикальной оси \(O\) (рис. 5.27), относительно которой его момент инерции равен \(I\). На стержне около оси вращения находится небольшая муфта

массы \(m\), соединенная с этой осью нитью. После пережигания нити муфта начинает скользить вдоль стержня. Найти скорость \(v^{\prime}\) муфты относительно стержня в зависимости от ее расстояния \(r\) до оси вращения.

\begin{figure}[h]
\begin{center}
  \includegraphics[alt={},max width=\textwidth]{66a35da9-3faa-4823-9ca9-40799f37b3db-194_178_430_138_732}
\captionsetup{labelformat=empty}
\caption{Рис. 5.27}
\end{center}
\end{figure}

Решение. У данной системы в процессе движения сохраняются кинетическая энергия и момент импульса относительно оси вращения. Отсюда следует, что

\[
I \omega_{0}^{2}=I \omega^{2}+m v^{2}, \quad I \omega_{0}=\left(I+m r^{2}\right) \omega
\]

где \(v^{2}=v^{\prime 2}+\omega^{2} r^{2}\) (рис. 5.27). Из этих уравнений получим

\[
v^{\prime}=\omega_{0} r / \sqrt{1+m r^{2} / I}
\]

5.6. Горизонтально летевшая пуля \(A\) попала, застряв, в вертикальный однородный стержень массы \(m\) и длины \(l_{0}\), верхний конец которого укреплен в шарнире \(O\) (рис. 5.28). Пуля имела импульс р и попала в стержень на расстоянии \(l\) от точки \(O\). Пренебрегая ее массой, найти:

\begin{enumerate}
  \item приращение импульса системы пуля стержень за время движения пули в стержне;
  \item угловую скорость, которую приобретет стержень, с учетом собственного момента импульса пули, равного \(\tilde{\mathbf{M}}\) и совпадающего по направлению с вектором p (пуля вращается вокруг
\end{enumerate}

\begin{figure}[h]
\begin{center}
  \includegraphics[alt={},max width=\textwidth]{66a35da9-3faa-4823-9ca9-40799f37b3db-194_347_290_790_875}
\captionsetup{labelformat=empty}
\caption{Рис. 5.28}
\end{center}
\end{figure}

направления ее движения).

Р е ш е н и е. 1. Система пуля - стержень незамкнутая: помимо сил, уравновешивающих друг друга, в процессе движения пули в стержне возникает горизонтальная составляющая силы реакции в точке \(O\) со стороны оси. Действие этой составляющей и вызовет приращение импульса системы:

\[
\Delta p=m v_{C}-p
\]

где \(v_{C}\) - скорость центра стержня после застревания пули.\\
Так как все внешние силы проходят через точку \(O\), то за время движения пули в стержне момент импульса системы будет оставаться постоянным относительно любой оси, проходящей через эту точку. Взяв ось перпендикулярной к плоскости рисунка, запишем

\[
l p=I \omega
\]

где \(I\) - момент инерции стержня относительно выбранной оси, а \(\omega\) - угловая скорость стержня непосредственно после остановки пули в нем.

Из этих уравнений с учетом того, что \(v_{C}=\omega r, r\) - расстояние от точки \(O\) до центра стержня, получим

\[
\Delta p=\left(3 l / 2 l_{0}-1\right) p .
\]

Отсюда видно, что знак приращения \(\Delta p\) зависит от отношения \(l / l_{0}\). В частности, при \(l / l_{0}=2 / 3\) величина \(\Delta p=0\), т. е. импульс системы не изменится за время движения пули в стержне. Это значит, что в данном случае горизонтальная составляющая реакции в точке \(O\) отсутствует.\\
2. В этом случае момент импульса системы относительно точки \(O\) также будет оставаться постоянным за время движения пули в стержне, поэтому, согласно (5.23),

\[
\tilde{\mathbf{M}}+[\mathbf{l p}]=\mathbf{M} .
\]

\begin{figure}[h]
\begin{center}
  \includegraphics[alt={},max width=\textwidth]{66a35da9-3faa-4823-9ca9-40799f37b3db-195_209_335_827_153}
\captionsetup{labelformat=empty}
\caption{Рис. 5.29}
\end{center}
\end{figure}

Слева записан момент импульса пули относительно точки \(O\), а справа - момент импульса стержня (с пулей) непосредственно после остановки пули в стержне (на рис. 5.29 все три вектора расположены в горизонтальной плоскости).\\
Найдем вектор M, когда стержень (с пулей) приобретет угловую скорость \(\omega\). Возьмем малый элемент стержня массы \(\mathrm{d} m\), находящийся на расстоянии \(\mathbf{r}\) от точки \(O\). Его момент импульса относительно точки \(O\) равен

\[
\mathrm{d} \mathbf{M}=[\mathbf{r}, \mathrm{d} m \mathbf{v}]=\mathrm{d} m \cdot r^{2} \boldsymbol{\omega}=\left(m \boldsymbol{\omega} / l_{0}\right) r^{2} \mathrm{~d} r,
\]

где \(\mathbf{v}\) - скорость данного элемента. Проинтегрировав это выражение по всем элементам, получим

\[
\mathbf{M}=m l_{0}^{2} \omega / 3 .
\]

Таким образом,

\[
\tilde{\mathbf{M}}+[\mathbf{l p}]=m l_{0}^{2} \omega / 3 .
\]

Из этой формулы, согласно рис. 5.29 , получим

\[
\omega=3 \sqrt{M^{2}+l^{2} p^{2}} / m l_{0}^{2} .
\]

С помощью того же рисунка можно найти и направление вектора \(\boldsymbol{\omega}(\) угол \(\alpha)\).\\
5.7. Динамика вращательного движения. Однородный сплошной цилиндр массы \(m_{0}\) и радиуса \(R\) может без трения вращаться вокруг неподвижной горизонтальной оси \(O\) (рис. 5.30). На цилиндр в один ряд плотно намотан тонкий нерастяжимый шнур длины \(l\) и массы \(m\). Найти угловое ускорение цилиндра в зависимости от длины \(x\) свешивающейся части шнура. Считать, что скольжения нет и центр масс намотанной части шнура находится на оси цилиндра.

\begin{figure}[h]
\begin{center}
  \includegraphics[alt={},max width=\textwidth]{66a35da9-3faa-4823-9ca9-40799f37b3db-196_322_168_185_956}
\captionsetup{labelformat=empty}
\caption{Рис. 5.30}
\end{center}
\end{figure}

Решение. Воспользуемся уравнением моментов (5.15) относительно оси \(O\). Для этого найдем момент импульса системы относительно данной оси \(M_{z}\) и соответствующий момент сил \(N_{z}\). Момент импульса

\[
M_{z}=I \omega_{z}+R m v=\left(m_{0} / 2+m\right) R^{2} \omega_{z},
\]

где учтено, что момент инерции цилиндра \(I=m_{0} R^{2} / 2\) и \(v=\omega_{z} R\) (отсутствие скольжения шнура). Момент внешних сил тяжести относительно оси \(O\)

\[
N_{z}=R m g x / l .
\]

Продифференцировав \(M_{z}\) по времени и подставив \(\mathrm{d} M_{z} / \mathrm{d} t\) и \(N_{z}\) в уравнение моментов, получим

\[
\beta_{z}=\frac{2 m g x}{l R\left(m_{0}+2 m\right)}
\]

5.8. На гладкой горизонтальной плоскости лежит однородный диск радиуса \(r_{0}\). На него осторожно опустили другой такой же диск, предварительно сообщив ему угловую скорость \(\omega_{0}\). Через сколько времени оба диска будут вращаться с одной и той же угловой скоростью, если коэффициент трения между дисками равен \(k\) ?\\
Решение. Сначала найдем установившуюся угловую скорость вращения ω. Из закона сохранения момента импульса следует, что

\[
I \omega_{0}=2 I \omega,
\]

где \(I\) - момент инерции каждого диска относительно общей оси вращения. Отсюда

\[
\omega=\omega_{0} / 2 .
\]

Теперь рассмотрим поведение одного из дисков, например нижнего. Его угловая скорость меняется под действием момента \(N\) сил трения. Вычислим \(N\). Для этого выделим на верхней поверхности диска элементарное кольцо с радиусами \(r, r+\mathrm{d} r\). Момент сил трения, действующих на данное кольцо, равен

\[
\mathrm{d} N=r k\left(m g / \pi r_{0}^{2}\right) 2 \pi r \mathrm{~d} r=2 k\left(m g / r_{0}^{2}\right) r^{2} \mathrm{~d} r,
\]

где \(m\) - масса каждого диска. Проинтегрировав это выражение по \(r\) от 0 до \(r_{0}\), получим

\[
N=\frac{2}{3} k m g r_{0} .
\]

Согласно уравнению (5.30), приращение угловой скорости нижнего диска на величину \(\mathrm{d} \omega\) происходит за время

\[
\mathrm{d} t=(I / N) \mathrm{d} \omega=\left(3 r_{0} / 4 k g\right) \mathrm{d} \omega .
\]

Интегрируя это уравнение по \(\omega\) от 0 до \(\omega_{0} / 2\), находим, что искомое время

\[
t=3 r_{0} \omega_{0} / 8 k g .
\]

5.9. Плоское движение твердого тела. Однородный цилиндр находится на горизонтальной доске (рис. 5.31). Коэффициент трения между ними равен \(k\). Доске сообщили ускорение \(a\) в горизонтальном направлении перпендикулярно оси цилиндра. Найти:

\begin{enumerate}
  \item ускорение оси цилиндра в отсутствие скольжения;
\end{enumerate}

\begin{figure}[h]
\begin{center}
  \includegraphics[alt={},max width=\textwidth]{66a35da9-3faa-4823-9ca9-40799f37b3db-197_192_358_1178_132}
\captionsetup{labelformat=empty}
\caption{Рис. 5.31}
\end{center}
\end{figure}

\begin{enumerate}
  \setcounter{enumi}{1}
  \item предельное значение \(a_{\text {пр }}\), при котором скольжение еще отсутствует.
\end{enumerate}

Решение. 1. Выбрав положительные направления \(X\) и \(\varphi\), как показано на рис. 5.31, запишем уравнение движения оси цилиндра и уравнение моментов в Ц-системе относительно этой оси:

\[
m a_{C}=F_{\text {тр }}, \quad I \beta=r F_{\text {тр }},
\]

где \(m\) и \(I\) - масса и момент инерции цилиндра относительно его оси, \(r\) - радиус цилиндра. Кроме того, отсутствие скольжения цилиндра дает кинематическую связь ускорений:

\[
a-a_{C}=\beta r .
\]

Из этих уравнений находим \(a_{C}=a / 3\).\\
2. Определим из предыдущих уравнений значение силы трения \(F_{\text {тр }}\), обеспечивающей качение цилиндра без скольжения: \(F_{\text {тр }}=m a / 3\). Максимально возможное значение этой силы равно \(k m g\). Отсюда

\[
a_{\text {пр }}=3 k g .
\]

5.10. Однородный шар радиуса \(r\) начинает скатываться без скольжения с вершины сферы радиуса \(R\) (рис. 5.32). Найти угловую скорость \(\omega\) шара после отрыва от поверхности сферы.

Решение. Прежде всего заметим, что угловая скорость шара после отрыва от поверхности сферы изменяться не будет. Поэтому задача сводится к нахождению ее значения в момент отрыва.

Запишем уравнение движения центра шара в момент отрыва:

\begin{figure}[h]
\begin{center}
  \includegraphics[alt={},max width=\textwidth]{66a35da9-3faa-4823-9ca9-40799f37b3db-198_273_350_512_810}
\captionsetup{labelformat=empty}
\caption{Рис. 5.32}
\end{center}
\end{figure}

\[
m v^{2} /(R+r)=m g \cos \vartheta,
\]

где \(v\) - скорость центра шара в момент отрыва, \(\vartheta\) - соответствующий угол (рис. 5.32). Скорость \(v\) можно найти из закона сохранения энергии:

\[
m g h=m v^{2} / 2+I \omega^{2} / 2,
\]

где \(I\) - момент инерции шара относительно оси, проходящей через его центр. Кроме того,

\[
v=\omega r, \quad h=(R+r)(1-\cos \vartheta) .
\]

Из этих четырех уравнений получим

\[
\omega=\sqrt{10 g(R+r) / 17 r^{2}} .
\]

5.11. Конический маятник. Тонкий однородный стержень массы \(m\) и длины \(l\) вращается с постоянной угловой скоростью \(\omega\) вокруг вертикальной оси, проходящей через его точку подвеса \(O\) (рис. 5.33). При этом стержень описывает коническую поверхность с некоторым углом полураствора Э. Найти Э, а также модуль и направление силы реакции \(\mathbf{R}\) в точке \(O\).

Решение. Рассмотрим систему отсчета, вращающуюся вокруг вертикальной оси вместе со стержнем. В этой системе отсчета на стержень действует кроме силы тяжести \(m \mathbf{g}\) и силы реакции \(\mathbf{R}\)

\begin{figure}[h]
\begin{center}
  \includegraphics[alt={},max width=\textwidth]{66a35da9-3faa-4823-9ca9-40799f37b3db-199_387_347_138_163}
\captionsetup{labelformat=empty}
\caption{Рис. 5.33}
\end{center}
\end{figure}

еще центробежная сила инерции \(\mathbf{F}_{\text {цб }}\). Так как стержень покоится в данной системе отсчета, т. е. находится в состоянии равновесия, то это значит, что результирующий момент всех сил относительно любой точки и результирующая всех сил равны нулю.

Относительно точки \(O\) момент создают только сила тяжести и центробежные силы инерции. Из равенства моментов этих сил следует


\begin{equation*}
\frac{1}{2} m g l \sin \vartheta=N_{\text {цб }} . \tag{1}
\end{equation*}


Вычислим \(N_{\text {цб }}\). Элементарный момент сил инерции, который действует на элемент стержня \(\mathrm{d} x\), находящийся на расстоянии \(x\) от точки \(O\), равен

\[
\mathrm{d} N_{\text {цб }}=\left(m \omega^{2} / l\right) \sin \vartheta \cos \vartheta x^{2} \mathrm{~d} x .
\]

Проинтегрировав это выражение по всей длине стержня, получим


\begin{equation*}
N_{\text {цб }}=1 / 3 m \omega^{2} l^{2} \sin \vartheta \cos \vartheta . \tag{2}
\end{equation*}


Из (1) и (2) следует, что


\begin{equation*}
\cos \vartheta=3 g / 2 \omega^{2} l . \tag{3}
\end{equation*}


Найдем модуль и направление вектора R. В системе отсчета, где стержень вращается с угловой скоростью \(\omega\), его центр масс (точка \(C\) ) движется по горизонтальной окружности. Поэтому из уравнения движения центра масс (3.11) сразу следует, что вертикальная составляющая вектора \(\mathbf{R}\) есть \(R_{\|}=m g\), а горизонтальная составляющая \(R_{\perp}\) определяется уравнением \(m a_{n}=R_{\perp}\), где \(a_{n}\) - нормальное ускорение центра масс \(C\). Отсюда


\begin{equation*}
R_{\perp}=1 / 2 m \omega^{2} l \sin \vartheta . \tag{4}
\end{equation*}


Модуль вектора \(\mathbf{R}\) есть

\[
R=\sqrt{(m g)^{2}+R_{\perp}^{2}}=\frac{m \omega^{2} l}{2} \sqrt{1+{ }^{7} / 4\left(g / \omega^{2} l\right)^{2}},
\]

а его направление - угол \(\vartheta^{\prime}\) между вектором \(\mathbf{R}\) и вертикалью определяется формулой \(\cos \vartheta^{\prime}=m g / R\). Интересно, что \(\vartheta^{\prime} \neq \vartheta\), т. е. вектор \(\mathbf{R}\) не совпадает по направлению со стержнем. В этом легко убедиться, выразив \(\cos \vartheta^{\prime}\) через \(\cos \vartheta\) :

\[
\cos \vartheta^{\prime}=\frac{4 \cos \vartheta}{\sqrt{9+7 \cos ^{2} \vartheta}}
\]

Отсюда видно, что \(\cos \vartheta^{\prime}>\cos \vartheta\), т. е. \(\vartheta^{\prime}<\vartheta\). Это и показано на рис. 5.33.\\
Заметим, что равнодействующая сил инерции \(F_{\text {цб }}\) проходит не через точку \(C\), а ниже. Действительно \(F_{\text {цб }}=R_{\perp}\) и определяется формулой (4), а результирующий момент \(N_{\text {цб }}\) - формулой (2). Из этих формул следует, что плечо вектора \(\mathbf{F}_{\text {цб }}\) относительно точки \(O\) равно \(\frac{2}{3} l \cos \vartheta\) (рис. 5.33).\\
5.12. Гироскоп. Волчок массы \(m\), ось которого составляет угол \(\vartheta\) с вертикалью, прецессирует вокруг вертикальной оси, проходящей через точку опоры \(O\). Момент импульса волчка равен \(M\), расстояние от его центра масс до точки \(O\) есть \(l\). Найти модуль и направление вектоpa F - горизонтальной составляющей силы реакции в точке \(O\).\\
Решение. Согласно (5.38), угловая скорость прецессии

\[
\omega^{\prime}=m g l / M
\]

\begin{figure}[h]
\begin{center}
  \includegraphics[alt={},max width=\textwidth]{66a35da9-3faa-4823-9ca9-40799f37b3db-200_374_323_789_839}
\captionsetup{labelformat=empty}
\caption{Рис. 5.34}
\end{center}
\end{figure}

Центр масс волчка движется по окружности. Следовательно, вектор \(\mathbf{F}\) направлен так, как показано на рис. 5.34 (этот вектор поворачивается вместе с осью волчка).\\
Из уравнения движения центра масс (3.11) получаем

\[
m \omega^{\prime 2} l \sin \vartheta=F
\]

В результате

\[
F=\left(m^{3} g^{2} l^{3} / M^{2}\right) \sin \vartheta .
\]

Заметим, что если бы точка опоры волчка находилась на гладкой плоскости, то волчок прецессировал бы с той же угловой скоростью, но вокруг вертикальной оси, проходящей через центр масс волчка - точку \(C\) на рис. 5.34.

\section*{Колебания}
\section*{§ 6.1. Гармонические колебания Кинематика гармонических колебаний}
Гармоническими называют колебания, в которых интересующая нас величина \(x\) (например, линейное или угловое смещение из положения равновесия) изменяется со временем \(t\) по закону


\begin{equation*}
x=a \cos \left(\omega_{0} t+\alpha\right), \tag{6.1}
\end{equation*}


где \(a\) - амплитуда, \(\left(\omega_{0} t+\alpha\right)\) - фаза, \(\alpha\) - начальная фаза, \(\omega_{0}\) - циклическая (круговая) частота колебаний. Эта частота связана с периодом \(T\) и линейной частотой \(v\) как


\begin{equation*}
\omega_{0}=2 \pi / T=2 \pi v . \tag{6.2}
\end{equation*}


Обратим внимание на различие наименований циклической и линейной частот: \(\omega, \mathrm{c}^{-1}\), а \(v\), Гц (гери).

Продифференцировав (6.1) по времени, найдем скорость \(\dot{x}\) и ускорение \(\ddot{x}\)


\begin{gather*}
\dot{x}=-a \omega_{0} \sin \left(\omega_{0} t+\alpha\right)=a \omega_{0} \cos \left(\omega_{0} t+\alpha+\pi / 2\right),  \tag{6.3}\\
\ddot{x}=-a \omega_{0}^{2} \cos \left(\omega_{0} t+\alpha\right)=a \omega_{0}^{2} \cos \left(\omega_{0} t+\alpha+\pi\right) . \tag{6.4}
\end{gather*}


Из этих выражений видно, что скорость \(\dot{x}\) и ускорение \(\ddot{x}\) также изменяются по гармоническому закону с амплитудами \(a \omega_{0}\) и \(a \omega_{0}^{2}\) соответственно. При этом скорость опережает смещение \(x\) по фазе на \(\pi / 2\), а ускорение - на \(\pi\), т. е. находится в противофазе со смещением \(x\). На рис. 6.1 приведены графики зависимостей \(x(t), \dot{x}(t)\) и \(\ddot{x}(t)\) для случая \(\alpha=0\).

Сопоставив (6.4) и (6.1), видим, что \(\ddot{x}=-\omega_{0}^{2} x\), или


\begin{equation*}
\ddot{x}+\omega_{0}^{2} x=0 . \tag{6.5}
\end{equation*}


Это дифференциальное уравнение называют уравнением гармонического осциллятора. Его решение (6.1)* содержит две произвольные постоянные: \(a\) и \(\alpha\). Для каждого конкретного колебания они определяются начальньлми условиями - смещением \(x_{0}\) и скоростью \(\dot{x}_{0}\) в начальный момент \(t=0\) :

\begin{figure}[h]
\begin{center}
  \includegraphics[alt={},max width=\textwidth]{66a35da9-3faa-4823-9ca9-40799f37b3db-202_296_388_138_766}
\captionsetup{labelformat=empty}
\caption{Рис. 6.1}
\end{center}
\end{figure}


\begin{equation*}
x_{0}=a \cos \alpha, \quad \dot{x}_{0}=-a \omega_{0} \sin \alpha . \tag{6.6}
\end{equation*}


Отсюда находим искомые постоянные:


\begin{equation*}
a=\sqrt{x_{0}^{2}+\left(\dot{x}_{0} / \omega_{0}\right)^{2}}, \quad \operatorname{tg} \alpha=-\dot{x}_{0} / \omega_{0} x_{0} . \tag{6.7}
\end{equation*}


Обычно рассматривают только значения \(\alpha\) в интервале \((-\pi,+\pi)\). Уравнение для \(\operatorname{tg} \alpha\) удовлетворяется двумя значениями \(\alpha\) в этом интервале. Из этих значений следует взять то, при котором получаются правильные знаки у \(\cos \alpha\) и \(\sin \alpha\) в (6.6).

Рассмотрим два примера на роль начальных условий.\\
Пример 1. Колебания совершаются по гармоническому закону, график которых приведен на рис. 6.2 (два случая: 1 и 2). Выясним, каковы начальные условия в обоих случаях и как они отражаются на характере колебаний.

В случае 1 начальное смещение \(x_{0}>0\) и скорость \(\dot{x}_{0}=0\) (напомним, что производная характеризует наклон графика \(x(t)\) в данной точке). При этом амплитуда колебаний \(a=x_{0}\).

\begin{figure}[h]
\begin{center}
  \includegraphics[alt={},max width=\textwidth]{66a35da9-3faa-4823-9ca9-40799f37b3db-202_230_382_1138_778}
\captionsetup{labelformat=empty}
\caption{Рис. 6.2}
\end{center}
\end{figure}

В случае же \(2 x_{0}>0\) и \(\dot{x}_{0}<0\). Амплитуда \(a\), согласно (6.7), будет больше \(x_{0}\).\\
Пример 2. Частица совершает колебания по закону (6.1) и в момент \(t=0 x_{0}<0\) и \(\dot{x}_{0}>0\). Найдем интервал возможных значений начальной фазы \(\alpha\).

\footnotetext{\begin{itemize}
  \item Заметим, что решение (6.1) может быть представлено и в ином виде, например \(x=A \sin \omega_{0} t+B \cos \omega_{0} t, \quad\) где \(\quad A \quad\) и \(\quad B \quad\) - постоянные, или как \(x=\operatorname{Re}\left\{a \mathrm{e}^{i\left(\omega_{0} t+\alpha\right)}\right\}\).
\end{itemize}
}Из (6.7) следует, что в данном случае \(\operatorname{tg} \alpha>0\). Значит, \(\alpha\) может находиться в первом или третьем квадрантах, т. е. в интервалах \((0, \pi / 2)\) или \((-\pi,-\pi / 2)\). Сопоставив это со знаками \(\cos \alpha\) и \(\sin \alpha\) из (6.6), приходим к выводу, что возможен только второй интервал ( \(-\pi,-\pi / 2\) ).

\section*{Динамика гармонических колебаний}
Для определения характера движения механической системы нужно, исходя из законов динамики или закона сохранения энергии, составить уравнение движения системы, и если оно приводится к виду (6.5), то можно однозначно утверждать, что данная система является гармоническим осциллятором, частота \(\omega_{0}\) которого равна корню квадратному из коэффициента при \(x\). Рассмотрим несколько примеров и затем обобщим полученные результаты.

Грузик на пружине. Пусть грузик массы \(m\), подвешенный на

\begin{figure}[h]
\begin{center}
  \includegraphics[alt={},max width=\textwidth]{66a35da9-3faa-4823-9ca9-40799f37b3db-203_360_241_830_72}
\captionsetup{labelformat=empty}
\caption{Рис. 6.3}
\end{center}
\end{figure}

невесомой пружине жесткости \(\varkappa\), совершает вертикальные колебания (рис. 6.3). Возьмем начало \(O\) оси \(X\) в положении равновесия, где \(m g=\varkappa \Delta l, \Delta l\) - растяжение пружины в этом положении. Тогда, согласно основному уравнению динамики, \(m \ddot{x}=m g-\varkappa(x+\Delta l)=-\varkappa x\), или

\[
\ddot{x}+(x / m) x=0 .
\]

Из сопоставления с (6.5) видим, что это уравнение гармонического осциллятора, колеблющегося около положения равновесия с частотой \(\omega_{0}\) и периодом \(T\), равными


\begin{equation*}
\omega_{0}=\sqrt{\varkappa / m}, \quad T=2 \pi \sqrt{m / \varkappa} . \tag{6.8}
\end{equation*}


Математический маятник. Материальная точка массы \(m\), подвешенная на нерастяжимой нити длиной \(l\), совершает колебания в вертикальной плоскости (рис. 6.4). Здесь удобнее всего использовать уравнение динамики (2.16) в проекции на орт \(\tau\), направление которого совпадает с положительным направлением отсчета дуговой координаты \(s\) (величина алгебраическая, на рисунке изображен момент, когда \(s>0\) ). Начало отсчета \(s\) возьмем в положении равновесия - в точке \(O\).

Имея в виду, что \(s=l \vartheta, \ddot{s}=l \ddot{\vartheta}\) и что проекция силы натяжения \(F_{\tau}=0\), запишем: \(m \ddot{s}=m l \ddot{\vartheta}=-m g \sin \vartheta\), или

\[
\ddot{\vartheta}+(g / l) \sin \vartheta=0 .
\]

Из сопоставления с (6.5) видим, что это уравнение, вообще говоря, не является уравнением гармонического осциллятора, поскольку в нем вместо смещения Э стоит \(\sin \vartheta\). Однако при мальιx колебаниях, когда \(\sin \vartheta \approx \vartheta\), уравнение совпадает с (6.5):

\begin{figure}[h]
\begin{center}
  \includegraphics[alt={},max width=\textwidth]{66a35da9-3faa-4823-9ca9-40799f37b3db-204_460_317_133_843}
\captionsetup{labelformat=empty}
\caption{Рис. 6.4}
\end{center}
\end{figure}

\[
\ddot{\vartheta}+(g / l) \vartheta=0,
\]

откуда следует, что частота \(\omega_{0}\) и период \(T\) математического маятника, совершающего малые колебания, равны


\begin{equation*}
\omega_{0}=\sqrt{g / l}, \quad T=2 \pi \sqrt{l / g} . \tag{6.9}
\end{equation*}


Физический маятник. Это твердое тело, совершающее колебания вокруг неподвижной оси, жестко связанной с телом. Рассмотрим колебания под действием силы тяжести (рис. 6.5). Выберем положительное направление отсчета угла \(\vartheta\) против часовой стрелки (ось \(Z\) направлена к нам). Тогда проекция момента силы тяжести на ось \(Z\) запишется как \(M_{z}=-m g l \sin \vartheta\) и уравнение динамики вращательного движения твердого тела (5.30) примет вид

\[
I \ddot{\vartheta}=-m g l \sin \vartheta,
\]

\begin{figure}[h]
\begin{center}
  \includegraphics[alt={},max width=\textwidth]{66a35da9-3faa-4823-9ca9-40799f37b3db-204_382_269_996_871}
\captionsetup{labelformat=empty}
\caption{Рис. 6.5}
\end{center}
\end{figure}

где \(I\) - момент инерции тела относительно оси \(O, l\) - расстояние между осью \(O\) и центром масс \(C\). Ограничимся рассмотрением маль к колебаний, при которых \(\sin \vartheta \approx \vartheta\). При этом условии предыдущее уравнение можно записать так:

\[
\ddot{\vartheta}+(m g l / I) \vartheta=0 .
\]

Колебания будут гармоническими с частотой \(\omega_{0}\) и периодом \(T\), равными


\begin{equation*}
\omega_{0}=\sqrt{m g l / I}, \quad T=2 \pi \sqrt{I / m g l} \tag{6.10}
\end{equation*}


Такую же частоту и период имеет математический маятник длины


\begin{equation*}
l_{\text {пр }}=I / m l, \tag{6.11}
\end{equation*}


которую называют приведенной длиной физического маятника.\\
Точку \(O^{\prime}\) (рис. 6.5), которая находится на прямой, проходящей через точку подвеса \(O\) и центр масс \(C\), и отстоит от точки \(O\) на расстоянии \(l_{\text {пр }}\), называют центром качания физического маятника. Центр качания \(O^{\prime}\) обладает замечательным свойством: если маятник перевернуть и заставить совершать малые колебания вокруг оси \(O^{\prime}\), то период колебаний не изменится. На этом свойстве основано определение ускорения свободного падения с помощью оборотного маятника: экспериментально устанавливают положения двух «сопряженных» точек (осей) \(O\) и \(O^{\prime}\), малые колебания вокруг которых происходят с одинаковой частотой. Это значит, что расстояние \(O O^{\prime}=l_{\text {пр }}\). Определив \(\omega_{0}\) и \(l_{\text {пр }}\), из формулы \(\omega_{0}=\sqrt{g / l_{\text {пр }}}\) находим \(g\).

Общие выводы. Рассмотренные примеры относятся к свободным колебаниям без трения, которые происходят в системе, предоставленной самой себе после того, как она была тем или иным способом выведена из состояния равновесия. Можно утверждать, что свободные колебания любого осциллятора в отсутствие трения будут гармоническими, если действующая в нем сила (или момент силы) является квазиупругой, т. е. силой, направленной к положению равновесия и зависящей от смещения из этого положения линейно.

Именно квазиупругий характер силы (или момента силы) служит и критерием мальιх колебаний.

Кроме того, частота и период свободных колебаний без трения зависят только от свойств самого осциллятора в отличие от амплитуды колебаний и начальной фазы, которые определяются начальными условиями.

Рассмотрим еще один пример на малые колебания.

Пример. Частица массы \(m\) совершает колебания в силовом поле, где ее потенциальная энергия зависит от координаты \(x\) как \(U=U_{0}(1-\cos \alpha x)\), где \(U_{0}\) и \(\alpha\) - постоянные. Найдем частоту \(\omega_{0}\) малых колебаний частицы около положения равновесия \(x=0\).

Согласно основному уравнению динамики,

\[
m \ddot{x}=F_{x}=-\partial U / \partial x=-\alpha U_{0} \sin \alpha x .
\]

Так как колебания малые, то \(\sin \alpha x \approx \alpha x\) и предыдущее уравнение можно привести к виду

\[
\ddot{x}+\left(\alpha^{2} U_{0} / m\right) x=0 .
\]

Отсюда следует, что \(\omega_{0}=\alpha \sqrt{U_{0} / m}\).\\
Энергия гармонического осциллятора. Рассмотрим этот вопрос на примере материальной точки массы \(m\), колеблющейся под действием квазиупругой силы \(F_{x}=-\varkappa x\). Потенциальная и кинетическая энергии частицы имеют в данном случае такой вид:


\begin{gather*}
U=\varkappa x^{2} / 2=\left(\varkappa a^{2} / 2\right) \cos ^{2}\left(\omega_{0} t+\alpha\right),  \tag{6.12}\\
K=m \dot{x}^{2} / 2=\left(m a^{2} \omega_{0}^{2} / 2\right) \sin ^{2}\left(\omega_{0} t+\alpha\right) .
\end{gather*}


Из этих соотношений видно, что значения \(U\) и \(K\) сдвинуты друг относительно друга по фазе на \(\pi / 2:\) когда \(U\) максимальна, \(K\) минимальна, и наоборот. При этом полная энергия сохраняется:


\begin{equation*}
E=U+K=\varkappa a^{2} / 2=m a^{2} \omega_{0}^{2} / 2, \tag{6.13}
\end{equation*}


где учтено, что \(\omega_{0}^{2}=x / m\). Принимая во внимание (6.13), формулы (6.12) можно переписать так:


\begin{equation*}
U=E \cos ^{2}\left(\omega_{0} t+\alpha\right), \quad K=E \sin ^{2}\left(\omega_{0} t+\alpha\right) . \tag{6.14}
\end{equation*}


Графики зависимостей \(U(t)\) и \(K(t)\) даны на рис. 6.6. Из рисунка видно, что в процессе колебаний происходит переход потенциальной энергии в кинетическую и обратно. Это иллюстрирует и рис. 6.7.

\begin{figure}[h]
\begin{center}
  \includegraphics[alt={},max width=\textwidth]{66a35da9-3faa-4823-9ca9-40799f37b3db-207_277_445_136_129}
\captionsetup{labelformat=empty}
\caption{Рис. 6.6}
\end{center}
\end{figure}

\begin{figure}[h]
\begin{center}
  \includegraphics[alt={},max width=\textwidth]{66a35da9-3faa-4823-9ca9-40799f37b3db-207_279_344_135_719}
\captionsetup{labelformat=empty}
\caption{Рис. 6.7}
\end{center}
\end{figure}

Средние (за период колебания) значения потенциальной и кинетической энергий одинаковы, и каждое из них равно \(E / 2\) :


\begin{equation*}
\langle U\rangle=\langle K\rangle=E / 2, \tag{6.15}
\end{equation*}


поскольку известно*, что средние (за период) значения квадратов синуса и косинуса равны \(1 / 2\).

Отметим в заключение, что, согласно (6.13), энергия колебаний осциллятора \(E c a^{2}\). Это весьма существенный факт, и его неоднократно придется учитывать в дальнейшем.

\section*{Энергия и уравнение движения}
Уравнение движения колебательной системы можно получить не только из уравнений динамики, но и из закона сохранения энергии \(E\) (иногда это бывает удобнее). Для этого нужно составить выражение для энергии \(E\), продифференцировать его по времени и потребовать, чтобы \(\mathrm{d} E / \mathrm{d} t=0\), поскольку \(E=\) const. Это и приведет к искомому уравнению.

Важно отметить, что колебательная система будет гармоническим осциллятором лишь при условии \(U \sim x^{2}\), т. е. когда потенциальная энергия пропорциональна квадрату смещения из положения равновесия. Это условие, кстати, является и «энергетическим» критерием мальх колебаний.

\footnotetext{\begin{itemize}
  \item Это сразу следует из тождества \(\sin ^{2} \varphi+\cos ^{2} \varphi=1\), усредняя которое в пределах периода \(2 \pi\) получим \(\left\langle\sin ^{2} \varphi\right\rangle+\left\langle\cos ^{2} \varphi\right\rangle=1\). Так как разница между \(\sin \varphi\) и \(\cos \varphi\) заключается только в сдвиге фаз на \(\pi / 2\), мы находим, что \(\left\langle\sin ^{2} \varphi\right\rangle=\left\langle\cos ^{2} \varphi\right\rangle=1 / 2\).
\end{itemize}
}Пример. Пусть в колебательной системе \(U=\alpha x^{2}\) и \(K=\beta \dot{x}^{2}\), где \(x\) смещение из положения равновесия, \(\alpha\) и \(\beta\) - положительные постоянные. Убедимся, что условие \(E=U+K=\) const приводит к уравнению гармонического осциллятора. Продифференцировав \(E\) по времени, получим

\[
\mathrm{d} E / \mathrm{d} t=2 \alpha x \dot{x}+2 \beta \dot{x} \ddot{x}=0 .
\]

Отсюда следует, что \(\ddot{x}+(\alpha / \beta) x=0\). Это и есть уравнение гармонического осциллятора с частотой \(\omega_{0}=\sqrt{\alpha / \beta}\).

\section*{§ 6.2. Сложение гармонических колебаний}
\section*{Сложение колебаний одного направления}
Векторная диаграмма. Решение ряда вопросов значительно облегчается и становится наглядным, если изображать колебания графически с помощью вектора-амплитуды \(\mathbf{a}\), вращающегося с угловой скоростью \(\omega\) против часовой стрелки. Если в момент \(t=0\) вектор а образует с осью \(X\) угол \(\alpha\) (рис. 6.8), то проекция вектора а на ось \(X\) изменяется со временем по гармоническому закону (6.1). Такой способ представ-

\begin{figure}[h]
\begin{center}
  \includegraphics[alt={},max width=\textwidth]{66a35da9-3faa-4823-9ca9-40799f37b3db-208_181_338_786_822}
\captionsetup{labelformat=empty}
\caption{Рис. 6.8}
\end{center}
\end{figure}

ления колебаний, называемый векторной диаграммой, удобно использовать при сложении колебаний одного направления.

Сложение колебаний. Рассмотрим два случая, когда частоты двух складываемых колебаний одинаковы или мало отличаются друг от друга.

\begin{enumerate}
  \item Случай, когда \(\omega_{1}=\omega_{2}=\omega\). В этом случае результирующее смещение
\end{enumerate}


\begin{equation*}
x=x_{1}+x_{2}=a_{1} \cos \left(\omega t+\alpha_{1}\right)+a_{2} \cos \left(\omega t+\alpha_{2}\right) . \tag{6.16}
\end{equation*}


Каждое из складываемых колебаний можно представить с помощью векторов \(\mathbf{a}_{1}\) и \(\mathbf{a}_{2}\), сумма проекций которых на ось \(X\) равна проекции суммы векторов \(\mathbf{a}_{1}+\mathbf{a}_{2}=\mathbf{a}\) (рис. 6.9). Поскольку векторы \(\mathbf{a}_{1}\) и \(\mathbf{a}_{2}\) вращаются с одной и той же угловой скоростью \(\omega\), с той же угловой скоростью вращается и вектор а. Значит результирующее колебание является тоже гармоническим и имеет вид


\begin{equation*}
x=a \cos (\omega t+\alpha), \tag{6.17}
\end{equation*}


где \(a\) и \(\alpha\) находим из рис. 6.9:


\begin{align*}
& a^{2}=a_{1}^{2}+a_{2}^{2}+2 a_{1} a_{2} \cos \delta  \tag{6.18}\\
& \operatorname{tg} \alpha=\frac{a_{1} \sin \alpha_{1}+a_{2} \sin \alpha_{2}}{a_{1} \cos \alpha_{1}+a_{2} \cos \alpha_{2}} . \tag{6.19}
\end{align*}


Разность фаз \(\delta\) в данном случае не зависит от времени и равна


\begin{equation*}
\delta=\alpha_{2}-\alpha_{1} . \tag{6.20}
\end{equation*}


Из рис. 6.9 и формулы (6.18) видно, что амплитуда \(a\) результирующего колебания существенно зависит от разности фаз \(\delta\). При сложении синфазных колебаний ( \(\delta=0\) ) a максимально, при сложении же «противофазных» колебаний ( \(\delta=\pi\) ) \(a\) минимально:

\[
a_{\text {макс }}=a_{1}+a_{2}, \quad a_{\text {мин }}=\left|a_{1}-a_{2}\right| .
\]

Поскольку энергия колебаний \(E\) っ \(a^{2}\), то при сложении колебаний одного направления \(E\), как и амплитуда \(a\), существенно зависит от разности фаз \(\delta\), достигая максимума при сложении синфазных колебаний и минимума при сложении «противофазных» колебаний. Из-за наличия последнего слагаемого в (6.18) энергия результирующего колебания не может быть представлена как сумма энергий складываемых колебаний, т. е. \(E \neq E_{1}+E_{2}\) (за исключением случая, когда \(\delta=\pi / 2\) ).\\
2. Случай, когда \(\left|\omega_{1}-\omega_{2}\right| \ll \omega_{1}\) и \(\omega_{2}\). Здесь также результи-

\begin{figure}[h]
\begin{center}
  \includegraphics[alt={},max width=\textwidth]{66a35da9-3faa-4823-9ca9-40799f37b3db-209_293_362_1359_68}
\captionsetup{labelformat=empty}
\caption{Рис. 6.9}
\end{center}
\end{figure}

рующее колебание записывается формулой (6.18) и справедлив рис. 6.9. Но поскольку теперь векторы \(\mathbf{a}_{1}\) и \(\mathbf{a}_{2}\) вращаются с немного отличающимися угловыми скоростями, модуль результирующего вектора а будет медленно изменяться от \(a_{\text {макс }}\) до \(a_{\text {мин }}\), причем сам вектор а вращается с угловой скоростью, близкой к \(\omega_{1}\) и \(\omega_{2}\). Результирующее колебание уже не является гармоническим, однако его все же можно рассматривать как гармоническое, но с медленно и

периодически меняющейся амплитудой. Такие колебания называют биениями. Они показаны на рис. 6.10 для случая \(a_{1}=a_{2}\).

\begin{figure}[h]
\begin{center}
  \includegraphics[alt={},max width=\textwidth]{66a35da9-3faa-4823-9ca9-40799f37b3db-210_219_638_235_294}
\captionsetup{labelformat=empty}
\caption{Рис. 6.10}
\end{center}
\end{figure}

Амплитуда колебаний описывается той же формулой (6.18), но в данном случае входящая в нее разность фаз \(\delta\) зависит от времени:


\begin{equation*}
\delta=\left(\alpha_{2}+\omega_{2} t\right)-\left(\alpha_{1}+\omega_{1} t\right)=\left(\alpha_{2}-\alpha_{1}\right)+\left(\omega_{2}-\omega_{1}\right) t . \tag{6.21}
\end{equation*}


Промежуток времени между соседними моментами, когда амплитуда \(a\) максимальна, называют периодом биений \(\tau_{б}\) (рис. 6.10). За это время разность фаз \(\delta\) изменяется на \(2 \pi\) (это следует и из векторной диаграммы). Значит, \(\left|\omega_{2}-\omega_{1}\right| \tau_{б}=2 \pi\). Отсюда период и частота биений:


\begin{equation*}
\tau_{\text {б }}=\frac{2 \pi}{\left|\omega_{2}-\omega_{1}\right|}=\frac{1}{\left|v_{2}-v_{1}\right|}, \quad v_{\text {б }}=\frac{1}{\tau_{\text {б }}}=\left|v_{2}-v_{1}\right| . \tag{6.22}
\end{equation*}


\section*{Сложение взаимно перпендикулярных колебаний}
Сначала рассмотрим случай, когда частоты складываемых колебаний одинаковы. Пусть координаты \(x\) и \(y\) частицы изменяются по закону


\begin{equation*}
x=a \cos \omega t, \quad y=b \cos (\omega t+\delta) . \tag{6.23}
\end{equation*}


Можно показать, что траекторией частицы при этом является эллипс (рис. 6.11), вид которого определяется отношением амплитуд \(a\) и \(b\) и разностью фаз \(\delta\).

Некоторые частные случаи:\\
а) \(\delta=0\), тогда \(y=(b / a) x\), т. е. частица движется по прямой в первом и третьем квадрантах (рис. 6.12, a);

\begin{figure}[h]
\begin{center}
  \includegraphics[alt={},max width=\textwidth]{66a35da9-3faa-4823-9ca9-40799f37b3db-210_305_309_1442_843}
\captionsetup{labelformat=empty}
\caption{Рис. 6.11}
\end{center}
\end{figure}

\begin{figure}[h]
\begin{center}
  \includegraphics[alt={},max width=\textwidth]{66a35da9-3faa-4823-9ca9-40799f37b3db-211_222_882_131_173}
\captionsetup{labelformat=empty}
\caption{Рис. 6.12}
\end{center}
\end{figure}

б) \(\delta=\pi\), тогда \(y=-(b / a) x\) и частица движется тоже по прямой, но во втором и четвертом квадрантах (рис. 6.12, б);\\
в) \(\delta=\pi / 2\). В этом случае \(x^{2} / a^{2}+y^{2} / b^{2}=1\), т. е. частица движется по эллипсу, полуоси которого \(a\) и \(b\) совпадают с осями координат. При \(a=b\) эллипс превращается в окружность. Так как колебания вдоль оси \(Y\) происходят с опережением по фазе на \(\pi / 2\) относительно колебаний по оси \(X\), то сначала \(y\) и лишь затем \(x\) достигают максимальных значений. Это значит, что движение частицы будет происходить по часовой стрелке (рис. 6.12, в);\\
г) \(\delta=3 \pi / 2\). Это то же, что и \(\delta=-\pi / 2\), поскольку изменение фазы на \(2 \pi\) несущественно (рис. 6.12, z).

Если частоты взаимно перпендикулярных колебаний не оди-

\begin{figure}[h]
\begin{center}
  \includegraphics[alt={},max width=\textwidth]{66a35da9-3faa-4823-9ca9-40799f37b3db-211_237_306_1020_68}
\captionsetup{labelformat=empty}
\caption{Рис. 6.13}
\end{center}
\end{figure}

наковы и относятся как целые числа, то траектории результирующего движения имеют более сложные формы. Их называют фигурами Лиссажу. Одна из этих фигур показана на рис. 6.13, она соответствует отношению частот \(\omega_{y}: \omega_{x}=3: 2\).

И последнее: при сложении взаимно перпендикулярных колебаний полная энергия


\begin{equation*}
E=\left(\frac{\varkappa_{1} x^{2}}{2}+\frac{\varkappa_{2} y^{2}}{2}\right)+\frac{m}{2}\left(\dot{x}^{2}+\dot{y}^{2}\right)=E_{x}+E_{y}, \tag{6.24}
\end{equation*}


т. е. складывается из энергий каждого колебания (в отличие от сложения колебаний одного направления). Согласно (6.13), эта энергия


\begin{equation*}
E=\frac{m}{2}\left(a^{2} \omega_{x}^{2}+b^{2} \omega_{y}^{2}\right) . \tag{6.25}
\end{equation*}


\section*{§ 6.3. Затухающие колебания}
\section*{Уравнение затухающих колебаний}
В любой реальной колебательной системе есть силы сопротивления (трения), действие которых приводит к уменьшению амплитуды и энергии колебаний. Такие свободные колебания называют затухающими.

Будем исходить из основного уравнения динамики, полагая, что на частицу массы \(m\) действует кроме квазиупругой силы \((-\varkappa x)\) сила сопротивления, пропорциональная скорости частицы, \(F_{x}=-r \dot{x}\), где \(r\) - коэффициент сопротивления (величина размерная). Тогда уравнение движения будет иметь вид


\begin{equation*}
m \ddot{x}=-\varkappa x-r \dot{x}, \tag{6.26}
\end{equation*}


или


\begin{equation*}
\ddot{x}+2 \beta \dot{x}+\omega_{0}^{2} x=0, \tag{6.27}
\end{equation*}


где \(2 \beta=r / m, \omega_{0}^{2}=\varkappa / m\). Отметим, что \(\omega_{0}\) - это частота свободных колебаний без трения. Частоту \(\omega_{0}\) называют собственной частотой осциллятора, а \(\beta\) - коэффициентом затухания.

Уравнение (6.27) при условии \(\beta<\omega_{0}\) описывает затухающие колебания. Его решение имеет вид


\begin{equation*}
x=a_{0} \mathrm{e}^{-\beta t} \cos \left(\omega^{\prime} t+\alpha\right), \tag{6.28}
\end{equation*}


где \(a_{0}\) и \(\alpha\) - постоянные, определяемые начальными условиями \(x(0)=x_{0}\) и \(\dot{x}(0)=\dot{x}_{0}\), \(\omega^{\prime}\) - частота затухающих колебаний:


\begin{equation*}
\omega^{\prime}=\sqrt{\omega_{0}^{2}-\beta^{2}} . \tag{6.29}
\end{equation*}


График функции (6.28) показан на рис. 6.14 для случая \(x_{0}>0\) и \(\dot{x}_{0}>0\). Видно, что эта функция не периодическая. Тем не менее величину \(T=2 \pi / \omega^{\prime}\) принято называть периодом затухающих колебаний:


\begin{equation*}
T=2 \pi / \sqrt{\omega_{0}^{2}-\beta^{2}} \tag{6.30}
\end{equation*}


Множитель \(a=a_{0} \mathrm{e}^{-\beta t}\) перед косинусом в (6.28) называют амплитудой затухающих колебаний (пунктир на рис. 6.14).

\begin{figure}[h]
\begin{center}
  \includegraphics[alt={},max width=\textwidth]{66a35da9-3faa-4823-9ca9-40799f37b3db-213_303_444_135_123}
\captionsetup{labelformat=empty}
\caption{Рис. 6.14}
\end{center}
\end{figure}

\begin{figure}[h]
\begin{center}
  \includegraphics[alt={},max width=\textwidth]{66a35da9-3faa-4823-9ca9-40799f37b3db-213_301_368_137_701}
\captionsetup{labelformat=empty}
\caption{Рис. 6.15}
\end{center}
\end{figure}

\section*{Энергия затухающих колебаний}
Эта энергия складывается из потенциальной и кинетической: \(E=\varkappa x^{2} / 2+m \dot{x}^{2} / 2\). После подстановки сюда выражений \(x(t)\) и \(\dot{x}(t)\), соответствующих затухающими колебаниям (6.28), получим зависимость \(E(t)\), которая графически показана на рис. 6.15. Уменьшение энергии колебаний обусловлено работой силы сопротивления. Мощность этой силы равна \(-r \dot{x} \cdot \dot{x}=-r \dot{x}^{2}\), тогда

\[
\mathrm{d} E / \mathrm{d} t=-r \dot{x}^{2} .
\]

Таким образом, \(\mathrm{d} E / \mathrm{d} t<0\), кроме тех моментов, когда \(\dot{x}=0\).\\
При малом затухании ( \(\beta \ll \omega_{0}\) ) зависимость \(E(t)\) становится практически экспоненциальной:


\begin{equation*}
E=E_{0} \mathrm{e}^{-2 \beta t} . \tag{6.31}
\end{equation*}


Отсюда убыль энергии в единицу времени


\begin{equation*}
-\mathrm{d} E / \mathrm{d} t=2 \beta E . \tag{*}
\end{equation*}


\section*{Характеристики затухания}
Кроме коэффициента \(\beta\) затухание характеризуют и другими величинами:

\begin{enumerate}
  \item Время релаксации \(\tau\) - это время, за которое амплитуда колебаний уменьшается в е раз. Из выражения \(a=a_{0} \mathrm{e}^{-\beta t}\) видно, что
\end{enumerate}


\begin{equation*}
\tau=1 / \beta . \tag{6.32}
\end{equation*}


\begin{enumerate}
  \setcounter{enumi}{1}
  \item Логарифмический декремент затухания. Его определяют как
\end{enumerate}


\begin{equation*}
\lambda=\ln \frac{a(t)}{a(t+T)}=\beta T, \tag{6.33}
\end{equation*}


где \(T\) - период затухающих колебаний. Из предыдущих двух формул следует, что


\begin{equation*}
\lambda=1 / N_{\mathrm{e}}, \tag{6.34}
\end{equation*}


где \(N_{\mathrm{e}}\) - число колебаний за время \(\tau\), в течение которого амплитуда уменьшается в е раз.

При малом затухании ( \(\beta \ll \omega_{0}\) ) \(\lambda\) характеризует относительное уменьшение амплитуды колебаний за период. Это следует из (6.33), поскольку в этом случае


\begin{equation*}
\lambda=\ln \frac{a+\delta a}{a}=\ln \left(1+\frac{\delta a}{a}\right) \approx \frac{\delta a}{a} . \tag{6.35}
\end{equation*}


Кроме того, при \(\beta \ll \omega_{0}\) относительное уменьшение энергии колебаний за период, согласно (6.31*), равно \(\delta E / E=2 \beta T=2 \lambda\), откуда


\begin{equation*}
\lambda=\delta E / 2 E . \tag{6.36}
\end{equation*}


\begin{enumerate}
  \setcounter{enumi}{2}
  \item Добротность осциллятора. По определению,
\end{enumerate}


\begin{equation*}
Q=\pi / \lambda=\pi N_{\mathrm{e}} . \tag{6.37}
\end{equation*}


При малом затухании ( \(\beta \ll \omega_{0}\) ), когда справедливо (6.36),


\begin{equation*}
Q \approx 2 \pi E / \delta E . \tag{6.38}
\end{equation*}


Пример. Найдем добротность осциллятора, у которого амплитуда смещения уменьшается в η раз через каждые \(N\) колебаний.\\
Поскольку \(Q=\pi / \lambda=\pi / \beta T\), задача сводится к определению \(\beta\) и \(T\). Пусть \(\Delta t\) - время, за которое амплитуда уменьшается в \(\eta\) раз, тогда \(\eta=\mathrm{e}^{\beta \Delta t}\) и \(\beta \Delta t=\ln \eta\). Кроме того, \(T=\Delta t / N\). После подстановки выражений для \(\beta\) и \(T\) в исходную формулу получим \(Q=\pi N / \ln \eta\).

В заключение отметим, что при достаточно большом затухании ( \(\beta \geqslant \omega_{0}\) ) система совершает апериодическое движение: вы-

веденная из положения равновесия, она возвращается в это положение, не совершая колебаний.

\section*{§ 6.4. Вынужденные колебания Уравнение вынужденных колебаний}
Свободные колебания реальной колебательной системы являются, как мы выяснили, затухающими. Чтобы возбудить в такой системе незатухающие колебания, необходимо компенсировать потери энергии, обусловленные силами сопротивления (трения). Это можно осуществить, воздействуя на систему переменной внешней силой \(F\), изменяющейся - в простейшем и практически наиболее важном случае - по гармоническому закону \(F_{x}=F_{m} \cos \omega t\). Возникающие при этом колебания и называют вынужденными.

Теперь на колеблющуюся частицу будут действовать одновременно три силы: квазиупругая ( \(-\varkappa x)\), сила сопротивления ( \(-r \dot{x}\) ) и внешняя, вынуждающая ( \(F_{x}\) ). Согласно основному уравнению динамики,


\begin{equation*}
m \ddot{x}=-\varkappa x-r \dot{x}+F_{m} \cos \omega t \tag{6.39}
\end{equation*}


или в более удобной форме


\begin{equation*}
\ddot{x}+2 \beta \dot{x}+\omega_{0}^{2} x=f_{m} \cos \omega t \tag{6.40}
\end{equation*}


где \(2 \beta=r / m, \omega_{0}^{2}=\varkappa / m, f_{m}=F_{m} / m\).\\
Опыт показывает, что по истечении некоторого времени (с момента начала действия вынуждающей силы) в системе устанавливаются гармонические колебания* с частотой вынуждающей силы, но отстающие по фазе от последней на \(\varphi\) :


\begin{equation*}
x=a \cos (\omega t-\varphi) . \tag{6.41}
\end{equation*}


\footnotetext{\begin{itemize}
  \item Решение уравнения (6.40), как доказывается в математике, представляет собой сумму общего решения однородного уравнения (когда правая часть равна нулю) и частного решения неоднородного:
\end{itemize}

\[
x=a_{0} \mathrm{e}^{-\beta t} \cos \left(\omega^{\prime} t+\alpha\right)+a \cos (\omega t-\varphi) .
\]

Нас будет интересовать только частное решение, соответствующее установившимся колебаниям. Общее решение однородного уравнения описывает затухающие колебания, которые по истечении некоторого времени практически исчезают.
}Наша задача - определить постоянные \(a\) и φ. Для этого продифференцируем (6.41) дважды по времени:


\begin{align*}
& \dot{x}=-a \omega \sin (\omega t-\varphi)=a \omega \cos (\omega t-\varphi+\pi / 2), \\
& \ddot{x}=-a \omega^{2} \cos (\omega t-\varphi)=a \omega^{2} \cos (\omega t-\varphi+\pi) \tag{6.42}
\end{align*}


и подставим выражения для \(x, \dot{x}\) и \(\ddot{x}\) в исходное уравнение (6.40). Сумма трех гармонических функций в левой части (6.40) должна быть равной функции \(f_{m} \cos \omega t\). Учитывая фазовые сдвиги между \(x, \dot{x}\) и \(\ddot{x}\), представим это равенство с помощью векторной диаграммь (рис. 6.16, для случая \(\omega<\omega_{0}\) ). В скобках на этой диаграмме указаны «происхождения» (или соответствие) векторов, модули которых имеют

\begin{figure}[h]
\begin{center}
  \includegraphics[alt={},max width=\textwidth]{66a35da9-3faa-4823-9ca9-40799f37b3db-216_319_495_514_665}
\captionsetup{labelformat=empty}
\caption{Рис. 6.16}
\end{center}
\end{figure}


\begin{equation*}
a=f_{m} / \sqrt{\left(\omega_{0}^{2}-\omega^{2}\right)^{2}+4 \beta^{2} \omega^{2}} . \tag{6.43}
\end{equation*}


Из этой диаграммы видно, что отставание смещения по фазе на φ от вынуждающей силы определяется как


\begin{equation*}
\operatorname{tg} \varphi=2 \beta \omega /\left(\omega_{0}^{2}-\omega^{2}\right) \tag{6.44}
\end{equation*}


Формулы (6.43) и (6.44) показывают, что амплитуда \(a\) колебаний и отставание смещения по фазе на φ от вынуждающей силы определяются свойствами самого осциллятора ( \(\omega_{0}, \beta\) ) и вынуждающей силы ( \(f_{m}, \omega\) ), но не начальными условиями.

\section*{Резонанс}
На рис. 6.17 приведены графики зависимости амплитуды вынужденных колебаний от частоты вынуждающей силы \(a(\omega)\) для трех коэффициентов затухания. Видно, что \(a(\omega)\) имеет максимум при частоте, которую легко найти из условия \(\mathrm{d} a / \mathrm{d} \omega=0\) (до-

\begin{figure}[h]
\begin{center}
  \includegraphics[alt={},max width=\textwidth]{66a35da9-3faa-4823-9ca9-40799f37b3db-217_337_407_139_67}
\captionsetup{labelformat=empty}
\caption{Рис. 6.17}
\end{center}
\end{figure}

статочно найти экстремум подкоренного выражения). Эту частоту называют резонансной:


\begin{equation*}
\omega_{\text {рез }}=\sqrt{\omega_{0}^{2}-2 \beta^{2}}, \tag{6.45}
\end{equation*}


а существование максимума амплитуды \(a\) - явлением резонанса. Соответственно приведенные на рис. 6.17 графики принято называть резонансными кривыми.

Выражение для амплитуды при резонансе получим, подставив (6.45) в (6.43):


\begin{equation*}
a_{\text {макс }}=\frac{f_{m}}{2 \beta \sqrt{\omega_{0}^{2}-\beta^{2}}} \tag{6.46}
\end{equation*}


Чем меньше затухание системы, тем более ярко выражен резонанс. Явление резонанса играет огромную роль в физике и технике. Его используют, если нужно усилить колебания, и, наоборот, всячески избегают, если резонанс может привести к нежелательным усилениям колебаний.

Зависимость фазового сдвига \(\varphi\) от частоты \(\omega\) показана на рис. 6.18 (для двух коэффициентов затухания). При слабом затухании \(\omega_{\text {рез }} \approx \omega_{0}\), и значение \(\varphi\) при резонансе практически равно \(\pi / 2\) (см. рис. 6.16).

На рис. 6.19 дан график зависимости средней (за период) мощности вынуждающей силы от ее частоты \(\langle P(\omega)\rangle\). Заметим, что \(\langle P(\omega)\rangle=\max\) при \(\omega=\omega_{0}\) независимо от коэффициента зату-

\begin{figure}[h]
\begin{center}
  \includegraphics[alt={},max width=\textwidth]{66a35da9-3faa-4823-9ca9-40799f37b3db-217_375_486_1381_119}
\captionsetup{labelformat=empty}
\caption{Рис. 6.18}
\end{center}
\end{figure}

\begin{figure}[h]
\begin{center}
  \includegraphics[alt={},max width=\textwidth]{66a35da9-3faa-4823-9ca9-40799f37b3db-217_375_519_1381_643}
\captionsetup{labelformat=empty}
\caption{Рис. 6.19}
\end{center}
\end{figure}

хания \(\beta\). Важным параметром резонансной кривой \(\langle P(\omega)\rangle\), характеризующим «остроту» резонанса, является ее ширина \(\Delta \omega\) на половине «высоты». Можно показать, что при малом затухании ( \(\beta \ll \omega_{0}\) ) «острота» резонанса, т. е. отношение \(\omega_{0} / \Delta \omega\), равно добротности осциллятора:


\begin{equation*}
\omega_{0} / \Delta \omega=\boldsymbol{Q} . \tag{6.47}
\end{equation*}


\section*{Энергия вынужденных колебаний}
Интересно проследить, как зависит энергия \(E\) осциллятора, совершающего установившиеся колебания, от времени. Так как \(E=U+K\), то


\begin{gather*}
E=\varkappa x^{2} / 2+m \dot{x}^{2} / 2= \\
=m a^{2}\left[\omega_{0}^{2} \cos ^{2}(\omega t-\varphi)+\omega^{2} \sin ^{2}(\omega t-\varphi)\right] / 2, \tag{6.48}
\end{gather*}


где учтено, что \(x=m \omega_{0}^{2}\). График зависимости \(E(t)\) для случая \(\omega>\omega_{0}\) показан на рис. 6.20. Колебания энергии \(E\) будут тем меньше, чем ближе частота \(\omega\) к \(\omega_{0}\), и при \(\omega=\omega_{0}\) энергия \(E\) не будет зависеть от времени \(t\) : \(E_{0}=m a^{2} \omega_{0}^{2} / 2=\) const.

В установившихся колебаниях при \(\omega \neq \omega_{0}\) работа вынуждающей силы за период будет компенсировать потери энергии в системе за счет работы

\begin{figure}[h]
\begin{center}
  \includegraphics[alt={},max width=\textwidth]{66a35da9-3faa-4823-9ca9-40799f37b3db-218_370_533_911_631}
\captionsetup{labelformat=empty}
\caption{Рис. 6.20}
\end{center}
\end{figure}

Пример. Найдем среднюю за период мощность \(\langle P\rangle\) вынуждающей силы, необходимую для поддержания среднего значения кинетической энергии на уровне \(\langle K\rangle\) у осциллятора с коэффициентом затухания \(\beta\).

Согласно закону сохранения энергии, \(\langle P\rangle\) должно быть равно модулю среднего значения мощности силы сопротивления:

\[
\langle P\rangle=|\langle-r \dot{x} \cdot \dot{x}\rangle|=\left\langle r \dot{x}^{2}\right\rangle .
\]

Так как \(\dot{x}^{2}=2 K / m\), то \(r \dot{x}^{2}=2(r / m) K=4 \beta K\) и \(\langle P\rangle=4 \beta\langle K\rangle\).

\section*{Задачи}
6.1. Свободные колебания без трения. Идеальная жидкость объемом \(V\) налита в U-образную трубку (рис. 6.21) с площадью поперечного сечения канала \(S\). Найти период малых колебаний\\
\includegraphics[max width=\textwidth, alt={}, center]{66a35da9-3faa-4823-9ca9-40799f37b3db-219_300_224_556_138}\\
жидкости.

Решение. Эту задачу наиболее просто решать с помощью дуговой координаты \(l\). Проецируя все силы, действующие на жидкость, на орт \(\tau\), получим, согласно основному уравнению динамики \(m \ddot{l}=F_{\tau}(2.16), m \ddot{l}=-\rho g S \cdot 2 l\), где справа записана проекция единственной некомпенсированной\\
Рис. 6.21 силы - силы тяжести, действующей справа на элемент жидкости длины \(2 l\). Отсюда, имея в виду, что \(m / \rho=V\), получаем

\[
\ddot{l}+(2 g S / V) l=0 .
\]

Значит, \(\omega_{0}^{2}=2 g S / V\) и \(T=\pi \sqrt{2 V / g S}\).\\
6.2. Крутильные колебания. Горизонтальный диск с моментом инер-

\begin{figure}[h]
\begin{center}
  \includegraphics[alt={},max width=\textwidth]{66a35da9-3faa-4823-9ca9-40799f37b3db-219_294_234_1172_143}
\captionsetup{labelformat=empty}
\caption{Рис. 6.22}
\end{center}
\end{figure}

ции \(I\) относительно его оси укреплен в центре тонкого упругого стержня (рис. 6.22). При повороте диска на него действует момент упругих сил \(M_{z}=-D \varphi\), где \(D\) - коэффициент кручения. Найти частоту \(\omega_{0}\) и амплитуду \(\varphi_{m}\) крутильных колебаний, если в начальный момент диск повернули на угол \(\varphi_{0}\) из положения равновесия и сообщили ему угловую скорость \(\dot{\varphi}_{0}\).\\
\(P\) ешен и е. Из уравнения движения \(I \ddot{\varphi}=-D \varphi\) находим \(\omega_{0}=\sqrt{D / I}\). Амплитуду колебаний проще всего найти из того, что энергия колебаний ( \(E=U+K\) ) здесь будет сохраняться. Значит, энергия в начальный момент будет равна энергии при максимальном отклонении из положения равновесия:

\[
D \varphi_{0}^{2} / 2+I \dot{\varphi}_{0}^{2} / 2=D \varphi_{m}^{2} / 2 .
\]

Отсюда \(\varphi_{m}=\sqrt{\varphi_{0}^{2}+(I / D) \dot{\varphi}_{0}^{2}}\).\\
Заметим, что выражение для потенциальной энергии ( \(U=D \varphi^{2} / 2\) ) следует из ее определения, а именно: убыль величины \(U\) равна работе упругой силы (в данном случае ее момента):

\[
U(0)-U(\varphi)=\int_{0}^{\varphi} M_{z} \mathrm{~d} \varphi .
\]

В положении равновесия ( \(\varphi=0\) ) полагаем \(U(0)=0\).\\
6.3. Физический маятник. На каком расстоянии \(x\) от центра \(C\) надо подвесить тонкий однородный стержень длины \(l\), чтобы период его малых колебаний был наименьшим?

Решение. Согласно (6.10), период колебаний физического маятника \(T=2 \pi \sqrt{I / m g x}\), где \(I\) - момент инерции стержня относительно искомой точки подвеса. По теореме Штейнера, \(I=I_{C}+m x^{2}\), где \(I_{C}\) - момент инерции относительно центра масс \(C\). Подставив это выражение в формулу для \(T\), получим

\[
T=2 \pi \sqrt{(l / 12 x+x / l) l / g}
\]

Период \(T\) будет наименьшим при условии \(\mathrm{d} T / \mathrm{d} x=0\) (или при равенстве нулю производной от подкоренного выражения):

\[
-l / 12 x^{2}+1 / l=0
\]

откуда \(x=l / \sqrt{12}\).\\
6.4. Однородный стержень положили на два быстро вращающихся блока (рис. 6.23). Известны расстояние \(l\) между осями блоков и коэффициент трения \(k\) между стержнем и блоками. Показать, что стержень будет совершать гармонические колебания. Найти их период.

\begin{figure}[h]
\begin{center}
  \includegraphics[alt={},max width=\textwidth]{66a35da9-3faa-4823-9ca9-40799f37b3db-220_224_434_1313_726}
\captionsetup{labelformat=empty}
\caption{Рис. 6.23}
\end{center}
\end{figure}

Решение. Согласно основному уравнению динамики,


\begin{equation*}
m \ddot{x}=F_{1}-F_{2}=k\left(R_{1}-R_{2}\right) . \tag{*}
\end{equation*}


Отсутствие вращения стержня означает, что алгебраическая сумма моментов всех сил, действующих на стержень, равна нулю. Относительно точки \(O\) (начала отсчета координаты \(x\) )

\[
\left(R_{1}-R_{2}\right) l / 2+m g x=0 .
\]

Найденное отсюда ( \(R_{1}-R_{2}\) ) подставим в (*) и после сокращения на \(m\) получим

\[
\ddot{x}+(2 k g / l) x=0 .
\]

Это есть уравнение гармонического осциллятора с частотой \(\omega_{0}=\sqrt{2 k g / l}\) и периодом \(T=\pi \sqrt{2 l / k g}\).

Заметим, что вращаться блоки должны достаточно быстро, чтобы при всех положениях стержня было обеспечено трение скольжения.\\
6.5. Найти период малых колебаний системы (рис. 6.24), если радиус блока \(R\), его момент инерции относительно

\begin{figure}[h]
\begin{center}
  \includegraphics[alt={},max width=\textwidth]{66a35da9-3faa-4823-9ca9-40799f37b3db-221_314_335_816_138}
\captionsetup{labelformat=empty}
\caption{Рис. 6.24}
\end{center}
\end{figure}

оси вращения \(I\), масса грузика \(m\) и жесткость пружины \(\varkappa\). Нить по блоку не скользит, трения в его оси нет.

Решение. Выбрав положительные направления координаты \(x\) и угла \(\varphi\) (для блока), запишем уравнения движения грузика и блока:

\[
m \ddot{x}=m g-F, \quad I \ddot{\varphi}=R F-R \varkappa(x+\Delta l),
\]

где \(\Delta l\) - растяжение пружины в положении равновесия. Учтем также кинематическую связь ускорений: \(\ddot{x}=R \ddot{\varphi}\).\\
Исключив \(F\) из первых двух уравнений, получим

\[
I \ddot{x} / R=R(m g-m \ddot{x})-R \varkappa(x+\Delta l) .
\]

В положении равновесия \(m g=x \Delta l\), и предыдущее уравнение примет вид \((I / R+R m) \ddot{x}=-R \varkappa x\), или

\[
\ddot{x}+\left[\varkappa R^{2} /\left(I+m R^{2}\right)\right] x=0 .
\]

Отсюда \(\omega_{0}=\sqrt{x R^{2} /\left(I+m R^{2}\right)}\) и период колебаний

\[
T=2 \pi / \omega_{0}=2 \pi \sqrt{\left(I / R^{2}+m\right) / \varkappa} .
\]

6.6. В гладком горизонтальном желобе находятся два цилиндра массами \(m_{1}\) и \(m_{2}\), соединенные пружиной жесткости \(\varkappa\) (рис. 6.25). Левому цилиндру толчком сообщили вдоль желоба начальную скорость \(v_{1}\). Найти: 1) частоту колебаний системы в процессе движения; 2) энергию и амплитуду колебаний.

\begin{figure}[h]
\begin{center}
  \includegraphics[alt={},max width=\textwidth]{66a35da9-3faa-4823-9ca9-40799f37b3db-222_113_380_218_782}
\captionsetup{labelformat=empty}
\caption{Рис. 6.25}
\end{center}
\end{figure}

Решение. 1. Пусть в некоторый момент координаты центров цилиндров \(x_{1}\) и \(x_{2}\). Тогда


\begin{equation*}
x_{2}-x_{1}=l+x, \tag{1}
\end{equation*}


где \(l\) - расстояние между центрами цилиндров при недеформированной пружине, \(x\) - ее деформация. Запишем уравнения движения обоих цилиндров в момент, когда пружина растянута ( \(x>0\) ), тогда

\[
m_{1} \ddot{x}_{1}=n x, \quad m_{2} \ddot{x}_{2}=-n x,
\]

где справа записаны проекции упругой силы со стороны пружины, действующие на каждый цилиндр. Разделим первое уравнение на \(m_{1}\), второе - на \(m_{2}\) и вычтем из второго первое:

\[
\ddot{x}_{2}-\ddot{x}_{1}=-\varkappa\left(1 / m_{1}+1 / m_{2}\right) x=-(\varkappa / \mu) x,
\]

где \(\mu\) - приведенная масса системы. Левая часть этого уравнения, согласно (1), равна \(\ddot{x}\), поэтому

\[
\ddot{x}+(x / \mu) x=0,
\]

откуда \(\omega_{0}=\sqrt{\varkappa / \mu}\).\\
2. Механическая энергия системы, согласно (4.57),

\[
E=\tilde{E}+\left(m_{1}+m_{2}\right) v_{C}^{2} / 2,
\]

где \(\widetilde{E}\) - механическая энергия в Ц-системе, она и является искомой энергией колебания системы: \(\tilde{E}=E_{\text {кол }}\). Итак,


\begin{equation*}
E_{\text {кол }}=E-\left(m_{1}+m_{2}\right) v_{C}^{2} / 2 . \tag{2}
\end{equation*}


Остается найти \(E\) и \(v_{C}\). В нашем случае (трения нет) энергия \(E\) будет сохраняться; значит, она равна кинетической энергии левого цилиндра в начальный момент: \(E=m_{1} v_{1}^{2} / 2\).

Сохраняться будет и импульс системы по той же причине, т. е.

\[
m_{1} v_{1}=\left(m_{1}+m_{2}\right) v_{C} .
\]

После подстановки выражений для \(E\) и \(v_{C}\) в (2) получим

\[
E_{\text {кол }}=\mu v_{1}^{2} / 2, \quad \mu=m_{1} m_{2} /\left(m_{1}+m_{2}\right) .
\]

Из формулы \(E_{\text {кол }}=\varkappa a^{2} / 2\) находим амплитуду колебаний:

\[
a=v_{1} \sqrt{\mu / \varkappa} .
\]

6.7. Затухающие колебания. Найти добротность математического маятника длины \(l=50 \mathrm{~cm}\), если за \(\Delta t=5,2\) мин его энергия колебаний уменьшается в \(\eta=4,0 \cdot 10^{4}\) раз.

Решение. Прежде всего выясним, можно ли в данном случае пользоваться формулой (6.31), справедливой для малого затухания \(\left(\beta \ll \omega_{0}\right)\). Если \(E \sim \mathrm{e}^{-2 \beta t}\), то из условия задачи следует, что \(\eta \approx \mathrm{e}^{2 \beta \Delta t}\) и \(\beta=\ln \eta / 2 \Delta t=10,6 / 624=0,017 \mathrm{c}^{-1}\), а \(\omega_{0}=\sqrt{g / l}=4,4 \mathrm{c}^{-1}\), т.е.

здесь действительно \(\beta \ll \omega_{0}\).\\
Добротность в данном случае

\[
Q=\pi / \beta T \approx \omega_{0} / 2 \beta=(\Delta t / \ln \eta) \sqrt{g / l}=1,3 \cdot 10^{2} .
\]

6.8. В начальный момент \(t=0\) смещение осциллятора равно \(x_{0}\), причем \(x_{0}>0\). Найти начальную скорость \(\dot{x}_{0}\), при которой данное смещение окажется равным амплитуде, если время релаксации осциллятора равно \(\tau\).\\
Р е ш е н и е. Амплитуда смещения изменяется по закону \(a=a_{0} \mathrm{e}^{-\beta t}\). Смещение в момент \(t=0\) равно амплитуде ( \(x_{0}=a\) ) лишь в том случае, если начальная скорость \(\dot{x}_{0}=\mathrm{d} a / \mathrm{d} t\), т. е. наклоны графиков \(x(t)\) и \(a(t)\) в момент \(t=0\) одинаковы. Отсюда \(\dot{x}_{0}=-\beta x_{0}=-x_{0} / \tau\).\\
6.9. Вынужденные колебания. Показать, что при малом затухании \(\beta \ll \omega_{0}\) отношение амплитуды \(a_{m}\) колебаний при резонансной частоте к амплитуде \(a_{0}\) при очень малых частотах равно добротности осциллятора.

Решения. При \(\beta \ll \omega_{0}\) амплитуда \(a_{m}\), согласно (6.46), равна \(a_{m}=f_{m} / 2 \beta \omega_{0}\), а при \(\omega \rightarrow 0\) амплитуда \(a_{0}=f_{m} / \omega_{0}^{2}\). Их отношение

\[
a_{m} / a_{0}=\omega_{0} / 2 \beta \approx \omega / 2 \beta=2 \pi / 2 \beta T=\pi / \lambda=Q .
\]

Видно, что для систем с малым затуханием это отношение (а значит, и \(Q)\) может быть очень большим.\\
6.10. Под действием вынуждающей силы \(F_{x}=F_{m} \cos \omega t\) осциллятор совершает установившиеся колебания по закону \(x=a \cos (\omega t-\varphi)\). Найти работу вынуждающей силы за период.\\
Решение. При установившихся колебаниях работа вынуждающей силы за период \(T\) равна работе сил сопротивления с обратным знаком:

\[
\begin{aligned}
& A_{F}=-A_{\text {coпp }}=-\left\langle P_{\text {coпp }}\right\rangle T=-\langle-r \dot{x} \cdot \dot{x}\rangle T= \\
& \quad=r a^{2} \omega^{2}\left\langle\sin ^{2}(\omega t-\varphi)\right\rangle T=2 \pi \beta m a^{2} \omega .
\end{aligned}
\]

Здесь \(\left\langle P_{\text {сопр }}\right\rangle\) - средняя мощность сил сопротивления, а кроме того, принято во внимание, что средний квадрат синуса за период равен \(1 / 2\). Из векторной диаграммы видно (см. рис. 6.16), что \(2 \beta a \omega=f_{m} \sin \varphi\). Поэтому выражение для \(A_{F}\) примет вид

\[
A_{F}=\pi a F_{m} \sin \varphi .
\]

\section*{Кинематика специальной теории относительности}
\section*{§ 7.1. Трудности дорелятивистской физики}
Специальная теория относительности, созданная Эйнштейном в 1905 г., означала пересмотр всех представлений классической физики и главным образом представлений о свойствах пространства и времени. Поэтому данная теория по своему основному содержанию может быть названа физическим учением о пространстве и времени. Физическим потому, что свойства пространства и времени в этой теории рассматриваются в теснейшей связи с законами совершающихся в них физических явлений. Термин «специальная» подчеркивает то обстоятельство, что эта теория рассматривает явления только в инерциальных системах отсчета.

Мы начнем этот раздел с краткого обзора дорелятивистской физики и остановимся на истоках тех трудностей, которые привели к появлению теории относительности.

\section*{Основные представления дорелятивистской физики}
Напомним сначала те представления о пространстве и времени, которые связаны с законами Ньютона, т. е. лежат в основе ньютоновской механики.

\begin{enumerate}
  \item Пространство, имеющее три измерения, подчиняется евклидовой геометрии.
  \item Наряду с трехмерным пространством существует независимое от него время (независимое в том смысле, в каком три измерения пространства не зависят друг от друга). Но вместе с тем время связано с пространством законами движения. Действительно, время измеряют часами, в принципе представляющими собой любой прибор, в котором используется тот или иной периодический процесс, дающий масштаб времени. Поэтому определить время безотносительно к како-му-либо периодическому процессу, т. е. вне связи с движением, невозможно.
  \item Размеры твердых тел (масштабы) и промежутки времени между данными событиями одинаковы в разных системах отсчета. Это соответствует ньютоновской концепции абсолютности пространства и времени, согласно которой их свойства считаются не зависящими от системы отсчета - пространство и время одинаковы для всех систем отсчета.
  \item Признается справедливость закона инерции Гали-лея-Ньютона, согласно которому тело, не подверженное действию со стороны других тел, движется прямолинейно и равномерно. Этот закон утверждает существование инерциальных систем отсчета, в которых выполняются законы Ньютона (а также принцип относительности Галилея).
  \item Из этих представлений вытекают преобразования Галилея, выражающие пространствен-но-временнэю связь любого события в разных инерциальных системах отсчета. Если \(K^{\prime}\)-система отсчета движется относительно K-системы со скоростью \(\mathbf{V}\) (рис. 7.1) и начало отсчета времени соответствует моменту, когда начала координат \(O^{\prime}\) и
\end{enumerate}

\begin{figure}[h]
\begin{center}
  \includegraphics[alt={},max width=\textwidth]{66a35da9-3faa-4823-9ca9-40799f37b3db-226_278_424_709_736}
\captionsetup{labelformat=empty}
\caption{Рис. 7.1}
\end{center}
\end{figure}


\begin{equation*}
x^{\prime}=x-V t ; \quad y^{\prime}=y ; \quad t^{\prime}=t . \tag{7.1}
\end{equation*}


Отсюда следует, что координаты любого события относительны, т. е. имеют разные значения в разных системах отсчета; момент же времени, когда событие произошло, одинаков в разных системах. Последнее означает, что время течет одинаковым образом в разных системах отсчета. Это обстоятельство казалось столь очевидным, что даже не оговаривалось как специальный постулат.

Из (7.1) непосредственно вытекает закон преобразования (сложения) скоростей:


\begin{equation*}
\mathbf{v}^{\prime}=\mathbf{v}-\mathbf{V}, \tag{7.2}
\end{equation*}


где \(\mathbf{v}^{\prime}\) и \(\mathbf{v}\) - скорости точки (частицы) в \(K^{\prime}\) - и \(K\)-системах отсчета.

\footnotetext{\begin{itemize}
  \item Здесь и в дальнейшем мы ограничимся только двумя пространственными координатами: \(x\) и \(y\). Координата \(z\) ведет себя во всех отношениях так же, как \(y\).
\end{itemize}
}
6. Выполняется принцип относительности Галилея: все инерциальные системы отсчета эквиваленты друг другу в механическом отношении, все законы механики одинаковы в этих системах отсчета, или, другими словами, инвариантны относительно преобразований Галилея.\\
7. Соблюдается принцип дальнодействия: взаимодействия тел распространяются мгновенно, т. е. с бесконечно большой скоростью.

Эти представления ньютоновской механики вполне соответствовали всей совокупности экспериментальных данных, имевшихся в то время (заметим, впрочем, что эти данные относились к изучению движения тел со скоростями, значительно меньшими скорости света). В их пользу говорило и весьма успешное развитие самой механики. Поэтому представления ньютоновской механики о свойствах пространства и времени стали считаться настолько фундаментальными, что никаких сомнений в их истинности ни у кого не возникало.

Первому испытанию подвергся принцип относительности Галилея, который, как известно, касался только механики единственного раздела физики, достигшего к тому времени достаточного развития. По мере развития других разделов физики, в частности оптики и электродинамики, возник естественный вопрос: распространяется ли принцип относительности и на другие явления? Если нет, то с помощью этих (немеханических) явлений можно в принципе различить инерциальные системы отсчета и в свою очередь поставить вопрос о существовании главной, или абсолютной, системы отсчета.

Одно из таких явлений, которое, как ожидали, по-разному протекает в разных системах отсчета, - это распространение света. Согласно господствовавшей в то время волновой теории, световые волны должны распространяться с определенной скоростью по отношению к некоторой гипотетической среде («светоносному эфиру»), о природе которой, правда, не было единого мнения. Но какова бы ни была природа этой среды, она не может покоиться во всех инерциальных системах сразу. Выделяется одна из инерциальных систем - абсолютная - та самая, которая неподвижна относительно «светоносного эфира». Полагали, что в этой - и только этой - системе отсчета свет распространяется с одинаковой скоростью \(c\) во всех направле-

ниях. Если некоторая инерциальная система отсчета движется по отношению к эфиру со скоростью \(\mathbf{V}\), то в этой системе отсчета скорость света \(\mathbf{c}^{\prime}\) должна подчиняться обычному закону сложения скоростей (7.2), т. е. \(\mathbf{c}^{\prime}=\mathbf{c}-\mathbf{V}\).

Это предположение оказалось возможным проверить на опыте, который и был осуществлен Майкельсоном (совместно с Морли).

\section*{Опыт Майкельсона}
Цель этого эксперимента заключалась в том, чтобы обнаружить «истинное» движение Земли относительно эфира. Было использовано движение Земли по ее орбите со скоростью 30 км/с. Идея эксперимента состояла в следующем.

Свет от источника \(S\) (рис. 7.2) посылался в двух взаимно перпендикулярных направлениях, отражался от зеркал \(A\) и \(B\), находящихся на одинаковом расстоянии \(l\) от источника \(S\), и возвращался в точку \(S\). В этом опыте сравнивалось время прохождения светом путей \(S A S\) и \(S B S\).

Предположим, что установка вместе с Землей движется так, что ее скорость V относительно эфира направлена вдоль \(S A\)

\begin{figure}[h]
\begin{center}
  \includegraphics[alt={},max width=\textwidth]{66a35da9-3faa-4823-9ca9-40799f37b3db-228_323_334_738_822}
\captionsetup{labelformat=empty}
\caption{Рис. 7.2}
\end{center}
\end{figure}

(в момент проведения опыта). Если скорость света подчиняется обычному закону сложения скоростей (7.2), то на пути \(S A\) скорость света относительно установки (Земли) равна \(c-V\), а на обратном пути \(c+V\). Тогда время прохождения пути \(S A S\)

\[
t_{\|}=\frac{l}{c-V}+\frac{l}{c+V}=\frac{2 l}{c} \frac{1}{1-(V / c)^{2}}
\]

На пути же \(S B S\) скорость света относительно установки равна \(c^{\prime}=\sqrt{c^{2}-V^{2}}\) (рис. 7.2), и время прохождения этого пути

\[
t_{\perp}=\frac{2 l}{\sqrt{c^{2}-V^{2}}}=\frac{2 l}{c} \frac{1}{\sqrt{1-(V / c)^{2}}}
\]

Из сравнения выражений для \(t_{| |}\)и \(t_{\perp}\) видно, что свет должен проходить оба пути за разное время. Измерив разность времен\\
\(t_{\|}-t_{\perp}\), можно определить скорость установки (Земли) относительно эфира.

Несмотря на то, что ожидаемая разность времен была чрезвычайно мала, установка была достаточно чувствительной, чтобы эту разность надежно обнаружить (это достигалось с помощью очень чувствительного интерференционного метода).

Тем не менее результат опыта оказался отрицательным: разность времен не была обнаружена. Конечно, случайно могло оказаться, что в момент проведения опыта Земля покоилась относительно эфира. Но тогда через полгода, например, скорость Земли относительно эфира достигла бы \(60 к м / с\). Однако повторение опыта через полгода по-прежнему не дало ожидаемого результата.

Более точные опыты того же рода, поставленные позднее, также подтвердили первоначальный результат.

Отрицательный результат опыта Майкельсона противоречил тому, что ожидалось на основании преобразований Галилея (преобразования скоростей). Он показал также, что нельзя обнаружить движение относительно эфира, что скорость света не зависит от движения источника света (ведь источник движется по-разному относительно эфира в разные времена года).

В пользу того, что скорость света не зависит от скорости источника, говорят и некоторые астрономические наблюдения (например, над двойными звездами), а также другие опыты, поставленные позднее специально с целью проверки этого факта.

К началу XX в. в теоретической и экспериментальной физике сложилась своеобразная ситуация. С одной стороны, теоретически были предсказаны различные эффекты, выделяющие из множества инерциальных систем главную (абсолютную). С другой стороны, настойчивые попытки обнаружить эти эффекты на опыте неизменно оканчивались неудачей. Опыт неуклонно подтверждал справедливость принципа относительности для всех явлений, включая и те, к которым теория считала его заведомо неприемлемым.

Был сделан целый ряд попыток объяснения отрицательного результата опыта Майкельсона и аналогичных ему в рамках ньютоновской механики. Однако все они оказались в конечном счете неудовлетворительными. Кардинальное решение этой проблемы было дано лишь в теории относительности Эйнштейна.

\section*{§ 7.2. Постулаты Эйнштейна}
Глубокий анализ всего экспериментального и теоретического материала, имеющегося к началу XX в., привел Эйнштейна к пересмотру исходных полжений классической физики, прежде всего представлений о свойствах пространства и времени. В результате им была создана специальная теория относительности, явившаяся логическим завершением всей классической физики.

Эта теория принимает без изменения такие положения ньютоновской механики, как евклидовость пространства и закон инерции Галилея - Ньютона. Что касается утверждения о неизменности размеров твердых тел и промежутков времени в разных системах отсчета, то Эйнштейн обратил внимание на то, что эти представления возникли в результате изучения движений тел с малыми скоростями, поэтому их экстраполяция в область больших скоростей ничем не оправдана, а следовательно, незаконна. Только опыт может дать ответ на вопрос, каковы их истинные свойства. Это же относится к преобразованиям Галилея и к принципу дальнодействия.

В качестве исходных позиций специальной теории относительности Эйнштейн принял два постулата, или принципа, в пользу которых говорит весь экспериментальный материал (и в первую очередь опыт Майкельсона):

\begin{enumerate}
  \item приниип относительности,
  \item независимость скорости света от скорости источника.
\end{enumerate}

Первый постулат представляет собой обобщение принципа относительности Галилея на любые физические процессы: все физические явления протекают одинаковым образом во всех инерциальных системах отсчета; все законы природы и уравнения, их описывающие, инвариантны, т. е. не меняются, при переходе от одной инерциальной системь отсчета к другой.

Другими словами, все инерциальные системы отсчета эквивалентны (неразличимы) по своим физическим свойствам; никаким опытом нельзя в принципе выделить ни одну из них как предпочтительную.

Второй постулат утверждает, что скорость света в вакууме не зависит от движения источника света и одинакова во всех направлениях.

Это значит, что скорость света в вакууме одинакова во всех инерциальных системах отсчета. Таким образом, скорость света занимает особое положение в природе. В отличие от всех других скоростей, меняющихся при переходе от одной системы отсчета к другой, скорость света в пустоте является инвариантной величиной. Как мы увидим, наличие такой скорости существенно изменяет представления о пространстве и времени.

Из постулатов Эйнштейна следует также, что скорость света в вакууме является предельной: никакой сигнал, никакое воздействие одного тела на другое не могут распространяться со скоростью, превышающей скорость света в вакууме. Именно предельный характер этой скорости и объясняет одинаковость скорости света во всех системах отсчета. В самом деле, согласно принципу относительности, законы природы должны быть одинаковы во всех инерциальных системах отсчета. Тот факт, что скорость любого сигнала не может превышать предельное значение, есть также закон природы. Следовательно, значение предельной скорости - скорости света в вакууме - должно быть одинаково во всех инерциальных системах отсчета: в противном случае эти системы можно было бы отличить друг от друга.

В частности, наличие предельной скорости автоматически предполагает ограничение скорости движения частиц величиной \(c\). Иначе эти частицы могли бы осуществлять передачу сигналов (или взаимодействий между телами) со скоростью, превышающей предельную. Таким образом, согласно постулатам Эйнштейна, значение всех возможных в природе скоростей движения тел и распространения взаимодействий ограничено величиной \(c\). Этим отвергается принцип дальнодействия ньютоновской механики.

Все содержание специальной теории относительности вытекает из этих двух ее постулатов. В настоящее время оба постулата Эйнштейна, как и все следствия из них, убедительно подтверждаются всей совокупностью накопленного экспериментального материала.

\section*{Синхронизация часов}
Прежде чем делать какие-либо выводы из этих постулатов, Эйнштейн тщательно проанализировал представления о способах измерения пространства и времени. В первую очередь он

обратил внимание на то, что физической реальностью обладает не точка пространства и не момент времени, когда что-либо произошло, а только само событие. Для описания события в данной системе отсчета нужно указать место, в котором оно происходит, и момент времени, когда оно происходит.

Положение точки, в которой происходит событие, может быть определено с помощью жестких масштабов методами евклидовой геометрии и выражено в декартовых координатах. Ньютоновская механика в этом отношении пользовалась вполне реальными приемами сравнения измеряемых величин с образцовыми эталонами.

Соответствующий момент времени можно определить с помощью часов, помещенных в ту точку системы отсчета, где происходит данное событие. Однако такое определение уже не является удовлетворительным, когда нам надо сопоставить друг с другом события, происходящие в различных местах, или, что то же самое, сравнить время для событий, происходящих в местах, удаленных от часов.

Действительно, чтобы сравнить время (показания часов) в различных точках системы отсчета, прежде всего необходимо установить способ, как определить общее для всех точек системы отсчета время. Другими словами, надо обеспечить синхронный ход всех часов данной системы отсчета.

Синхронизировать часы, помещенные в различные точки системы отсчета, можно только с помощью каких-нибудь сигналов. Наиболее быстрые сигналы, пригодные для этой цели, это световые или радиосигналы, распространяющиеся с известной скоростью \(c\). Выбор именно этих сигналов обусловлен еще и тем, что их скорость не зависит от направления в пространстве, а также одинакова во всех инерциальных системах отсчета.

Далее можно поступить следующим образом. Наблюдатель, находящийся, например, в начале координат \(O\) данной системы отсчета, сообщает по радио: «Передаем сигнал точного времени. Сейчас по моим часам время \(t_{0}\) ». В момент, когда этот сигнал достигнет часов, находящихся на известном расстоянии \(r\) от точки \(O\), их устанавливают так, чтобы они показывали время \(t=t_{0}+r / c\), т. е. с учетом времени запаздывания сигнала. Повторение сигнала через определенные промежутки времени даст

возможность каждому наблюдателю установить синхронный ход его часов с часами в точке \(O\). В результате такой операции можно утверждать, что все часы данной системы отсчета показывают в каждый момент одно и то же общее время.

Существенно отметить, что определенное таким образом время относится лишь к той системе отсчета, относительно которой синхронизированные часы покоятся.

\section*{Соотношения между событиями}
Обратимся к вопросу о пространственных и временны́х соотношениях между данными событиями в разных инерциальных системах отсчета.

Уже в ньютоновской механике пространственные соотношения между различными событиями зависят от того, к какой системе отсчета они относятся. Например, две последовательные вспышки лампочки в движущемся поезде происходят в одной и той же точке системы отсчета, связанной с поездом, но в разных точках системы отсчета, связанной с полотном дороги. Утверждение, что два разновременных события происходят в одном и том же месте или на таком-то расстоянии друг от друга, приобретает смысл только тогда, когда указано, к какой системе отсчета это утверждение относится.

В противоположность этому временны́е соотношения между событиями в ньютоновской механике считаются не зависящими от системы отсчета. Это значит, что если какие-нибудь два события происходят одновременно в одной системе отсчета, то они являются одновременными и во всех других системах отсчета. Вообще промежуток времени между двумя данными событиями считается одинаковым во всех системах отсчета.

Легко, однако, убедиться, что в действительности это не так - одновременность (а следовательно, и течение времени) является понятием относительным, приобретающим смысл только тогда, когда указано, к какой системе отсчета это понятие относится. Покажем с помощью простого рассуждения, что два события, одновременные в одной системе отсчета, в другой системе отсчета оказываются неодновременными.

Представим себе стержень \(A B\), движущийся с постоянной скоростью \(V\) относительно \(K\)-системы отсчета. В середине стер-

жня находится лампочка \(O\), по концам - в точках \(A\) и \(B\) - фотоэлементы (рис. 7.3). Пусть в некоторый момент лампочка \(O\) дала кратковременную

\begin{figure}[h]
\begin{center}
  \includegraphics[alt={},max width=\textwidth]{66a35da9-3faa-4823-9ca9-40799f37b3db-234_102_366_153_792}
\captionsetup{labelformat=empty}
\caption{Рис. 7.3}
\end{center}
\end{figure}

вспышку света. Так как скорость распространения света в системе отсчета, связанной со стержнем (как и во всякой инерциальной системе отсчета), равна \(c\) в обоих направлениях, то световые импульсы достигнут равноудаленных от \(O\) фотоэлементов \(A\) и \(B\) в один и тот же момент времени (в системе отсчета «стержень») и оба фотоэлемента сработают одновременно.

Иначе обстоит дело в \(K\)-системе. В этой системе отсчета скорость световых импульсов в обоих направлениях равна также \(c\), однако проходимые ими пути различны. Действительно, пока световые импульсы идут к точкам \(A\) и \(B\), последние переместятся вправо (рис. 7.3) и, следовательно, фотоэлемент \(A\) сработает раньше, чем фотоэлемент \(B\).

Таким образом, события, одновременные в одной системе отсчета, не являются одновременными в другой системе отсчета, т. е. одновременность в отличие от представлений ньютоновской механики является понятием относительным. А это в свою очередь означает, что время в разных системах отсчета течет неодинаково.

Если бы в нашем распоряжении имелись мгновенно распространяющиеся сигналы, то события, одновременные в одной системе отсчета, были бы одновременными и в любой другой системе. Это непосредственно следует из только что рассмотренного примера. В этом случае течение времени не зависело бы от системы отсчета и можно было бы говорить об абсолютном времени, которое фигурирует в преобразованиях Галилея. Таким образом, преобразования Галилея, по существу, исходят из предположения, что синхронизация часов осуществляется с помощью мгновенно распространяющихся сигналов. Однако таких сигналов в действительности нет.

\section*{§ 7.3. Замедление времени и сокращение длины}
В этом параграфе мы рассмотрим три важнейших следствия, которые вытекают из постулатов Эйнштейна, - это равенство

поперечных размеров движущихся тел в разных системах отсчета, замедление хода движущихся часов и сокращение продольных размеров движущихся тел, а затем (в § 7.4) обобщим полученные результаты в виде соответствующих формул преобразования координат и времени.

Приступая к решению этих вопросов, напомним прежде всего, что под системой отсчета подразумевается тело отсчета, с которым связаны координатная сетка и ряд неподвижных одинаковых часов, синхронизированных между собой. Предполагается, что во всех инерциальных системах отсчета координатные сетки и часы проградуированы одинаковым образом. Это можно осуществить только с помощью эталонов длины и времени, реализованных также одинаковым образом во всех системах отсчета.

Для этого достаточно использовать какой-либо природный периодический процесс, дающий естественный масштаб как длины, так и времени, например одну из монохроматических волн, испускаемых определенными атомами, неподвижными в данной системе отсчета. Тогда в этой системе отсчета эталоном длины можно взять длину волны, а эталоном времени - соответствующий период колебания. С помощью этих эталонов можно построить эталон один метр как определенное число данных длин волн и эталон одна секунда как тоже определенное число периодов данных колебаний (заметим, что в настоящее время так и сделано).

Аналогичную операцию можно проделать в каждой инерциальной системе отсчета, используя одну и ту же монохроматическую волну одних и тех же атомов, неподвижных в каждой из этих систем отсчета. Основанием для этого служит то, что, по принципу относительности, физические свойства покоящихся атомов не зависят от того, в какой инерциальной системе отсчета они покоятся.

Реализовав в каждой системе отсчета эталоны длины и времени, можно перейти к решению такого фундаментального вопроса, как сравнение этих эталонов в разных системах отсчета, или, другими словами, к сравнению размеров тел и течения времени в этих системах.

\section*{Равенство поперечных размеров тел}
Начнем с вопроса о сравнении поперечных размеров тел в разных инерциальных системах отсчета. Представим себе две инерциальные системы отсчета \(K\) и \(K^{\prime}\), оси \(Y\) и \(Y^{\prime}\) которых параллельны друг другу и перпендикулярны направлению движения одной системы относительно другой (рис. 7.4), причем начало отсчета \(O^{\prime} K^{\prime}\)-системы движется по прямой, проходящей через начало отсчета \(O K\)-системы. Установим вдоль осей \(Y\) и \(Y^{\prime}\)

\begin{figure}[h]
\begin{center}
  \includegraphics[alt={},max width=\textwidth]{66a35da9-3faa-4823-9ca9-40799f37b3db-236_317_313_274_839}
\captionsetup{labelformat=empty}
\caption{Рис. 7.4}
\end{center}
\end{figure}

Принцип относительности позволяет сразу ответить на этот вопрос: да, совпадет. Если бы это было не так, то с точки зрения обеих систем отсчета один из стержней оказался бы, например, короче другого и, следовательно, имелась бы возможность экспериментально отличить одну из инерциальных систем отсчета от другой по более коротким поперечным размерам. Однако это противоречит принципу относительности.

Отсюда следует, что поперечные размеры тел одинаковы во всех инерциальных системах отсчета. Это означает также, что при указанном выборе начал отсчета \(K^{\prime}\) - и \(K\)-систем координаты \(y^{\prime}\) и \(y\) любой точки или события совпадают, т. е.


\begin{equation*}
y^{\prime}=y . \tag{7.3}
\end{equation*}


Это соотношение представляет собой одно из искомых преобразований координат.

\section*{Замедление времени}
Наша следующая задача - сравнить течение времени в разных инерциальных системах отсчета. Как уже говорилось, время измеряется часами, причем под часами имеется в виду любой прибор, в котором используется тот или иной периодический процесс. Поэтому в теории относительности принято

обычно говорить о сравнении хода идентичных часов в разных инерциальных системах отсчета.

Наиболее просто этот вопрос можно решить с помощью следующего мысленного (т. е. в принципе возможного) эксперимента. Возьмем световые чась - стержень с зеркалами на обоих концах, между которыми «бегает» короткий световой импульс. Период таких часов равен интервалу времени между двумя последовательными моментами, когда световой импульс достигает какого-то определенного конца стержня.

\begin{figure}[h]
\begin{center}
  \includegraphics[alt={},max width=\textwidth]{66a35da9-3faa-4823-9ca9-40799f37b3db-237_336_334_544_68}
\captionsetup{labelformat=empty}
\caption{Рис. 7.5}
\end{center}
\end{figure}

Далее, представим себе две инерциальные системы отсчета \(K^{\prime}\) и \(K\), движущиеся относительно друг друга со скоростью \(V\). Пусть световые часы \(A B\) неподвижны в \(K^{\prime}\)-системе и ориентированы перпендикулярно направлению ее движения относительно \(K\)-системы (рис. 7.5). Проследим за «ходом» этих часов в системах отсчета \(K^{\prime}\) и \(K\).

В \(K^{\prime}\)-системе часы неподвижны и их период

\[
\Delta t_{0}=2 l / c
\]

где \(l\) - расстояние между зеркалами, \(c\) - скорость света.\\
В \(K\)-системе, относительно которой часы движутся, расстояние между зеркалами также \(l\), так как поперечные размеры тел одинаковы в разных инерциальных системах отсчета. Однако путь светового импульса в этой системе отсчета будет уже иным - зигзагообразным (рис. 7.5): пока световой импульс распространяется от нижнего зеркала к верхнему, последнее переместится на некоторое расстояние вправо и т. д. Поэтому световой импульс, чтобы вернуться к нижнему зеркалу, проходит в \(K\)-системе больший путь, причем с той же скоростью \(c\). Значит, свету понадобится на это больше времени - больше, чем когда часы неподвижны. Другими словами, период движущихся часов удлинится - с точки зрения \(K\)-системы отсчета они будут идти медленее.

Обозначим период движущихся часов через \(\Delta t\) в \(K\)-системе. Из прямоугольного треугольника \(A B^{\prime} A^{\prime}\) (рис. 7.5) следует, что \(l^{2}+(V \Delta t / 2)^{2}=(c \Delta t / 2)^{2}\), откуда

\[
\Delta t=(2 l / c) / \sqrt{1-(V / c)^{2}} .
\]

А так как \(2 l / c=\Delta t_{0}\), то


\begin{equation*}
\Delta t=\frac{\Delta t_{0}}{\sqrt{1-\beta^{2}}} \tag{7.4}
\end{equation*}


где \(\beta=V / c, V\) - скорость часов в \(K\)-системе.\\
Отсюда видно, что \(\Delta t>\Delta t_{0}\), т. е. одни и теже часы в разных инерциальных системах отсчета идут по-разному: в той системе отсчета, относительно которой часы движутся, они идут медленнее, чем в системе отсчета, где они покоятся. Другими словами, движущиеся часы идут медленнее, чем покоящиеся. Это явление называют замедлением времени.

Время, отсчитываемое по часам, движущимся вместе с телом, в котором происходит какой-либо процесс, называют собственным временем этого тела. Его обозначают \(\Delta t_{0}\). Как следует из (7.4), собственное время самое короткое. Время \(\Delta t\) того же процесса в другой системе отсчета зависит от скорости \(V\) этой системы относительно тела, в котором происходит процесс. Такая зависимость особенно сильно проявляется для значений скорости \(V\), сравнимых со скоростью света (рис. 7.6).

\begin{figure}[h]
\begin{center}
  \includegraphics[alt={},max width=\textwidth]{66a35da9-3faa-4823-9ca9-40799f37b3db-238_501_485_738_665}
\captionsetup{labelformat=empty}
\caption{Рис. 7.6}
\end{center}
\end{figure}

Пример. Часы движутся в \(K\)-системе отсчета прямолинейно и равномерно со скоростью \(v\). В начальный момент \(t=0\) их показания совпадали с часами \(K\)-системы. На сколько секунд отстанут движущиеся часы за время \(t=60\) мин (это время по часам \(K\)-системы), если: 1) \(v=1800\) км/ч (реактивный самолет); 2) \(v=\frac{4}{5} c\), где \(c\) - скорость света в вакууме?

Пусть в момент \(t\) по часам \(K\)-системы движущиеся часы показывали \(t_{0}\), причем, согласно (7.4), \(t_{0}=t \sqrt{1-(v / c)^{2}}\), тогда искомое время

\[
t-t_{0}=t\left(1-\sqrt{1-(v / c)^{2}}\right)
\]

\begin{enumerate}
  \item При \(v \ll c\), согласно формуле бинома Ньютона \(\sqrt{1-(v / c)^{2}} \approx 1-\frac{1}{2}(v / c)^{2}\) и \(t-t_{0}=\frac{1}{2}(v / c)^{2} t=5 \cdot 10^{-9}\) с.
  \item \(t-t_{0}=\frac{2}{5} t=24\) мин.
\end{enumerate}

Таким образом, в отличие от ньютоновской механики течение времени в действительности зависит от состояния движения. Не существует единого мирового времени, и понятие «промежуток времени между двумя данными событиями» оказывается относительным. Утверждение, что между двумя данными событиями прошло столько-то секунд, приобретает смысл только тогда, когда указано, к какой системе отсчета это утверждение относится.

Абсолютное время ньютоновской механики является в теории относительности приближенным понятием, справедливым только при малых (по сравнению со скоростью света) относительных скоростях систем отсчета. Это сразу следует из (7.4) и видно из рис. 7.6 : при \(V \ll c \Delta t \approx \Delta t_{0}\).

Итак, мы пришли к фундаментальному выводу: время в системе отсчета, движущейся с часами, течет медленнее (для наблюдателя, относительно которого данные часы движутся). Это относится и ко всем процессам, протекающим в движущихся относительно наблюдателя системах отсчета.

Естественно, возникает вопрос: заметит ли наблюдатель в \(K^{\prime}\)-системе, движущейся относительно \(K\)-системы, что его часы идут медленнее, чем часы \(K\)-системы? Нет, не заметит. Это сразу же следует из принципа относительности. Если бы \(K^{\prime}\)-наблюдатель тоже обнаружил замедление времени в своей системе отсчета, то это означало бы, что для обоих наблюдателей \(K^{\prime}\) и \(K\) - время течет медленнее в одной из инерциальных систем отсчета. Из этого они заключили бы, что одна из инерциальных систем отсчета отличается от другой - в противоречии с принципом относительности.

Отсюда следует, что эффект замедления времени является взаимным, симметричным относительно обеих инерциальных систем отсчета \(K\) и \(K^{\prime}\). Иначе говоря, если с точки зрения \(K\)-системы медленнее идут часы \(K^{\prime}\)-системы, то с точки зрения \(K^{\prime}\) - системы, наоборот, медленнее идут часы \(K\)-системы (причем в том же отношении). Это обстоятельство указывает на то, что явление замедления времени является чисто кинематическим. Оно

представляет собой обязательное следствие инвариантности скорости света и никак не может быть приписано какому-либо изменению в свойствах часов, обусловленному их движением.

Формула (7.4) сразу же нашла экспериментальное подтверждение, объяснив «загадочное» на первый взгляд поведение мюонов при прохождении земной атмосферы. Мюоны - это нестабильные частицы, которые самопроизвольно распадаются в среднем через \(2 \cdot 10^{-6}\) с (это время измерено в условиях, когда они неподвижны или движутся с малыми скоростями). Мюоны образуются в верхних слоях атмосферы на высоте \(20-30 к м\). Если бы время жизни мюонов не зависело от их скорости, то, двигаясь даже со скоростью света, они не смогли бы проходить путь больше чем

\[
c \Delta t=3 \cdot 10^{8} \mathrm{~m} / \mathrm{c} \cdot 2 \cdot 10^{-6} \mathrm{c}=600 \text { м. }
\]

Однако наблюдения показывают, что значительное число мюонов все-таки достигает земной поверхности.

Это объясняется тем, что время \(2 \cdot 10^{-6}\) с - это собственное время ( \(\Delta t_{0}\) ) жизни мюонов, т. е. время по часам, движущимся вместе с мюонами. Время же по земным часам должно быть, согласно (7.4), гораздо больше (скорость этих частиц близка к скорости света) и оказывается достаточным, чтобы мюоны могли достигнуть поверхности Земли.

В заключение несколько слов о «парадоксе часов», или «парадоксе близнецов». Пусть имеются двое одинаковых часов \(A\) и \(B\), из которых часы \(A\) неподвижны в некоторой инерциальной системе отсчета, а часы \(B\) сначала удаляются от часов \(A\) и затем возвращаются к ним. Предполагается, что в начальный момент, когда часы находились вместе, они показывали одно и то же время.

С «точки зрения» часов \(A\) движущимися являются часы \(B\), поэтому они идут медленнее и по возвращении отстанут от часов \(A\). С «точки же зрения» часов \(B\), наоборот, движутся часы \(A\), поэтому по возвращении отстанут именно они. Явное противоречие - в этом суть «парадокса».

В действительности в этих рассуждениях допущена принципиальная ошибка. Эта ошибка касается рассуждения с «точки зрения» часов \(B\), так как система отсчета, связанная с этими часами, является неинерциальной (она сначала удаляется с ускорением, а затем приближается), и мы не имеем права в данном случае использовать результаты, относящиеся только к инерциальным системам отсчета. Детальный расчет, выходящий за

рамки специальной теории относительности, показывает, что часы, движущиеся с ускорением (в нашем случае часы \(B\) ), идут медленнее, поэтому при возвращении отстанут именно они.

\section*{Лоренцево сокращение}
Пусть стержень \(A B\) движется относительно \(K\)-системы отсчета с постоянной скоростью \(V\)

\begin{figure}[h]
\begin{center}
  \includegraphics[alt={},max width=\textwidth]{66a35da9-3faa-4823-9ca9-40799f37b3db-241_183_396_435_68}
\captionsetup{labelformat=empty}
\caption{Рис. 7.7}
\end{center}
\end{figure}

(рис. 7.7) и длина стержня равна \(l_{0}\) в системе отсчета \(K^{\prime}\), связанной со стержнем. Наша задача - определить длину \(l\) данного стержня в \(K\)-системе.

Проделаем для этого следующий мысленный эксперимент. Сделаем на оси \(X K\)-системы метку \(M\) и установим около нее часы. Зафиксируем по этим часам время пролета \(\Delta t_{0}\) стержня мимо метки \(M\). Тогда можно утверждать, что искомая длина стержня в \(K\)-системе

\[
l=V \Delta t_{0} .
\]

Для наблюдателя, связанного со стержнем, время пролета будет иным. Действительно, для него часы, показавшие пролетное время \(\Delta t_{0}\), движутся со скоростью \(V\), а значит, показывают «чужое» время. «Свое» время пролета \(\Delta t\) для этого наблюдателя будет, согласно (7.4), больше. Это время он может найти из соотношения

\[
l_{0}=V \Delta t .
\]

Из этих двух уравнений, с учетом (7.4), получим

\[
l / l_{0}=\Delta t_{0} / \Delta t=\sqrt{1-\beta^{2}},
\]

или


\begin{equation*}
l=l_{0} \sqrt{1-\beta^{2}}, \tag{7.5}
\end{equation*}


где \(\beta=V / c\). Длину \(l_{0}\), измеренную в системе отсчета, где стержень неподвижен, называют собственной длиной.

Таким образом, продольный размер движущегося стержня оказывается меньше его собственной длины, т. е. \(l<l_{0}\). Это явление называют лоренцевым сокращением. Заметим, что данное сокращение относится только к продольным размерам тел (размерам в направлении движения), поперечные же размеры, как было установлено, не меняются. Сравнительно с формой тела в системе отсчета, где оно покоится, его форма в движущейся системе отсчета может характеризоваться как сплющенная в направлении движения.

Из формулы (7.5) следует, что степень сокращения зависит от

\begin{figure}[h]
\begin{center}
  \includegraphics[alt={},max width=\textwidth]{66a35da9-3faa-4823-9ca9-40799f37b3db-242_487_455_286_697}
\captionsetup{labelformat=empty}
\caption{Рис. 7.8}
\end{center}
\end{figure}

Пример 1. Стержень, собственная длина которого \(l_{0}=5,0 м\), движется в продольном направлении со скоростью \(V\) относительно \(K\)-системы отсчета. При каком значении \(V\) длина стержня в \(K\)-системе будет

\begin{figure}[h]
\begin{center}
  \includegraphics[alt={},max width=\textwidth]{66a35da9-3faa-4823-9ca9-40799f37b3db-242_176_376_1008_784}
\captionsetup{labelformat=empty}
\caption{Рис. 7.9}
\end{center}
\end{figure}

Чтобы наблюдать такое сокращение длины, скорость стержня, согласно (7.5), должна быть \(V=c \sqrt{1-\left(l / l_{0}\right)^{2}}=\frac{4}{5} c\).

Пример 2. Стержень \(A\) движется мимо неподвижного в \(K\)-системе отсчета стержня \(B\) со скоростью \(v\), как показано на рис. 7.10. Оба стержня имеют одинаковую собственную длину \(l_{0}\). Найдем в \(K\)-системе отсчета промежуток времени \(\Delta t\) между моментами

\begin{figure}[h]
\begin{center}
  \includegraphics[alt={},max width=\textwidth]{66a35da9-3faa-4823-9ca9-40799f37b3db-242_126_380_1508_784}
\captionsetup{labelformat=empty}
\caption{Рис. 7.10}
\end{center}
\end{figure}

совпадения левых и правых концов стержней.

Длина движущегося в \(K\)-системе стержня \(A\) равна \(l=l_{0} \sqrt{1-(v / c)^{2}}\), и с помощью рис. 7.10 нетрудно сообразить, что искомый промежуток времени

\[
\Delta t=\left(l_{0}-l\right) / v=\left(1-\sqrt{1-(v / c)^{2}}\right) l_{0} / v .
\]

Пример 3. Две частицы, двигавшиеся в \(K\)-системе отсчета по одной прямой с одинаковой скоростью \(v=\frac{4}{5} c\), попали в неподвижную мишень с промежутком времени \(\Delta t=5 \cdot 10^{-9}\) с (в данной системе отсчета). Каким было собственное расстояние между частицами до попадания в мишень?\\
Расстояние между частицами в \(K\)-системе отсчета \(l=v \Delta t\). Поэтому искомое расстояние, согласно формуле (7.5),

\[
l_{0}=v \Delta t / \sqrt{1-(v / c)^{2}}=2 \text { м. }
\]

Итак, в разных инерциальных системах отсчета длина одного и того же стержня оказывается различной. Иными словами, длина - понятие относительное, имеющее смысл только по отношению той или иной системы отсчета. Утверждение, что длина тела столько-то метров, не имеет смысла, пока не указано, к какой именно системе отсчета отнесена эта величина.

При малых же скоростях ( \(V \ll c\) ), как следует из (7.5) и видно из рис. \(7.8, l \approx l_{0}\) и длина тела приобретает практически абсолютный смысл.

Необходимо отметить, что лоренцево сокращение, как и замедление времени, должно быть взаимным. Это значит, что если мы будем сравнивать два движущихся относительно друг друга стержня, собственная длина которых одинакова, то с «точки зрения» каждого из этих стержней длина другого стержня будет короче, причем в одинаковом отношении. Если бы это было не так, то имелась бы возможность экспериментально отличить инерциальные системы отсчета, связанные с этими стержнями, что, однако, противоречит принципу относительности.

Это говорит о том, что лорениево сокращение является также чисто кинематическим эффектом - в теле не возникает каких-либо напряжений, вызывающих деформацию.

Подчеркнем, что лоренцево сокращение тел в направлении их движения, равно как и замедление времени, представляет собой реальный и объективный факт, отнюдь не связанный с какими-либо иллюзиями наблюдателя. Все значения размеров данного тела или промежутков времени, полученные в разных системах отсчета, являются равноправными (все они «правильные»). Трудность понимания этих утверждений связана исключительно с нашей привычкой, основанной на повседневном опыте, считать понятия длины и промежутков времени абсолютными понятиями, когда в действительности это не так. Понятия длины и промежутка времени столь же относительны, как понятия движения и покоя.

\section*{§ 7.4. Преобразования Лоренца}
Теперь нам предстоит решить фундаментальный вопрос о формулах преобразования координат и времени (имеются в виду формулы, связывающие координаты и моменты времени одного и того же события в разных инерциальных системах отсчета).

Преобразования Галилея? Напомним, что эти преобразования основаны на предположениях, что длина тел не зависит от движения и время течет одинаково в различных инерциальных системах отсчета. Однако в предыдущем параграфе было показано, что в действительности это не так: течение времени и длина тел зависят от системы отсчета - выводы, являющиеся неизбежным следствием постулатов Эйнштейна. Поэтому мы вынуждены отказаться от преобразований Галилея, или, говоря точнее, признать, что они - лишь частный случай каких-то более общих преобразований.

Возникает задача отыскания таких формул преобразования, которые, во-первых, учитывали бы замедление времени и лоренцево сокращение (т. е. были бы в конечном счете следствиями постулатов Эйнштейна), и, во-вторых, переходили бы в предельном случае малых скоростей в преобразования Галилея. Перейдем к решению этой задачи.

Рассмотрим две инерциальные системы отсчета \(K\) и \(K^{\prime}\). Пусть \(K^{\prime}\)-система движется относительно \(K\)-системы со скоростью V. Направим координатные оси обеих систем отсчета так,

\begin{figure}[h]
\begin{center}
  \includegraphics[alt={},max width=\textwidth]{66a35da9-3faa-4823-9ca9-40799f37b3db-245_271_406_139_70}
\captionsetup{labelformat=empty}
\caption{Рис. 7.11}
\end{center}
\end{figure}

как показано на рис. 7.11 : оси \(X\) и \(X^{\prime}\) совпадают и направлены параллельно вектору \(\mathbf{V}\), а оси \(Y\) и \(Y^{\prime}\) параллельны друг другу. Установим в разных точках обеих систем отсчета одинаковые часы и синхронизируем их - отдельно часы \(K\)-системы и отдельно часы \(K^{\prime}\)-системы. И наконец, возьмем за начало отсчета времени в обеих системах момент, когда начала координат \(O\) и \(O^{\prime}\) совпадают ( \(t=t^{\prime}=0\) ).

Предположим теперь, что в момент времени \(t\) (в \(K\)-системе) в точке с координатами \(x, y\) произошло некоторое событие \(A\), например вспыхнула лампочка. Наша задача - найти координаты \(x^{\prime}, y^{\prime}\) и момент времени \(t^{\prime}\) этого события в \(K^{\prime}\)-системе.

Вопрос относительно координаты \(y^{\prime}\) был уже решен в начале предыдущего параграфа, где было показано [см. формулу (7.3)], что \(y^{\prime}=y\). Поэтому сразу перейдем к нахождению координаты \(x^{\prime}\) события. Координата \(x^{\prime}\) характеризует собственную длину отрезка \(O^{\prime} P\), неподвижного в \(K^{\prime}\)-системе (рис. 7.11). Длина же этого отрезка в \(K\)-системе, где отсчет производится в момент \(t\), равна \(x-V t\). Связь между этими длинами дается формулой (7.5), согласно которой \(x-V t=x^{\prime} \sqrt{1-\beta^{2}}\). Отсюда


\begin{equation*}
x^{\prime}=(x-V t) / \sqrt{1-\beta^{2}} . \tag{7.6}
\end{equation*}


С другой стороны, координата \(x\) характеризует собственную длину отрезка \(O P\), неподвижного в \(K\)-системе. Длина же этого отрезка в \(K^{\prime}\)-системе, где измерение проводится в момент \(t^{\prime}\), равна \(x^{\prime}+V t^{\prime}\). Учитывая опять (7.5), получим \(x^{\prime}+V t^{\prime}=x \sqrt{1-\beta^{2}}\), откуда


\begin{equation*}
x=\left(x^{\prime}+V t^{\prime}\right) / \sqrt{1-\beta^{2}} . \tag{7.6'}
\end{equation*}


Полученные формулы позволяют также установить и связь между моментами времени \(t\) и \(t^{\prime}\) события \(A\) в обеих системах отсчета. Для этого достаточно исключить из (7.6) и (7.6') \(x^{\prime}\) или \(x\), после чего найдем:


\begin{equation*}
t^{\prime}=\left(t-x V / c^{2}\right) / \sqrt{1-\beta^{2}} ; \quad t=\left(t^{\prime}+x^{\prime} V / c^{2}\right) / \sqrt{1-\beta^{2}} . \tag{7.7}
\end{equation*}


Формулы (7.3), (7.6), (7.6') и (7.7) называют преобразованиями Лоренца. Они играют фундаментальную роль в теории относительности. По этим формулам осуществляется преобразование координат и времени любого события при переходе от одной инерциальной системы отсчета к другой.

Итак, преобразования Лоренца при переходе от \(K\) - к \(K^{\prime}\)-системе имеют вид:


\begin{equation*}
x^{\prime}=\frac{x-V t}{\sqrt{1-\beta^{2}}} ; \quad y^{\prime}=y ; \quad t^{\prime}=\frac{t-x V / c^{2}}{\sqrt{1-\beta^{2}}}, \tag{7.8}
\end{equation*}


а при обратном переходе от \(K^{\prime}\) - к \(K\)-системе -


\begin{equation*}
x=\frac{x^{\prime}+V t^{\prime}}{\sqrt{1-\beta^{2}}} ; \quad y=y^{\prime} ; \quad t=\frac{t^{\prime}+x^{\prime} V / c^{2}}{\sqrt{1-\beta^{2}}}, \tag{7.9}
\end{equation*}


где \(\beta=V / c, V\) - скорость \(K^{\prime}\)-системы относительно \(K\)-системы.\\
Следует сразу же обратить внимание на симметрию (одинаковый вид) формул (7.8) и (7.9), что является следствием полного равноправия обеих систем отсчета (различный знак при \(V\) в этих формулах обусловлен лишь противоположным направлением движения систем \(K\) и \(K^{\prime}\) относительно друг друга).

Преобразования Лоренца сильно отличаются от преобразований Галилея (7.1), однако последние могут быть получены из (7.8) и (7.9), если в них формально положить \(c=\infty\). Что это значит?

В конце предыдущего параграфа было отмечено, что в основе преобразований Галилея лежит допущение о синхронизации часов с помощью мгновенно распространяющихся сигналов. Из этого обстоятельства вытекает, что величина \(c\) в преобразованиях Лоренца играет роль скорости тех сигналов, которые используют для синхронизации часов. Если эта скорость бесконечно велика, то получаются преобразования Галилея; если же она равна скорости света, то - преобразования Лоренца. Таким образом, в основе преобразований Лоренца лежит допущение о

синхронизации часов с помощью световых сигналов, имеющих предельную скорость.

Замечательной особенностью преобразований Лоренца является то, что при \(V \ll c\) они переходят* в преобразования Галилея (7.1). Таким образом, в предельном случае \(V \ll c\) законы преобразования теории относительности и ньютоновской механики совпадают. Это означает, что теория относительности не отвергает преобразований Галилея как неправильные, но включает их в истинные законы преобразования как частный случай, справедливый при \(V \ll c\). В дальнейшем мы увидим, что это отражает общую взаимосвязь между теорией относительности и ньютоновской механикой - законы и соотношения теории относительности переходят в законы ньютоновской механики в предельном случае малых скоростей.

Далее, из преобразований Лоренца видно, что при \(V>c\) подкоренные выражения становятся отрицательными и формулы теряют физический смысл. Это соответствует тому факту, что движение тел со скоростью, большей скорости света в вакууме, невозможно. Нельзя даже пользоваться системой отсчета, движущейся со скоростью \(V=c\); при этом подкоренные выражения обращаются в нуль и формулы также теряют физический смысл. Это значит, что, например, с фотоном, движущимся со скоростью \(c\), принципиально не может быть связана система отсчета. Или иначе: не существует такой системы отсчета, в которой фотон был бы неподвижньлм.

И наконец, необходимо обратить внимание на то, что в формулы преобразования времени входит пространственная координата. Это важное обстоятельство указывает на неразрывную связь между пространством и временем. Другими словами, речь должна идти не отдельно о пространстве и времени, а о едином пространстве - времени, в котором протекают все физические явления.

\footnotetext{\begin{itemize}
  \item Строго говоря, необходимо еще, чтобы \(x / c \ll t\), т. е. чтобы времена распространения световых сигналов на расстояния, фигурирующие в рассматриваемых задачах ( \(x / c\) ), были малы по сравнению с интересующими нас промежутками времени. При этом условии можно считать, что сигналы распространяются практически мгновенно.
\end{itemize}
}\section*{§ 7.5. Следствия из преобразований Лоренца}
\section*{Понятие одновременности}
Пусть в \(K\)-системе отсчета происходят два каких-то события: \(A_{1}\left(x_{1}, y_{1}, t_{1}\right)\) и \(A_{2}\left(x_{2}, y_{2}, t_{2}\right)\). Найдем интервал времени между этими событиями в \(K^{\prime}\)-системе, движущейся со скоростью \(V\) вдоль оси \(X\), как показано на рис. 7.11. Согласно формуле преобразования времени (7.8), искомый интервал времени


\begin{equation*}
t_{2}^{\prime}-t_{1}^{\prime}=\frac{\left(t_{2}-t_{1}\right)-\left(x_{2}-x_{1}\right) V / c^{2}}{\sqrt{1-\beta^{2}}} . \tag{7.10}
\end{equation*}


Отсюда следует, что события, одновременные в \(K\)-системе ( \(t_{2}=t_{1}\) ), не одновременны в \(K^{\prime}\)-системе ( \(t_{2}^{\prime}-t_{1}^{\prime} \neq 0\) ). Исключением является случай, когда оба события происходят в \(K\)-системе одновременно в точках с одинаковыми значениями координаты \(x\) (координата \(y\) может быть различной).

Итак, одновременность - понятие относительное: то, что одновременно в одной системе отсчета, в общем случае не одновременно в другой системе отсчета. Говоря об одновременности событий, необходимо указывать систему отсчета, относительно которой эта одновременность имеет место. В противном случае понятие одновременности теряет смысл и могут возникнуть разного рода недоразумения и «парадоксы».\\
Пример. «Парадокс» стержня и трубки. Сквозь неподвижную в \(K\)-системе отсчета трубку \(A B\) a) длины \(l_{0}\) пролетает стержень \(A^{\prime} B^{\prime}\), собственная длина которого равна \(2 l_{0}\).\\
\includegraphics[max width=\textwidth, alt={}, center]{66a35da9-3faa-4823-9ca9-40799f37b3db-248_110_489_1290_649}\\
Скорость стержня такова, что его длина в \(K\)-системе равна длине трубки, \(l=l_{0}\) (рис. 7.12, \(a\) ), и в некоторый момент стерб)

\begin{figure}[h]
\begin{center}
  \includegraphics[alt={},max width=\textwidth]{66a35da9-3faa-4823-9ca9-40799f37b3db-248_118_461_1438_649}
\captionsetup{labelformat=empty}
\caption{Рис. 7.12}
\end{center}
\end{figure}

жень, пролетая сквозь трубку, целиком в ней уместится. Однако «с точки зрения стержня» лоренцево сокращение вдвое претерпевает трубка (рис. 7.12 , б), поэтому стержень (длины \(2 l_{0}\) ) не поместится в трубке (длины \(l_{0} / 2\) ). Есть ли здесь противоречие?

Противоречия нет, и вот почему. «С точки зрения трубки» концы пролетающего стержня совместятся с концами трубки одновременно. «С точки же зрения стержня» совпадения концов ( \(A\) с \(A^{\prime}, B\) с \(B^{\prime}\) ) произойдут не одновременно: сначала совпадут концы \(B\) и \(B^{\prime}\) (рис. 7.12, б), а затем, через некоторый промежуток времени, концы \(A\) и \(A^{\prime}\).

Из относительности понятия одновременности следует, что часы \(K^{\prime}\)-системы, расставленные вдоль оси \(X^{\prime}\) и синхронизированные между собой в этой системе отсчета, в \(K\)-системе будут показывать разное время. В самом деле, возьмем для простоты момент, когда начала координат \(O\) и \(O^{\prime}\) обеих систем отсчета совпадают и часы в этих точках показывают одно время: \(t=t^{\prime}=0\). Тогда в \(K\)-системе в точке с координатой \(x\) часы \(K\)-системы показывают в этот момент время \(t=0\), часы же \(K^{\prime}\)-системы в этой точке - иное время, \(t^{\prime}\). Действительно, согласно формуле преобразования времени (7.8),

\[
t^{\prime}=-x V / c^{2} \sqrt{1-\beta^{2}} .
\]

\begin{figure}[h]
\begin{center}
  \includegraphics[alt={},max width=\textwidth]{66a35da9-3faa-4823-9ca9-40799f37b3db-249_267_433_956_66}
\captionsetup{labelformat=empty}
\caption{Рис. 7.13}
\end{center}
\end{figure}

Отсюда видно, что в момент \(t=0\) (в \(K\)-системе) часы \(K^{\prime}\)-системы будут показывать разное время, зависящее от координаты \(x\) (местное время). Это показано на рис. 7.13, \(a\). Относительно \(K^{\prime}\)-системы картина будет обратной (рис. 7.13, б), как и должно быть в соответствии с равноправием обеих инерциальных систем отсчета.\\
Из формулы (7.10) видно, что для одновременных в \(K\)-системе событий знак разности \(t_{2}^{\prime}-t_{1}^{\prime}\) определяется знаком выражения \(-\left(x_{2}-x_{1}\right) V\). Следовательно, в разных системах отсчета (при разных значениях скорости \(V\) ) разность \(t_{2}^{\prime}-t_{1}^{\prime}\) будет различной не только по модулю, но и по знаку. Последнее означает, что порядок событий \(A_{1}\) и \(A_{2}\) может быть любым (как прямым, так и обратным).

Сказанное, однако, не относится к причинно-связанным событиям. Порядок следования таких событий (причина → следствие) будет одинаков во всех системах отсчета. В этом легко убедиться из следующего рассуждения. Рассмотрим, например,

выстрел - событие \(A_{1}\left(x_{1}, t_{1}\right)\) и попадание пули в мишень - событие \(A_{2}\left(x_{2}, t_{2}\right)\), предполагая, что оба события происходят на оси \(X\). В \(K\)-системе отсчета \(t_{2}>t_{1}\) скорость пули \(v\) и пусть для определенности \(x_{2}>x_{1}\), причем ясно, что \(x_{2}-x_{1}=v\left(t_{2}-t_{1}\right)\). После подстановки этого равенства в формулу (7.10) получим

\[
t_{2}^{\prime}-t_{1}^{\prime}=\frac{\left(t_{2}-t_{1}\right)\left(1-v V / c^{2}\right)}{\sqrt{1-\beta^{2}}} .
\]

Величина, стоящая во второй круглой скобке числителя, всегда положительна в связи с тем, что \(V<c\) (даже при \(v=c\), когда причинно-следственная связь обусловлена максимально возможной скоростью передачи сигналов или взаимодействий). Отсюда следует, что если \(t_{2}>t_{1}\), то и \(t_{2}^{\prime}>t_{1}^{\prime}\), т. е. порядок следования причинно-следственных событий одинаков во всех инерциальных системах отсчета.

\section*{Лоренцево сокращение}
Расположим неподвижный в \(K^{\prime}\)-системе стержень вдоль оси \(X^{\prime}\), т. е. вдоль направления движения этой системы отсчета относительно \(K\)-системы. Пусть длина стержня в \(K^{\prime}\)-системе \(l_{0}=x_{2}^{\prime}-x_{1}^{\prime}\) (собственная длина).

В \(K\)-системе, относительно которой стержень движется, его длину определяют как расстояние \(l\) между координатами \(x_{2}\) и \(x_{1}\) его концов, взятыми в один и тот же момент ( \(t_{2}=t_{1}\) ). Воспользовавшись преоборазованиями Лоренца (7.8) для координат \(x^{\prime}\) и \(x\), запишем

\[
l_{0}=x_{2}^{\prime}-x_{1}^{\prime}=\left(x_{2}-x_{1}\right) / \sqrt{1-\beta^{2}}=l / \sqrt{1-\beta^{2}},
\]

откуда


\begin{equation*}
l=l_{0} \sqrt{1-\beta^{2}} . \tag{7.11}
\end{equation*}


Таким образом, длина \(l\) движущегося стержня оказывается меньше его собственной длины \(l_{0}\), и в разных инерциальных системах отсчета она будет иметь свое значение. Этот результат полностью согласуется с полученным ранее (7.5).

Из определения длины следует, что относительность длины данного стержня является следствием относительности понятия одновременности. Это же относится и к форме любого тела - его размеры в направлении движения также различны в разных инерциальных системах отсчета.

\section*{Длительность процессов}
Пусть в точке с координатой \(x^{\prime} K^{\prime}\)-системы отсчета протекает некоторый процесс, длительность которого в этой системе \(\Delta t_{0}=t_{2}^{\prime}-t_{1}^{\prime}\) (собственное время процесса). Найдем длительность данного процесса \(\Delta t=t_{2}-t_{1}\) в \(K\)-системе, относительно которой \(K^{\prime}\)-система движется.

Воспользуемся с этой целью преобразованиями Лоренца для времени. Так как процесс происходит в точке с фиксированной координатой \(x^{\prime} K^{\prime}\)-системы, то наиболее удобно использовать формулы (7.9):

\[
t_{2}-t_{1}=\left(t_{2}^{\prime}-t_{1}^{\prime}\right) / \sqrt{1-\beta^{2}},
\]

или


\begin{equation*}
\Delta t=\Delta t_{0} / \sqrt{1-\beta^{2}} . \tag{7.12}
\end{equation*}


Отсюда видно, что длительность одного и того же процесса различна в разных инерциальных системах отсчета. В \(K\)-системе его длительность больше ( \(\Delta t>\Delta t_{0}\) ), а следовательно, в этой системе отсчета он протекает медленнее, чем в \(K^{\prime}\)-системе. Это вполне согласуется с результатом, относящимся к ходу одних и тех же часов в разных инерциальных системах отсчета, - формулой (7.4).

\section*{Интервал}
Относительный характер пространственных и временны́х промежутков отнюдь не означает, что теория относительности вообще отрицает существование каких бы то ни было абсолютных величин. В действительности дело обстоит как раз наоборот. Задача, которую ставит перед собой теория относительности, заключается в нахождении таких величин (и законов), которые не зависели бы от выбора инерциальной системы отсчета.

Первой из этих величин является универсальная скорость распространения взаимодействий, равная скорости света в вакууме. Другой, также весьма важной инвариантной величиной является интервал \(s_{12}\) между событиями 1 и 2 , квадрат которого определяется как


\begin{equation*}
s_{12}^{2}=c^{2} t_{12}^{2}-l_{12}^{2}=\mathrm{inv}, \tag{7.13}
\end{equation*}


где \(t_{12}\) - промежуток времени между событиями, \(l_{12}\) - расстояние между двумя точками, в которых происходят данные события ( \(l_{12}^{2}=x_{12}^{2}+y_{12}^{2}+z_{12}^{2}\) ).

В инвариантности интервала можно легко убедиться, вычислив его непосредственно в \(K^{\prime}\) - и \(K\)-системах отсчета. Воспользовавшись преобразованиями Лоренца (7.8) и учитывая, что \(y_{12}^{\prime}=y_{12} \quad\) и \(z_{12}^{\prime}=z_{12}\), запишем:

\[
c^{2} t_{12}^{\prime 2}-x_{12}^{\prime 2}=c^{2} \frac{\left(t_{12}-x_{12} V / c^{2}\right)^{2}}{1-\beta^{2}}-\frac{\left(x_{12}-V t_{12}\right)^{2}}{1-\beta^{2}}=c^{2} t_{12}^{2}-x_{12}^{2} .
\]

Таким образом, действительно, интервал является величиной инвариантной. Иначе говоря, утверждение «два события разделены таким-то интервалом \(s\) » имеет абсолютный характер - оно справедливо во всех инерциальных системах отсчета. Инвариантность интервала играет фундаментальную роль в теории относительности и служит весьма эффективным инструментом при анализе и решении многих вопросов (см., например, задачу 7.4).

Типы интервалов. В зависимости от того, какая составляющая в интервале преобладает, пространственная или временна́я, соответствующие интервалы называют: пространственноподобными ( \(l_{12}>c t_{12}\) ), времениподобными ( \(c t_{12}>l_{12}\) ). Кроме этих двух типов интервалов существует еще третий светоподобный ( \(c t_{12}=l_{12}\) ).

Если интервал между двумя событиями пространственноподобный, то всегда можно найти такую \(K^{\prime}\)-систему отсчета, в которой оба события происходят одновременно ( \(t_{12}^{\prime}=0\) ):

\[
c^{2} t_{12}^{2}-l_{12}^{2}=-l_{12}^{2}
\]

Если же интервал времениподобный, то всегда можно найти такую \(K^{\prime}\)-систему отсчета, в которой оба события происходят в одной точке ( \(l_{12}^{\prime}=0\) ):

\[
c^{2} t_{12}^{2}-l_{12}^{2}=c^{2} t_{12}^{\prime 2}
\]

В случае пространственноподобных интервалов \(l_{12}>c t_{12}\), т. е. ни в одной системе отсчета события не могут оказать влияния друг на друга, даже если бы связь между событиями осуществлялась с предельной скоростью \(c\). Иначе обстоит дело в случае времениподобных или светоподобных интервалов, для которых \(l_{12} \leqslant c t_{12}\). Следовательно, события, разделенные времениподобными или светоподобными интервалами, могут быть причинно-связанными друг с другом.

\section*{Преобразование скорости}
Пусть в \(K\)-системе в плоскости \(X, Y\) движется частица со скоростью \(\mathbf{v}\), проекции которой \(v_{x}\) и \(v_{y}\). Найдем с помощью преобразований Лоренца (7.8) проекции скорости этой частицы \(v_{x}^{\prime}\) и \(v_{y}^{\prime}\) в \(K^{\prime}\)-системе, движущейся со скоростью \(\mathbf{V}\), как показано на рис. 7.11.

Для этого проведем расчет по следующей схеме:

\[
v_{x}^{\prime}=\frac{\mathrm{d} x^{\prime}}{\mathrm{d} t^{\prime}}=\frac{\mathrm{d} x^{\prime} / \mathrm{d} t}{\mathrm{~d} t^{\prime} / \mathrm{d} t}, \quad v_{y}^{\prime}=\frac{\mathrm{d} y^{\prime}}{\mathrm{d} t^{\prime}}=\frac{\mathrm{d} y^{\prime} / \mathrm{d} t}{\mathrm{~d} t^{\prime} / \mathrm{d} t} .
\]

Продифференцируем выражения (7.8) для \(x^{\prime}, y^{\prime}\) и \(t^{\prime}\) по времени \(t\) и результаты подставим в предыдущие формулы для \(v_{x}^{\prime}\) и \(v_{y}^{\prime}\). После несложных преобразований получим


\begin{equation*}
v_{x}^{\prime}=\frac{v_{x}-V}{1-v_{x} V / c^{2}}, \quad v_{y}^{\prime}=\frac{v_{y} \sqrt{1-\beta^{2}}}{1-v_{x} V / c^{2}}, \tag{7.14}
\end{equation*}


где \(\beta=V / c\). Отсюда скорость частицы в \(K^{\prime}\)-системе


\begin{equation*}
v^{\prime}=\sqrt{v_{x}^{\prime 2}+v_{y}^{\prime 2}}=\frac{\sqrt{\left(v_{x}-V\right)^{2}+v_{y}^{2}\left(1-\beta^{2}\right)}}{1-v_{x} V / c^{2}} . \tag{7.15}
\end{equation*}


Эти формулы выражают релятивистский закон преобразования скорости. При малых скоростях ( \(V \ll c\) и \(v \ll c\) ) они переходят, как нетрудно убедиться, в формулы преобразования скорости ньютоновской механики:

\[
v_{x}^{\prime}=v_{x}-V ; \quad v_{y}^{\prime}=v_{y},
\]

или в векторном виде

\[
\mathbf{v}^{\prime}=\mathbf{v}-\mathbf{V} .
\]

Обратим внимание на то, что последняя формула оказывается справедливой только в ньютоновском приближении; в релятивистской же области она не имеет смысла здесь нет простого закона сложения скоростей. В этом можно легко убедиться хотя бы на таком примере. Пусть вектор скорости \(\mathbf{v}\) частицы в \(K\)-системе перпендикулярен оси \(X\), т. е. имеет проекции \(v_{x}=0\) и \(v_{y}=v\). Тогда, согласно (7.14), проекции скорости этой час-

\begin{figure}[h]
\begin{center}
  \includegraphics[alt={},max width=\textwidth]{66a35da9-3faa-4823-9ca9-40799f37b3db-254_334_392_625_768}
\captionsetup{labelformat=empty}
\caption{Рис. 7.14}
\end{center}
\end{figure}


\begin{equation*}
v_{x}^{\prime}=-V ; \quad v_{y}^{\prime}=v_{y} \sqrt{1-\beta^{2}} . \tag{7.16}
\end{equation*}


Это значит, что в данном случае \((\mathbf{v} \perp\) оси \(X) v_{y}^{\prime}\)-проекция скорости уменьшается при переходе к \(K^{\prime}\)-системе, и ясно, что \(\mathbf{v}^{\prime} \neq \mathbf{v}-\mathbf{V}\) (рис. 7.14).

Рассмотрим еще один пример использования формул преобразования скорости (7.14) - при движении двух частиц (см. также задачу 7.7).\\
Пример. Пусть две релятивистские частицы движутся в \(K\)-системе отсчета навстречу друг другу по одной прямой с одинаковой скоростью \(v\). Найдем: 1) скорость сближения частиц в этой системе отсчета; 2) их относительную скорость.\\
Прежде всего необходимо уточнить, что́ понимается под каждой из этих скоростей.

\begin{enumerate}
  \item Скорость сближения - это скорость, с которой изменяется (уменьшается) расстояние между частицами в данной системе
\end{enumerate}

отсчета. В нашем случае она просто равна \(2 v\), причем эта скорость может быть и больше скорости света - это ничему не противоречит.\\
2. Под относительной скоростью имеется в виду скорость, с

\begin{figure}[h]
\begin{center}
  \includegraphics[alt={},max width=\textwidth]{66a35da9-3faa-4823-9ca9-40799f37b3db-255_233_423_314_205}
\captionsetup{labelformat=empty}
\caption{Рис. 7.15}
\end{center}
\end{figure}

которой одна из частиц движется в системе отсчета, связанной с другой частицей и перемещающейся поступательно по отношению к исходной \(K\)-системе. Чтобы найти эту скорость, выберем ось \(X\) вдоль направления движения частиц. Свяжем с одной из частиц, например частицей 1 , которая движется в положительном направлении оси \(X, K^{\prime}\)-систему отсчета (рис. 7.15). Тогда задача сводится к нахождению скорости частицы 2 в этой системе отсчета. Подставив в формулу (7.14) для \(v_{x}\)-проекции скорости \(v_{x}=-v, V=v\), получим

\[
v_{x}^{\prime}=-\frac{2 v}{1+(v / c)^{2}} .
\]

Знак минус означает, что в данном случае частица 2 движется в отрицательном направлении оси \(X^{\prime} K^{\prime}\)-системы отсчета.

Следует отметить, что даже в том случае, когда обе частицы движутся с максимально возможной скоростью \(v \approx c\), скорость \(v_{x}^{\prime}\) не может превзойти \(c\) - это сразу видно из последней формулы.

И наконец, проверим непосредственно, что релятивистские формулы преобразования скоростей соответствуют утверждению второго постулата Эйнштнейна относительно неизменности скорости света \(c\) во всех инерциальных системах отсчета. Пусть вектор с имеет в \(K\)-системе проекции \(c_{x}\) и \(c_{y}\), т. е. \(c^{2}=c_{x}^{2}+c_{y}^{2}\). Воспользуемся формулой (7.15), преобразовав в ней подкоренное выражение следующим образом:

\[
c_{x}^{2}-2 c_{x} V+V^{2}+\left(c^{2}-c_{x}^{2}\right)\left(1-\frac{V^{2}}{c^{2}}\right)=\left(c-\frac{c_{x} V}{c}\right)^{2} .
\]

После этого нетрудно получить, что \(v^{\prime}=c\). При этом, конечно, вектор \(\mathbf{c}^{\prime}\) в \(K^{\prime}\)-системе будет иметь в общем случае другое направление.

\section*{Задачи}
7.1. Преобразование длины. В \(K\)-системе отсчета находится неподвижный стержень длины \(l=100 \mathrm{~cm}\), ориентированный под углом \(\vartheta=45^{\circ}\) к оси \(O X\) (рис. 7.16). Найти его длину \(l^{\prime}\) и соответствующий угол \(\vartheta^{\prime}\) в \(K^{\prime}\)-системе, движущейся относительно \(K\)-системы со скоростью \(V=c / 2\) вдоль оси \(O X\).\\
\(P\) ешение. Длина стержня в \(K^{\prime}\)-системе

\[
l^{\prime}=\sqrt{\left(\Delta x^{\prime}\right)^{2}+\left(\Delta y^{\prime}\right)^{2}}=\sqrt{(\Delta x)^{2}\left(1-\beta^{2}\right)+(\Delta y)^{2}}
\]

где \(\beta=V / c\). Имея в виду, что \(\Delta x=l \cos \vartheta\) и \(\Delta y=l \sin \vartheta\), получим

\[
l^{\prime}=l \sqrt{1-\beta^{2} \cos ^{2} \vartheta}=94 \mathrm{~cm} .
\]

Угол \(\vartheta^{\prime}\) в \(K^{\prime}\)-системе найдем через тангенс:

\[
\operatorname{tg} \vartheta^{\prime}=\frac{\Delta y^{\prime}}{\Delta x^{\prime}}=\frac{\Delta y}{\Delta x \sqrt{1-\beta^{2}}}=\frac{\operatorname{tg} \vartheta}{\sqrt{1-\beta^{2}}}=1,155 .
\]

Отсюда \(\vartheta^{\prime}=49^{\circ}\). Следует обратить внимание на то, что полученные результаты не зависят от направления скорости \(K^{\prime}\)-системы: она может двигаться или в положительном направлении оси \(x\), или в противоположном.\\
7.2. Собственная длина. Стержень движется вдоль линейки с некоторой постоянной скоростью. Если зафиксировать положение обоих концов стержня одновременно в системе отсчета, связанной с линейкой, то разность отсчетов по линейке \(\Delta x_{1}=4,0\) м. Если же положение обоих концов зафиксировать одновременно в системе отсчета, связанной со стержнем, то разность отсчетов по тойже линейке \(\Delta x_{2}=9,0 \mathrm{~m}\). Определить собственную длину \(l_{0}\) стержня и

\begin{figure}[h]
\begin{center}
  \includegraphics[alt={},max width=\textwidth]{66a35da9-3faa-4823-9ca9-40799f37b3db-256_264_421_1172_739}
\captionsetup{labelformat=empty}
\caption{Рис. 7.16}
\end{center}
\end{figure}

\[
\Delta x_{1}=l_{0} \sqrt{1-\beta^{2}}
\]

где \(\beta\) - скорость стержня (в единицах скорости света).

Во втором же случае \(l_{0}\) - это измеренная в системе отсчета, связанной со стержнем, длина участка движущейся линейки, собственный размер которого (участка) равен \(\Delta x_{2}\). Поэтому

\[
l_{0}=\Delta x_{2} \sqrt{1-\beta^{2}}
\]

Из этих формул легко найти, что

\[
l_{0}=\sqrt{\Delta x_{1} \cdot \Delta x_{2}}=6,0 \mathrm{~m}, \quad \beta=\sqrt{1-\Delta x_{1} / \Delta x_{2}} \approx 0,75,
\]

или \(v \approx 0,75 c\).\\
7.3. Преобразование времени. Две нестабильные частицы движутся в \(K\)-системе отсчета по некоторой прямой в одном направлении с одинаковой скоростью \(v=0,990 c\). Расстояние между частицами в этой системе отсчета \(l=12 \mathrm{~m}\). В некоторый момент обе частицы распались одновременно в \(K^{\prime}\)-системе отсчета, связанной с ними. Найти: 1) промежуток времени между моментами распада обеих частиц в исходной \(K\)-системе отсчета; 2) какая частица распалась позже в \(K\)-системе.\\
Решение. 1. Пусть распад частицы, двигавшейся впереди, событие 1 , а распад частицы, двигавшейся сзади, - событие 2. Тогда, согласно преобразованиям Лоренца (7.9) для времени,

\[
t_{1}-t_{2}=\frac{\left(x_{1}^{\prime}-x_{2}^{\prime}\right) v / c^{2}}{\sqrt{1-(v / c)^{2}}}
\]

где учтено, что \(t_{1}^{\prime}=t_{2}^{\prime}\) (по условию). Разность ( \(x_{1}^{\prime}-x_{2}^{\prime}\) ) - это собственное расстояние \(l_{0}\) между частицами. Согласно (7.5), оно равно \(l_{0}=l / \sqrt{1-(v / c)^{2}}\). Поэтому

\[
t_{1}-t_{2}=\frac{l v / c^{2}}{1-(v / c)^{2}}=2,0 \text { мкс. }
\]

\begin{enumerate}
  \setcounter{enumi}{1}
  \item Так как \(\left(t_{1}-t_{2}\right)>0\), то \(t_{1}>t_{2}\); другими словами, частица, двигавшаяся впереди, распалась позже.\\
Замечание. Нередко эту задачу решают так: согласно (7.8),
\end{enumerate}

\[
t_{1}^{\prime}-t_{2}^{\prime}=\frac{\left(t_{1}-t_{2}\right)-\left(x_{1}-x_{2}\right) v / c^{2}}{\sqrt{1-(v / c)^{2}}}=0
\]

откуда

\[
t_{1}-t_{2}=\left(x_{1}-x_{2}\right) v / c^{2}=l v / c^{2} .
\]

Полученный результат отличается от приведенного выше и является неверным. Дело в том, что мы не имеем права разность \(x_{1}-x_{2}\) заменить на \(l\), ибо \(x_{1}\) и \(x_{2}\) - это координаты событий (распадов), происшедших в \(K\)-системе в разные моменты времени. Расстояние же \(l\) между частицами в \(K\)-системе равно, по определению, разности координат частиц, зафиксированных одновременно.\\
7.4. Найти расстояние, которое пролетела в \(K\)-системе отсчета нестабильная частица от момента ее рождения до распада, если ее время жизни в этой системе отсчета \(\Delta t=3,0\) мкс, а собственное время жизни \(\Delta t_{0}=2,2\) мкс.\\
Р е ш е н и е. Воспользовавшись формулой (7.12), найдем скорость \(V\) частицы и затем искомое расстояние как

\[
l=\Delta t \cdot V=\Delta t c \sqrt{1-\left(\Delta t_{0} / \Delta t\right)^{2}}=0,6 \text { км. }
\]

Другой способ решения основан на использовании инвариантности интервала:

\[
c^{2}\left(\Delta t_{0}\right)^{2}=c^{2}(\Delta t)^{2}-l^{2}
\]

где квадрат интервала записан слева в системе отсчета, связанной с самой частицей, а справа - в \(K\)-системе отсчета. Отсюда получается тот же результат для \(l\).\\
7.5. Әффект Доплера. В \(K\)-системе отсчета находится неподвижный приемник \(P\) (рис. 7.17). К нему со скоростью \(V\) приближается источник \(S\) световых сигналов. В системе отсчета, связанной с источником, сигналы испускаются периодически с частотой \(v_{0}\) (собственная частота). С какой частотой \(v\) будет воспринимать эти сигналы приемник \(P\) ?

\begin{figure}[h]
\begin{center}
  \includegraphics[alt={},max width=\textwidth]{66a35da9-3faa-4823-9ca9-40799f37b3db-258_208_430_1135_732}
\captionsetup{labelformat=empty}
\caption{Рис. 7.17}
\end{center}
\end{figure}

Решение. Промежуток времени между двумя последовательными сигналами (импульсами) в \(K^{\prime}\)-системе, связанной с источником, равен \(T_{0}=1 / v_{0}\). Так как эта система движется со скоростью \(V\), то соответствующий промежуток времени в \(K\)-системе, согласно (7.12), будет больше:

\[
T=T_{0} / \sqrt{1-\beta^{2}}, \quad \beta=V / c
\]

Расстояние между соседними импульсами в \(K\)-системе


\begin{equation*}
\lambda=c T-V T=(c-V) T=(c-V) \frac{T_{0}}{\sqrt{1-\beta^{2}}} . \tag{1}
\end{equation*}


Поэтому воспринимаемая приемником частота \(v=c / \lambda\), или


\begin{equation*}
v=v_{0} \frac{\sqrt{1-\beta^{2}}}{1-\beta} . \tag{2}
\end{equation*}


Если источник приближается (как в нашем случае), то \(v>v_{0}\), если же удаляется, то \(v<v_{0}\) (в этом слу-\\
\includegraphics[max width=\textwidth, alt={}, center]{66a35da9-3faa-4823-9ca9-40799f37b3db-259_456_451_595_136}

Удаление Сближение\\
Рис. 7.18 чае знак \(\beta\) меняется на противоположный). Зависимость \(v / v_{0}\) от \(\beta\) показана на рис. 7.18. Полученная формула (2) для частоты \(v\) соответствует продольному эффекту Доплера. Как видно из приведенного вывода, эффект Доплера является следствием двух явлений: замедления хода движущихся часов [корень в числителе формулы (2)] и «уплотнения» (или разрежения) импульсов, связанного с изменением расстояния между источником и приемником [это учтено в первом равенстве формулы (1)].\\
Заметим, что в нерелятивистском случае \(T=T_{0}\), поэтому формула для эффекта Доплера не содержит корня \(\sqrt{1-\beta^{2}}\) (вместо него стоит единица):

\[
v=v_{0} /(1-\beta) \approx v_{0}(1+V / c)
\]

Рассмотрим попутно более общий слу-

\begin{figure}[h]
\begin{center}
  \includegraphics[alt={},max width=\textwidth]{66a35da9-3faa-4823-9ca9-40799f37b3db-259_114_390_1476_141}
\captionsetup{labelformat=empty}
\caption{Рис. 7.19}
\end{center}
\end{figure}

чай: в \(K\)-системе вектор скорости \(\mathbf{V}\) источника составляет угол \(\alpha\) с линией наблюдения, как показано на рис. 7.19. В этом случае в формуле (1) достаточно заменить \(V\) на \(V \cos \alpha\). Тогда

\[
v=v_{0} \frac{\sqrt{1-\beta^{2}}}{1-\beta \cos \alpha} .
\]

В частности, при \(\alpha=\pi / 2\) наблюдается поперечный эффект Доплеpa

\[
v=v_{0} \sqrt{1-\beta^{2}}
\]

при котором воспринимаемая приемником частота \(v\) оказывается всегда меньше собственной частоты \(v_{0}\).\\
7.6. Соотношения между событиями. На рис. 7.20 изображена диаграмма пространства - времени. Каждая точка этой диаграммы (мировая точка) характеризует некоторое событие - его координату и момент времени, когда оно произошло. Рассмотрим три события, соответствующие мировым точкам \(A, B\) и \(C\). Убедиться, что между этими событиями имеют место следую-

\begin{figure}[h]
\begin{center}
  \includegraphics[alt={},max width=\textwidth]{66a35da9-3faa-4823-9ca9-40799f37b3db-260_344_400_470_739}
\captionsetup{labelformat=empty}
\caption{Рис. 7.20}
\end{center}
\end{figure}

\begin{center}
\begin{tabular}{|l|l|l|l|l|}
\hline
\multirow{2}{*}{Пара событий} & \multirow{2}{*}{Тип интервала} & \multicolumn{2}{|c|}{Собственное} & \multirow{2}{*}{Возможность причинноследственной связи} \\
\hline
 &  & время \(c \Delta t_{0}\), м & расстояние \(\Delta x_{0}, \mathrm{~m}\) &  \\
\hline
\(A B\) & Времениподобный & 4 & - & \(A \rightarrow B\) \\
\hline
\(A C\) & Пространственноподобный & - & 4 & Нет \\
\hline
\(B C\) & Светоподобный & 0 & 0 & \(C \rightarrow B\) \\
\hline
\end{tabular}
\end{center}

Ук аз з н и е: воспользоваться инвариантностью интервала.\\
7.7. Две частицы движутся в \(K\)-системе отсчета под прямым углом друг к другу, причем первая частица со скоростью \(v_{1}\), а вторая со скоростью \(v_{2}\). Найти скорость одной частицы относительно другой.

Решение. Возьмем оси координат \(K\)-системы, как показано на рис. 7.21. Свяжем с частицей \(1 K^{\prime}\)-систему, тогда скорость частицы 2 в этой системе отсчета и есть искомая скорость. С помощью формулы (7.15), положив \(V=v_{1}\) и \(v_{x}=0\), получим

\[
v_{2}^{\prime}=\sqrt{v_{2 x}^{\prime 2}+v_{2 y}^{\prime 2}}=\sqrt{v_{1}^{2}+v_{2}^{2}-\left(v_{1} v_{2} / c\right)^{2}} .
\]

\begin{figure}[h]
\begin{center}
  \includegraphics[alt={},max width=\textwidth]{66a35da9-3faa-4823-9ca9-40799f37b3db-261_244_384_136_141}
\captionsetup{labelformat=empty}
\caption{Рис. 7.21}
\end{center}
\end{figure}

Заметим, что по классическому закону сложения скоростей,

\[
v_{2}^{\prime}=\sqrt{v_{1}^{2}+v_{2}^{2}} .
\]

\section*{7.8. Преобразование направления скорости. Частица движется в}
\begin{figure}[h]
\begin{center}
  \includegraphics[alt={},max width=\textwidth]{66a35da9-3faa-4823-9ca9-40799f37b3db-261_230_364_506_136}
\captionsetup{labelformat=empty}
\caption{Рис. 7.22}
\end{center}
\end{figure}

Решение. Пусть в \(K\)-системе проекции вектора \(\mathbf{v}\) равны \(v_{x}\) и \(v_{y}\). Тогда для угла \(\vartheta\) можно записать следующее соотношение:

\[
\operatorname{tg} \vartheta=v_{y} / v_{x} .
\]

В \(K^{\prime}\)-системе с учетом формул (7.14) получим

\[
\operatorname{tg} \vartheta^{\prime}=v_{y}^{\prime} / v_{x}^{\prime}=v_{y} \sqrt{1-\beta^{2}} /\left(v_{x}-V\right) .
\]

После подстановки \(v_{x}=v \cos \vartheta\) и \(v_{y}=v \sin \vartheta\) найдем

\[
\operatorname{tg} \vartheta^{\prime}=\frac{\sin \vartheta \sqrt{1-\beta^{2}}}{\cos \vartheta-V / v}
\]

Как видно из этой формулы, закон преобразования углов для скорости иной, нежели для отрезков (см. задачу 7.1).\\
7.9. Стержень, ориентированный параллельно оси \(X K\)-системы отсчета, движется в этой системе со скоростью \(v\) в положительном направлении оси \(Y\). Найти угол \(\vartheta^{\prime}\) между стержнем и осью \(X^{\prime} K^{\prime}\)-системы, перемещающейся со скоростью \(V\) относительно \(K\)-системы в положительном направлении ее оси \(X\). Оси \(X\) и \(X^{\prime}\) совпадают, оси \(Y\) и \(Y^{\prime}\) параллельны друг другу.

Решение. Пусть в некоторый момент концы стержня совпадают с осью \(X\) в \(K\)-системе. Эти два события, одновременные в \(K\)-системе, будут неодновременными в \(K^{\prime}\)-системе. Согласно (7.10), они произойдут через промежуток времени

\[
\Delta t^{\prime}=\Delta x V / c^{2} \sqrt{1-\beta^{2}}
\]

где \(\Delta x\) - собственная длина стержня. За это время правый конец стержня окажется «выше» левого на \(\Delta y^{\prime}=v_{y}^{\prime} \Delta t^{\prime}\), где \(v_{y}^{\prime}=v \sqrt{1-\beta^{2}}\) [см. (7.16)]. Таким образом, в \(K^{\prime}\)-системе данный стержень будет повернут против часовой стрелки на некоторый угол \(\vartheta^{\prime}\), который можно определить по формуле

\[
\operatorname{tg} \vartheta^{\prime}=\Delta y^{\prime} / \Delta x^{\prime}=\beta v / c \sqrt{1-\beta^{2}},
\]

где \(\Delta x^{\prime}=\Delta x \sqrt{1-\beta^{2}}\) - проекция стержня на ось \(X^{\prime}\) в \(K^{\prime}\)-системе, \(\beta=V / c\).\\
7.10. Релятивистское преобразование ускорения. В \(K\)-системе движется частица со скоростью \(\mathbf{v}\) и ускорением \(\mathbf{a}\). Найти ускорение этой частицы в \(K^{\prime}\)-системе, которая перемещается со скоростью \(\mathbf{V}\) в положительном направлении оси \(X K\)-системы. Рассмотреть случаи, когда частица движется вдоль следующих осей \(K\)-системы: 1) \(X\); 2) \(Y\).\\
P е шен и е. 1. Запишем каждую проекцию ускорения частицы в \(K^{\prime}\)-системе таким образом:

\[
a_{x}^{\prime}=\frac{\mathrm{d} v_{x}^{\prime}}{\mathrm{d} t^{\prime}}=\frac{\mathrm{d} v_{x}^{\prime}}{\mathrm{d} t} \frac{1}{\mathrm{~d} t^{\prime} / \mathrm{d} t}
\]

Воспользовавшись первой из формул (7.14) и последней из (7.8), получим после дифференцирования

\[
a_{x}^{\prime}=\frac{\left(1-\beta^{2}\right)^{3 / 2}}{\left(1-\beta v_{x} / c\right)^{3}} a_{x}, \quad a_{y}^{\prime}=0
\]

\begin{enumerate}
  \setcounter{enumi}{1}
  \item Аналогичные расчеты приводят к следующим результатам:
\end{enumerate}

\[
a_{x}^{\prime}=0, \quad a_{y}^{\prime}=\left(1-\beta^{2}\right) a_{y}
\]

В этих формулах \(\beta=V / c\).

\section*{§ 8.1. Релятивистский импульс}
Напомним сначала два основных положения ньютоновской механики об импульсе:

\begin{enumerate}
  \item импульс частицы определяется как \(\mathbf{p}=m \mathbf{v}\), где масса \(m\) частицы считается не зависящей от ее скорости;
  \item импульс замкнутой системы частиц сохраняется во времени в любой инерциальной системе отсчета.
\end{enumerate}

Теперь обратимся к релятивистской динамике. Оказывается (это будет видно уже из простого примера, который мы сейчас рассмотрим), для замкнутой системы из релятивистских частиц закон сохранения ньютоновского импульса не выполняется. Возникает альтернатива: отказаться или от ньютоновского определения импульса, или от закона сохранения этой величины.

Учитывая громадную роль, которую играют законы сохранения, в теории относительности за фундаментальный принимают именно закон сохранения импульса и уже отсюда находят выражение для самого импульса*.

\footnotetext{\begin{itemize}
  \item Возникает естественный вопрос: как же закон сохранения импульса может представлять какую-либо ценность, если импульс определяют именно так, чтобы он сохранялся? Для ответа на этот вопрос представим себе частицу, которая при своем движении сталкивается с другими частицами. Рассмотрев первое столкновение, определим импульс так, чтобы выполнялся его закон сохранения в данном столкновении. Но при последующих столкновениях положение изменится: мы уже будем знать импульсы частиц, участвующих в этих столкновениях, и теперь закон сохранения импульса (если он действительно есть) будет выполняться уже не по определению, а в силу глубинных законов природы.\\
Опыт показывает, что так определенный импульс действительно подчиняется закону сохранения. По крайней мере до сих пор не обнаружено ни одного явления, где бы этот закон нарушался.
\end{itemize}
}Покажем прежде всего, что требование, чтобы закон сохранения импульса выполнялся в любой инерциальной системе отсчета, и учет релятивистского преобразования скоростей при переходе от одной инерциальной системы отсчета к другой приводят к выводу, что масса частицы должна зависеть от ее скорости (в отличие от ньютоновской механики). Для этого рассмотрим абсолютно неупругое столкновение двух частиц - система предполагается замкнутой.

\begin{figure}[h]
\begin{center}
  \includegraphics[alt={},max width=\textwidth]{66a35da9-3faa-4823-9ca9-40799f37b3db-264_340_971_508_133}
\captionsetup{labelformat=empty}
\caption{Рис. 8.1}
\end{center}
\end{figure}

Пусть в некоторой инерциальной \(K\)-системе отсчета навстречу друг другу движутся две одинаковые частицы 1 и 2 с одинаковой скоростью \(v_{0}\), но под углом \(\alpha\) к оси \(X\) (рис. 8.1, \(a\) ). В этой системе отсчета суммарный импульс обеих частиц, очевидно, сохраняется: до и после столкновения он равен нулю (образовавшаяся частица, как следует из соображений симметрии, оказывается неподвижной).

Выясним, как будет обстоять дело в другой инерциальной системе отсчета. Для этого выберем сначала две системы отсчета: \(K_{1}\)-систему, движущуюся вправо со скоростью \(v_{1 x}\), и \(K_{2}\)-систему, движущуюся влево со скоростью \(v_{2 x}\) (рис. 8.1, \(a\) ). Ясно, что частица 1 в \(K_{1}\)-системе и частица 2 в \(K_{2}\)-системе движутся только вдоль оси \(Y\), причем с одинаковыми по модулю скоростями, которые мы обозначим \(u\).

Рассмотрим картину столкновения в \(K_{1}\)-системе (рис. 8.1, \(б\) ), где частица 1 имеет скорость \(u\). Найдем \(y\)-составляющую скорости частицы 2 в этой системе отсчета, обозначив ее \(u^{\prime}\). Эта частица, как было сказано, движется со скоростью \(u\) вдоль оси \(Y\) в \(K_{2}\)-системе и, кроме того, вместе с \(K_{2}\)-системой перемещается влево со скоростью \(V\) относительно \(K_{1}\)-системы. Поэтому,

согласно (7.16), \(y\)-составляющая скорости частицы 2 в \(K_{1}\)-системе равна


\begin{equation*}
u^{\prime}=u \sqrt{1-(V / c)^{2}} . \tag{8.1}
\end{equation*}


Запишем \(y\)-составляющие импульсов обеих частиц в \(K_{1}\)-системе: \(m_{1} u\) и \(m_{2} u^{\prime}\). Согласно (8.1), \(u^{\prime}<u\), поэтому легко видеть, что закон сохранения импульса в его обычной (ньютоновской) формулировке не выполняется. Действительно, в нашем случае \(m_{1}=m_{2}\) (частицы одинаковые) и, следовательно, \(y\)-составляющая суммарного импульса частиц до столкновения отлична от нуля, а после столкновения равна нулю (образовавшаяся частица будет двигаться только вдоль оси \(X\) ).

Потребуем, однако, чтобы закон сохранения импульса выполнялся и в \(K_{1}\)-системе, т. е. положим, что \(m_{1} u=m_{2} u^{\prime}\). Отсюда с учетом (8.1) получим

\[
m_{2}=m_{1} / \sqrt{1-(V / c)^{2}} .
\]

При \(\alpha \rightarrow 0\) (рис. 8.1) \(u \rightarrow 0\) и \(m_{1}\) представляет собой массу покоящейся частицы; ее обозначают \(m_{0}\) и называют массой покоя. Скорость же \(V\) при этом условии оказывается равной скорости \(v\) частицы 2 относительно частицы 1. Поэтому последнюю формулу можно переписать так:


\begin{equation*}
m=m_{0} / \sqrt{1-(v / c)^{2}}, \tag{8.2}
\end{equation*}


\begin{figure}[h]
\begin{center}
  \includegraphics[alt={},max width=\textwidth]{66a35da9-3faa-4823-9ca9-40799f37b3db-265_314_354_1294_68}
\captionsetup{labelformat=empty}
\caption{Рис. 8.2}
\end{center}
\end{figure}

где \(m\) - «масса» движущейся частицы (напомним, обе частицы одинаковые). Величину \(m\) называют релятивистской массой. Она, как видно из формулы (8.2), больше массы покоя и зависит от скорости частицы (рис. 8.2). Другими словами, релятивистская масса одной и той же частицы различна в разных инерциальных системах отсчета.\\
В отличие от релятивистской массы масса покоя \(m_{0}\) частицы - величина инвариантная, т. е. одинаковая во всех системах отсчета. По этой причине именно масса покоя является

характеристикой частицы. В дальнейшем, однако, мы будем использовать и релятивистскую массу \(m\), имея в виду при этом, что \(m\) представляет собой просто сокращенное обозначение отношения \(m_{0} / \sqrt{1-(v / c)^{2}}\), и не более. Использование релятивистской массы продиктовано только стремлением упростить ряд выводов, рассуждений и расчетов.

Массу же покоя \(m_{0}\) будем называть в дальнейшем просто массой.

Теперь сделаем последний шаг - напишем выражение для импульса релятивистской частицы. С учетом (8.2) этот импульс записывают в виде


\begin{equation*}
\mathbf{p}=m \mathbf{v}=\frac{m_{0} \mathbf{v}}{\sqrt{1-(v / c)^{2}}} . \tag{8.3}
\end{equation*}


Это и есть релятивистский импульс частицы. Опыт подтверждает, что так определенный импульс действительно подчиняется закону сохранения независимо от выбора инерциальной системы отсчета.

Отметим, что при \(v \ll c\) из (8.3) следует ньютоновское определение импульса: \(\mathbf{p}=m_{0} \mathbf{v}\), где \(m_{0}\) не зависит от скорости \(v\). На рис. 8.3 показаны для сравнения графики зависимостей релятивистского \(p_{\text {рел }}\) и ньютоновского \(p_{\text {н }}\) импульсов частицы от ее скорости. Различие между обоими импульсами становится весьма значительным по

\begin{figure}[h]
\begin{center}
  \includegraphics[alt={},max width=\textwidth]{66a35da9-3faa-4823-9ca9-40799f37b3db-266_353_458_1024_698}
\captionsetup{labelformat=empty}
\caption{Рис. 8.3}
\end{center}
\end{figure}

Пример 1. В современных гигантских ускорителях протоны ускоряются до скоростей, отличающихся от скорости света на \(0,0003 \%\). Найдем, во сколько раз релятивистская масса таких протонов превышает их массу (покоя).\\
Согласно (8.2), \(m / m_{0}=1 / \sqrt{1-\beta^{2}}\), где \(\beta=v / c\). Так как \(\beta\) мало отличается от единицы в данном случае, то подкорен-

ное выражение следует представить в виде

\[
1-\beta^{2}=(1+\beta)(1-\beta) \approx 2(1-\beta) .
\]

Тогда искомое отношение

\[
m / m_{0} \approx 1 / \sqrt{2(1-\beta)} \approx 4 \cdot 10^{2} .
\]

Пример 2. Выясним, при каких значениях скорости частицы ее ньютоновский импульс отличается от релятивистского на \(1 \%\); на \(10 \%\).\\
Из условия \(\eta=\left(p-p_{\mathrm{H}}\right) / p=1-\sqrt{1-(v / c)^{2}}\) получим

\[
v / c=\sqrt{\eta(2-\eta)}=\left\{\begin{array}{l}
0,14 \text { при } \eta=0,01, \\
0,45 \text { при } \eta=0,10 .
\end{array}\right.
\]

Использование нерелятивистской формулы для импульса гарантирует точность не хуже \(1 \%\) при \(v / c \leqslant 0,14\) и не хуже \(10 \%\) при \(v / c \leqslant 0,45\).

\section*{§ 8.2. Основное уравнение релятивистской динамики}
Согласно принципу относительности Эйнштейна, все законы природы должны быть инвариантны по отношению к инерциальным системам отсчета. Другими словами, математические формулировки законов должны иметь один и тот же вид во всех этих системах отсчета. В частности, это относится и к законам динамики.

Однако, как показывает более детальное рассмотрение, уже основное уравнение динамики Ньютона \(m \mathbf{a}=\mathbf{F}\) не удовлетворяет принципу относительности Эйнштейна. Преобразования Лоренца при переходе к другой инерциальной системе придают ему совершенно иную форму.

Чтобы удовлетворить требованиям принципа относительности, основное уравнение динамики должно иметь другой вид и лишь при \(v \ll c\) переходить в ньютоновское уравнение. Этим требованиям, как доказывается в теории относительности, удовлетворяет уравнение


\begin{equation*}
\mathrm{d} \mathbf{p} / \mathrm{d} t=\mathbf{F}, \tag{8.4}
\end{equation*}


где \(\mathbf{F}\) - сила, действующая на частицу. Данное уравнение по виду полностью совпадает с основным уравнением ньютоновской динамики (3.1). Однако физический смысл здесь уже другой: слева стоит производная по времени от релятивистского импульса, определяемого формулой (8.3). Подставив (8.3) в (8.4), получим


\begin{equation*}
\frac{\mathrm{d}}{\mathrm{~d} t}\left(\frac{m_{0} \mathbf{v}}{\sqrt{1-(v / c)^{2}}}\right)=\mathbf{F} \tag{8.5}
\end{equation*}


Это и есть основное уравнение релятивистской динамики.\\
В таком виде уравнение динамики приводит к сохранению импульса для свободной частицы и при малых скоростях ( \(v \ll c\) ) принимает форму основного уравнения ньютоновской динамики ( \(m \mathbf{a}=\mathbf{F}\) ).

Кроме того, именно в таком виде основное уравнение динамики оказывается инвариантным по отношению к преобразованиям Лоренца и, следовательно, удовлетворяет принципу относительности Эйнштейна. Не останавливаясь на способе доказательства этого, отметим только, что при переходе от одной инерциальной системы отсчета к другой необходимо принять, что сила \(\mathbf{F}\) преобразуется по определенным законам. Другими словами, сила \(\mathbf{F}\) в теории относительности - величина неинвариантная, в разных системах отсчета ее числовое значение и направление будут различны*.

Из основного уравнения релятивистской динамики следует неожиданный вывод: вектор ускорения а частицы в общем случае не совпадает по направлению с вектором силы \(\mathbf{F}\). Чтобы это показать, запишем (8.5) в такой форме:

\[
\mathrm{d}(m \mathbf{v}) / \mathrm{d} t=\mathbf{F},
\]

\footnotetext{\begin{itemize}
  \item В отличие от ньютоновской механики, где силы абсолютны, в теории относительности проекции силы, перпендикулярные направлению вектора относительной скорости систем отсчета, различны в разных системах. Эти проекции имеют максимальные значения в той системе отсчета, где частица в данный момент покоится:
\end{itemize}

\[
F_{x}^{\prime}=F_{x}, \quad F_{y}^{\prime}=F_{y} \sqrt{1-(v / c)^{2}} .
\]
}где \(m\) - релятивистская масса частицы. Выполнив дифференцирование по времени, получим


\begin{equation*}
(\mathrm{d} m / \mathrm{d} t) \mathbf{v}+m(\mathrm{~d} \mathbf{v} / \mathrm{d} t)=\mathbf{F} . \tag{8.6}
\end{equation*}


\begin{figure}[h]
\begin{center}
  \includegraphics[alt={},max width=\textwidth]{66a35da9-3faa-4823-9ca9-40799f37b3db-269_258_351_316_69}
\captionsetup{labelformat=empty}
\caption{Рис. 8.4}
\end{center}
\end{figure}

Это выражение графически представлено на рис. 8.4. Таким образом, действительно, вектор ускорения а в общем случае не коллинеарен вектору силы \(\mathbf{F}\).

Вектор ускорения а совпадает по направлению с вектором \(\mathbf{F}\) только в двух случаях:

\begin{enumerate}
  \item если \(\mathbf{F} \perp \mathbf{v}\) (поперечная сила). При этом вектор скорости \(\mathbf{v}\) по модулю не меняется, т. е. \(v=\) const, и уравнение (8.5) принимает вид
\end{enumerate}

\[
m_{0} \mathbf{a} / \sqrt{1-(v / c)^{2}}=\mathbf{F}
\]

откуда ускорение

\[
\mathbf{a}=\left(\mathbf{F} / m_{0}\right) \sqrt{1-(v / c)^{2}} ;
\]

\begin{enumerate}
  \setcounter{enumi}{1}
  \item если \(\mathbf{F} \| \mathbf{|}\) (продольная сила). В данном случае уравнение (8.5) можно записать в скалярном виде. Выполнив в его левой части дифференцирование по времени, получим
\end{enumerate}

\[
\left(\frac{m_{0}}{\sqrt{1-(v / c)^{2}}}+\frac{m_{0} v^{2} / c^{2}}{\left(1-(v / c)^{2}\right)^{3 / 2}}\right) \frac{\mathrm{d} v}{\mathrm{~d} t}=F
\]

откуда ускорение (в векторном виде) есть

\[
\mathbf{a}=\left(\mathbf{F} / m_{0}\right)\left(1-(v / c)^{2}\right)^{3 / 2} .
\]

Нетрудно заметить, что при одинаковых в обоих случаях значениях силы \(F\) и скорости \(v\) поперечная сила сообщает частице большее ускорение, чем продольная сила.

Основное уравнение релятивистской динамики позволяет найти закон действующей на частицу силы \(\mathbf{F}\), если известна зависимость от времени релятивистского импульса \(\mathbf{p}(t)\), а с другой стороны, найти уравнение движения частицы \(\mathbf{r}(t)\), если известны действующая сила и начальные условия - скорость \(\mathbf{v}_{0}\) и положение \(\mathbf{r}_{0}\) частицы в начальный момент времени.

В качестве примеров на применение уравнения (8.5) могут служить задачи 8.1-8.3.

\section*{§ 8.3. Закон взаимосвязи массы и энергии}
\section*{Кинетическая энергия релятивистской частицы}
Определим эту величину таким же путем, как и в ньютоновской механике, т. е. как величину, приращение которой равно работе действующей на частицу силы. Сначала найдем приращение кинетической энергии \(\mathrm{d} K\) частицы под действием силы \(\mathbf{F}\) на элементарном пути \(\mathrm{d} \mathbf{r}=\mathbf{v} \mathrm{d} t\) :

\[
\mathrm{d} K=\mathbf{F v} \mathrm{d} t .
\]

Согласно основному уравнению релятивистской динамики (8.4), \(\mathbf{F} \mathrm{d} t=\mathrm{d}(m \mathbf{v})=\mathrm{d} m \cdot \mathbf{v}+m \mathrm{~d} \mathbf{v}\), где \(m\) - релятивистская масса. Поэтому

\[
\mathrm{d} K=\mathbf{v}(\mathrm{d} m \cdot \mathbf{v}+m \mathrm{~d} \mathbf{v})=v^{2} \mathrm{~d} m+m v \mathrm{~d} v
\]

где \(\mathbf{v} \mathbf{d} \mathbf{v}=v \mathrm{~d} v\) (см. с. 112). Это выражение можно упростить, используя формулу (8.2). Возведем эту формулу в квадрат и приведем ее к виду

\[
m^{2} c^{2}=m^{2} v^{2}+m_{0}^{2} c^{2} .
\]

Найдем дифференциал этого выражения, имея в виду, что \(m\) и \(c\) - постоянные величины:

\[
2 m c^{2} \mathrm{~d} m=2 m v^{2} \mathrm{~d} m+2 m^{2} v \mathrm{~d} v .
\]

Если разделить это равенство на \(2 m\), то его правая часть совпадет с выражением для \(\mathrm{d} K\), отсюда следует


\begin{equation*}
\mathrm{d} K=c^{2} \mathrm{~d} m \tag{8.7}
\end{equation*}


Таким образом, приращение кинетической энергии частицы пропорционально приращению ее релятивистской массы. Кинетическая энергия покоящейся частицы равна нулю, а ее релятивистская масса \(m=m_{0}\). Поэтому, проинтегрировав (8.7), получим


\begin{equation*}
K=\left(m-m_{0}\right) c^{2}, \tag{8.8}
\end{equation*}


или


\begin{equation*}
K=m_{0} c^{2}\left(\frac{1}{\sqrt{1-\beta^{2}}}-1\right) \tag{8.9}
\end{equation*}


где \(\beta=v / c\). Это и есть выражение для релятивистской кинетической энергии частицы. Оно сильно отличается от ньютоновского \(m_{0} v^{2} / 2\). Убедимся, что при малых скоростях ( \(\beta \ll 1\) ) выражение (8.9) переходит в ньютоновское. Для этого воспользуемся формулой бинома Ньютона, согласно которой

\[
\frac{1}{\sqrt{1-\beta^{2}}}=\left(1-\beta^{2}\right)^{-1 / 2}=1+\frac{1}{2} \beta^{2}+\frac{3}{8} \beta^{4}+\ldots
\]

При \(\beta \ll 1\) можно ограничиться первыми двумя членами этого ряда и тогда

\begin{figure}[h]
\begin{center}
  \includegraphics[alt={},max width=\textwidth]{66a35da9-3faa-4823-9ca9-40799f37b3db-271_443_410_836_64}
\captionsetup{labelformat=empty}
\caption{Рис. 8.5}
\end{center}
\end{figure}

\[
K=m_{0} c^{2} \beta^{2} / 2=m_{0} v^{2} / 2 .
\]

Таким образом, при больших скоростях кинетическая энергия частицы определяется релятивистской формулой (8.9), отличной от \(m_{0} v^{2} / 2\). Заметим, что (8.9) нельзя представить и в виде \(m v^{2} / 2\), где \(m\) - релятивистская масса.

На рис. 8.5 показаны для сравнения графики зависимостей от \(\beta\) релятивистской \(K_{\text {рел }}\) и ньютоновской \(K_{\text {н }}\) кинетических энергий. Их различие особенно сильно проявляется в области скоростей, сравнимых со скоростью света.\\
Пример. Какую работу необходимо совершить, чтобы увеличить скорость частицы с массой \(m_{0}\) от 0,6 до \(0,8 c\) ? Сравним полученный результат со значением, вычисленным по нерелятивистской формуле.

Искомая работа в соответствии с формулой (8.9) равна

\[
A=K_{2}-K_{1}=m_{0} c^{2}\left(\frac{1}{\sqrt{1-\beta_{2}^{2}}}-\frac{1}{\sqrt{1-\beta_{1}^{2}}}\right)=0,42 m_{0} c^{2} .
\]

Соответствующая же работа, по нерелятивистской формуле,

\[
A=m_{0}\left(v_{2}^{2}-v_{1}^{2}\right) / 2=0,14 m_{0} c^{2} .
\]

Различие между обоими результатами весьма значительное.

\section*{Закон взаимосвязи массы и энергии}
Прежде всего перепишем полученное выше соотношение (8.8) в такой форме:


\begin{equation*}
m c^{2}=m_{0} c^{2}+K, \tag{8.8'}
\end{equation*}


где \(m\) - релятивистская масса частицы (тела). Мы уже знаем, что \(K\) - это кинетическая энергия частицы. Две остальные величины - тоже энергии. Но каков их физический смысл?

Глубокий анализ этого вопроса привел Эйнштейна к весьма важному выводу: \(m_{0} c^{2}\) - это общая внутренняя энергия тела, из каких бы видов она ни состояла (кинетическая, электрическая, химическая и др.). Эту энергию назвали энергией покоя \(E_{0}\) :


\begin{equation*}
E_{0}=m_{0} c^{2} . \tag{8.10}
\end{equation*}


Величину же \(m c^{2}\), равную сумме \(m_{0} c^{2}+K\), назвали полной энергией \(E\) тела (частицы):


\begin{equation*}
E=m c^{2}=m_{0} c^{2}+K . \tag{8.11}
\end{equation*}


Во избежание недоразумений обратим внимание на то, что в полную энергию \(E\) не включена потенциальная энергия тела во внешнем силовом поле, если таковое действует на тело.

Вернемся, однако, к соотношению (8.10). Как выяснилось, оно выражает один из наиболее фундаментальных законов природы - закон взаимосвязи (пропорциональности) массьι \(m_{0} u\) энергии покоя \(E_{0}\) тела.

Мы видели, что масса тела, которая в нерелятивистской механике выступала как мера инертности (во втором законе Ньютона) или как мера гравитационного действия (в законе всемирного тяготения), теперь выступает в новой функции - как мера энергосодержания тела.

Изменение энергии покоя тела сопровождается эквивалентным изменением его массы \(\Delta m_{0}=\Delta E_{0} / c^{2}\), и наоборот. При обычных макроскопических процессах изменение массы тел оказывается чрезвычайно малым, недоступным для измерений. Это можно проиллюстрировать на следующих примерах.\\
Пример 1. Пружину жесткости \(\varkappa=1,0 к \mathrm{H} / с м\) сжали на \(\Delta l=1,0 \mathrm{~cm}\). При этом пружина приобрела энергию \(U=x(\Delta l)^{2} / 2\). Эквивалентное приращение ее массы

\[
\Delta m_{0}=U / c^{2}=0,5 \cdot 10^{-16} \text { кг. }
\]

Пример 2. При нагревании 1 л воды от 0 до \(100^{\circ} \mathrm{C}\) ей сообщают энергию \(Q=m c_{p} \Delta t\), где \(c_{p}=4,2\) Дж/(г \(\cdot К\) ) - удельная теплоемкость воды, \(\Delta t\) - разность температур. Соответствующее увеличение массы воды

\[
\Delta m_{0}=Q / c^{2}=0,47 \cdot 10^{-10} \text { кг. }
\]

Обычно - и это нетрудно видеть из этих двух примеров изменение массы тела лежит далеко за пределами точности эксперимента. Однако уже в астрономических явлениях, связанных, например, с излучением звезд, изменение массы представляет собой весьма внушительную величину. В этом можно убедиться на примере излучения Солнца.\\
Пример. Из астрономических наблюдений установлено, что количество энергии, которое приносит на Землю солнечное излучение за 1 с на площадку \(1 м^{2}\), перпендикулярную солнечным лучам, составляет около 1,4 кДж/(с•m²). Это позволяет вычислить суммарную энергию, излучаемую Солнцем за 1 с:

\[
P=1,4 \cdot 10^{3} \cdot 4 \pi R^{2}=4 \cdot 10^{26} \text { Дж } / \mathrm{c},
\]

где \(R\) - расстояние от Земли до Солнца. Следовательно, Солнце ежесекундно теряет массу

\[
\Delta m_{0}=P / c^{2}=4,4 \cdot 10^{9} \text { кг } / \text { с. }
\]

Величина грандиозная с точки зрения земных масштабов, однако по сравнению с массой Солнца эта потеря ничтожно мала: \(\Delta m_{0} / m_{0}=2 \cdot 10^{-21} \mathrm{c}^{-1}\).\\
Совершенно иначе обстоит дело в ядерной физике. Именно здесь впервые оказалось возможным экспериментально прове-

рить и подтвердить закон взаимосвязи массы и энергии. Это обусловлено тем, что ядерные процессы и процессы взаимного превращения элементарных частиц сопровождаются весьма большими изменениями энергии, сравнимыми с энергией покоя самих частиц. Но к этому вопросу мы еще вернемся в § 8.5.

\section*{§ 8.4. Связь между энергией и импульсом частицы}
Ясно, что полная энергия \(E\) и импульс \(p\) частицы имеют разные значения в разных системах отсчета. Оказывается, однако, что существует величина - некоторая комбинация \(E\) и \(p\), которая является инвариантной, т. е. имеет одно и то же значение в разных системах отсчета. Эта величина есть \(E^{2}-p^{2} c^{2}\). Убедимся, что это так.

Воспользовавшись формулами \(E=m c^{2}\) и \(p=m v\), запишем

\[
E^{2}-p^{2} c^{2}=m^{2} c^{2}-m^{2} v^{2} c^{2}=\frac{m_{0}^{2} c^{4}}{1-(v / c)^{2}}\left[1-(v / c)^{2}\right]
\]

или после сокращения


\begin{equation*}
E^{2}-p^{2} c^{2}=m_{0}^{2} c^{4} \tag{8.12}
\end{equation*}


Тот факт, что скорость \(v\) в правой части сократилась, означает независимость величины \(E^{2}-p^{2} c^{2}\) от скорости частицы, а следовательно, и от системы отсчета. Другими словами, величина \(E^{2}-p^{2} c^{2}\) действительно является инвариантом и имеет одно и то же значение \(m_{0}^{2} c^{4}\) во всех инерциальных системах отсчета:


\begin{equation*}
E^{2}-p^{2} c^{2}=\text { inv. } \tag{8.13}
\end{equation*}


Этот вывод чрезвычайно важен: он позволяет, как будет видно из дальнейшего, во многих случаях резко упростить анализ и решение различных вопросов.

Приведем еще два полезных соотношения, с которыми приходится часто встречаться. Первое:


\begin{equation*}
\mathbf{p}=m \mathbf{v}=E \mathbf{v} / c^{2} \tag{8.14}
\end{equation*}


второе - связь между импульсом и кинетической энергией \(K\) частицы; его легко получить, подставив в (8.12) \(E=m_{0} c^{2}+K\), тогда


\begin{equation*}
p c=\sqrt{K\left(K+2 m_{0} c^{2}\right)} . \tag{8.15}
\end{equation*}


Последнее соотношение при \(K \ll m_{0} c^{2}\) переходит в ньютоновское: \(p=\sqrt{2 m_{0} K}\), а при \(K \gg m_{0} c^{2}\) приобретает вид \(p=K / c\).\\
Пример. Считая, что энергия покоя электрона равна 0,51 МэВ, вычис-\\
лим:

\begin{enumerate}
  \item импульс* электрона с кинетической энергией, равной его энергии покоя;
  \item кинетическую энергию электрона с импульсом \(0,51 \mathrm{M}\) В \(/ c\), где \(c\) - скорость света.
\end{enumerate}

\begin{enumerate}
  \item Согласно (8.15), при \(K=m_{0} c^{2}\) получим \(p=\sqrt{3} m_{0} c= =0,9 \mathrm{M} \ni \mathrm{B} / c\).
  \item Этот вопрос можно решить также с помощью (8.15). Но можно и проще, воспользовавшись (8.12):
\end{enumerate}

\[
K=E-m_{0} c^{2}=\sqrt{p^{2} c^{2}+m_{0}^{2} c^{4}}-m_{0} c^{2}=0,21 \text { МэВ. }
\]

Рассмотрим весьма интересный вопрос о возможности существования частиц с нулевой массой покоя ( \(m_{0}=0\) ). Из формул

\[
E=\frac{m_{0} c^{2}}{\sqrt{1-(v / c)^{2}}}, \quad p=\frac{m_{0} v}{\sqrt{1-(v / c)^{2}}}
\]

следует, что частица с массой \(m_{0}=0\) может иметь энергию и импульс в том и только в том случае, если она движется со скоростью света \(c\). При этом обе последние формулы принимают вид \(0 / 0\). Однако это не означает неопределенности энергии и импульса такой частицы. Дело в том, что обе эти величины, оказывается, не зависят от скорости, причем связь между импульсом \(p\) и энергией \(E\) дается формулой (8.14), где \(v=c\), т. е.


\begin{equation*}
p=E / c \tag{8.16}
\end{equation*}


\footnotetext{\begin{itemize}
  \item Заметим, что в настоящее время импульсы релятивистских частиц выражают в единицах энергия \(/ c\) ( \(c\) - скорость света). Например, если энергия выражается в МэВ ( \(1 \mathrm{M} \ni \mathrm{B}=1,6 \cdot 10^{-13}\) Дж), то импульс - в МэВ \(/ c\). Использование такой единицы импульса заметно упрощает многие расчеты.
\end{itemize}
}Таким образом, согласно теории относительности, существование частиц с нулевой массой возможно, причем эти частицы могут двигаться только со скоростью \(c\). Это движение не есть результат предшествующего ускорения, а вообще единственное состояние, в котором такие частицы могут существовать. Остановка подобной частицы равносильна ее поглощению (исчезновению). Как сейчас известно, такими частицами являются фотон и, по-видимому, нейтрино.

\section*{Преобразования импульса и энергии}
Пусть частица движется со скоростью \(v=\mathrm{d} l / \mathrm{d} t\) в \(K\)-системе отсчета. Из формулы (7.13) следует, что элементарный интервал между двумя событиями, которые происходят с частицей, есть

\[
\mathrm{d} s=\sqrt{c^{2}(\mathrm{~d} t)^{2}-(\mathrm{d} l)^{2}}=c \mathrm{~d} t \sqrt{1-(v / c)^{2}}
\]

Имея в виду это выражение, представим проекции импульса и энергию частицы в следующем виде

\[
\begin{gathered}
p_{x}=\frac{m_{0}}{\sqrt{1-(v / c)^{2}}} \frac{c \mathrm{~d} x}{c \mathrm{~d} t}=m_{0} c \frac{\mathrm{~d} x}{\mathrm{~d} s} ; \quad p_{y}=m_{0} c \frac{\mathrm{~d} y}{\mathrm{~d} s} ; \\
E=\frac{m_{0} c^{2}}{\sqrt{1-(v / c)^{2}}} \frac{c \mathrm{~d} t}{c \mathrm{~d} t}=m_{0} c \frac{c^{2} \mathrm{~d} t}{\mathrm{~d} s} .
\end{gathered}
\]

Из инвариантности \(m_{0}, c\) и интервала \(\mathrm{d} s\) сразу следует, что при переходе к другой инерциальной системе отсчета \(p_{x}\) и \(p_{y}\) преобразуются подобно \(\mathrm{d} x\) и \(\mathrm{d} y\), т. е. подобно координатам \(x\) и \(y\), а энергия \(E\) - подобно времени \(t\).

Поскольку координаты и время входят в преобразования Лоренца (7.8) линейно, мы выделили в предыдущих выражениях для \(p\) и \(E\) одинаковую часть ( \(m_{0} c\) ). Тогда можно сделать следующее сопоставление:

\[
p_{x} \leftrightarrow x, \quad p_{y} \leftrightarrow y, \quad E / c^{2} \leftrightarrow t .
\]

Делая эту замену в преобразованиях Лоренца (7.8), получим сразу искомые преобразования импульса и энергии:


\begin{equation*}
p_{x}^{\prime}=\frac{p_{x}-E V / c^{2}}{\sqrt{1-(V / c)^{2}}}, \quad p_{y}^{\prime}=p_{y}, \quad E^{\prime}=\frac{E-p_{x} V}{\sqrt{1-(V / c)^{2}}}, \tag{8.17}
\end{equation*}


где \(V\) - скорость \(K^{\prime}\)-системы относительно \(K\)-системы.\\
Эти формулы выражают закон преобразования проекций импульса и энергии частицы при переходе от \(K\) - \(\kappa K^{\prime}\)-системе.

\section*{Запись формул в более компактном виде}
В настоящее время все формулы релятивистской механики принято записывать в более компактном виде с помощью использования следующих сокращенных обозначений:

\begin{enumerate}
  \item величины \(m c^{2}\) и \(p c\) обозначают просто \(m\) и \(p\), их соответственно выражают в энергетических единицах (например, в МэВ);
  \item все скорости выражают в единицах скорости света и обозначают \(\beta\) :
\end{enumerate}


\begin{equation*}
\beta=v / c ; \tag{8.18}
\end{equation*}


\begin{enumerate}
  \setcounter{enumi}{2}
  \item часто встречающийся множитель \(1 / \sqrt{1-\beta^{2}}\) обозначают \(\gamma\) - ло-рени-фактор:
\end{enumerate}


\begin{equation*}
\gamma=1 / \sqrt{1-\beta^{2}} . \tag{8.19}
\end{equation*}


Эти обозначения резко упрощают как вид самих формул, так и все преобразования и расчеты. Приведем основные формулы релятивистской динамики в этих обозначениях:

релятивистский импульс (8.3)


\begin{equation*}
\mathbf{p}=\frac{m_{0} \boldsymbol{\beta}}{\sqrt{1-\beta^{2}}}=\gamma m_{0} \boldsymbol{\beta} ; \tag{8.20}
\end{equation*}


кинетическая (8.9) и полная (8.11) энергии:


\begin{gather*}
K=m_{0}\left(\frac{1}{\sqrt{1-\beta^{2}}}-1\right)=m_{0}(\gamma-1),  \tag{8.21}\\
E=m_{0}+K=m=\gamma m_{0} ; \tag{8.22}
\end{gather*}


связь между энергией и импульсом (8.12)-(8.15):


\begin{gather*}
E^{2}-p^{2}=m_{0}^{2}=\operatorname{inv}  \tag{8.23}\\
\mathbf{p}=E \boldsymbol{\beta}  \tag{8.24}\\
p=\sqrt{K\left(K+2 m_{0}\right)} \tag{8.25}
\end{gather*}


преобразования импульса и энергии (8.17):


\begin{align*}
& p_{x}^{\prime}=\frac{p_{x}-\beta E}{\sqrt{1-\beta^{2}}}=\gamma\left(p_{x}-\beta E\right) \\
& E^{\prime}=\frac{E-\beta p_{x}}{\sqrt{1-\beta^{2}}}=\gamma\left(E-\beta p_{x}\right) \tag{8.26}
\end{align*}


\section*{§ 8.5. Система релятивистских частиц}
\section*{Об энергии и импульсе системы}
До сих пор мы ограничивались рассмотрением поведения одной частицы. В отличие от динамики одной частицы построение динамики системы частиц в теории относительности является гораздо более сложной задачей. Тем не менее и в этом случае можно установить ряд важных общих законов.

Если нас интересует движение системы как целого, то, отвлекаясь от внутренних процессов в системе и пренебрегая ее пространственной протяженностью, систему можно считать одной материальной точкой - частицей. Поскольку это так, систему релятивистских частиц как целое можно характеризовать полной энергией \(E\), импульсом \(\mathbf{p}\), массой покоя \(M_{0}\) и утверждать, что полученные ранее выражения справедливы и для системы частиц как целого.

Остается выяснить, что следует понимать под полной энергией \(E\), импульсом \(\mathbf{p}\), массой покоя \(M_{0}\) системы как целого. В общем случае, если система состоит из взаимодействующих релятивистских частиц, ее полная энергия


\begin{equation*}
E=\sum m_{i} c^{2}+W, \tag{8.27}
\end{equation*}


где \(m_{i} c^{2}\) - полная энергия \(i\)-й частицы (напомним, что в эту величину не включается энергия взаимодействия с другими частицами); \(W\) - суммарная энергия взаимодействия всех частиц системы.

В ньютоновской механике \(W\) представляет собой потенциальную энергию взаимодействия частиц системы - величину, зависящую при данном характере взаимодействий только

от конфигурации системы. В релятивистской же динамике, оказывается, не существует понятия потенциальной энергии взаимодействия частиц. Это обусловлено тем обстоятельством, что само понятие потенциальной энергии тесно связано с представлением о дальнодействии (мгновенной передаче взаимодействий). Являясь функцией конфигурации системы, потенциальная энергия в каждый момент времени определяется относительным расположением частиц системы в этот момент. Изменение конфигурации системы должно мгновенно вызвать изменение и потенциальной энергии. Так как в действительности этого нет (взаимодействия передаются с конечной скоростью), то для системы релятивистских частиц понятие потенциальной энергии взаимодействия не может быть введено.

В общем случае написать выражение для энергии взаимодействия \(W\), а следовательно, и для полной энергии \(E\) системы взаимодействующих релятивистских частиц не представляется возможным. Это же относится и к импульсу системы, так как в релятивистской динамике импульс не является величиной, независимой от энергии \(E\). Так же сложно обстоит дело и с массой \(M_{0}\) системы, о которой в общем случае можно сказать только одно: это масса в системе отсчета, где данная механическая система как целое покоится (т. е. в Ц-системе).

Вследствие указанных трудностей построение динамики системы релятивистских частиц ограничено сравнительно немногими простейшими случаями, на двух из которых мы и остановимся. Это система из невзаимодействующих релятивистских частиц и важный в практическом отношении случай столкновения двух частиц.

\section*{Система невзаимодействующих частиц}
В данном случае полная энергия \(E\) и импульс p обладают аддитивными свойствами и для системы их можно представить в виде


\begin{equation*}
E=\sum m_{i} c^{2}, \quad \mathbf{p}=\sum \mathbf{p}_{i}, \tag{8.28}
\end{equation*}


где \(m_{i}\) и \(\mathbf{p}_{i}\) - релятивистская масса и импульс \(i\)-й частицы системы. Так как взаимодействий в данном случае нет, то скоро-

сти всех частиц постоянны, а следовательно, постоянны во времени полная энергия и импульс всей системы.

Введем понятие энергии покоя \(E_{0}\) системы частиц как полную энергию ее в Ц-системе, где суммарный импульс \(\tilde{\mathbf{p}}=\sum \tilde{\mathbf{p}}_{i}=0\) и система как целое покоится. Таким образом,


\begin{equation*}
E_{0}=\sum \tilde{E}_{i}, \tag{8.29}
\end{equation*}


где \(\tilde{E}_{i}\) - полная энергия \(i\)-й частицы в Ц-системе. Это значит, что в энергию покоя входит кроме энергии покоя каждой частицы и их кинетическая энергия \(\tilde{K}_{i}\) в Ц-системе:

\[
\tilde{E}_{i}=m_{0 i} c^{2}+\tilde{K}_{i} .
\]

Это же относится, очевидно, и к массе покоя системы:


\begin{equation*}
M_{0}=E_{0} / c^{2} . \tag{8.30}
\end{equation*}


Отсюда, в частности, следует, что масса покоя системы не равна сумме масс покоя отдельных частиц, а именно:

\[
M_{0}>\sum m_{0 i} .
\]

Введение энергии и массы покоя системы ( \(E_{0}\) и \(M_{0}\) ) позволяет рассматривать систему невзаимодействующих релятивистских частиц как одну частицу с полной энергией \(E=\sum m_{i} c^{2}\), импульсом \(\mathbf{p}=\sum \mathbf{p}_{i}\), массой покоя \(M_{0}=E_{0} / c^{2}\) и утверждать, что выражения (8.12) и (8.14) справедливы и для системы частиц:


\begin{gather*}
E^{2}-p^{2} c^{2}=M_{0}^{2} c^{4}=\operatorname{inv}  \tag{8.31}\\
\mathbf{p}=E \mathbf{V} / c^{2} \tag{8.32}
\end{gather*}


где \(\mathbf{V}\) - скорость системы частиц как целого, т. е. скорость Ц-системы. Эту скорость, согласно (8.32), можно представить в таком виде:


\begin{equation*}
\mathbf{V}=\frac{\sum \mathbf{p}_{i}}{\sum m_{i}}, \tag{8.33}
\end{equation*}


где \(m_{i}\) - релятивистская масса \(i\)-й частицы системы. Заметим, что (8.33) по форме совпадает с соответствующим нерелятивистским выражением (3.9) для скорости центра масс системы.

\section*{Столкновение двух частиц}
Рассмотрим процесс столкновения происходящим в два этапа: сначала образование некоторой составной частицы \(A^{*}\) и затем ее распад на какие-то в общем случае другие частицы:

\[
A_{1}+A_{2} \rightarrow A^{*} \rightarrow A_{3}+A_{4}+\ldots .
\]

В процессе сближения частиц \(A_{1}\) и \(A_{2}\) взаимодействие между ними может становиться не малым, и формулы (8.28) теряют свою применимость. Однако после того, как возникшие частицы разойдутся на большое расстояние друг от друга, эти формулы опять применимы.

В данном случае можно показать, что сумма полных энергий двух исходных частиц (когда они находятся настолько далеко друг от друга, что их взаимодействие пренебрежимо мало) равна полной энергии составной частицы. Это же относится и ко второй стадии процесса - распаду. Другими словами, можно показать, что для этого процесса оказывается справедливым закон сохранения полной энергии в таком виде:


\begin{equation*}
E_{1}+E_{2}=E^{*}=E_{3}+E_{4}+\ldots . \tag{8.34}
\end{equation*}


Убедимся, что это именно так, на следующем простом примере.

Представим себе столкновение двух одинаковых частиц 1 и 2 ,

\begin{figure}[h]
\begin{center}
  \includegraphics[alt={},max width=\textwidth]{66a35da9-3faa-4823-9ca9-40799f37b3db-281_370_378_1387_68}
\captionsetup{labelformat=empty}
\caption{Рис. 8.6}
\end{center}
\end{figure}

в результате которого образуется некоторая составная частица. Пусть частицы до столкновения движутся навстречу друг другу в \(K\)-системе с одинаковыми скоростями \(v\), как показано на рис. 8.6. Рассмотрим теперь этот процесс в \(K^{\prime}\)-системе, движущейся влево со скоростью \(\mathbf{V}\) относительно \(K\)-системы. Так как в \(K\)-системе скорость каждой частицы перпендикулярна вектору V, то, согласно\\
(7.14), обе частицы в \(K^{\prime}\)-системе имеют \(x\)-компоненту скорости, равную \(V\). Такую же скорость в \(K^{\prime}\)-системе будет иметь и образовавшаяся частица, релятивистскую массу которой обозначим \(m^{*}\). Из закона сохранения импульса до и после столкновения получим для \(x\)-составляющей импульса \(2 m\left(v^{\prime}\right) V=m^{*} V\), где \(v^{\prime}\) - скорость каждой частицы в \(K^{\prime}\)-системе. Отсюда

\[
2 m\left(v^{\prime}\right)=m^{*},
\]

т. е. сумма релятивистских масс исходных частиц равна релятивистской массе образовавшейся частицы. Аналогично дело обстоит и в \(K\)-системе. Действительно, при очень малом значении скорости \(V\) скорость \(v^{\prime}\) практически равна \(v\), а релятивистская масса \(m^{*}\) - массе покоя \(m_{0}^{*}\) образовавшейся частицы, так что в \(K\)-системе

\[
2 m(v)=m_{0}^{*} .
\]

Отсюда видно, что масса покоя образовавшейся частицы больше суммы масс покоя исходных частиц. Кинетическая энергия исходных частиц претерпела превращение, в результате которого масса образовавшейся частицы превысила суму масс исходных частиц.

Итак, мы показали, что вследствие сохранения импульса системы сумма релятивистских масс исходных частиц равна релятивистской массе образовавшейся частицы. Это же, очевидно, относится и к полной энергии. Поэтому можно утверждать, что сохранение полной энергии в форме (8.34) действительно имеет место для рассматриваемых стадий этого процесса.

Применение закона сохранения энергии к ядерным процессам позволило, как уже говорилось в § 8.3, экспериментально проверить справедливость одного из фундаментальных законов теории относительности - закона взаимосвязи массы и энергии. Рассмотрим примеры.

Пример 1. Энергетический выход ядерных реакций. Возьмем ядерную реакцию типа

\[
A_{1}+A_{2} \rightarrow A_{3}+A_{4},
\]

где слева - исходные ядра, справа - ядра - продукты реакции. Применим к этой реакции закон сохранения полной энергии:

\[
E_{1}+E_{2}=E_{3}+E_{4} .
\]

Имея в виду, что полная энергия каждой частицы может быть представлена как \(E=m_{0} c^{2}+K\), где \(m_{0}\) - масса покоя соответствующего ядра, \(K\) - его кинетическая энергия, перепишем предыдущее равенство так:

\[
\left(m_{1}+m_{2}\right) c^{2}+K_{12}=\left(m_{3}+m_{4}\right) c^{2}+K_{34},
\]

где \(K_{12}\) и \(K_{34}\) - суммарные кинетические энергии ядер до и после реакции. Отсюда

\[
K_{34}-K_{12}=\left(m_{1}+m_{2}\right) c^{2}-\left(m_{3}+m_{4}\right) c^{2} .
\]

Левая часть этого равенства есть приращение суммарной кинетической энергии ядер данной системы - то, что называют энергетическим выходом ядерной реакции и обозначают \(Q\). Итак,

\[
Q=\left[\left(m_{1}+m_{2}\right)-\left(m_{3}+m_{4}\right)\right] c^{2} .
\]

Эта величина может иметь любой знак - в зависимости от характера той или иной ядерной реакции. Таким образом, энергетический выход ядерной реакции определяется разностью суммарных масс покоя ядер до и после реакции. Все величины, входящие в это соотношение, могут быть экспериментально измерены с достаточно высокой точностью, тем самым можно проверить и само равенство.\\
Рассмотрим конкретную ядерную реакцию:

\[
{ }^{7} \mathrm{Li}+{ }^{1} \mathrm{H} \rightarrow 2{ }^{4} \mathrm{He} \text {. }
\]

Измеренные массы покоя этих ядер (в атомных единицах массы а.е. м.) равны соответственно \(7,0160,1,0078\) и 4,0024 а. е. м. Отсюда нетрудно подсчитать, что сумма масс покоя ядер в результате ядерной реакции уменьшилась на 0,019 а. е. м. Учитывая, что 1 а. е. м. соответствует энергия 931,4 МэВ, найдем \(Q=0,019 \cdot 931,4\) МэВ \(=17,7\) МэВ. Этот результат с большой точностью совпадает с данными эксперимента.

Пример 2. Распад частицы. Пусть покоящаяся частица \(A_{1}\) самопроизвольно распадается на частицы \(A_{2}\) и \(A_{3}\). Согласно закону сохранения полной энергии,

\[
E_{1}=E_{2}+E_{3} .
\]

Так как полная энергия каждой частицы \(E=m_{0} c^{2}+K\), то

\[
m_{1} c^{2}=\left(m_{2}+m_{3}\right) c^{2}+K_{23},
\]

где \(K_{23}\) - суммарная кинетическая энергия образовавшихся частиц. Эту энергию называют энергией распада \(Q\). Таким образом,

\[
\boldsymbol{Q}=\left[m_{1}-\left(m_{2}+m_{3}\right)\right] c^{2} .
\]

Поскольку \(Q\) - величина существенно положительная, самопроизвольный распад частицы возможен только при условии

\[
m_{1}>m_{2}+m_{3},
\]

т. е. если масса покоя первичной частицы больше суммы масс возникающих частиц. В противном случае самопроизвольный распад невозможен. Эксперимент полностью подтверждает этот вывод.

Рассмотрим, например, распад \(\pi\)-мезона. Экспериментально установлено, что заряженные \(\pi\)-мезоны распадаются на мюон и нейтрино: \(\pi \rightarrow \mu+\nu\). Согласно табличным данным, массы покоя этих частиц (в единицах массы электрона) равны соответственно \(273,2,206,8\) и 0 . Отсюда следует, что масса покоя в результате распада уменьшается на 66,4 электронной массы. Так как массе покоя электрона соответствует энергия \(0,51 \mathrm{M} \ni \mathrm{B}\), то энергия данного распада \(Q=66,4 \cdot 0,51 \mathrm{M} \ni \mathrm{B}=34 \mathrm{M} \ni \mathrm{B}\), что находится в точном соответствии с результатами эксперимента.

Тот факт, что в результате столкновения частиц и последующего затем распада составной частицы полная энергия системы (а значит, и ее импульс) не меняется, приводит к другому важному выводу: величина \(E^{2}-p^{2} c^{2}\) для системы будет инвариантной не только по отношению к разным инерциальным системам отсчета, но и для указанных выше стадий процесса столкновения.

Пусть, например, две релятивистские частицы испытали столкновение, в результате которого образовалась новая частица с массой покоя \(M_{0}\). Если в \(K\)-системе отсчета полные энергии частиц до столкновения равны \(E_{1}\) и \(E_{2}\), а их импульсы соответственно \(\mathbf{p}_{1}\) и \(\mathbf{p}_{2}\), то мы сразу можем записать, что при переходе от \(K\)-системы (до столкновения) к Ц-системе (после столкновения) будет выполняться следующее равенство:


\begin{equation*}
\underbrace{\left(E_{1}+E_{2}\right)^{2}-\left(\mathbf{p}_{1}+\mathbf{p}_{2}\right)^{2}}_{K \text {-система }}=\underbrace{M_{0}^{2} c^{4}}_{\text {Ц-система }}, \tag{8.35}
\end{equation*}


где учтено, что в Ц-системе образовавшаяся частица покоится.\\
Инвариантность величины \(E^{2}-p^{2} c^{2}\) дает нам незаменимый инструмент при изучении различных процессов распада и столкновения релятивистских частиц, с помощью которого чрезвычайно упрощается как анализ самих процессов, так и соответствующие расчеты.

Пример. В \(K\)-системе отсчета частица массы \(m_{0}\) с кинетической энергией \(K\) налетает на другую, покоящуюся, частицу той же массы. Найдем массу \(M_{0}\) и скорость \(V\) составной частицы, образовавшейся в результате столкновения.\\
Воспользовавшись инвариантностью величины \(E^{2}-p^{2} c^{2}\), запишем

\[
E^{2}-p^{2} c^{2}=M_{0}^{2} c^{4}
\]

где левая часть равенства относится к \(K\)-системе отсчета (до столкновения), а правая - к Ц-системе (после столкновения). В данном случае \(E=K+2 m_{0} c^{2}\). Кроме того, согласно (8.15), \(p^{2} c^{2}=K\left(K+2 m_{0} c^{2}\right)\). Поэтому

\[
\left(K+2 m_{0} c^{2}\right)^{2}-K\left(K+2 m_{0} c^{2}\right)=M_{0}^{2} c^{4}
\]

Отсюда

\[
M_{0}=\frac{1}{c} \sqrt{2 m_{0}\left(K+2 m_{0} c^{2}\right)}
\]

Скорость образовавшейся частицы - это скорость Ц-системы. Согласно (8.32),

\[
V=p c^{2} / E=c \sqrt{K /\left(K+2 m_{0} c^{2}\right)} .
\]

\section*{Задачи}
Внимание! В задачах 8.4-8.11 использованы сокращенные обозначения, приведенные в конце § 8.4 (например, \(p\) и \(m\) - это сокращенные записи величин \(p c\) и \(m c^{2}\) ).\\
8.1. Движение под действием продольной силы. Частица массы \(m\) начала двигаться под действием постоянной силы F. Найти зависимость скорости частицы от времени.\\
Решение. Умножим обе части уравнения (8.5) на \(\mathrm{d} t\), тогда

\[
\mathrm{d}\left(\frac{m_{0} v}{\sqrt{1-(v / c)^{2}}}\right)=F \mathrm{~d} t .
\]

Проинтегрировав это выражение с учетом того, что в начальный момент \(v=0\), получим \(m_{0} v / \sqrt{1-(v / c)^{2}}=F t\). Отсюда

\[
v(t)=\frac{F t / m_{0}}{\sqrt{1+\left(F t / m_{0} c\right)^{2}}} .
\]

Сравним полученное выражение с ньютоновским. Согласно второму закону Ньютона, \(a=F / m\) и скорость \(v_{\mathrm{H}}=F t / m_{0}\). Поэтому предыдущее выражение для скорости \(v(t)\) можно представить так:

\[
v(t)=\frac{v_{\mathrm{H}}}{\sqrt{\left(1+v_{\mathrm{H}} / c\right)^{2}}}
\]

Отсюда видно, что \(v<v_{\text {н }}\), т. е. действительная скорость \(v\) частицы растет со временем медленнее, чем \(v_{\mathrm{H}}\), причем при \(t \rightarrow \infty\) скорость \(v \rightarrow c\) (рис. 8.7).\\
Интересно, что импульс частицы при этом будет расти линейно со временем: из уравнения \(\mathrm{d} \mathbf{p} / \mathrm{d} t=\mathbf{F}\) следует, что \(p=F t\). В этом характерная особенность релятивистского движения: в то время как ско-

\begin{figure}[h]
\begin{center}
  \includegraphics[alt={},max width=\textwidth]{66a35da9-3faa-4823-9ca9-40799f37b3db-286_381_461_1202_701}
\captionsetup{labelformat=empty}
\caption{Рис. 8.7}
\end{center}
\end{figure}

8.2. Движение под действием поперечной силы. Релятивистская частица массы \(m_{0}\) с зарядом \(q\) движется в постоянном однородном магнитном поле, индукция которого В. Движение происходит по окружности радиуса \(\rho\) в плоскости, перпендикулярной вектору \(\mathbf{B}\). Найти импульс и круговую частоту обращения частицы по окружности.\\
Решение. В данном случае частица движется под действием сильл Лорениа \(\mathbf{F}=q[\mathbf{v B}]\), где \(\mathbf{v}\) - скорость частицы. Так как \(\mathbf{F} \perp \mathbf{v}\), то модуль скорости частицы \(v=\) const и уравнение (8.5) принимает вид

\[
m \mathbf{a}=q[\mathbf{v B}],
\]

где \(m\) - релятивистская масса частицы. Имея в виду, что а представляет собой нормальное ускорение, равное по модулю \(v^{2} / \rho\), перепишем предыдущее уравнение так: \(m v^{2} / \rho=q v B\). Отсюда импульс частицы


\begin{equation*}
p=m v=q \rho B . \tag{1}
\end{equation*}


Видно, что произведение \(\rho B\) может служить мерой релятивистского импульса частицы.

Период обращения частицы по окружности \(T=2 \pi \rho / v\), откуда круговая частота обращения \(\omega=2 \pi / T=v / \rho\). Учитывая (1), получим

\[
\omega=q B / m .
\]

Значит, круговая частота \(\omega\) зависит от скорости частицы: чем больше скорость частицы, а следовательно, и ее релятивистская масса \(m\), тем меньше частота \(\omega\). Однако при малых скоростях \((v \ll c) m \rightarrow m_{0}\) и

\[
\omega=q B / m_{0}=\mathrm{const},
\]

т. е. при нерелятивистских скоростях частота \(\omega\) практически не зависит от скорости частицы.\\
8.3. Релятивистский протон с импульсом \(\mathbf{p}_{0}\) влетел в момент \(t=0\) в область, где имеется поперечное однородное электрическое поле с напряженностью \(\mathbf{E}\), причем \(\mathbf{p}_{0} \perp \mathbf{E}\). Найти зависимость от времени угла \(\vartheta\), на который протон будет отклоняться от первоначального направления движения.

Решение. Выбрав оси координат ( \(X\) - вдоль вектора \(\mathbf{p}_{0}, Y\) вдоль вектора Е), запишем уравнение (8.4) в проекциях на эти оси:

\[
\mathrm{d} p_{x} / \mathrm{d} t=0, \quad \mathrm{~d} p_{y} / \mathrm{d} t=e E
\]

где \(e\) - заряд протона. Из этих уравнений следует, что \(p_{x}=p_{0}\), \(p_{y}=e E t\), или


\begin{equation*}
\frac{m_{0} v_{x}}{\sqrt{1-(v / c)^{2}}}=p_{0}, \quad \frac{m_{0} v_{y}}{\sqrt{1-(v / c)^{2}}}=e E t . \tag{1}
\end{equation*}


Взяв отношение последних двух равенств, найдем

\[
\operatorname{tg} \vartheta=v_{y} / v_{x}=e E t / p_{0} .
\]

Интересно отметить, что в отличие от нерелятивистского случая здесь \(v_{x}\) уменьшается со временем. Чтобы в этом убедиться, возведем оба равенства (1) в квадрат и затем сложим отдельно их левые и правые части:

\[
\frac{m_{0}^{2}\left(v_{x}^{2}+v_{y}^{2}\right)}{1-(v / c)^{2}}=p_{0}^{2}+(e E t)^{2} .
\]

Заметив, что \(v_{x}^{2}+v_{y}^{2}=v^{2}\), получим

\[
\left(\frac{v}{c}\right)^{2}=\left(1+\frac{m_{0}^{2} c^{2}}{p_{0}^{2}+(e E t)^{2}}\right)^{-1} .
\]

Подставив это выражение в первое из (1), найдем

\[
v_{x}=\frac{c}{\sqrt{1+\left(m_{0} c / p_{0}\right)^{2}+\left(e E t / p_{0}\right)^{2}}} \text {, }
\]

т. е. действительно, \(v_{x}\) уменьшается с ростом \(t\).\\
8.4. Симметричное упругое рассеяние. Релятивистский протон с кинетической энергией \(K\) испытал упругое столкновение с покоившимся протоном, в результате чего оба протона разлетелись симметрично относительно первоначального направления движения. Найти угол между направлениями разлета протонов после столкновения.

\begin{figure}[h]
\begin{center}
  \includegraphics[alt={},max width=\textwidth]{66a35da9-3faa-4823-9ca9-40799f37b3db-289_276_406_135_138}
\captionsetup{labelformat=empty}
\caption{Рис. 8.8}
\end{center}
\end{figure}

Решение. При симметричном разлете протонов их импульсы и энергии должны быть одинаковы (импульсы - по модулю). Это сразу видно из треугольника импульсов (рис. 8.8), выражающего закон сохранения импульса. Из этого треугольника, согласно теореме косинусов, следует, что \(p^{2}=2 p^{\prime 2}+2 p^{\prime 2} \cos \theta\), откуда

\[
\cos \theta=p^{2} / 2 p^{\prime 2}-1
\]

Воспользовавшись формулой (8.25) и учитывая, что \(K=2 K^{\prime}\), где \(K^{\prime}\) - кинетическая энергия каждого протона после столкновения, найдем

\[
\frac{p^{2}}{p^{\prime 2}}=\frac{K\left(K+2 m_{0}\right)}{K^{\prime}\left(K^{\prime}+2 m_{0}\right)}=4 \frac{K+2 m_{0}}{K+4 m_{0}} .
\]

Здесь \(m_{0}\) - масса протона. После подстановки этого выражения в формулу для \(\cos \theta\) получим

\[
\cos \theta=K /\left(K+4 m_{0}\right) .
\]

Заметим, что в отличие от нерелятивистского случая, когда \(\theta=90^{\circ}\), здесь \(\theta<90^{\circ}\).\\
8.5. Рассеяние фотона на электроне. Фотон с энергией \(\varepsilon\) испытал рассеяние на покоившемся свободном электроне. Найти энергию \(\varepsilon^{\prime}\) рассеянного фотона, если угол между направлениями движения рассеянного и налетающего фотонов равен Э.\\
Решение. Воспользуемся законами сохранения энергии и импульса. В данном процессе

\[
K_{e}=\varepsilon-\varepsilon^{\prime}, \quad \mathbf{p}_{e}=\mathbf{p}-\mathbf{p}^{\prime},
\]

\begin{figure}[h]
\begin{center}
  \includegraphics[alt={},max width=\textwidth]{66a35da9-3faa-4823-9ca9-40799f37b3db-289_178_399_1479_138}
\captionsetup{labelformat=empty}
\caption{Рис. 8.9}
\end{center}
\end{figure}

где \(K_{e}\) и \(\mathbf{p}_{e}\) - кинетическая энергия и импульс электрона отдачи, \(\mathbf{p}\) и \(\mathbf{p}^{\prime}\) импульсы налетающего и рассеянного фотонов. Из треугольника импульсов (рис. 8.9), согласно теореме косинусов, следует, что

\[
p_{e}^{2}=p^{2}+p^{\prime 2}-2 p p^{\prime} \cos \vartheta .
\]

Так как \(p=\varepsilon, p^{\prime}=\varepsilon^{\prime}\) и

\[
p_{e}=\sqrt{K_{e}\left(K_{e}+2 m_{e}\right)}=\sqrt{\left(\varepsilon-\varepsilon^{\prime}\right)\left(\varepsilon-\varepsilon^{\prime}+2 m_{e}\right)},
\]

где \(m_{e}\) - масса электрона, то после несложных преобразований получим

\[
\varepsilon^{\prime}=\frac{\varepsilon}{1+2\left(\varepsilon / m_{e}\right) \sin ^{2}(\vartheta / 2)}
\]

8.6. К методу встречных пучков. Два протона движутся навстречу друг другу с одинаковыми кинетическими энергиями \(K\) (в \(K\)-системе отсчета). Найти кинетическую энергию \(K^{\prime}\) одного протона в \(K^{\prime}\)-системе отсчета, где другой протон покоится.\\
Решение. Воспользуемся инвариантностью величины \(E^{2}-p^{2}\), записав ее в \(K\)-системе (она здесь является одновременно и \(L\)-системой), а также в \(K^{\prime}\)-системе:

\[
\left[2\left(K+m_{0}\right)\right]^{2}=\left(K^{\prime}+2 m_{0}\right)^{2}-K^{\prime}\left(K^{\prime}+2 m_{0}\right),
\]

где \(m_{0}\) - масса покоя протона. Отсюда

\[
K^{\prime}=2 K\left(K+2 m_{0}\right) / m_{0} .
\]

Например, для протонов ( \(m_{0} \approx 1\) ГэВ) при \(K=50\) ГэВ величина \(K^{\prime}=5 \cdot 10^{3}\) ГэВ. Возможность получения такого большого «выигрыша» в энергии лежит в основе метода встречных пучков.\\
8.7. Энергетическая схема ядерной реакции. Частица \(A_{1}\) с кинетической энергией \(K_{1}\) налетает на покоящееся ядро \(A_{2}\) (в \(K\)-системе отсчета). В результате реакции образуются ядра \(A_{3}\) и \(A_{4}\) :

\[
A_{1}+A_{2} \rightarrow A_{3}+A_{4}
\]

Массы частиц равны соответственно \(m_{1}, m_{2}, m_{3}, m_{4}\). Изобразить энергетическую схему ядерной реакции для случаев:\\
a) \(\left(m_{1}+m_{2}\right)>\left(m_{3}+m_{4}\right)\),\\
б) \(\left(m_{1}+m_{2}\right)<\left(m_{3}+m_{4}\right)\).

Найти для второго случая пороговую кинетическую энергию \(K_{1 \text { пор }}\) налетающей частицы в \(K\)-системе отсчета.\\
Решение. Из закона сохранения полной энергии следует, что в Ц-системе

\[
\tilde{K}_{12}+m_{1}+m_{2}=\tilde{K}_{34}+m_{3}+m_{4},
\]

\begin{figure}[h]
\begin{center}
  \includegraphics[alt={},max width=\textwidth]{66a35da9-3faa-4823-9ca9-40799f37b3db-291_358_785_132_223}
\captionsetup{labelformat=empty}
\caption{Рис. 8.10}
\end{center}
\end{figure}

где \(\tilde{K}_{12}\) и \(\tilde{K}_{34}\) - суммарные кинетические энергии частиц до и после реакции. Обозначив приращение кинетической энергии \(\tilde{K}_{34}-\tilde{K}_{12}\) через \(Q\), запишем предыдущее выражение так:

\[
Q=\left(m_{1}+m_{2}\right)-\left(m_{3}+m_{4}\right) .
\]

Величину \(Q\) называют энергетическим выходом ядерной реакции или, короче, энергией реакции.\\
Энергетическая схема ядерной реакции показана на рис. 8.10. В случае \(a\) эффект будет положительным, \(Q>0\) : суммарная кинетическая энергия увеличивается за счет уменьшения суммы масс покоя частиц системы. В случае б - наоборот.\\
В последнем случае, как видно из рис. 8.10, б, ядерная реакция возможна лишь при условии \(\tilde{K}_{12} \geqslant|Q|\), где знак равенства соответствует пороговому значению энергии \(\tilde{K}_{12}\). При нерелятивистских скоростях, согласно (4.16), \(\tilde{K}_{12}=\mu v_{\text {отн }}^{2} / 2\), или

\[
\tilde{K}_{12}=\frac{m_{2}}{m_{1}+m_{2}} \frac{m_{1} v_{1}^{2}}{2}=\frac{m_{2}}{m_{1}+m_{2}} K_{1} .
\]

Отсюда, имея в виду, что \(\tilde{K}_{12} \geqslant|Q|\) и \(K_{1}=K_{1 \text { пор }}\), получим

\[
K_{1 \text { пор }}=\frac{m_{1}+m_{2}}{m_{2}}|Q| .
\]

8.8. Пороговая энергия. Релятивистская частица массы \(m\) налетает на покоящуюся частицу массы \(M\). В результате столкновения возникают частицы с массами \(m_{1}, m_{2}\), ... по схеме

\[
m+M \rightarrow m_{1}+m_{2}+\ldots
\]

Найти пороговую (минимальную) кинетическую энергию \(K_{\text {пор }}\) налетающей частицы, необходимую для осуществления данного процесса.

Р е ш е н и е. Прежде всего ясно, что о пороговой энергии может идти речь только в том случае, когда сумма масс возникших частиц превышает сумму масс первичных частиц. Чтобы найти \(K_{\text {пор }}\), воспользуемся инвариантностью величины \(E^{2}-p^{2}\). Запишем эту величину до столкновения при \(K=K_{\text {пор }}\) в системе отсчета, где частица \(M\) покоилась, и после столкновения - в Ц-системе: \(E^{2}-p^{2}=\widetilde{E}^{2}\), или

\[
\left(K_{\text {пор }}+m+M\right)^{2}-K_{\text {пор }}\left(K_{\text {пор }}+2 m\right)=\left(m_{1}+m_{2}+\ldots\right)^{2} .
\]

Здесь учтено, что в Ц-системе кинетическая энергия возникших частиц равна нулю на пороге реакции, поэтому их полная энергия равна просто сумме масс отдельных частиц. Из последнего уравнения находим

\[
K_{\text {пор }}=\frac{\left(m_{1}+m_{2}+\ldots\right)^{2}-(m+M)^{2}}{2 M} .
\]

8.9. Найти пороговую энергию фотона для рождения пары электрон позитрон в поле покоящегося протона, если массы покоя электрона и позитрона равны \(m_{0}\), а протона \(-M_{0}\).\\
Решение. Воспользуемся инвариантностью величины \(E^{2}-p^{2}\) и запишем ее до взаимодействия в системе отсчета, где протон покоится, а после взаимодействия - в Ц-системе. При пороговом значении энергии \(\varepsilon\) налетающего фотона

\[
\left(\varepsilon_{\text {пор }}+M_{0}\right)^{2}-\varepsilon_{\text {пор }}^{2}=\left(M_{0}+2 m_{0}\right)^{2} .
\]

Отсюда

\[
\varepsilon_{\text {пор }}=2 m_{0}\left(1+m_{0} / M_{0}\right) .
\]

Видно, что для рождения пары необходимо, чтобы энергия фотона была больше \(2 m_{0}\) (этого требует закон сохранения импульса).\\
8.10. Энергия частиц в Ц-системе. Фотон с энергией ε в лабораторной системе отсчета налетает на неподвижную частицу \(A\), масса которой равна \(m_{0}\). Найти: 1) скорость Ц-системы этих двух частиц; 2) энергию фотона и частицы \(A\) в данной Ц-системе.

Решение. 1. Согласно формуле (8.32), скорость Ц-системы

\[
\beta_{C}=p / E=\varepsilon /\left(m_{0}+\varepsilon\right) .
\]

\begin{enumerate}
  \setcounter{enumi}{1}
  \item Из преобразования (8.26) для энергии следует, что в Ц-системе энергия фотона
\end{enumerate}

\[
\tilde{\varepsilon}=\left(\varepsilon-\beta_{C} p\right) / \sqrt{1-\beta_{C}^{2}},
\]

где \(\beta_{C}\) - скорость Ц-системы (в единицах \(c\) ). Подставив сюда \(p=\varepsilon\) и выражение для \(\beta_{C}\) из предыдущего пункта, получим

\[
\tilde{\varepsilon}=\frac{\varepsilon}{\sqrt{1+2 \varepsilon / m_{0}}}
\]

Частица \(A\) движется в Ц-системе со скоростью \(\beta=\beta_{C}\), поэтому ее полная энергия в Ц-системе

\[
\tilde{E}_{A}=\frac{m_{0}}{\sqrt{1-\beta_{C}^{2}}}=\frac{m_{0}+\varepsilon}{\sqrt{1+2 \varepsilon / m_{0}}} .
\]

В правильности полученных формул можно убедиться, воспользовавшись инвариантностью величины \(E^{2}-p^{2}\) при переходе от лабораторной к Ц-системе отсчета:

\[
\left(\varepsilon+m_{0}\right)^{2}-\varepsilon^{2}=\left(2 \tilde{\varepsilon}+\tilde{E}_{A}\right)^{2}
\]

8.11. Распад движущейся частицы. Релятивистский \(\pi\)-мезон массы \(m_{0}\) распался на лету на два \(\gamma\)-фотона с энергиями \(\varepsilon_{1}\) и \(\varepsilon_{2}\) (в \(K\)-системе отсчета). Найти угол \(\theta\) между направлениями разлета этих фотонов.\\
Решение. Исходя из инвариантности величины \(E^{2}-p^{2}\), запишем ее до распада в \(L\)-системе, а после распада - в \(K\)-системе:

\[
m_{0}^{2}=\left(\varepsilon_{1}+\varepsilon_{2}\right)^{2}-\left(\mathbf{p}_{1}+\mathbf{p}_{2}\right)^{2},
\]

где \(\mathbf{p}_{1}\) и \(\mathbf{p}_{2}\) - импульсы фотонов. Преобразуем правую часть этого уравнения, учитывая, что \(p_{1}=\varepsilon_{1}\) и \(p_{2}=\varepsilon_{2}\). Тогда

\[
m_{0}^{2}=2 \varepsilon_{1} \varepsilon_{2}-2 \mathbf{p}_{1} \mathbf{p}_{2}=2 \varepsilon_{1} \varepsilon_{2}(1-\cos \theta)
\]

откуда

\[
\sin \frac{\theta}{2}=\frac{m_{0}}{2 \sqrt{\varepsilon_{1} \varepsilon_{2}}}
\]

\section*{1. Движение точки в полярных координатах}
В полярных координатах \(\rho\), \(\varphi\) положение точки \(A\) на плоскости определено, если заданы ее расстояние \(\rho\) от начала отсчета \(O\) и угол φ между радиусом-вектором \(\rho\) точки и выбранным направлением \(O O^{\prime}\) - началом отсчета угловой координаты φ (рис. 1, a).

\begin{figure}[h]
\begin{center}
  \includegraphics[alt={},max width=\textwidth]{66a35da9-3faa-4823-9ca9-40799f37b3db-294_318_802_534_213}
\captionsetup{labelformat=empty}
\caption{Рис. 1}
\end{center}
\end{figure}

Введем единичные векторы - орты \(\mathbf{e}_{\rho}\) и \(\mathbf{e}_{\varphi}\), связанные с движущейся точкой \(A\) и направленные в сторону возрастания соответствующих координат \(\rho\) и \(\varphi\), как показано на рис. \(1, a\). В отличие от ортов декартовой системы координат орты \(\mathbf{e}_{\rho}\) и \(\mathbf{e}_{\varphi}\) - подвижные (при движении точки \(A\) они меняют свое направление). Найдем сразу же их производные по времени они понадобятся ниже. При движении точки \(A\) за промежуток времени \(\mathrm{d} t\) оба орта повернутся в одну сторону на один и тот же угол dφ (рис. 1, б) и получат приращения:

\[
d \mathbf{e}_{\rho}=1 \cdot d \varphi \mathbf{e}_{\varphi}, \quad d \mathbf{e}_{\varphi}=1 \cdot d \varphi\left(-\mathbf{e}_{\rho}\right) .
\]

Разделив оба выражения на \(\mathrm{d} t\), получим


\begin{equation*}
\dot{\mathbf{e}}_{\rho}=\dot{\varphi} \mathbf{e}_{\varphi}, \quad \dot{\mathbf{e}}_{\varphi}=-\dot{\varphi} \mathbf{e}_{\rho}, \tag{1}
\end{equation*}


где точка сверху над буквой означает дифференцирование по времени.

Теперь найдем скорость и ускорение точки \(A\), записав ее ра-диус-вектор ρ в виде


\begin{equation*}
\boldsymbol{\rho}=\rho \mathbf{e}_{\rho} . \tag{2}
\end{equation*}


Скорость точки v. Продифференцируем (2) по времени с учетом (1):


\begin{equation*}
\mathbf{v}=\dot{\rho} \mathbf{e}_{\rho}+\rho \dot{\varphi} \mathbf{e}_{\varphi} . \tag{3}
\end{equation*}


Отсюда видно, что проекции вектора \(\mathbf{v}\) на подвижные орты \(\mathbf{e}_{\rho}\) и \(\mathbf{e}_{\varphi}\) равны


\begin{equation*}
v_{\rho}=\dot{\rho}, \quad v_{\varphi}=\rho \dot{\varphi}, \tag{4}
\end{equation*}


а модуль вектора скорости \(v=\sqrt{\dot{\rho}^{2}+\rho^{2} \dot{\varphi}^{2}}\).\\
Ускорение точки а. Продифференцировав (3) по времени еще раз, получим

\[
\mathbf{a}=\ddot{\rho} \mathbf{e}_{\rho}+\rho \dot{\mathbf{e}}_{\rho}+\frac{\mathrm{d}}{\mathrm{~d} t}(\rho \dot{\varphi}) \mathbf{e}_{\varphi}+\rho \dot{\varphi} \dot{\mathbf{e}}_{\varphi} .
\]

Учитывая (1), после несложных преобразований найдем


\begin{equation*}
\mathbf{a}=\left(\ddot{\rho}-\rho \dot{\varphi}^{2}\right) \mathbf{e}_{\rho}+(2 \dot{\rho} \dot{\varphi}+\rho \ddot{\varphi}) \mathbf{e}_{\varphi}, \tag{5}
\end{equation*}


т. е. проекции вектора \(\mathbf{a}\) на орты \(\mathbf{e}_{\rho}\) и \(\mathbf{e}_{\varphi}\) имеют вид


\begin{equation*}
a_{\rho}=\dot{\rho}-\rho \dot{\varphi}^{2}, \quad a_{\varphi}=2 \dot{\rho} \dot{\varphi}+\rho \ddot{\varphi}=\frac{1}{\rho} \frac{d}{d t}\left(\rho^{2} \dot{\varphi}\right) . \tag{6}
\end{equation*}


Основное уравнение динамики в полярных координатах. Основное уравнение динамики \(m \mathbf{a}=\mathbf{F}\) в проекциях на подвижные орты \(\mathbf{e}_{\rho}\) и \(\mathbf{e}_{\varphi}\) легко получить сразу, воспользовавшись формулами (6):


\begin{align*}
& m\left(\ddot{\rho}-\rho \dot{\varphi}^{2}\right)=F_{\rho}, \\
& m \frac{1}{\rho} \frac{\mathrm{~d}}{\mathrm{~d} t}\left(\rho^{2} \dot{\varphi}\right)=F_{\varphi}, \tag{7}
\end{align*}


\begin{figure}[h]
\begin{center}
  \includegraphics[alt={},max width=\textwidth]{66a35da9-3faa-4823-9ca9-40799f37b3db-295_251_380_1500_82}
\captionsetup{labelformat=empty}
\caption{Рис. 2}
\end{center}
\end{figure}

где \(F_{\rho}\) и \(F_{\varphi}\) - проекции вектора \(\mathbf{F}\) на орты \(\mathbf{e}_{\rho}\) и \(\mathbf{e}_{\varphi}\) (рис. 2). На этом рисунке \(F_{\rho}<0\), а \(F_{\varphi}>0\).

\section*{2. О задаче Кеплера}
В задаче Кеплера рассматривается вопрос о движении частицы в центральном поле сил, убывающих обратно пропорционально квадрату расстояния от центра поля. Этому закону подчиняются силы гравитационного притяжения между материальными точками (или телами, обладающими сферической симметрией), а также кулоновские силы между точечными зарядами.

В таком поле потенциальная энергия частицы \(U=-\alpha / \rho\), где \(\alpha\) - постоянная, \(\rho\) - расстояние от центра поля. Рассмотрим случай, когда \(\alpha>0\), т. е. сила, действующая на частицу массы \(m\), направлена к центру поля (притяжение). Какой вид будет иметь траектория частицы в полярных координатах \(\rho(φ)\), если при \(\varphi=0 \quad \rho(0)=\rho_{0}\), а скорость частицы перпендикулярна радиусу-векто-

\begin{figure}[h]
\begin{center}
  \includegraphics[alt={},max width=\textwidth]{66a35da9-3faa-4823-9ca9-40799f37b3db-296_310_416_564_744}
\captionsetup{labelformat=empty}
\caption{Рис. 3}
\end{center}
\end{figure}

ру и равна \(v_{0}\) (рис. 3)?

Для решения этой задачи обычно используют законы сохранения энергии и момента импульса. В полярных координатах \(\rho\) и ф из этих законов следует:

\[
1 / 2 m\left(\dot{\rho}^{2}+\rho^{2} \dot{\varphi}^{2}\right)-\alpha / \rho=E, \quad m \rho^{2} \dot{\varphi}=M,
\]

где \(E\) и \(M\) - полная механическая энергия и момент импульса частицы относительно точки \(O\) - центра поля. Обе эти величины легко найти из начальных условий.

Решение данных уравнений проводят следующим образом. Сначала в первом уравнении переходят от дифференцирования по времени к дифференцированию по \(φ\) - это можно сделать с помощью второго уравнения: \(\mathrm{d} t=\left(m \rho^{2} / M\right) \mathrm{d} \varphi\). Затем разделяют переменные \(\rho\) и \(\varphi\), т. е. приводят полученное выражение к виду \(\mathrm{d} \varphi=f(\rho) \mathrm{d} \rho\). И наконец, интегрируют это уравнение с учетом начальных условий. Результат интегрирования и дает искомое решение \(\rho(φ)\).

Мы не будем здесь воспроизводить довольно громоздкий ход решения этих уравнений (при желании его можно найти почти в

любом курсе теоретической физики или механики). Ограничимся лишь анализом полученного решения, которое имеет вид


\begin{equation*}
\rho(\varphi)=\frac{\rho_{0}}{a+(1-a) \cos \varphi}, \tag{1}
\end{equation*}


где \(a=\alpha / m \rho_{0} v_{0}^{2}\).\\
Из математики известно, что уравнение (1) определяет кривую второго порядка. В зависимости от значения параметра \(a\) это может быть эллипс (окружность), парабола или гипербола.

\begin{enumerate}
  \item Сразу видно, что при \(a=1\) величина \(\rho\) не зависит от \(\varphi\), т. е. траекторией является окружность. Такую траекторию частица будет иметь при скорости \(v_{0}\), равной
\end{enumerate}


\begin{equation*}
v_{\mathrm{I}}=\sqrt{\alpha / m \rho_{0}} . \tag{2}
\end{equation*}


\begin{enumerate}
  \setcounter{enumi}{1}
  \item Для всех значений параметра \(a\), при которых \(\rho\) конечно вплоть до \(\varphi=\pi\), траектория будет иметь форму эллипса. Как следует из (1), при \(\varphi=\pi\)
\end{enumerate}

\[
\rho(\pi)=\rho_{0} /(2 a-1) .
\]

Отсюда видно, что \(\rho(\pi)\) будет конечным лишь при \(2 a>1\), т. е. при скорости \(v_{0}<v_{\text {II }}\), где


\begin{equation*}
v_{\mathrm{II}}=\sqrt{2 \alpha / m \rho_{0}} . \tag{3}
\end{equation*}


\begin{figure}[h]
\begin{center}
  \includegraphics[alt={},max width=\textwidth]{66a35da9-3faa-4823-9ca9-40799f37b3db-297_449_398_1234_68}
\captionsetup{labelformat=empty}
\caption{Рис. 4}
\end{center}
\end{figure}

\begin{enumerate}
  \setcounter{enumi}{2}
  \item Если же \(2 a=1\), т. е. \(v_{0}=v_{\mathrm{II}}\), то эллипс вырождается в параболу - частица обратно не вернется.
  \item При \(v_{0}>v_{\text {II }}\) траектория будет иметь форму гиперболы.
\end{enumerate}

Все эти случаи показаны на рис. 4. Следует обратить внимание на то, что для эллиптических орбит центр поля совпадает с одним из фокусов эллипса: при \(v_{0}<v_{\text {I }}-\) с задним фокусом, а при \(v_{0}>v_{\text {I }}-\) с передним.

Заметим, что уравнение (1) описывает, например, траектории планет Сол-

нечной системы (при этом \(\alpha=\gamma m M\), где \(M\) - масса Солнца). Применительно к движению космических аппаратов скорости \(v_{\text {I }}\) и \(v_{\text {II }}\) являются соответственно первой и второй космическими скоростями. Ясно, что их значения зависят от массы тела, являющегося источником поля.

\section*{3. Доказательство теоремы Штейнера}
Теорема: момент инерции \(I\) твердого тела относительно произвольной оси \(O\) равен моменту инерции \(I_{C}\) этого тела относительно оси \(C\), параллельной данной и проходящей через центр масс тела, плюс произведение массы \(m\) тела на квадрат расстояния \(a\) между осями:

\[
I=I_{C}+m a^{2} .
\]

Доказательство. Пусть положение \(i\)-го элемента твердого тела относительно осей \(O\) и \(C\) характеризуется векторами \(\boldsymbol{\rho}_{i}\) и \(\boldsymbol{\rho}_{i}^{\prime}\), а положение оси \(C\) относительно оси \(O\) - вектором а (рис. 5, плоскость которого перпендикулярна осям \(O\) и \(C\) ). Воспользовавшись связью между этими векторами ( \(\boldsymbol{\rho}_{i}=\boldsymbol{\rho}_{i}^{\prime}+\mathbf{a}\) ), преобразуем выражение для момента инер-

\begin{figure}[h]
\begin{center}
  \includegraphics[alt={},max width=\textwidth]{66a35da9-3faa-4823-9ca9-40799f37b3db-298_266_354_830_806}
\captionsetup{labelformat=empty}
\caption{Рис. 5}
\end{center}
\end{figure}

ции тела относительно оси \(O\) следующим образом:

\[
I=\sum m_{i} \boldsymbol{\rho}_{i}^{2}=\sum m_{i}\left(\boldsymbol{\rho}_{i}^{\prime}+\mathbf{a}\right)^{2},
\]

или

\[
I=\sum m_{i} \rho_{i}^{2}+2 \mathbf{a} \sum m_{i} \boldsymbol{\rho}_{i}^{\prime}+\sum m_{i} a^{2} .
\]

В правой части этого равенства первая сумма представляет собой момент инерции тела \(I_{C}\) относительно оси \(C\), а последняя сумма просто равна \(m a^{2}\). Остается показать, что средняя сумма равна нулю.

Пусть \(\mathbf{r}_{i}^{\prime}\) - радус-вектор \(i\)-го элемента тела относительно центра масс, тогда относительно последнего суммарный вектор, со-

гласно (3.8), \(\sum m_{i} \mathbf{r}_{i}^{\prime}=0\). Но \(\boldsymbol{\rho}_{i}^{\prime}\) - это составляющая вектора \(\mathbf{r}_{i}^{\prime}\), перпендикулярная осям \(O\) и \(C\). Отсюда ясно, что если суммарный вектор равен нулю, то сумма его составляющих в плоскости, перпендикулярной осям \(O\) и \(C\), также равна нулю, т. е. \(\sum m_{i} \rho_{i}^{\prime}=0\). Теорема, таким образом, доказана.

\section*{4. Греческий алфавит}
\begin{center}
\begin{tabular}{|l|l|l|}
\hline
А, \(\alpha\) - альфа & I, г - йота & P, \(\rho\) - po \\
\hline
B, \(\beta\) - бета & K, \(x\) - каппа & Σ, \(\sigma\) - сигма \\
\hline
Г, \(\gamma\) - гамма & \(\Lambda, \lambda\) - ламбда & T, \(\tau\) - тау \\
\hline
\(\Delta, \delta\) - дельта & M, μ - мю & Y, v - ипсилон \\
\hline
Ε, \(ε\) - эпсилон & N, \(v\) - ню & Ф, φ - фи \\
\hline
\(\mathrm{Z}, \zeta\) - дзета & \(\Xi, \xi\) - кси & Х, \(\chi\) - хи \\
\hline
H, η - эта & О, о - омикрон & Ч, у - пси \\
\hline
\(\Theta, \theta, \vartheta\) - тета & П, π - пи & \(\Omega, \omega\) - омега \\
\hline
\end{tabular}
\end{center}

\section*{5. Основные единицы СИ в механике}
Секунда - это промежуток времени, в течение которого совершается 9192631770 колебаний электромагнитного излучения, соответствующего переходу между двумя определенными сверхтонкими уровнями основного состояния атома цезия-133. Эталон времени и частоты состоит из атомно-лучевой трубки с пучком атомов цезия и радиотехнического устройства, которое дает набор электрических сигналов фиксированной частоты. Секунда приблизительно равна \(1 / 86400\) средних солнечных суток.

Метр - это длина, равная 1650 763,73 длин волн в вакууме оранжевой линии атома криптона-86 (линии, соответствующей переходу между уровнями \(2 p_{10}\) и \(5 d_{5}\) данного атома). Эталон для воспроизведения метра представляет собой комплекс аппаратуры, включающей интерферометры для точного измерения длин. Метр приблизительно равен \(1 / 40000000\) доле длины земного меридиана.

Килограмм - это масса платино-иридиевого эталона, хранящегося в Международном бюро мер и весов (в Севре, близ Парижа). Масса эталона близка к массе 1 дм \({ }^{3}\) чистой воды при \(4{ }^{\circ} \mathrm{C}\).

\section*{6. Формулы алгебры и тригонометрии}
Корни квадратного уравнения \(a x^{2}+b x+c=0\) :

\[
x_{1,2}=\frac{-b \pm \sqrt{b^{2}-4 a c}}{2 a} .
\]

Некоторые приближенные формулы. Если \(\alpha \ll 1\), то

\[
\begin{array}{ll}
(1+\alpha)^{n} \approx 1+n \alpha ; & \sin \alpha \approx \alpha ; \\
\mathrm{e}^{\alpha} \approx 1+\alpha ; & \cos \alpha \approx 1-\alpha^{2} / 2 ; \\
\ln (1+\alpha) \approx \alpha ; & \operatorname{tg} \alpha \approx \alpha .
\end{array}
\]

Основные тригонометрические формулы:

\[
\begin{array}{ll}
\sin ^{2} \alpha+\cos ^{2} \alpha=1 ; & \\
\sin \alpha=1 / \sqrt{1+\operatorname{ctg}^{2} \alpha} ; & \operatorname{tg} 2 \alpha=\frac{2 \operatorname{tg} \alpha}{1-\operatorname{tg}^{2} \alpha} ; \\
\cos \alpha=1 / \sqrt{1+\operatorname{tg}^{2} \alpha} ; & \sin ^{2}\left(\frac{\alpha}{2}\right)=\frac{1-\cos \alpha}{2} ; \\
\sin 2 \alpha=2 \sin \alpha \cos \alpha ; & \cos ^{2}\left(\frac{\alpha}{2}\right)=\frac{1+\cos \alpha}{2} ; \\
\cos 2 \alpha=\cos ^{2} \alpha-\sin ^{2} \alpha ; & \\
& \sin (\alpha \pm \beta)=\sin \alpha \cos \beta \pm \cos \alpha \sin \beta ; \\
& \cos (\alpha \pm \beta)=\cos \alpha \cos \beta \mp \sin \alpha \sin \beta .
\end{array}
\]

\section*{7. Таблица производных и интегралов}
\begin{center}
\begin{tabular}{|l|l|l|l|}
\hline
Функция & Производная & Функция & Производная \\
\hline
\(1 / x\) & \(-1 / x^{2}\) & \(\sin x\) & \(\cos x\) \\
\hline
\(\sqrt{x}\) & \(1 /(2 \sqrt{x})\) & \(\cos x\) & \(-\sin x\) \\
\hline
\(x^{n}\) & \(x^{n-1}\) & \(\operatorname{tg} x\) & \(1 / \cos ^{2} x\) \\
\hline
\(\mathbf{e}^{n x}\) & \(n \mathrm{e}^{n x}\) & \(\operatorname{ctg} x\) & \(-1 / \sin ^{2} x\) \\
\hline
\(a^{x}\) & \(a^{x} \ln a\) & \(\arcsin x\) & \(1 / \sqrt{1-x^{2}}\) \\
\hline
\(\ln x\) & \(1 / x\) & \(\arccos x\) & \(-1 / \sqrt{1-x^{2}}\) \\
\hline
\(\frac{u(x)}{v(x)}\) & \(\frac{v u_{x}^{\prime}-v_{x}^{\prime} u}{v^{2}}\) & \(\operatorname{arctg} x\) & \(1 /\left(1+x^{2}\right)\) \\
\hline
\end{tabular}
\end{center}

\section*{8. Некоторые сведения о векторах}
Скалярное произведение векторов:

\[
\mathbf{a b}=\mathbf{b a}=a b \cos \alpha ; \quad \mathbf{a}(\mathbf{b}+\mathbf{c})=\mathbf{a b}+\mathbf{a c} .
\]

Векторное произведение векторов:

\[
\begin{gathered}
{[\mathbf{a b}]=-[\mathbf{b a}] ; \quad|[\mathbf{a b}]|=a b \sin \alpha ;} \\
{[\mathbf{a}, \mathbf{b}+\mathbf{c}]=[\mathbf{a b}]+[\mathbf{a c}] .}
\end{gathered}
\]

Смешанное, или векторно-скалярное, произведение трех векторов является скаляром и численно равно объему параллелепипеда, построенного на этих векторах:

\[
\mathbf{a}[\mathbf{b c}]=\mathbf{b}[\mathbf{c a}]=\mathbf{c}[\mathbf{a b}] ; \quad \mathbf{a}[\mathbf{b c}]=-\mathbf{b}[\mathbf{a c}]=-\mathbf{a}[\mathbf{c b}] .
\]

Двойное векторное произведение:

\[
[\mathbf{a}[\mathbf{b c}]]=\mathbf{b}(\mathbf{a c})=-\mathbf{c}(\mathbf{a b}) .
\]

Произведение векторов. Если

\[
\mathbf{a}=a_{1} \mathbf{e}_{1}+a_{2} \mathbf{e}_{2}+a_{3} \mathbf{e}_{3}, \quad \mathbf{b}=b_{1} \mathbf{e}_{1}+b_{2} \mathbf{e}_{2}+b_{3} \mathbf{e}_{3},
\]

где \(\mathbf{e}_{1}, \mathbf{e}_{2}, \mathbf{e}_{3}\) - координатные орты (взаимно перпендикулярные и образующие правую тройку), то

\[
\begin{gathered}
\mathbf{a}=a_{1} b_{1}+a_{2} b_{2}+a_{3} b_{3} ; \\
{[\mathbf{a} \mathbf{b}]=\left|\begin{array}{lll}
\mathbf{e}_{1} & \mathbf{e}_{2} & \mathbf{e}_{3} \\
a_{1} & a_{2} & a_{3} \\
b_{1} & b_{2} & b_{3}
\end{array}\right|=} \\
=\left(a_{2} b_{3}-a_{3} b_{2}\right) \mathbf{e}_{1}+\left(a_{3} b_{1}-a_{1} b_{3}\right) \mathbf{e}_{2}+\left(a_{1} b_{2}-a_{2} b_{1}\right) \mathbf{e}_{3} .
\end{gathered}
\]

Правила дифференцирования векторов, зависящих от некоторой скалярной переменной \(t\) :

\[
\begin{array}{ll}
\frac{\mathrm{d}}{\mathrm{~d} t}(\mathbf{a}+\mathbf{b})=\frac{\mathrm{d} \mathbf{a}}{\mathrm{~d} t}+\frac{\mathrm{d} \mathbf{b}}{\mathrm{~d} t} ; & \frac{\mathrm{d}}{\mathrm{~d} t}(\mathbf{a b})=\frac{\mathrm{d} \mathbf{a}}{\mathrm{~d} t} \mathbf{b}+\mathbf{a} \frac{\mathrm{d} \mathbf{b}}{\mathrm{~d} t} ; \\
\frac{\mathrm{d}}{\mathrm{~d} t}(\alpha \mathbf{a})=\frac{\mathrm{d} \alpha}{\mathrm{~d} t} \mathbf{a}+\alpha \frac{\mathrm{d} \mathbf{a}}{\mathrm{~d} t} ; & \frac{\mathrm{d}}{\mathrm{~d} t}[\mathbf{a} \mathbf{b}]=\left[\frac{\mathrm{d} \mathbf{a}}{\mathrm{~d} t} \mathbf{b}\right]+\left[\mathbf{a} \frac{\mathrm{d} \mathbf{b}}{\mathrm{~d} t}\right] .
\end{array}
\]

\section*{9. Единицы механических величин в СИ и системе СГС}
\begin{center}
\begin{tabular}{|l|l|l|l|}
\hline
\multirow{2}{*}{Величина} & \multicolumn{2}{|c|}{Единица} & \multirow{2}{*}{Отношение ед. СИ ед. СГС} \\
\hline
 & СИ & СГС &  \\
\hline
Длина & M & CM & \(10^{2}\) \\
\hline
Время & c & c & 1 \\
\hline
Угол & рад & рад & 1 \\
\hline
Площадь & M \({ }^{2}\) & CM \({ }^{2}\) & \(10^{4}\) \\
\hline
Объем & M \(^{3}\) & см \({ }^{3}\) & \(10^{6}\) \\
\hline
Скорость & m/c & см/с & \(10^{2}\) \\
\hline
Ускорение & \(\mathrm{m} / \mathrm{c}^{2}\) & см/c \({ }^{2}\) & \(10^{2}\) \\
\hline
Частота колебаний & Гц & Гц & 1 \\
\hline
Круговая частота колебаний & \(\mathrm{c}^{-1}\) & \(\mathrm{c}^{-1}\) & 1 \\
\hline
Угловая скорость & рад/с & рад/с & 1 \\
\hline
Угловое ускорение & рад/c \({ }^{2}\) & рад/c \({ }^{2}\) & 1 \\
\hline
Macca & кг & г & \(10^{3}\) \\
\hline
Плотность & кг/м \({ }^{3}\) & г/см \({ }^{3}\) & \(10^{-3}\) \\
\hline
Сила & H & дин & \(10^{5}\) \\
\hline
Давление & Па & дин/см \({ }^{2}\) & 10 \\
\hline
Работа, энергия & Дж & эрг & \(10^{7}\) \\
\hline
Мощность & Вт & эрг/с & \(10^{7}\) \\
\hline
Импульс & кг•м/с & \(\boldsymbol{\Gamma} \boldsymbol{\cdot} \mathbf{c m} \boldsymbol{/} \mathbf{c}\) & \(10^{5}\) \\
\hline
Импульс силы & H•c & дин•с & \(10^{5}\) \\
\hline
Момент силы & \(\mathbf{H} \cdot \mathbf{m}\) & дин•см & \(10^{7}\) \\
\hline
Момент импульса & \(\boldsymbol{\kappa} \boldsymbol{\Gamma} \boldsymbol{\cdot} \mathbf{M}^{\mathbf{2}} \boldsymbol{/} \mathbf{c}\) & г.см \({ }^{2}\) /с & \(10^{7}\) \\
\hline
Момент инерции & \(\boldsymbol{\kappa \Gamma} \boldsymbol{\cdot} \mathbf{M}^{\mathbf{2}}\) & \(\boldsymbol{\Gamma} \boldsymbol{\cdot} \mathbf{C M}^{\mathbf{2}}\) & \(10^{7}\) \\
\hline
Импульс момента силы & \(\mathbf{H} \boldsymbol{\cdot} \mathbf{m} \boldsymbol{\cdot} \mathbf{c}\) & дин ⋅ \(\cdot \mathrm{cm} \cdot \mathrm{c}\) & \(10^{7}\) \\
\hline
\end{tabular}
\end{center}

\section*{10. Десятичные приставки к названиям единиц}
\begin{center}
\begin{tabular}{|l|l|l|l|}
\hline
T - & тера ( \(10^{12}\) ) & c - & санти ( \(10^{-2}\) ) \\
\hline
Г - & гига ( \(10^{9}\) ) & M - & милли ( \(10^{-3}\) ) \\
\hline
M - & мега ( \(10^{6}\) ) & мк - & микро ( \(10^{-6}\) ) \\
\hline
к - & кило ( \(10^{3}\) ) & H - & нано ( \(10^{-9}\) ) \\
\hline
г - & гекто ( \(10^{2}\) ) & п - & пико ( \(10^{-12}\) ) \\
\hline
да - & дека ( \(10^{1}\) ) & ф - & фемто ( \(10^{-15}\) ) \\
\hline
д - & деци ( \(10^{-1}\) ) & a - & атто ( \(10^{-18}\) ) \\
\hline
\end{tabular}
\end{center}

Примеры: нм - нанометр ( \(10^{-9}\) м),\\
\(\kappa \mathrm{H}\) - килоньютон ( \(10^{3} \mathrm{H}\) ),\\
МэВ - мегаэлектрон-вольт ( \(10^{6}\) эВ),\\
мкВт - микроватт ( \(10^{-6}\) Вт).

\section*{11. Некоторые внесистемные единицы}
Длина ...... 1 А (ангстрем) \(=10^{-10}\) м\\
1 a. е. (астрономическая единица) \(\approx 1,5 \cdot 10^{11}\) м\\
1 св. год (световой год) \(\approx 0,95 \cdot 10^{16}\) м\\
1 пк (парсек) \(\approx 3,1 \cdot 10^{16}\) м\\
Время ....... 1 сут (сутки) \(=86400\) с\\
\(1 г(\) год \()=3,11 \cdot 10^{7}\) с\\
Масса . . . . . . 1 а.е.м. (атомная единица массы) = \(=1,66 \cdot 10^{-27}\) кг

\[
1 \text { т (тонна) }=10^{3} \text { кг }
\]

Сила . . . . . . 1 кгс (килограмм-сила) \(=9,81 \mathrm{H}\)\\
Давление ..... 1 бар \(=10^{5}\) Па (точно)\\
1 атм \(=1,013 \cdot 10^{5}\) Па\\
1 мм рт. ст. (торр.) = 133 Па\\
Энергия ...... 1 эВ \(=1,60 \cdot 10^{-19}\) Дж\\
1 Вт.ч \(=3,6 \cdot 10^{3}\) Дж\\
Мощность . . . . 1 л. с. (лошадиная сила) \(=736\) Вт

\section*{12. Астрономические величины}
\begin{center}
\begin{tabular}{|l|l|l|l|}
\hline
Космическое тело & Масса, кг & Средний радиус, м & Средний радиус орбиты, м \\
\hline
Солнце & \(1,99 \cdot 10^{30}\) & \(6,96 \cdot 10^{8}\) & - \\
\hline
Земля & \(5,98 \cdot 10^{24}\) & \(6,37 \cdot 10^{6}\) & \(1,50 \cdot 10^{11}\) \\
\hline
Луна & \(7,35 \cdot 10^{22}\) & \(1,74 \cdot 10^{6}\) & \(3,84 \cdot 10^{8}\) \\
\hline
\end{tabular}
\end{center}

\section*{13. Физические постоянные}
\begin{center}
\begin{tabular}{|l|l|}
\hline
Скорость света в вакууме & \(c=2,998 \cdot 10^{8} \mathrm{~m} / \mathrm{c}\) \\
\hline
Гравитационная постоянная & \(\gamma=6,67 \cdot 10^{-11} \mathrm{M}^{3} /\left(к г \cdot \mathrm{c}^{2}\right)\) \\
\hline
Ускорение свободного падения & \(g=9,807 \mathrm{~m} / \mathrm{c}^{2}\) \\
\hline
Постоянная Авогадро & \(N_{\text {А }}=6,022 \cdot 10^{23}\) моль \(^{-1}\) \\
\hline
Элементарный заряд & \(e=\left\{\begin{array}{l}1,602 \cdot 10^{-19} \text { Кл } \\ 4,80 \cdot 10^{-10} \text { СГСЭ }\end{array}\right.\) \\
\hline
Масса покоя электрона & \(m_{e}=\left\{\begin{array}{l}0,911 \cdot 10^{-30} \text { кг } \\ 0,511 \text { МэВ }\end{array}\right.\) \\
\hline
Удельный заряд электрона & \(\frac{e}{m_{e}}=\left\{\begin{array}{l}1,76 \cdot 10^{11} \text { Кл } / \text { кг } \\ 5,27 \cdot 10^{17} \text { СГСЭ }\end{array}\right.\) \\
\hline
Масса покоя протона & \(m_{p}=1,673 \cdot 10^{-27}\) кг \\
\hline
Атомная единица массы & 1 а.е. м. \(=\left\{\begin{array}{l}1,660 \cdot 10^{-27} \\ 931,4 \text { МэВ }\end{array}\right.\) кг \\
\hline
\end{tabular}
\end{center}

Амплитуда 200

\begin{itemize}
  \item затухающих колебаний 211
\end{itemize}

Биения 209

Вектор аксиальный 18

\begin{itemize}
  \item перемещения 9
  \item полярный 18
\end{itemize}

Вес 47\\
Время релаксации 212

\begin{itemize}
  \item собственное 237
\end{itemize}

Вязкость 140

Гироскоп 185\\
Градиент 104

Движение апериодическое 213

\begin{itemize}
  \item вращательное 17
  \item плоское 21
  \item поступательное 16
  \item тела переменной массы 82
\end{itemize}

Декремент затухания логарифмический 213\\
Диаграмма векторная 207

\begin{itemize}
  \item импульсов векторная 131
  \item пространства - времени 259
\end{itemize}

Динамика гармонических колебаний 202\\
Длина приведенная 204

\begin{itemize}
  \item собственная 240
\end{itemize}

Добротность 213\\
Жидкость идеальная 136\\
Задача Кеплера 295\\
Закон взаимосвязи массы и энергии 271

\begin{itemize}
  \item всемирного тяготения 45,46
  \item Гука 47
  \item Ньютона второй 42
  \item 
  \begin{itemize}
    \item первый 37
  \end{itemize}
  \item 
  \begin{itemize}
    \item третий 44
  \end{itemize}
  \item сохранения импульса 73,74
  \item 
  \begin{itemize}
    \item момента импульса 165
  \end{itemize}
  \item 
  \begin{itemize}
    \item энергии 111, 121
  \end{itemize}
\end{itemize}

Законы сохранения 68-70\\
Замедление времени 236, 237\\
Изотропность пространства 69\\
Импульс 70

\begin{itemize}
  \item момента силы 160,165
  \item релятивистский 265
  \item силы 71
  \item системы 72
\end{itemize}

Интервал 251

\begin{itemize}
  \item времениподобный 251
  \item пространственноподобный 251
  \item светоподобный 251
\end{itemize}

Килограмм (эталон) 298\\
Кинематика твердого тела 16

\begin{itemize}
  \item гармонических колебаний 200
  \item точки 9
\end{itemize}

Колебания вынужденные 214

\begin{itemize}
  \item гармонические 200
  \item затухающие 211
  \item малые 203, 204
  \item свободные 204
\end{itemize}

Координата дуговая 13\\
Кривые резонансные 216\\
Линия тока 136\\
Лоренц-фактор 276\\
Macca 41

\begin{itemize}
  \item покоя 264
  \item приведенная 127
  \item релятивистская 264
\end{itemize}

Маятник математический 202

\begin{itemize}
  \item оборотный 204
  \item физический 203
\end{itemize}

Метр (эталон) 298\\
Момент гироскопический 188

\begin{itemize}
  \item импульса 157
  \item 
  \begin{itemize}
    \item относительно оси 163
  \end{itemize}
  \item 
  \begin{itemize}
    \item системы 163
  \end{itemize}
  \item 
  \begin{itemize}
    \item собственный 171
  \end{itemize}
  \item инерции 176
  \item силы 158
  \item 
  \begin{itemize}
    \item относительно оси 163
  \end{itemize}
  \item угловой 157
\end{itemize}

Мощность 97\\
Напряженность поля 106

Одновременность 232, 247\\
Однородность времени 38

\begin{itemize}
  \item пространства 37
\end{itemize}

Опыт Майкельсона 227\\
Оси тела главные 184

\begin{itemize}
  \item 
  \begin{itemize}
    \item свободные 184
  \end{itemize}
\end{itemize}

Ось вращения мгновенная 23\\
Парадокс близнецов 239\\
Параметр прицельный 133\\
Пара сил 170\\
Период биений 209

\begin{itemize}
  \item колебаний 200\\
— - затухающих 211\\
Плечо импульса 157
  \item пары 170
  \item силы 158
\end{itemize}

Поверхность эквипотенциальная 105\\
Поле стационарное 98\\
Порог 134, 135\\
Постулаты Эйнштейна 229\\
Потенциал 107\\
Потенциальный барьер 112\\
Преобразования Галилея 39

\begin{itemize}
  \item Лоренца 245
  \item импульса и энергии релятивистской 275
  \item скорости 25, 26
  \item 
  \begin{itemize}
    \item релятивистской 252
  \end{itemize}
  \item ускорения 25, 27
\end{itemize}

Прецессия гироскопа 185\\
Принцип дальнодействия 44,226

\begin{itemize}
  \item относительности Галилея 38
  \item 
  \begin{itemize}
    \item Эйнштейна 229
  \end{itemize}
  \item суперпозиции 43,106
  \item эквивалентности 56
\end{itemize}

Приращение 6, 101\\
Работа 93

\begin{itemize}
  \item гравитационной силы 95
  \item однородной силы тяжести 96
  \item при вращении твердого тела 180
  \item упругой силы 94
\end{itemize}

Резонанс 215, 216\\
Секунда (эталон) 298\\
Сила 42, 43

\begin{itemize}
  \item внешняя 72
  \item внутренняя 71
  \item гироскопическая 188
  \item гравитационная 46
  \item диссипативная 117, 118
  \item инерции центробежная 53
  \item квазиупругая 204
  \item консервативная 98,117
  \item Кориолиса 53
  \item кулоновская 46
  \item Лоренца 286
  \item нормальная 50
  \item потенциальная 98
  \item равнодействующая 174
  \item реактивная 83
  \item сопротивления 48
  \item сторонняя 109, 124
  \item тангенциальная 50
  \item трения 47
  \item тяжести 47
  \item упругая 47
  \item центральная 99
\end{itemize}

Симметрия пространства и времени 37\\
Синхронизация часов 230,231\\
Система замкнутая 73

\begin{itemize}
  \item консервативная 121
  \item отсчета 7
  \item 
  \begin{itemize}
    \item гелиоцентрическая 37
  \end{itemize}
  \item 
  \begin{itemize}
    \item инерциальная 36
  \end{itemize}
  \item 
  \begin{itemize}
    \item неинерциальная 37
  \end{itemize}
  \item центра масс 81
\end{itemize}

Скорости космические 149\\
Скорость точки 9

\begin{itemize}
  \item угловая 18
  \item 
  \begin{itemize}
    \item прецессии 185
  \end{itemize}
  \item центра масс 78
\end{itemize}

Сложение колебаний одного направления 207

\begin{itemize}
  \item взаимно перпендикулярных колебаний 209
\end{itemize}

Событие 231\\
Сокращение лоренцево 240,249\\
Столкновение абсолютно неупругое 128

\begin{itemize}
  \item 
  \begin{itemize}
    \item упругое 129
  \end{itemize}
  \item лобовое 129
  \item нелобовое 130
  \item неупругое 133-134
  \item экзоэнергетическое 134
  \item эндоэнергетическое 134
\end{itemize}

Схема энергетическая ядерной реакции 289

Теорема Кёнига 125\\
Теорема Штейнера 177,297\\
Течение ламинарное 140

\begin{itemize}
  \item турбулентное 140
\end{itemize}

Трубка тока 136\\
Убыль 6, 101\\
Уравнение Бернулли 138

\begin{itemize}
  \item вынужденных колебаний 214
  \item гармонического осциллятора 201
  \item движения центра масс 79
  \item динамики вращения твердого тела 177
  \item затухающих колебаний 211
  \item Мещерского 83
  \item моментов 157, 158
  \item 
  \begin{itemize}
    \item в Ц-системе 172, 173
  \end{itemize}
  \item неразрывности струи 137
  \item ньютоновской динамики основное 48
  \item релятивистской динамики основное 267
\end{itemize}

Ускорение 10

\begin{itemize}
  \item кориолисово 27
  \item нормальное 15
  \item осестремительное 27
  \item тангенциальное 15
  \item угловое 19
\end{itemize}

Условия начальные \(10,48,178,201\)

\begin{itemize}
  \item равновесия твердого тела 175
\end{itemize}

Фаза 200\\
Фигуры Лиссажу 210\\
Формула Пуазейля 142

\begin{itemize}
  \item Торричелли 139
\end{itemize}

Центр качания 204

\begin{itemize}
  \item кривизны 14
  \item масс 77\\
\(Ц\)-система 81\\
Частица с нулевой массой 274\\
Частота затухающих колебаний 211
  \item резонансная 216
  \item собственная 211
\end{itemize}

Часы «световые» 236\\
Число Рейнольдса 141\\
Энергия внутренняя механическая 125

\begin{itemize}
  \item вынужденных колебаний 217
  \item гармонического осциллятора 205
  \item затухающих колебаний 212
  \item кинетическая 108, 119, 125
  \item 
  \begin{itemize}
    \item твердого тела 178, 182
  \end{itemize}
  \item 
  \begin{itemize}
    \item релятивистская 269,270
  \end{itemize}
  \item механическая 120,124
  \item покоя 271
  \item полная механическая 110, 122-124
  \item пороговая 290
  \item потенциальная 100, 112
  \item 
  \begin{itemize}
    \item «внешняя» 116
  \end{itemize}
  \item 
  \begin{itemize}
    \item собственная 112-114
  \end{itemize}
  \item собственная механическая 119, 120
  \item ядерной реакции 290
\end{itemize}

Эффект гироскопический 185

\begin{itemize}
  \item Доплера 257
  \item 
  \begin{itemize}
    \item поперечный 259
  \end{itemize}
  \item 
  \begin{itemize}
    \item продольный 258
  \end{itemize}
\end{itemize}

Учебное электронное издание

МЕХАНИКА. ОСНОВНЫЕ ЗАКОНЫ

Минимальные системные требования определяются соответствующими требованиями программы Adobe Reader версии не ниже 10-й для операционных систем Windows, Android, iOS, Windows Phone и BlackBerry

\begin{itemize}
  \item 
\end{itemize}

\begin{itemize}
  \item 
\end{itemize}

\begin{itemize}
  \item 
\end{itemize}

\begin{itemize}
  \item 
\end{itemize}

\begin{itemize}
  \item 
\end{itemize}

\begin{itemize}
  \item 
\end{itemize}

\begin{itemize}
  \item 
\end{itemize}

\begin{itemize}
  \item 
\end{itemize}

\begin{itemize}
  \item 
\end{itemize}

\begin{itemize}
  \item 
\end{itemize}

\begin{itemize}
  \item 
\end{itemize}

\begin{itemize}
  \item 
\end{itemize}

\begin{itemize}
  \item 
\end{itemize}

\begin{itemize}
  \item 
\end{itemize}

\begin{itemize}
  \item 
\end{itemize}

\begin{itemize}
  \item 
\end{itemize}

\begin{itemize}
  \item 
\end{itemize}

\begin{itemize}
  \item 
\end{itemize}

\begin{itemize}
  \item 
\end{itemize}

\begin{itemize}
  \item 
\end{itemize}

\begin{itemize}
  \item 
\end{itemize}

\begin{itemize}
  \item 
\end{itemize}

\begin{itemize}
  \item 
\end{itemize}

\begin{itemize}
  \item 
\end{itemize}

\begin{itemize}
  \item 
\end{itemize}


\end{document}